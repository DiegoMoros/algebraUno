\ejercicio Sea $A = \set{a,b,c,d,e,f,g,h}$. Para cada uno de los siguientes gráficos describit por
extensión la relación en $A$ que representa y determinar si es \textit{reflexiva, simétrica, antisimétrica o transitiva}.
\begin{enumerate}
	\item  \textit{Reflexiva}: $(x, x) \en \relacion \paratodo x \en A$ o $x \relacion x.\, \paratodo x \en A$. Gráficamente,
	      cada elemento tiene que tener un bucle.
	      \begin{tikzpicture}[baseline=0, >=Latex, draw=Aquamarine]
		      \node[](a) {$\bullet$} edge [in=0,out=90,loop] ();
		      \node[] at (a.west) {$x$};
	      \end{tikzpicture}

	\item  \textit{Simétrica}: $(x, y) \en \relacion$, entonces el par $(y, x) \en \relacion$, también si $\paratodo x, y \en A, x\relacion y \entonces y \relacion x$.
	      Gráficamente tiene que haber un ida y vuelta en cada elemento de la relación.
	      \begin{tikzpicture}[baseline=-10, >=Latex, draw=Aquamarine]
		      \node[] (x) {$\bullet$};
		      \node[] at (x.west) {$x$};
		      \node[below right =0.2 of x] (y) {$\bullet$};
		      \node[] at (y.south east) {$y$};
		      \draw[->, bend left=1cm] (x.center) to (y);
		      \draw[->, bend left=1cm] (y.center) to (x);
	      \end{tikzpicture}

	\item  \textit{Antisimétrica}: $(x, y) \en \relacion$, con $x \distinto y$ entonces el par $(y, x) \notin \relacion$, también se puede pensar
	      como $\paratodo x,y \en A, x\relacion y$ e $ y \relacion x \entonces x = y$. Gráficamente \textbf{no} tiene que haber ningún ida y vuelta en el gráfico.
	      Solo en una dirección.
	      \begin{tikzpicture}[baseline=-10, >=Latex, draw=Aquamarine]
		      \node[] (x) {$\bullet$};
		      \node[] at (x.west) {$x$};
		      \node[below right =0.2 of x] (y) {$\bullet$};
		      \node[] at (y.south east) {$y$};
		      \draw[->] (x.center) to (y.center);
	      \end{tikzpicture}
	\item  \textit{Transitiva}: Para toda terna $x, y, z \en A$ tales que $(x, y) \en \relacion$ e $(y,z) \en \relacion$, se tiene que $(x, z) \en \relacion$.
	      Otra manera sería si $\paratodo x, y, z \en A$, $x \relacion y$ e $y \relacion z \entonces x\relacion z$. Gráficamente tiene que haber flecha directa
	      entre las puntas de cualquier camino que vaya por más de dos nodos.
	      \begin{tikzpicture}[baseline=0, >=Latex, draw=Aquamarine]
		      \node[] (x) {$\bullet$};
		      \node[] at (x.north east) {$x$};
		      \node[below right = 0.6cm of x] (y) {$\bullet$};
		      \node[] at (y.north east) {$y$};
		      \node[below right = 0.6cm of y] (z) {$\bullet$};
		      \node[] at (z.north east) {$z$};

		      \draw[->, bend left] (x.center) to (y.center);
		      \draw[->, bend left] (y.center) to (z.center);
		      \draw[->,magenta, bend right]
		      (x.center)
		      to node[midway,below left] {atajo}
		      (z.center);
	      \end{tikzpicture}
\end{enumerate}

\begin{enumerate}[label=\roman*)]

	\item
	      \begin{minipage}{0.37\textwidth}
		      \diecinuevei
	      \end{minipage}
	      \begin{minipage}{0.5\textwidth}
		      \begin{itemize}
			      \item No es reflexiva, porque no hay bucles en todos los vértices, en particular $a \norelacion a$.
			      \item No es simétrica, porque $d \norelacion c$.
			      \item No es antisimétrica, porque $a \relacion b$ y $b \relacion a$ con $a \neq b$.
			      \item No es transitiva, porque $c \relacion h$ y $h \relacion g$, pero $c \norelacion h$.
		      \end{itemize}
	      \end{minipage}

	\item \hacer

	\item \hacer

	\item
	      \begin{minipage}{0.37\textwidth}
		      \diecinueveiv
	      \end{minipage}
	      \begin{minipage}{0.5\textwidth}
		      \begin{itemize}
			      \item Reflexiva, porque hay bucles en todos los elementos de $A$.
			      \item Es simétrica, porque hay ida y vuelta en todos los pares de vértices.
			      \item No es antisimétrica, porque $a \relacion b$ y $b \relacion a$ con $a \neq b$.
			      \item Es transitiva, porque hay \textit{atajos} en todas las relaciones de ternas.
		      \end{itemize}
	      \end{minipage}
\end{enumerate}
