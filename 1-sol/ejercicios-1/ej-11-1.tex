\ejercicio Hallar contraejemplos para mostrar que las siguientes proposiciones son falsas:
\begin{enumerate}[label=\roman*)]
	\item $\paratodo a \en \naturales,\, \frac{a-1}{a}$ no es un número entero.\\
	      La proposición es falsa, dado que si $a = 1 \entonces \frac{\green{1} - 1}{1} = 0 \en \enteros$

	\item $\paratodo x, y \en \reales$ con $x, y$ positivos, $\sqrt{x+y} = \sqrt{x} + \sqrt{y}$.\\
	      La proposición es falsa, dado que si.\\
	      \[
		      \llaves{c}{
			      x = 4.\\
			      y = 4
		      } \to \sqrt{4+4} = \sqrt{8} = 2\sqrt{2} \stacktext{?}{\leftrightarrow} \sqrt{4} + \sqrt{4} = 2 + 2 = 4
	      \]

	\item $\paratodo x \en \reales,\, x^2 > 4 \entonces x > 2$.\\
	      La  proposición es falsa, dado que si $x = -3$, queda $9 > 4 \entonces -3 > 2$, lo cual es falso,
\end{enumerate}
