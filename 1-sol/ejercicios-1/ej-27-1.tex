\ejercicio
Sean $A = \set{n \en \naturales \talque n \leq 92}$ y
$\relacion$ la relación en $A$ definida por
$x \relacion y \sisolosi x^2 - y^2 = 93x - 93y$
\begin{enumerate}[label=\roman*)]
	\item Probar que $\relacion$ es una relación de equivalencia. ¿Es antisimétrica?
	\item Hallar la clase de equivalencia de cada $x \en A$. Deducir cuántas clases de equivalencia \textbf{distintas} determina la relación $\relacion$
\end{enumerate}

\separadorCorto

\begin{enumerate}[label=\roman*)]
	\item Primero acomodo la condición de la relación:\\
	      $x^2 - y^2 = 93x - 93y \flecha{acomodo}
		      \llave{rrcl}{
			      \cyan{*^1} & x & = & y  \\
			      & & \o &\\
			      \cyan{*^2} & x + y & = & 93
		      }$\\
	      Para ser relación de equivalencia es necesario que:\\
	      $\relacion \to
		      \llaves{cl}{
			      R: & \text{Reflexiva} \to x \relacion x \sisolosi x \llamada{1} = x  \Tilde\\
			      S: & \text{Simétrica} \to
			      \llave{l}{
				      x \relacion y \sisolosi x + y\llamada{1}I = 93\\
				      y \relacion x \sisolosi y + x \llamada{1}I = 93  \Tilde
			      }\\
			      T: & \text{Transitiva} \to
			      \llave{l}{
				      x \relacion y \sisolosi x \llamada{1}I = 93 - y\\
				      y \relacion z \sisolosi y \llamada{1}I = 93 - z \\
				      \flecha{resto}[m.a.m] x - y = -y + z \to x \llamada{1} = z \sisolosi x \relacion z
			      }\\
		      }$\\
	      La $\relacion$ no es antisimétrica, como contraejemplo se ve que $1 \relacion 92$ y $92 \relacion 1$ con $1 \distinto 92$

	\item
	      \begin{minipage}{0.2\textwidth}
		      \veintisiete
	      \end{minipage}
	      \begin{minipage}{0.7\textwidth}
		      A priori no sé como encontrar las clases de equivalencia, pero solo buscando la relación del $1$ con algún número (excepto el mismo) veo que únicamente
		      se puede relacionar con el $92$ por la condición de la relación. De ahí se pueden inferir que todas las clases van a ser así chiquitas.Las clases de equivalencia :

		      $\llave{ccccc}{
				      \clase{1} & = & \clase{92} & = & \set{1, 92}\\
				      \clase{2} & = & \clase{91} & = & \set{2, 91}\\
				      \vdots  &   & \vdots   &   & \vdots\\
				      \clase{45} & = & \clase{47} & = & \set{45, 47}\\
				      \clase{46} & = & \set{46}
			      }$\\
		      Hay entonces 46 clases. $A = \set{\clase{1},\, \clase{2},\, \dots,\, \clase{45},\, \clase{46}}$
	      \end{minipage}
\end{enumerate}
