\ejercicio
\begin{enumerate}[label=\roman*)]
	\item
	      Sea $A = \set{1, 2, 3, 4, 5, 6, 7, 8, 9, 10}$. Consideremos en $\partes(A)$ la relación de equivalencia dada
	      por el cardinal (es decir, la cantidad de elementos): Dos subconjuntos de $A$ están relacionados si y solo si tienen la misma cantidad
	      de elementos ¿Cuántas clases de equivalencia \textbf{distintas} determina la relación? Hallar un representante par acada clase.\\

	      \separadorCorto

	      $\partes(A) = \set{\vacio, \set{1}, \set{1,2}, \cdots, \set{1, 2, 3, 4, 5, 6, 7, 8, 9, 10}}$, el conjunto $\partes(A)$ tiene un total de
	      $2^{10} = 1024$ elementos. La relación determina 11 \textit{clases de equivalencia} distintas.\\
	      $\llave{lcl}{
			      \text{Conjuntos con 0 elementos: } & \clase{0}  & \vacio\\
			      \text{Conjuntos con 1 elemento: } & \clase{1} & \set{3}\\
			      \text{Conjuntos con 2 elementos: }& \clase{2} & \set{5,2}\\
			      \text{Conjuntos con 3 elementos: }& \clase{3} & \set{1,6, 3}\\
			      \text{Conjuntos con 4 elementos: }& \clase{4} & \set{1,8, 10,4}\\
			      \vdots                            &\vdots  & \vdots\\
			      \text{Conjuntos con 10 elementos: } & \clase{10} & \set{1,2,3,4,5,6,7,8,9,10} = A\\
		      }$

	\item
	      En el conjunto de todos los subconjuntos finitos de $\naturales$, consideremos nuevamente la relación de equivalencia dada por el cardinal: Dos subconjuntos finitos
	      de $\naturales$ están relacionados si y solo si tienen la misma cantidad de elementos ¿Cuántas clases de equivalencia \textbf{distintas} determina la relación?
	      Hallar un representante para cada clase.\\

	      \separadorCorto

	      Es parecido al inciso anterior, donde ahora $A = \set{1,2,3, \cdots, N-1, N}$, donde $\partes(\naturales_N)$ tiene $2^N$ elementos.\\
	      La relación determina $N+1$ \textit{clases de equivalencia} distintas.\\
	      $\llave{lcl}{
			      \text{Conjuntos con 0 elementos: } & \clase{0}  & \vacio\\
			      \text{Conjuntos con 1 elemento: } & \clase{1} & \set{3}\\
			      \text{Conjuntos con 2 elementos: }& \clase{2} & \set{5,2}\\
			      \text{Conjuntos con 3 elementos: }& \clase{3} & \set{1,6, 3}\\
			      \text{Conjuntos con 4 elementos: }& \clase{4} & \set{1,8, 10,4}\\
			      \vdots                            &\vdots  & \vdots\\
			      \text{Conjuntos con 10 elementos: } & \clase{N} & \set{1,2,3,4, \cdots, N-1, N} \naturales_N\\
		      }$
\end{enumerate}

\noindent\separador

\noindent\underline{\textit{Funciones}}\\

Sean $A$ y  $B$ conjuntos, y sea $\relacion$ de $A$ en $B$. Se dice que
$\relacion$ es una \textit{función} cuando todo elemento $x \en A$ está relacionado con algún
$y \en B$, y este elemento $y$ es único. Es decir:\\

$\llave{l}{
		\paratodo x \en A, \existe! y \en B \talque x \relacion y\\
		\paratodo x \en A, \existe y \en B \talque x\relacion y, \\
		\text{si } y, z \en B \text{ son tales que } x \relacion y \text{ y } x \relacion z \entonces y = z.
	}
$
\begin{itemize}
	\item Dada $f:A\,(dominio) \to B\,(codominio)$ el conjunto \textit{imagen} es: $\imagen(f)= \set{y \en B : \existe x \en A \talque f(x) = y}$
	\item \begin{itemize}
		      \item \textit{inyectiva:} si $\forall x, x' \en A \text{ tales que } f(x) = f(x')$ se tiene que $ x = x'$
		      \item \textit{sobreyectiva:} si $\forall y \en B, \existe x \en A \text{ tal que } f(x) = y.\: f\text{ es sobreyectiva si } \imagen(f) = B$
		      \item \textit{biyectiva:} Cuando es inyectiva y sobreyectiva.
	      \end{itemize}
	\item $A, B, C$ conjuntos y $f: A \to B \to C,\, g: B \to C$ funciones. Entonces la \textit{composición} de $f$ con $g$, que se nota:\\
	      $g \comp f = g\left(f(x)\right),\, \paratodo x \en A$, resulta ser una función de $A$ en $C$.
	\item $f$ es biyectiva cuando: $f\inv : B \to A$ es la función que satisface que:\\
	      $\paratodo y \en B: f\inv(y) = x \sisolosi f(x) = y$
\end{itemize}
