\subsubsection*{Notas teóricas}
Voy a estar usando:
\begin{enumerate}[label=(\alph*)]
	\item $A \union B = B \union A \to \text{Conmuta}$
	\item $A \inter B = B \inter A \to \text{Conmuta}$
	\item $(A \union B)^c = A^c \inter B^c \to \text{De Morgan 1}$
	\item $(A \inter B)^c = A^c \union B^c \to \text{De Morgan 2}$
	\item $A \inter(B \union C) = (A \inter B) \union (A \inter C) \to \text{Distributiva 1}$
	\item $A \union(B \inter C) = (A \union B) \inter (A \union C) \to \text{Distributiva 2}$
	\item $A - B = A \inter B^c$
	\item $A \triangle B =
		      \begin{cases}
			      (A - B) \union (B - A)             \\
			      (A \union B) \inter (A \inter B)^c \\
			      (A \inter B^c) \union (B \inter A^c)
		      \end{cases}
	      $
	\item $A^c = \set{x \en \universo \talque x \notin A}$
	\item
	      \def\subconjuntoYequivalente{
	      \begin{array}{|c|}
			      A \subseteq B \\
			      \hline
			      A^c \union B
		      \end{array}
            }

	      \[
		      \begin{array}{|c|c|c|c|c|c|c|c|}
			      \hline
			      x \en A & x \en B & x \en A^c & x \en A \inter B & x \en A \union B & x \en \subconjuntoYequivalente & x \en A \triangle B & A - B \\
			      \hline
			      V       & V       & F         & V                & V                & V                              & F                   & F     \\
			      V       & F       & F         & F                & V                & F                              & V                   & V     \\
			      F       & V       & V         & F                & V                & V                              & V                   & F     \\
			      F       & F       & V         & F                & F                & V                              & F                   & F     \\
			      \hline
		      \end{array}
	      \]
	\item
	      \emph{Cuando para probar $p \entonces q$ se prueba en su lugar $\neg q \entonces \neg p$ se dice que es una demostración
		      por contrarrecíproco, mientras que cuando se prueba en su lugar que suponer que vale $p \land \neg q$
		      lleva a una contradicción, se dice que es una demostración por reducción al absurdo.}

\end{enumerate}
