\documentclass[12pt,a4paper, spanish]{article}
% Sacar draft para que aparezcan las imagenes.
% Opciones: 12pt, 10pt, 11pt, landscape, twocolumn, fleqn, leqno...
% Opciones de clase: article, report, letter, beamer...

% Paquetes:
% =========
\usepackage[headheight=110pt, top = 2cm, bottom = 2cm, left=1cm, right=1cm]{geometry} %modifico márgenes
\usepackage[T1]{fontenc} % tildes
\usepackage[utf8]{inputenc} % Para poder escribir con tildes en el editor.
\usepackage[english]{babel} % Para cortar las palabras en silabas, creo.
\usepackage[ddmmyyyy]{datetime}
\usepackage{amsmath} % Soporte de mathmatics
\usepackage{amssymb} % fuentes de mathmatics
\usepackage{array} % Para tablas y eso
\usepackage{caption} % Configuracion de figuras y tablas
\usepackage[dvipsnames]{xcolor} % Para colorear el texto: black, blue, brown, cyan, darkgray, gray, green, lightgray, lime, magenta, olive, orange, pink, purple, red, teal, violet, white, yellow.
\usepackage{graphicx} % Necesario para poner imagenes
\usepackage{enumitem} % Cambiar labels y más flexibilidad para el enumerate
\usepackage{multicol} 
\usepackage{tikz} % para graficar
\usepackage{cancel} % cancelar fórmulas
\usepackage{titlesec} % para editar titulos y hacer secciones con formato a medida
\usepackage{ulem}
\usepackage{centernot} % tacha cosas
\usepackage{bbding} % símbolos de donde uso FiveStar
\usepackage{skull} % símbolos de donde uso Skull
\usepackage{soul} % Para tachar texto en text y math mode

% \usepackage{lipsum} % dummy text

% para hacer los graficos tipo grafos
\usetikzlibrary{shapes,arrows.meta, chains, matrix, calc, trees, positioning, fit}
\usetikzlibrary{external}

\setlength{\parindent}{0pt} % Para que no haya indentación en las nuevas líneas.

% Definiciones y macros para que se me haga
% más ameno el codeo.
% Definiciones y nuevos comandos:def
% =============
% Conjuntos
\DeclareMathOperator{\partes}{\mathcal P}
\DeclareMathOperator{\relacion}{\,\mathcal{R}\,}
\DeclareMathOperator{\norelacion}{\,\cancel{\relacion}\,}
\DeclareMathOperator{\universo}{\mathcal U}
\DeclareMathOperator{\reales}{\mathbb R}
\DeclareMathOperator{\naturales}{\mathbb N}
\DeclareMathOperator{\enteros}{\mathbb Z}
\DeclareMathOperator{\racionales}{\mathbb Q}
\DeclareMathOperator{\irracionales}{\mathbb I}
\DeclareMathOperator{\complejos}{\mathbb C}


\DeclareMathOperator{\K}{\mathbb K} % cuerpo K
\DeclareMathOperator{\vacio}{\varnothing}
\DeclareMathOperator{\union}{\cup}
\DeclareMathOperator{\inter}{\cap}
\DeclareMathOperator{\diferencia}{\ \setminus \ }
\DeclareMathOperator{\y}{\land}
\def\o{\lor}
\def\neg{\sim}

\def\entonces{\Rightarrow}
\def\noEntonces{\centernot\Rightarrow}

\def\sisolosi{\iff} % largo
\def\sii{\Leftrightarrow} % corto

\def\clase{\overline}
\def\ord{\text{ord}}


\def\existe{\,\exists\,}
\def\noexiste{\,\nexists\,}
\def\paratodo{\ \, \forall}
\def\distinto{\neq}
\def\en{\in}
\def\talque{\;/\;}

% =====
\def\qvq{\text{ quiero ver que }}

%funciones
\DeclareMathOperator{\dom}{Dom}
\DeclareMathOperator{\cod}{Cod}
\def\F{\mathcal F}
\def\comp{\circ}
\def\inv{^{-1}}
\def\infinito{\infty}

% Llaves, paréntesis, contenedores
\newcommand{\llave}[2]{ \left\{ \begin{array}{#1} #2 \end{array}\right. }
\newcommand{\llaveInv}[2]{ \left\} \begin{array}{#1} #2 \end{array}\right. }
\newcommand{\llaves}[2]{ \left\{ \begin{array}{#1} #2 \end{array} \right\} }
\newcommand{\matriz}[2]{\left( \begin{array}{#1} #2 \end{array} \right)}
\newcommand{\deter}[2]{\left| \begin{array}{#1} #2 \end{array} \right|}
\newcommand{\lista}[2][(1)]{\begin{enumerate}[\bf #1]\setlength\itemsep{-0.6ex} #2 \end{enumerate}}
\newcommand{\listal}[2][-0.6ex]{\begin{enumerate}[\bf(a)]\setlength\itemsep{#1} #2 \end{enumerate}}

% naturales
\newcommand{\sumatoria}[2]{\sum\limits_{#1}^{#2}}
\newcommand{\productoria}[2]{\prod\limits_{#1}^{#2}}
\newcommand{\kmasuno}[1]{\underbrace{#1}_{k+1\text{-ésimo}}}
\newcommand{\HI}[1]{\underbrace{#1}_{\text{HI}}}

% % enteros
\def\divideA{\, | \,}
\def\noDivide{\centernot\divideA}
\def\congruente{\, \equiv \,}
\newcommand{\congruencia}[3]{#1 \equiv #2 \;(#3)}
\newcommand{\noCongruencia}[3]{#1 \not\equiv #2 \;(#3)}
\newcommand{\conga}[1]{\stackrel{(#1)}{\congruente}}
\newcommand{\divset}[2]{\mathcal{D}(#1) = \set{#2}}
\newcommand{\divsetP}[2]{\mathcal{D_+}(#1) = \set{#2}}
\newcommand{\ub}[2]{ \underbrace{\textstyle #1}_{\mathclap{#2}} }
\newcommand{\ob}[2]{ \overbrace{\textstyle #1}^{\mathclap{#2}} }
\def\cop{\, \perp \, }

% complejos
\DeclareMathOperator{\re}{Re}
\DeclareMathOperator{\im}{Im}
\DeclareMathOperator{\argumento}{arg}
\newcommand{\conj}[1]{\overline{#1}}

% Polinomios
\DeclareMathOperator{\cp}{cp}
\DeclareMathOperator{\gr}{gr}
\DeclareMathOperator{\mult}{mult}
\newcommand{\divPol}[2]{\polylongdiv[style=D]{#1}{#2}}
\newcommand{\mcd}[2]{\polylonggcd{#1}{#2}}


% =====
% Miscelanea
% =====
\def\ot{\leftarrow}
\newcommand{\estabien}{{\color{blue} Consultado, está bien. \checkmark}}
\newcommand{\hacer}{
  {\color{red!80!black}{\Large \faIcon{radiation} Falta hacerlo!}}\par
  {\color{black!70!white}
    \small Si querés mandarlo: Telegram $\to$ \href{https://t.me/+1znt2GV1i8cwMTNh}{\small\faIcon{telegram}},
    o  mejor aún si querés subirlo en \LaTeX $\to$ \href{https://github.com/nad-garraz/algebraUno}{\small \faIcon{github}}.
  }\par
}

\newcommand{\Hacer}{{\color{black!30!red}\Large Hacer!}}
\def\Tilde{\quad\checkmark}
\def\ytext{\text{ y }}
\def\otext{\text{ o }}

% Estrellita para hacer llamadas de atención, viene en divertidos colores
% para coleccionar.
\newcommand{\llamada}[1]{
  \textcolor{
    \ifcase \numexpr#1 mod 6\relax
      cyan\or magenta\or OliveGreen\or YellowOrange\or Cerulean\or Violet\or Purple\or
    \fi
  }
  {\text{\FiveStar}^{\scriptscriptstyle#1}}
}


% separadores
\def\separador{\par\medskip\rule{\linewidth}{0.4pt}\par\medskip}
\def\separadorCorto{\par\medskip\rule{0.5\linewidth}{0.4pt}\par\medskip}


% Colores
\newcommand{\red}[1]{\textcolor{red}{#1}}
\newcommand{\green}[1]{\textcolor{OliveGreen}{#1}}
\newcommand{\blue}[1]{\textcolor{Cerulean}{#1}}
\newcommand{\cyan}[1]{\textcolor{cyan}{#1}}
\newcommand{\yellow}[1]{\textcolor{YellowOrange}{#1}}
\newcommand{\magenta}[1]{\textcolor{magenta}{#1}}
\newcommand{\purple}[1]{\textcolor{purple}{#1}}

% Conjuntos entre llaves y paréntesis
% te ahorrás escribir los \left y \right, así dejando el código más legible.
\newcommand{\set}[1] { \left\{ #1 \right\} }
\newcommand{\parentesis}[1]{ \left( #1 \right) }

% Stackrel text. Es para ahorrarse ecribir el \text
\newcommand{\stacktext}[2]{ \stackrel{\text{#1}}{#2} }

% Dado que muchas veces ponemos cosas sobre un signo '='
%  acá está el comando para escribir \igual{arriba}[abajo] con texto!
\NewDocumentCommand{\igual}{m o}{
  \IfNoValueTF{#2}{
    \overset{\mathclap{\text{#1}}}=
  }{
    \overset{\mathclap{\text{#1}}}{\underset{\mathclap{\text{#2}}}=}
  }
}
% Dado que muchas veces ponemos cosas sobre un signo '='
%  acá está el comando para escribir \igual{arriba}[abajo] con texto!
\NewDocumentCommand{\mayorIgual}{m o}{
  \IfNoValueTF{#2}{
    \overset{\mathclap{\text{#1}}}\geq
  }{
    \overset{\mathclap{\text{#1}}}{\underset{\mathclap{\text{#2}}}\geq}
  }
}
% Dado que muchas veces ponemos cosas sobre un signo '='
%  acá está el comando para escribir \igual{arriba}[abajo] con texto!
\NewDocumentCommand{\menorIgual}{m o}{
  \IfNoValueTF{#2}{
    \overset{\mathclap{\text{#1}}}\leq
  }{
    \overset{\mathclap{\text{#1}}}{\underset{\mathclap{\text{#2}}}\leq}
  }
}


%=======================================================
% Comandos con flechas extensibles.
%=======================================================
% *Flechita* extensible con texto {arriba} y [abajo] 
\NewDocumentCommand{\flecha}{m o}{%
  \IfNoValueTF{#2}{%
    \xrightarrow[]{\text{#1}}
  }{
    \xrightarrow[\text{#2}]{\text{#1}}
  }
}
% *Si solo si* extensible con texto {arriba} y [abajo] 
\NewDocumentCommand{\Sii}{m o}{%
  \IfNoValueTF{#2}{%
    \xLeftrightarrow[]{\text{#1}}
  }{
    \xLeftrightarrow[\text{#2}]{\text{#1}}
  }
}

% *Si solo si* extensible con texto {arriba} y [abajo] 
\NewDocumentCommand{\Entonces}{m o}{%
  \IfNoValueTF{#2}{%
    \xRightarrow[]{\text{#1}}
  }{
    \xRightarrow[\text{#2}]{\text{#1}}
  }
}

%=======================================================
% fin comandos con flechas extensibles.


% como el stackrel pero también se puede poner algo debajo
\newcommand{\taa}[3]{ % [t]exto [a]rriba y [a]bajo
  \overset{\mathclap{#1}}{\underset{\mathclap{#2}}{#3}}
}

%Update time
\def\update{
  actualizado: \today
}


%=======================================================
% sección ejercicio con su respectivo formato y contador
%=======================================================
\newcounter{ejercicio}[section] % contador que se resetea en cada sección
\renewcommand{\theejercicio}{\arabic{ejercicio}} % el contador es un número arabic
\newcommand{\ejercicio}{%
  \stepcounter{ejercicio}% incremento en uno
  \titleformat{\section}[runin]{\bfseries}{\theejercicio}{1em}{}%
  \section*{\theejercicio.}\labelEjercicio{ej:\theejercicio}
}

% Label y refencia para ejercicio hay alguna forma más elegante de hacer esto?
\newcommand{\labelEjercicio}[1]{
  \addtocounter{ejercicio}{-1} % counter - 1
  \refstepcounter{ejercicio} % referencia al anterior y luego + 1
  \label{#1}}
\newcommand{\refEjercicio}[1]{{ \bf\ref{#1}.}}

\def\fueguito{{\color{orange}{\faIcon{fire}}}}
\newcounter{ejExtra}[section] % contador que se resetea en cada sección
\renewcommand{\theejExtra}{\arabic{ejExtra}} % el contador es un número arabic
\newcommand{\ejExtra}{%
  \stepcounter{ejExtra}% incremento en uno
  \titleformat{\section}[runin]{\bfseries}{\theejExtra}{1em}{}%
  % Es como una sección. Le pongo un ícono, luego el número del ejercicio con la etiqueta para poder
  % linkearlo en el índice u otro lugar.
  % con \ref{ejExtra:{numero del ejercicio}} es que salto al ejercicio.
  \section*{\fueguito\theejExtra.}\labelEjExtra{ejExtra:\theejExtra}
}

% Label y refencia para ejercicio hay alguna forma más elegante de hacer esto?
\newcommand{\labelEjExtra}[1]{
  \addtocounter{ejExtra}{-1} % counter - 1
  \refstepcounter{ejExtra} % referencia al anterior y luego + 1
  \label{#1} % etiqueta para cada ejercicio extra
}
% Con esto llamos al ejercicio extra
\newcommand{\refEjExtra}[1]{
  {\fueguito\bf\ref{#1}.}
}

%=======================================================
% fin sección ejercicio con su respectivo formato y contador
%=======================================================

\newenvironment{enunciado}[1]{ % Toma un parametro obligatorio: \ejExtra o \ejercicio 
  \separador % linea sobre el enunciado
  \begin{minipage}{\textwidth}
    #1
    }% contenido
    {
  \end{minipage}
    \separadorCorto % linea debajo del enunciado
}



% Definiciones y macros para que se me haga
% más ameno el codeo.
% Definiciones y nuevos comandos:def
% =============
% Conjuntos
\DeclareMathOperator{\partes}{\mathcal P}
\DeclareMathOperator{\relacion}{\,\mathcal{R}\,}
\DeclareMathOperator{\norelacion}{\,\cancel{\relacion}\,}
\DeclareMathOperator{\universo}{\mathcal U}
\DeclareMathOperator{\reales}{\mathbb R}
\DeclareMathOperator{\naturales}{\mathbb N}
\DeclareMathOperator{\enteros}{\mathbb Z}
\DeclareMathOperator{\racionales}{\mathbb Q}
\DeclareMathOperator{\irracionales}{\mathbb I}
\DeclareMathOperator{\complejos}{\mathbb C}


\DeclareMathOperator{\K}{\mathbb K} % cuerpo K
\DeclareMathOperator{\vacio}{\varnothing}
\DeclareMathOperator{\union}{\cup}
\DeclareMathOperator{\inter}{\cap}
\DeclareMathOperator{\diferencia}{\ \setminus \ }
\DeclareMathOperator{\y}{\land}
\def\o{\lor}
\def\neg{\sim}

\def\entonces{\Rightarrow}
\def\noEntonces{\centernot\Rightarrow}

\def\sisolosi{\iff} % largo
\def\sii{\Leftrightarrow} % corto

\def\clase{\overline}
\def\ord{\text{ord}}


\def\existe{\,\exists\,}
\def\noexiste{\,\nexists\,}
\def\paratodo{\ \, \forall}
\def\distinto{\neq}
\def\en{\in}
\def\talque{\;/\;}

% =====
\def\qvq{\text{ quiero ver que }}

%funciones
\DeclareMathOperator{\dom}{Dom}
\DeclareMathOperator{\cod}{Cod}
\def\F{\mathcal F}
\def\comp{\circ}
\def\inv{^{-1}}
\def\infinito{\infty}

% Llaves, paréntesis, contenedores
\newcommand{\llave}[2]{ \left\{ \begin{array}{#1} #2 \end{array}\right. }
\newcommand{\llaveInv}[2]{ \left\} \begin{array}{#1} #2 \end{array}\right. }
\newcommand{\llaves}[2]{ \left\{ \begin{array}{#1} #2 \end{array} \right\} }
\newcommand{\matriz}[2]{\left( \begin{array}{#1} #2 \end{array} \right)}
\newcommand{\deter}[2]{\left| \begin{array}{#1} #2 \end{array} \right|}
\newcommand{\lista}[2][(1)]{\begin{enumerate}[\bf #1]\setlength\itemsep{-0.6ex} #2 \end{enumerate}}
\newcommand{\listal}[2][-0.6ex]{\begin{enumerate}[\bf(a)]\setlength\itemsep{#1} #2 \end{enumerate}}

% naturales
\newcommand{\sumatoria}[2]{\sum\limits_{#1}^{#2}}
\newcommand{\productoria}[2]{\prod\limits_{#1}^{#2}}
\newcommand{\kmasuno}[1]{\underbrace{#1}_{k+1\text{-ésimo}}}
\newcommand{\HI}[1]{\underbrace{#1}_{\text{HI}}}

% % enteros
\def\divideA{\, | \,}
\def\noDivide{\centernot\divideA}
\def\congruente{\, \equiv \,}
\newcommand{\congruencia}[3]{#1 \equiv #2 \;(#3)}
\newcommand{\noCongruencia}[3]{#1 \not\equiv #2 \;(#3)}
\newcommand{\conga}[1]{\stackrel{(#1)}{\congruente}}
\newcommand{\divset}[2]{\mathcal{D}(#1) = \set{#2}}
\newcommand{\divsetP}[2]{\mathcal{D_+}(#1) = \set{#2}}
\newcommand{\ub}[2]{ \underbrace{\textstyle #1}_{\mathclap{#2}} }
\newcommand{\ob}[2]{ \overbrace{\textstyle #1}^{\mathclap{#2}} }
\def\cop{\, \perp \, }

% complejos
\DeclareMathOperator{\re}{Re}
\DeclareMathOperator{\im}{Im}
\DeclareMathOperator{\argumento}{arg}
\newcommand{\conj}[1]{\overline{#1}}

% Polinomios
\DeclareMathOperator{\cp}{cp}
\DeclareMathOperator{\gr}{gr}
\DeclareMathOperator{\mult}{mult}
\newcommand{\divPol}[2]{\polylongdiv[style=D]{#1}{#2}}
\newcommand{\mcd}[2]{\polylonggcd{#1}{#2}}


% =====
% Miscelanea
% =====
\def\ot{\leftarrow}
\newcommand{\estabien}{{\color{blue} Consultado, está bien. \checkmark}}
\newcommand{\hacer}{
  {\color{red!80!black}{\Large \faIcon{radiation} Falta hacerlo!}}\par
  {\color{black!70!white}
    \small Si querés mandarlo: Telegram $\to$ \href{https://t.me/+1znt2GV1i8cwMTNh}{\small\faIcon{telegram}},
    o  mejor aún si querés subirlo en \LaTeX $\to$ \href{https://github.com/nad-garraz/algebraUno}{\small \faIcon{github}}.
  }\par
}

\newcommand{\Hacer}{{\color{black!30!red}\Large Hacer!}}
\def\Tilde{\quad\checkmark}
\def\ytext{\text{ y }}
\def\otext{\text{ o }}

% Estrellita para hacer llamadas de atención, viene en divertidos colores
% para coleccionar.
\newcommand{\llamada}[1]{
  \textcolor{
    \ifcase \numexpr#1 mod 6\relax
      cyan\or magenta\or OliveGreen\or YellowOrange\or Cerulean\or Violet\or Purple\or
    \fi
  }
  {\text{\FiveStar}^{\scriptscriptstyle#1}}
}


% separadores
\def\separador{\par\medskip\rule{\linewidth}{0.4pt}\par\medskip}
\def\separadorCorto{\par\medskip\rule{0.5\linewidth}{0.4pt}\par\medskip}


% Colores
\newcommand{\red}[1]{\textcolor{red}{#1}}
\newcommand{\green}[1]{\textcolor{OliveGreen}{#1}}
\newcommand{\blue}[1]{\textcolor{Cerulean}{#1}}
\newcommand{\cyan}[1]{\textcolor{cyan}{#1}}
\newcommand{\yellow}[1]{\textcolor{YellowOrange}{#1}}
\newcommand{\magenta}[1]{\textcolor{magenta}{#1}}
\newcommand{\purple}[1]{\textcolor{purple}{#1}}

% Conjuntos entre llaves y paréntesis
% te ahorrás escribir los \left y \right, así dejando el código más legible.
\newcommand{\set}[1] { \left\{ #1 \right\} }
\newcommand{\parentesis}[1]{ \left( #1 \right) }

% Stackrel text. Es para ahorrarse ecribir el \text
\newcommand{\stacktext}[2]{ \stackrel{\text{#1}}{#2} }

% Dado que muchas veces ponemos cosas sobre un signo '='
%  acá está el comando para escribir \igual{arriba}[abajo] con texto!
\NewDocumentCommand{\igual}{m o}{
  \IfNoValueTF{#2}{
    \overset{\mathclap{\text{#1}}}=
  }{
    \overset{\mathclap{\text{#1}}}{\underset{\mathclap{\text{#2}}}=}
  }
}
% Dado que muchas veces ponemos cosas sobre un signo '='
%  acá está el comando para escribir \igual{arriba}[abajo] con texto!
\NewDocumentCommand{\mayorIgual}{m o}{
  \IfNoValueTF{#2}{
    \overset{\mathclap{\text{#1}}}\geq
  }{
    \overset{\mathclap{\text{#1}}}{\underset{\mathclap{\text{#2}}}\geq}
  }
}
% Dado que muchas veces ponemos cosas sobre un signo '='
%  acá está el comando para escribir \igual{arriba}[abajo] con texto!
\NewDocumentCommand{\menorIgual}{m o}{
  \IfNoValueTF{#2}{
    \overset{\mathclap{\text{#1}}}\leq
  }{
    \overset{\mathclap{\text{#1}}}{\underset{\mathclap{\text{#2}}}\leq}
  }
}


%=======================================================
% Comandos con flechas extensibles.
%=======================================================
% *Flechita* extensible con texto {arriba} y [abajo] 
\NewDocumentCommand{\flecha}{m o}{%
  \IfNoValueTF{#2}{%
    \xrightarrow[]{\text{#1}}
  }{
    \xrightarrow[\text{#2}]{\text{#1}}
  }
}
% *Si solo si* extensible con texto {arriba} y [abajo] 
\NewDocumentCommand{\Sii}{m o}{%
  \IfNoValueTF{#2}{%
    \xLeftrightarrow[]{\text{#1}}
  }{
    \xLeftrightarrow[\text{#2}]{\text{#1}}
  }
}

% *Si solo si* extensible con texto {arriba} y [abajo] 
\NewDocumentCommand{\Entonces}{m o}{%
  \IfNoValueTF{#2}{%
    \xRightarrow[]{\text{#1}}
  }{
    \xRightarrow[\text{#2}]{\text{#1}}
  }
}

%=======================================================
% fin comandos con flechas extensibles.


% como el stackrel pero también se puede poner algo debajo
\newcommand{\taa}[3]{ % [t]exto [a]rriba y [a]bajo
  \overset{\mathclap{#1}}{\underset{\mathclap{#2}}{#3}}
}

%Update time
\def\update{
  actualizado: \today
}


%=======================================================
% sección ejercicio con su respectivo formato y contador
%=======================================================
\newcounter{ejercicio}[section] % contador que se resetea en cada sección
\renewcommand{\theejercicio}{\arabic{ejercicio}} % el contador es un número arabic
\newcommand{\ejercicio}{%
  \stepcounter{ejercicio}% incremento en uno
  \titleformat{\section}[runin]{\bfseries}{\theejercicio}{1em}{}%
  \section*{\theejercicio.}\labelEjercicio{ej:\theejercicio}
}

% Label y refencia para ejercicio hay alguna forma más elegante de hacer esto?
\newcommand{\labelEjercicio}[1]{
  \addtocounter{ejercicio}{-1} % counter - 1
  \refstepcounter{ejercicio} % referencia al anterior y luego + 1
  \label{#1}}
\newcommand{\refEjercicio}[1]{{ \bf\ref{#1}.}}

\def\fueguito{{\color{orange}{\faIcon{fire}}}}
\newcounter{ejExtra}[section] % contador que se resetea en cada sección
\renewcommand{\theejExtra}{\arabic{ejExtra}} % el contador es un número arabic
\newcommand{\ejExtra}{%
  \stepcounter{ejExtra}% incremento en uno
  \titleformat{\section}[runin]{\bfseries}{\theejExtra}{1em}{}%
  % Es como una sección. Le pongo un ícono, luego el número del ejercicio con la etiqueta para poder
  % linkearlo en el índice u otro lugar.
  % con \ref{ejExtra:{numero del ejercicio}} es que salto al ejercicio.
  \section*{\fueguito\theejExtra.}\labelEjExtra{ejExtra:\theejExtra}
}

% Label y refencia para ejercicio hay alguna forma más elegante de hacer esto?
\newcommand{\labelEjExtra}[1]{
  \addtocounter{ejExtra}{-1} % counter - 1
  \refstepcounter{ejExtra} % referencia al anterior y luego + 1
  \label{#1} % etiqueta para cada ejercicio extra
}
% Con esto llamos al ejercicio extra
\newcommand{\refEjExtra}[1]{
  {\fueguito\bf\ref{#1}.}
}

%=======================================================
% fin sección ejercicio con su respectivo formato y contador
%=======================================================

\newenvironment{enunciado}[1]{ % Toma un parametro obligatorio: \ejExtra o \ejercicio 
  \separador % linea sobre el enunciado
  \begin{minipage}{\textwidth}
    #1
    }% contenido
    {
  \end{minipage}
    \separadorCorto % linea debajo del enunciado
}


% Voy a definir graficos fuera del documento para que
% el código quede un poco más limpio
%3
\def\tresiiiUno{
	\begin{tikzpicture}[scale=0.8, baseline=0]
		% Number line
		\draw[thick, <->,] (-3.5,0) -- (3.5,0);
		% Interval
		\draw[fill=white] (2,0) circle (2pt);
		\draw[fill=white] (3,0) circle (2pt);
		\draw[fill=white] (-2,0) circle (2pt);
		\draw[fill=white] (-3,0) circle (2pt);
		\draw[-, magenta, ultra thick] (2,0) -- (3,0);
		\draw[-, magenta, ultra thick] (-2,0) -- (-3,0);
		\node at (2,-0.3) {2};
		\node at (3,-0.3) {3};
		\node at (-2,-0.3) {-2};
		\node at (-3,-0.3) {-3};
	\end{tikzpicture}
}
\def\tresiiiDos{
	\begin{tikzpicture}[scale=0.8, baseline=0]
		% Number line
		\draw[thick, <->,] (-3.5,0) -- (3.5,0);
		% Interval
		\draw[fill=white] (1.732,0) circle (2pt);
		\draw[fill=white] (-1.732,0) circle (2pt);
		\draw[-, cyan, ultra thick] (1.732,0) -- (-1.732,0);
		\node at (1.732,-0.3) {$\sqrt{3}$};
		\node at (-1.732,-0.3) {$-\sqrt{3}$};
		\node at (0,-0.3) {0};
	\end{tikzpicture}
}

%12
\def\doceiA{
	\begin{tikzpicture}[scale=0.8, baseline=0]
		% Number line
		\draw[thick, ->,] (1,0) -- (10,0);
		% Interval
		\draw[fill=magenta, color=magenta] (5,0.1) circle (3pt);
		\draw[fill=green, color=green] (8,0.2) circle (3pt);
		\draw[fill=green, color=green] (1,0.2) circle (3pt);
		\draw[fill=black] (1,0) circle (3pt);
		\draw[-, green, thick] (1,0.2) -- (8,0.2);
		\draw[->, magenta, thick] (5,0.1) -- (9,0.1);
		\node at (1,-0.3) {1};
		\node [color=magenta]at (5,-0.3) {5};
		\node[color=green] at (8,-0.3) {8};
	\end{tikzpicture}
}
\def\doceiiA{
	\begin{tikzpicture}[scale=0.8, baseline=0]
		% Number line
		\draw[thick, ->,] (1,0) -- (10,0);
		% Interval
		\draw[color=magenta] (5,0.1) circle (3pt);
		\draw[color=green] (8,0.1) circle (3pt);
		\draw[fill=magenta, color=magenta] (1,0.1) circle (3pt);
		\draw[fill=black] (1,0) circle (3pt);
		\draw[-, magenta, thick] (1,0.1) -- (5,0.1);
		\draw[->, green, thick] (8,0.1) -- (10,0.1);
		\node at (1,-0.3) {1};
		\node [color=magenta]at (5,-0.3) {5};
		\node[color=green] at (8,-0.3) {8};
	\end{tikzpicture}
}

\def\doceiiE{
	\begin{tikzpicture}[scale=0.8, baseline=0]
		% Number line
		\draw[thick, <->,] (-4,0) -- (5,0);
		% Interval
		\draw[fill=magenta, color=magenta] (3,0.1) circle (3pt);
		\draw[fill=green, color=green] (2,0.2) circle (3pt);
		\draw[fill=green, color=green] (-2,0.2) circle (3pt);
		\draw[<-, magenta, thick] (-3,0.1) -- (3,0.1);
		\draw[-, green, thick] (-2,0.2) -- (2,0.2);
		\node [color=magenta] at (3,-0.3) {3};
		\node[color=green] at (2,-0.3) {2};
		\node[color=green] at (-2,-0.3) {-2};
	\end{tikzpicture}
}

\def\doceiiicero{
	\begin{tikzpicture}[scale=0.6, baseline=0]
		% Number line
		\draw[thick, <->,] (-4.5,0) -- (4.5,0);
		% Interval
		\draw[color=magenta] (3,0.1) circle (3pt);
		\draw[color=green] (2,0.2) circle (3pt);
		\draw[color=green] (-2,0.2) circle (3pt);

		\draw[->, magenta, thick] (3,0.1) -- (4.5,0.1);
		\draw[<-, green, thick] (-4.5,0.2) -- (-2,0.2);
		\draw[->, green, thick] (2,0.2) -- (4.5,0.2);

		\node[color=magenta] at (3,-0.3) {3};
		\node[color=green] at (2,-0.3) {2};
		\node[color=green] at (-2,-0.3) {-2};
	\end{tikzpicture}
}

\def\doceiiiuno{
	\begin{tikzpicture}[scale=0.6, baseline=0]
		% Number line
		\draw[thick, <->,] (-4.5,0) -- (4.5,0);
		% Interval
		\draw[fill=magenta, color=magenta] (3,0.1) circle (3pt);
		\draw[fill=green, color=green] (2,0.2) circle (3pt);
		\draw[fill=green, color=green] (-2,0.2) circle (3pt);

		\draw[<-, magenta, thick] (-3,0.1) -- (3,0.1);
		\draw[-, green, thick] (-2,0.2) -- (2,0.2);

		\node[color=magenta] at (3,-0.3) {3};
		\node[color=green] at (2,-0.3) {2};
		\node[color=green] at (-2,-0.3) {-2};
	\end{tikzpicture}
}

\def\doceiiidos{
	\begin{tikzpicture}[scale=0.6, baseline=0]
		% Number line
		\draw[thick, <->,] (-4.5,0) -- (4.5,0);
		% Interval
		\draw[fill=magenta, color=magenta] (3,0.1) circle (3pt);
		\draw[color=green] (2,0.2) circle (3pt);
		\draw[color=green] (-2,0.2) circle (3pt);

		\draw[<-, magenta, thick] (-4,0.1) -- (3,0.1);
		\draw[<-, green, thick] (-4,0.2) -- (-2,0.2);
		\draw[->, green, thick] (2,0.2) -- (4,0.2);

		\node[color=magenta] at (3,-0.3) {3};
		\node[color=green] at (2,-0.3) {2};
		\node[color=green] at (-2,-0.3) {-2};
	\end{tikzpicture}
}

\def\doceiiitres{
	\begin{tikzpicture}[scale=0.6, baseline=0]
		% Number line
		\draw[thick, <->,] (-4.5,0) -- (4.5,0);
		% Interval
		\draw[color=magenta] (3,0.1) circle (3pt);
		\draw[color=green] (2,0.2) circle (3pt);
		\draw[color=green] (-2,0.2) circle (3pt);

		\draw[->, magenta, thick] (3,0.1) -- (4.5,0.1);
		\draw[-, green, thick] (-2,0.2) -- (2,0.2);

		\node[color=magenta] at (3,-0.3) {3};
		\node[color=green] at (2,-0.3) {2};
		\node[color=green] at (-2,-0.3) {-2};
	\end{tikzpicture}
}

% 17
\def\diecisietei{
	\begin{tikzpicture}[scale=0.5, >=Latex, draw=Aquamarine]
		%A vértices
		\node (1a) {$\bullet$};
		\node[] at (1a.west) {1};
		\node[below=of 1a] (2a) {$\bullet$};
		\node[] at (2a.west) {2};
		\node[below=of 2a] (3a) {$\bullet$};
		\node[] at (3a.west) {3};
		\node[shape=ellipse, draw, black, minimum size=2cm,fit={(1a) (3a)}] {};

		%B vértices
		\node[right=2cm of 1a] (1b) {$\bullet$};
		\node[] at (1b.east) {1};
		\node[below=of 1b] (3b) {$\bullet$};
		\node[] at (3b.east) {3};
		\node[below=of 3b] (5b) {$\bullet$};
		\node[] at (5b.east) {5};
		\node[below=of 5b] (7b) {$\bullet$};
		\node[] at (7b.east) {7};
		\node[shape=ellipse, draw, black, minimum size=2cm,fit={(1b) (7b)}] {};

		% Elipses
		\node[below=1cm of 3a] {$A$};
		\node[below=1.2cm of 7b] {$B$};

		% Aristas
		\draw[->, bend left] (1a) to (1b);
		\draw[->, bend right] (1a) to (3b);
		\draw[->, bend right] (1a) to (7b);
		\draw[->, bend right] (3a) to (1b);
		\draw[->, bend right] (3a) to (5b);
	\end{tikzpicture}
}

%19
\def\diecinuevei{
	\begin{tikzpicture}[scale=0.5, baseline=0, >=Latex, draw=Aquamarine]

		\node[] (a) {$\bullet$};
		\node[] at (a.west) {$a$};

		\node[above right= of a] (b) {$\bullet$};
		\node[] at (b.east) {$b$};

		\node[below right= of b] (c) {$\bullet$};
		\node[] at (c.east) {$c$};

		\node[below= of a] (d) {$\bullet$};
		\node[] at (d.west) {$d$};

		\node[below= of c] (e) {$\bullet$};
		\node[] at (e.west) {$e$};

		\node[right= of d] (f) {$\bullet$};
		\node[] at (f.west) {$f$};

		\node[right= of c] (g) {$\bullet$};
		\node[] at (g.north) {$g$};

		\node[below= of g] (h) {$\bullet$};
		\node[] at (h.south) {$h$};

		% Universo
		\node[shape=ellipse, draw, black, fit={ (b) (d) (g) (e)}] (universo) {};
		\node[above left = 0.1cm of universo] {$A$};

		% Aristas
		\draw[->, bend left] (a.center) to (b.center);
		\draw[->, bend left] (b.center) to (a.center);
		\draw[->, bend right] (c.center) to (d.center);
		\draw[->, loop above] (c) to (c);
		\draw[->, loop below ] (f) to (f);
		\draw[->, bend right] (c.center) to (h.center);
		\draw[->, bend left] (e.center) to (c.center);
		\draw[->, bend right] (h.center) to (g.center);
	\end{tikzpicture}
}

% 19 ii
\def\diecinueveiv{
	\begin{tikzpicture}[scale=0.5, baseline=0, >=Latex, draw=Aquamarine]

		\node[] (a) {$\bullet$};
		\node[] at (a.west) {$a$};

		\node[above right= of a] (b) {$\bullet$};
		\node[] at (b.east) {$b$};

		\node[below right= of b] (c) {$\bullet$};
		\node[] at (c.east) {$c$};

		\node[below= of a] (d) {$\bullet$};
		\node[] at (d.west) {$d$};

		\node[below= of c] (e) {$\bullet$};
		\node[] at (e.west) {$e$};

		\node[right= of d] (f) {$\bullet$};
		\node[] at (f.west) {$f$};

		\node[right= of c] (g) {$\bullet$};
		\node[] at (g.east) {$g$};

		\node[below= of g] (h) {$\bullet$};
		\node[] at (h.east) {$h$};

		% Universo
		\node[shape=ellipse, draw, black, fit={ (b) (d) (g) (e)}] (universo) {};
		\node[above left = 0.1cm of universo] {$A$};

		% Aristas
		\draw[->, loop below] (a) to (a);
		\draw[->, loop above ] (b) to (b);
		\draw[->, loop above] (c) to (c);
		\draw[->, loop below ] (d) to (d);
		\draw[->, loop below] (e) to (e);
		\draw[->, loop below] (f) to (f);
		\draw[->, loop above] (g) to (g);
		\draw[->, loop below ] (h) to (h);

		\draw[->, bend left] (a.center) to (b);
		\draw[->, bend left] (b.center) to (a);

		\draw[->, bend right] (e.center) to (h);
		\draw[->, bend right] (e.center) to (g);
		\draw[->, bend right] (h.center) to (g);
		\draw[->, bend right] (h.center) to (e);
		\draw[->, bend right] (g.center) to (h);
		\draw[->, bend right] (g.center) to (e);
	\end{tikzpicture}
}

%20

\def\veinte{
	\begin{tikzpicture}[scale=0.5, baseline=0, >=Latex, draw=Aquamarine]

		\node[] (1) {$\bullet$};
		\node[] at (1.west) {$1$};

		\node[above right= of 1] (2) {$\bullet$};
		\node[] at (2.east) {$2$};

		\node[below right= of 2] (3) {$\bullet$};
		\node[] at (3.east) {$3$};

		\node[below= of 1] (4) {$\bullet$};
		\node[] at (4.west) {$4$};

		\node[right= of 2] (5) {$\bullet$};
		\node[] at (5.west) {$5$};

		\node[right= of d] (6) {$\bullet$};
		\node[] at (6.east) {$6$};


		% Universo
		\node[shape=ellipse, draw, black, fit={ (1) (2) (3) (4)}] (universo) {};
		\node[above left = 0.1cm of universo] {$A$};

		% Aristas
		\draw[->, loop below] (1) to (1);
		\draw[->, loop above ] (3) to (3);
		\draw[->, loop above] (4) to (4);
		\draw[->, loop below ] (6) to (6);

		\draw[->, bend left] (6.center) to (4);
		\draw[->, bend left] (4.center) to (6);

		\draw[->, bend right] (1.center) to (3);
		\draw[->, bend right] (3.center) to (1);
	\end{tikzpicture}
}

%24
\def\veinticuatro{
	\begin{tikzpicture}[scale=0.5, baseline=0, >=Latex, draw=Aquamarine]

		\node[] (a) {$\bullet$};
		\node[] at (a.north west) {$a$};

		\node[below left = 1cm of a] (b) {$\bullet$};
		\node[] at (b.south) {$b$};

		\node[below right = 1cm of a] (f) {$\bullet$};
		\node[] at (f.south) {$f$};

		\node[above right = 1cm of a] (d) {$\bullet$};
		\node[] at (d.west) {$d$};

		\node[right=1cm of a] (c) {$\bullet$};
		\node[] at (c.south) {$c$};

		\node[right= of c] (e) {$\bullet$};
		\node[] at (e.south) {$e$};


		% Universo
    \node[shape=ellipse, draw, black, fit={ (a) (b) (d) (f) (e)}] (universo) {};
		\node[above left = 0.1cm of universo] {$A$};

		% Aristas
		\draw[->, loop above] (a) to (a);
		\draw[->, loop left ] (b) to (b);
		\draw[->, loop left] (c) to (c);
		\draw[->, loop above ] (d) to (d);
		\draw[->, loop right ] (e) to (e);
		\draw[->, loop right ] (f) to (f);

		\draw[->, bend left] (a.center) to (b);
		\draw[->, bend left] (b.center) to (a);
		\draw[->, bend left] (a.center) to (f);
		\draw[->, bend left] (f.center) to (a);
		\draw[->, bend left] (b.center) to (f);
		\draw[->, bend left] (f.center) to (b);

		\draw[->, bend right] (c.center) to (e);
		\draw[->, bend right] (e.center) to (c);
	\end{tikzpicture}
}

%25
\def\veintisiete{
	\begin{tikzpicture}[scale=0.5, baseline=0, >=Latex, draw=Aquamarine]

		\node[] (1) {$\bullet$};
		\node[] at (1.west) {$1$};

		\node[right = of 1] (92) {$\bullet$};
		\node[] at (92.east) {$92$};

		\node[below = of 1] (2) {$\bullet$};
		\node[] at (2.west) {$2$};

		\node[right = of 2] (91) {$\bullet$};
		\node[] at (91.east) {$91$};

		\node[below right = .5 of 2] (puntos) {$\vdots$};

		\node[below left = .5 of puntos] (45) {$\bullet$};
		\node[] at (45.west) {$45$};

		\node[right = of 45] (47) {$\bullet$};
		\node[] at (47.east) {$47$};


		\node[below right =.5 of 45] (46) {$\bullet$};
		\node[] at (46.east) {$46$};


		% Universo
    \node[shape=ellipse, draw, black, fit={ (1) (92) (45) (47) (46)}] (universo) {};
		\node[above left = 0.1cm of universo] {$A$};

		% Aristas
		\draw[->, loop below] (1) to (1);
		\draw[->, loop below ] (92) to (92);

		\draw[->, loop below] (2) to (2);
		\draw[->, loop below ] (91) to (91);

		\draw[->, loop below] (45) to (45);
		\draw[->, loop below ] (47) to (47);

		\draw[->, loop below ] (46) to (46);

		\draw[->, bend left] (1.center) to (92);
		\draw[->, bend left] (92.center) to (1);
		\draw[->, bend left] (2.center) to (91);
		\draw[->, bend left] (91.center) to (2);
		\draw[->, bend left] (45.center) to (47);
		\draw[->, bend left] (47.center) to (45);
	\end{tikzpicture}
}
 

\begin{document}

\pagestyle{empty} % Para que no muestre el número en pie de página

% Info para armar título.
\title{Práctica 1 de álgebra 1} % título
\author{Nad Garraz} % autor
\date{last update: \today} % Cambiar de ser necesario
% \maketitle  % Para que aprezca el título en el documento

\section*{Colección de resultados, fórmulas, definiciones}
Voy a estar usando:
\begin{enumerate}[label=(\alph*)]
	\item $A \union B = B \union A \to \text{Conmuta}$
	\item $A \inter B = B \inter A \to \text{Conmuta}$
	\item $(A \union B)^c = A^c \inter B^c \to \text{De Morgan 1}$
	\item $(A \inter B)^c = A^c \union B^c \to \text{De Morgan 2}$
	\item $A \inter(B \union C) = (A \inter B) \union (A \inter C) \to \text{Distributiva 1}$
	\item $A \union(B \inter C) = (A \union B) \inter (A \union C) \to \text{Distributiva 2}$
	\item $A - B = A \inter B^c$
	\item $A \triangle B =
		      \begin{cases}
			      (A - B) \union (B - A)             \\
			      (A \union B) \inter (A \inter B)^c \\
			      (A \inter B^c) \union (B \inter A^c)
		      \end{cases}
	      $
	\item $A^c = \set{x \en \universo \talque x \notin A}$
	\item
	      \def\subconjuntoYequivalente{
		      \begin{array}{|c|}
			      A \subseteq B \\
			      \hline
			      A^c \union B
		      \end{array}
	      }
	      \[
		      \begin{array}{|c|c|c|c|c|c|c|c|}
			      \hline
			      x \en A & x \en B & x \en A^c & x \en A \inter B & x \en A \union B & x \en \subconjuntoYequivalente & x \en A \triangle B & A - B \\
			      \hline
			      V       & V       & F         & V                & V                & V                              & F                   & F     \\
			      V       & F       & F         & F                & V                & F                              & V                   & V     \\
			      F       & V       & V         & F                & V                & V                              & V                   & F     \\
			      F       & F       & V         & F                & F                & V                              & F                   & F     \\
			      \hline
		      \end{array}
	      \]
	\item
	      \emph{Cuando para probar $p \entonces q$ se prueba en su lugar $\neg q \entonces \neg p$ se dice que es una demostración
		      por contrarrecíproco, mientras que cuando se prueba en su lugar que suponer que vale $p \land \neg q$
		      lleva a una contradicción, se dice que es una demostración por reducción al absurdo.}

\end{enumerate}

\section*{Ejercicios sueltos que me parecen relevantes}
\subsubsection*{Distributiva}
Probar la propiedad distributiva: $X \inter (Y \union Z) = (X \inter Y) \union (X \inter Z)$\\
Tengo que hacer una doble inclusión:
$\to \begin{cases}
		1) & X \inter (Y \union Z) \subseteq (X \inter Y) \union (X \inter Z) \\
		2) & (X \inter Y) \union (X \inter Z) \subseteq X \inter (Y \union Z)
	\end{cases}$

\begin{enumerate}[label=\arabic*)]
	\item
	      $x \en X \inter (Y \union Z)$ quiere decir que $x \en X$ y
	      $\llaves{c}{
			      x \en Y \\
			      \o      \\
			      x \en Z
		      } $.
	      Por lo tanto $\to
		      \llaves{c}{
			      x \en X \inter Y\\
			      \o \\
			      x \en X \inter Z
		      }$, lo que equivale a $x \en (X \inter Y) \union (X \inter Z)$ \Tilde.\\

	\item
	      Ahora hay que probar la vuelta. Uso razonamiento análogo.
	      $x \en (X \inter Y) \union (X \inter Z)$, por lo que $x \en X$ y
	      $
		      \llaves{c}{
			      x \en X \inter Y \blue{\flecha{dado que}[$Y \subseteq Y \union Z$] x \en X \inter (Y \union Z) }\\
			      \o               \\
			      x \en X \inter Z \blue{\flecha{dado que}[$Z \subseteq Y \union Z$] x \en X \inter (Y \union Z)}
		      }$.
	      Lo que quiere decir \blue{que $x \en X \inter (Y \union Z)$ \Tilde}\\
	      \red{¿Estoy suponiendo cosas que debería demostrar, me estoy salteando pasos?}\\
	      \blue{Para que la solución quede más creíble usé que $S \subseteq S \union T$ fue el dealbreaker. }

\end{enumerate}

\subsubsection*{De Morgan}
Probar la propiedad $(A \inter B)^c = A^c \union B^c$.\\
Tengo que hacer una doble inclusión
$\to \begin{cases}
		1) & (A \inter B)^c \subseteq A^c \union B^c \\
		2) & A^c \union B^c \subseteq (A \inter B)^c
	\end{cases}
$
\begin{enumerate}[label=\arabic*)]
	\item Prueba directa: Si $x \en (A \inter B)^c \entonces x \en A^c \union B^c $\\
	      Por hipótesis $x \en (A \inter B)^c \stacktext{def}{\sisolosi} x \notin A \o x \notin B
		      \entonces x \en A^c \o x \en B^c \entonces x \en A^c \union B^c$\\
	      $\begin{array}{|c|c|c|c|}
			      \hline
			      A & B & A^c \union B^c & (A \inter B)^c \\ \hline
			      V & V & F              & F              \\
			      V & F & V              & V              \\
			      F & V & V              & V              \\
			      F & F & V              & V              \\ \hline
		      \end{array}
	      $

	      \blue{Uso la tabla para ver la definición $x \en (A \inter B)^c \stacktext{def}{\sisolosi} x \notin A \o x \notin B$}

	\item Pruebo por absurdo. Si $\paratodo x \en A^c \union B^c \entonces x \en (A \inter B)^c$\\
	      \green{Supongo} que $ x \notin (A \inter B)^c \stacktext{def}{\sisolosi} x \en (A \inter B) \flecha{por}[hipótesis] x \en A^c \union B^c \to
		      \llaves{c}{
			      x \notin A\\
			      \o \\
			      x \notin B\\
		      }$, por lo que $x \notin A \union B \entonces x \notin A \inter B$ contradiciendo el \green{supuesto}, absurdo. Debe ocurrir que $x \en (A \inter B)^c   $

	      $\begin{array}{|c|c|c|c|c|}
			      \hline
			      A & B & A \inter B & (A \union B) & (A \inter B) \subseteq (A \union B) \\ \hline
			      V & V & V          & V            & V                                   \\
			      V & F & F          & V            & V                                   \\
			      F & V & F          & V            & V                                   \\
			      F & F & F          & F            & V                                   \\ \hline
		      \end{array}
	      $
\end{enumerate}

\newpage
\section*{Ejercicios de la guía}

\ejercicio
%1
$A = \set{1,2,3}$
\begin{multicols}{2}
	\begin{enumerate}[label=(\roman*)]
		\item $1 \en A \flecha{respueta} \text{V}$
		\item $\set{1} \subseteq A \flecha{respueta} \text{V}$
		\item $\set{2,1} \subseteq A \flecha{respuesta} \text{V}$
		\item $\set{1,3} \en A \flecha{respuesta} \text{F}$
		\item $\set{2} \en A \flecha{respuesta} \text{F}$
	\end{enumerate}
\end{multicols}

\ejercicio
%2
$A = \set{1,2,\set{3},\set{1,2}}$
\begin{multicols}{2}
	\begin{enumerate}[label=(\roman*)]
		\item $3 \en A       \flecha{respueta} \text{F}  $
		\item $\set{3} \subseteq A \flecha{respueta} \text{F}$
		\item $\set{3} \en A    \flecha{respueta} \text{V}$
		\item $\set{\set{3}} \en A \flecha{respueta} \text{V}$
		\item $\set{1,2} \en A \flecha{respueta} \text{V}$
		\item $\set{1,2} \subseteq A \flecha{respueta} \text{V} $
		\item $\set{\set{1,2}} \subseteq A \flecha{respueta} \text{V} $
		\item $\set{\set{1,2}, 3} \subseteq A \flecha{respueta} \text{F} $
		\item $\vacio \en A \flecha{respueta} \text{F} $
		\item $\vacio \subseteq A \flecha{respueta} \text{V} $
		\item $A \en A \flecha{respueta} \text{F} $
		\item $A \subseteq A \flecha{respueta} \text{V} $
	\end{enumerate}
\end{multicols}

%3
\ejercicio
Inclusión :
$
	\begin{cases}
		\text{Definición}      & A \subseteq B \text{ si } \paratodo x, x \en A \entonces x \en B   \\
		\text{Contrarecíproco} & A \nsubseteq B \text{ si } \exists x, x \en A \entonces x \notin B
	\end{cases}
$
\begin{enumerate}[label=(\roman*)]
	\item $\begin{cases}
			      A = \set{1, 2, 3} \\
			      B = \set{5,4,3,2,1}
		      \end{cases}
		      \flecha{respueta} \quad
		      A \stacktext{\Tilde}{\subseteq} B$
	\item $\begin{cases}
			      A = \set{1, 2, 3} \\
			      B = \set{1,2,\set{3},-3}
		      \end{cases}
		      \flecha{respueta} \quad
		      A \nsubseteq B \flecha{dado}[que] \set{3} \notin B$
	\item $
		      \llaves{ll}{
			      A = \set{x \en \reales \talque 2<|x|<3} & \tresiiiUno \\
			      B = \set{x \en \reales \talque x^2 < 3 } & \tresiiiDos
		      }\\
		      \flecha{respueta} A \nsubseteq B \flecha{dado}[que] 2.5 \en A \text{ y } 2.5 \notin B
	      $
	\item$
		      \begin{cases}
			      A = \set{\vacio} \\
			      B = \vacio
		      \end{cases}\\
		      \flecha{respueta}
		      A \nsubseteq B \flecha{dado}[que] \text{$B$ no tiene ningún elemento, sin embargo $A$ tiene un elemento: $\vacio$.}
	      $
\end{enumerate}

%4
\ejercicio
Subconjuntos: $A = \set{1, -2, 7, 3}$, $B = \set{1, \set{3}, 10}$, $C = \set{-2,\set{1,2,3},3}$,\\
Conjunto referencial: $V = \set{1,\set{3}, -2,7, 10, \set{1,2,3},3}$, hallar:

\begin{multicols}{2}
	\begin{enumerate}[label=(\roman*)]
		\item $A \inter (B \triangle C)$
		\item $(A \inter B) \triangle (A \inter C)$
		\item $A^c \inter B^c \inter C^c$
	\end{enumerate}
\end{multicols}

%5
\ejercicio
Describir
$
	\begin{cases}
		i) \quad (A \union B \union C)^c  & \text{en términos de intersecciones y complementos} \\
		ii) \quad (A \inter B \inter C)^c & \text{en términos de uniones y complementos}        \\
	\end{cases}
$
\begin{enumerate}[label=\roman*)]
	\item $\quad (A \union B \union C)^c \stacktext{(c)}{=} (A \union B)^c \inter C^c \stacktext{(c)}{=} A^c \inter B^c \inter C^c$
	\item $\quad (A \inter B \inter C)^c \stacktext{(d)}{=} (A \inter B)^c \union C^c \stacktext{(d)}{=} A^c \union B^c \union C^c$
\end{enumerate}
\red{Solo eso? Falta demostrar algo de alguna manera más rigurosa?}\\
\blue{No, con eso sería suficiente \Tilde}

%6
\ejercicio

%7
\ejercicio
$\flecha{del}[gráfico] (C \inter B) - A) \union (A - B) =\\
	\flecha{uso que}[$A - B = A \inter B^c$] = (C \inter B \inter A^c) \union (A \inter B^c) \stackrel{\blue{*}}{=}\\
	\flecha{distributiva} C \union (A \inter B^c) \inter B \union (A \inter B^c) \inter A^c \union (A \inter B^c) =\\
	= (C \union A) \inter (C \union B^c) \inter (B \union A) \inter (B \union B^c) \inter (A^c \union A) \inter (A^c \union B^c)=\\
	= (C \union A) \inter (C \union B^c) \inter (B \union A) \inter \universo \inter \universo \inter (A^c \union B^c)=\\
	= (C \union A) \inter (C \union B^c) \inter (B \union A) \inter (A^c \union B^c)=\\
	= C \inter C \union C \inter B^c \union A\inter C \union A \inter B^c ... \\
	= C \union C \inter B^c \union A\inter C \union A \inter B^c \red{... esto no puede ser así}\\
	\blue{efectivamente, no hay que enloquecer, corto en \blue{*} \Tilde }
$

%8
\ejercicio
Hallar $\partes(A)$, el conjunto partes formado por todos los subconjuntos de A.\\
$
	\partes(A) \to
	\begin{cases}
		B \en \partes(A) \sisolosi B \subseteq A \\
		\partes(A) = \set{B \talque B \subseteq A}
	\end{cases}
$
\begin{enumerate}[label=(\roman*)]
	\item $A = \set{1} \to \partes(A) = \set{\vacio, A}$ \red{controlar}
	\item $A = \set{a, b} \to \partes(A) = \set{\vacio, \set{a}, \set{b}, A}$
	\item $A = \set{1, \set{1,2}, 3} \to \partes(A) = \set{\vacio, \set{1},\set{\set{1,2}}, \set{3}, \set{1,\set{1,2}}, \set{1,3},\set{\set{1,2},3}, A }$
\end{enumerate}

%9
\ejercicio
Sean $A$ y $B$ conjuntos, Probar que $\partes(A) \subseteq \partes(B) \sisolosi A \subseteq B$
\begin{itemize}
	\item[$\entonces$)] Pruebo por absurdo. Supongo que $A \nsubseteq B \entonces \exists x \en A \talque x \notin B$.\\
	      Si $x \en A \entonces \set{x} \en \partes(A)
		      \flecha{hipótesis}[$\partes(A) \subseteq \partes(B)$] \set{x} \en \partes(B)
		      \entonces x \en B .$ Absurdo.\\
	      $\to \boxed{\partes(A) \subseteq \partes(B) \entonces A \subseteq B}$\Tilde\red{controlar} \estabien

	\item[$\Leftarrow$)]
	      $\qvq A \subseteq B \entonces \partes(A) \subseteq \partes(B),\\
		      \paratodo S \en \partes(A) \sisolosi S \subseteq A \stacktext{hip}{\subseteq} B \entonces S \en \partes(B).\\
		      \to \boxed{A \subseteq B \entonces \partes(A) \subseteq \partes(B) }$\Tilde\red{controlar} \estabien
\end{itemize}

%10
\ejercicio
\begin{enumerate}[label=\roman*)]
	\item
	      Sean p, q proposiciones. Verificar que las siguientes expresiones tienen la misma tabla de verdad para
	      concluir que son equivalentes:
	      \[
		      \begin{array}{|c|c|c|c|c|c|c|c|}
			      \hline
			      p & q & \neg p & \neg q & p \entonces q & \neg q \entonces \neg p & \neg p \lor q & \neg(p \land \neg q) \\
			      \hline  \hline
			      V & V & F      & F      & V             & V                       & V             & V                    \\
			      V & F & F      & V      & F             & F                       & F             & F                    \\
			      F & V & V      & F      & V             & V                       & V             & V                    \\
			      F & F & V      & V      & V             & V                       & V             & V                    \\
			      \hline
		      \end{array}
	      \]
	\item
	      \[
		      \begin{array}{|c|c|c|c|c|c|}
			      \hline
			      p & q & \neg q & p \entonces q & \neg (p \entonces q) & p\, \y \neg q \\ \hline  \hline
			      V & V & F      & V             & F                    & F             \\
			      V & F & V      & F             & V                    & V             \\
			      F & V & F      & V             & F                    & F             \\
			      F & F & V      & V             & F                    & F             \\ \hline
		      \end{array}
	      \]
\end{enumerate}

%11
\ejercicio Hallar contraejemplos para mostrar que las siguientes proposiciones son falsas:
\begin{enumerate}[label=\roman*)]
	\item $\paratodo a \en \naturales,\, \frac{a-1}{a}$ no es un número entero.\\
	      La proposición es falsa, dado que si $a = 1 \entonces \frac{\green{1} - 1}{1} = 0 \en \enteros$

	\item $\paratodo x, y \en \reales$ con $x, y$ positivos, $\sqrt{x+y} = \sqrt{x} + \sqrt{y}$.\\
	      La proposición es falsa, dado que si.\\
	      \[
		      \llaves{c}{
			      x = 4.\\
			      y = 4
		      } \to \sqrt{4+4} = \sqrt{8} = 2\sqrt{2} \stacktext{?}{\leftrightarrow} \sqrt{4} + \sqrt{4} = 2 + 2 = 4
	      \]

	\item $\paratodo x \en \reales,\, x^2 > 4 \entonces x > 2$.\\
	      La  proposición es falsa, dado que si $x = -3$, queda $9 > 4 \entonces -3 > 2$, lo cual es falso,
\end{enumerate}


%12
\ejercicio
\begin{enumerate}[label=\roman*)]
	\item
	      Decidir si las siguientes proposiciones son verdaderas o falsas, justificando debidamente:

	      \begin{enumerate}[label=(\alph*)]
		      \item $\paratodo n \en \naturales,\, n \geq 5 \o n \leq 8$.\\
		            La proposición es verdadera. El conjunto descrito por $\set{ n \en \naturales \talque n \leq 8 \o n \geq 5} = \naturales$\\
		            \doceiA                        \\
		            \red{¿Se puede justificar con un gráfico?}

		      \item $\existe n \en \naturales \talque n \geq 5 \y n\leq 8.$\\
		            La proposición es verdadera, en este caso es cuestión de encontrar solo un valor que cumpla, $n = 6$

		      \item $\paratodo n \en \naturales, \existe m \en \naturales \talque m > n$.\\
		            La proposición es verdadera, si se elige por ejemplo a $m = n+1$

		      \item $\existe n \en \naturales \talque \paratodo m \en \naturales, m >n$.\\
		            La proposición es falsa, el único $n \en \naturales$ que no tiene un número menor estricto es el 1. Pero la condición
		            dice que $\paratodo m \en \naturales$ se debe cumplir y si m $1 \nless 1$

		      \item $\paratodo x \en \reales,\, x > 3 \entonces x^2 > 4$.\\
		            La proposición es verdadera. Si $x > 3 \entonces x^2 > 9 \flecha{en}[particular] x^2 > 9 > 4 \entonces x^2 > 4$

		      \item Si $z$ es un número real, entonces $z \en \complejos$.\\
		            Están proponiendo que dado $z \en \reales \entonces z \en \complejos$. Dado que $\reales \subseteq \complejos = \set{a \en \reales,\, b\en \reales \talque a + i b}$, con $i^2 = -1$
		            Por lo tanto para $b = 0$, podría generar todo $\reales$.
	      \end{enumerate}

	\item
	      \begin{enumerate}[label=(\alph*)]
		      \item $\existe n \en \naturales,\, n < 5 \y n > 8$.\\
		            $A =  \set{n \en \naturales \talque n < 5 \y n > 8} = \vacio \entonces \noexiste n$ que cumpla lo pedido.\\
		            \doceiiA \\
		            \red{¿Se puede justificar con un gráfico?}

		      \item $\paratodo n \en \naturales \talque n < 5 \o n > 8$.\\
		            La proposición es falsa, $n = 6$ no cumple estar en ese conjunto.

		      \item $\existe n \en \naturales, \paratodo m \en \naturales \talque m \leq n$.\\
		            La proposición es falsa, porque el conjunto $\naturales$ no tiene un máximo. $n = m+1$.

		      \item $\paratodo n \en \naturales \talque \existe m \en \naturales, m \leq n$.\\
		            La proposición es verdadera, el único $m \en \naturales$ que cumple eso es el $m = 1$.

		      \item $\existe x \en \reales,\, x \leq 3 \entonces x^2 \leq 4$.\\
		            La proposición es falsa. Dado dos conjunto:\\
		            \[
			            \llaves{c}{
				            \magenta{A = \set{x \en \reales \talque x \leq 3}} \\
				            \green{B = \set{x \en \reales \talque x^2 \leq 2}}
			            } \to \doceiiE
		            \]
		            {
		            \color{red}Si lo pienso como conjuntos, entiendo que $A \nsubseteq B$ entonces es falso.
		            Pero si leo el enunciado, me confunde el $\existe$, porque 3 sería un contraejemplo
		            y no se usaban para los $\paratodo$?
		            }

		      \item Si $z$ no es un número real, entonces $z \notin \complejos$.\\
		            La proposición es falsa. Están proponiendo que dado $z \notin \reales \entonces z \notin \complejos$. Si $z = i$, se prueba lo contrario.
		            Dado que $i \notin \reales$, pero  $i \en \complejos$
	      \end{enumerate}


	\item Reescribir las proposiciones (e), (f) usando las equivalencias del ejercicio 10 i)\\
	      $
		      \begin{array}{|c|c|c|c|}
			      \hline
			      p \entonces q           & \paratodo x \en \reales, x > 3 \entonces x^2 > 4 & \doceiiicero & A \stacktext{?}{\subseteq} B \to \Tilde                     \\
			      \hline
			      \sim q \entonces \sim p & x^2 \leq 4 \entonces x \leq 3                    & \doceiiiuno  & A \stacktext{?}{\subseteq} B \to \Tilde                     \\
			      \hline
			      \sim p \lor q           & x \leq 3 \o x^2 > 4                              & \doceiiidos  & A \union B \stackrel{?}{=} \universo \to \Tilde             \\
			      \hline
			      \sim (p \lor \sim q)    & \sim (x > 3 \y x^2 \leq 4 )                      & \doceiiitres & (A \inter B)^c \stackrel{?}{=}\vacio^c=\universo \to \Tilde \\
			      \hline
		      \end{array}
	      $
\end{enumerate}

%13
\ejercicio Determinar cuáles de las siguientes afirmaciones son verdaderas y cualesquiera sean los subconjuntos $A$, $B$ y $C$ de un conjunto referencial $\universo$ y cuáles no.
Para las que sean verdaderas, dar una demostración, para las otras dar un contraejemplo.

\begin{enumerate}[label=\roman*)]
	\item $(A \triangle B) - C = (A-C) \triangle (B - C)$. Es verdadera. Pruebo con tabla de verdad.\\
	      $
		      \begin{array}{|c|c|c|c|c|c|c|c|c|}
			      \hline
			      A & B & C & C^c & A - C & B - C & A \triangle B & (A \triangle B) - C & (A - C) \triangle (B - C) \\
			      \hline  \hline
			      V & V & V & F   & F     & F     & F             & F                   & F                         \\
			      V & V & F & V   & V     & V     & F             & F                   & F                         \\
			      \hline
			      V & F & V & F   & F     & F     & V             & F                   & F                         \\
			      V & F & F & V   & V     & F     & V             & V                   & V                         \\
			      \hline
			      F & V & V & F   & F     & F     & V             & F                   & F                         \\
			      F & V & F & V   & F     & V     & V             & V                   & V                         \\
			      \hline
			      F & F & V & F   & F     & F     & F             & F                   & F                         \\
			      F & F & F & V   & F     & F     & F             & F                   & F                         \\
			      \hline
		      \end{array}
	      $\\
	      Hay distribución entre la resta y una diferencias simétrica.

	\item $(A \inter B) \triangle C = (A \triangle C) \inter (B \triangle C)$
	\item $C \subseteq A \entonces B \inter C \subseteq (A \triangle B)^c$
	\item $A \triangle B = \vacio \sisolosi A = B$
\end{enumerate}

%14
\ejercicio Sean $A$, $B$ y $C$ subconjuntos de un conjunto referencial $\universo$. Probar que:
\begin{enumerate}[label=\roman*)]
	\item $A \inter (B \triangle C) = (A \inter B) \triangle (A \inter C)$

	\item $A - (B - C) = (A-B) \union (A \inter C)$\\
	      $(A - B) \union (A \inter C) = A \inter [(A \inter B^c) \union C] = A \inter [(A \union C) \inter (B^c \union C)] =\\
		      A \inter (B^c \union C) = A \union(B-C)^c = A - (B - C)$

	\item $A \triangle B \subseteq (A \triangle C) \union (B \triangle C)$ \Hacer
	\item $(A \inter C ) - B = (A - B) \inter C$
	\item $A \subseteq B \entonces A \triangle B = B \inter A^c$
	\item $A \subseteq C \sisolosi B^c \subseteq A^c$
	\item $A \inter C = \vacio \entonces A \inter (B \triangle C) = A \inter B$
\end{enumerate}

%15

\ejercicio Sean $A = \set{1, 2, 3},\, B = \set{1, 3, 5, 7}.$ Hallar $A \times A, A \times B, (A \inter B) \times (A \union B).$

\begin{itemize}
	\item $A \times A =
		      \llave{l}{
			      \set{a \en A, b \en A \talque (a, b) \en A \times A} \to \text{Comprensión}\\
			      \set{(1,1), (1,2), (1,3),(2,1),(2,2), (2,3), (3,1), (3,2), (3,3)} \to \text{Extensión}
		      }$

	\item  $A \times B = \cdots$

	\item $(A \inter B) \times (A \union B) = \\
		      \llave{l}{
			      \set{1, 3} \times \set{1,2,3,5,7} =
			      \begin{array}{c|c|c|c|c|c}
				      \times & 1     & 2      & 3      & 5      & 7     \\
				      \hline
				      1      & (1,1) & \cdots & \cdots & \cdots & (1,7) \\
				      \hline
				      3      & (3,1) & \cdots & \cdots & \cdots & (3,7) \\
				      \hline
			      \end{array}\\

			      (A \inter B) \times (A \union B) = \set{s \en (A \inter B), t \en (A \union B) \talque (s, t) \en (A \inter B) \times (A \union B)}
		      }$
\end{itemize}

\ejercicio Sean $A, B$ y $C$ conjuntos. Probar que:
\begin{enumerate}[label=\roman*)]
	\item $(A \union B) \times C = (A \times C) \union (B \times C)$
	\item $(A \inter B) \times C = (A \times C) \inter (B \times C)$
	\item $(A - B) \times C = (A \times C) - (B \times C)$
	\item $(A \triangle B) \times C = (A \times C) \triangle (B \times C)$
\end{enumerate}


\noindent\separador

\underline{\textit{Relaciones}}

Definción de Relación, $\relacion$:
\begin{center}
	\fbox{
		\parbox{.9\textwidth}{
			Sean $A$ y $B$ conjuntos. Una \textit{relación $\relacion$ de $A$ en $B$}
			es un subconjunto cualquiera $\relacion$ del producto cartesiano $A \times B$. Es decir $\relacion$
			de $A$ en $B$ si $\relacion \en \partes(A \times B)$.
		}
	}
\end{center}

\ejercicio Sean $A = \set{1, 2, 3}$ y $B = \set{1, 3, 5, 7}$. Verificar las siguientes
relaciones de $A$ y $B$ y en caso afirmativo graficarlas por medio de un diagrama
con flechas de $A$ en $B$ y por medio de puntos en el producto cartesiano $A \times B$.

\begin{multicols}{2}
	\begin{enumerate}[label=\roman*)]
		\item $\relacion = \set{(1,1), (1,3), (1,7), (3,1), (3,5)}\\
			      \diecisietei $

		\item $\relacion = \set{(1,1), (1,3), (2,7), (3,2), (3,5)}\\
			      \to 3 \relacion 2 \notin \partes(A \times B) $

		\item $\relacion = \set{(1,1), (2,7), (3,7)}\hacer$

		\item $\relacion = \set{(1,3), (2,1), (3,7)}\hacer$
	\end{enumerate}
\end{multicols}

%18
\ejercicio Sean $A = \set{1, 2, 3}$ y $B = \set{1, 3, 5, 7}$. Describir por extensión cada una de las
siguientes relaciones de $A$ en $B$:
\begin{enumerate}[label=\roman*)]
	\item $(a,b) \en \relacion \sisolosi a\leq b =\\
		      \set{(1,1), (1,3), (1,5), (1,7), (2,3), (2,5), (2,7), (3,3), (3,5), (3,7)}$

	\item $(a,b) \en \relacion \sisolosi a > b = \set{}\hacer$

	\item $(a,b) \en \relacion \sisolosi a \cdot b = \set{(2,1), (2,3), (2,5), (2,7)}$

	\item $(a,b) \en \relacion \sisolosi a + b > 6 = \set{}\hacer$
\end{enumerate}
%19
\ejercicio Sea $A = \set{a,b,c,d,e,f,g,h}$. Para cada uno de los siguientes gráficos describit por
extensión la relación en $A$ que representa y determinar si es \textit{reflexiva, simétrica, antisimétrica o transitiva}.
\begin{enumerate}
	\item  \textit{Reflexiva}: $(x, x) \en \relacion \paratodo x \en A$ o $x \relacion x.\, \paratodo x \en A$. Gráficamente,
	      cada elemento tiene que tener un bucle.
	      \begin{tikzpicture}[baseline=0, >=Latex, draw=Aquamarine]
		      \node[](a) {$\bullet$} edge [in=0,out=90,loop] ();
		      \node[] at (a.west) {$x$};
	      \end{tikzpicture}

	\item  \textit{Simétrica}: $(x, y) \en \relacion$, entonces el par $(y, x) \en \relacion$, también si $\paratodo x, y \en A, x\relacion y \entonces y \relacion x$.
	      Gráficamente tiene que haber un ida y vuelta en cada elemento de la relación.
	      \begin{tikzpicture}[baseline=-10, >=Latex, draw=Aquamarine]
		      \node[] (x) {$\bullet$};
		      \node[] at (x.west) {$x$};
		      \node[below right =0.2 of x] (y) {$\bullet$};
		      \node[] at (y.south east) {$y$};
		      \draw[->, bend left=1cm] (x.center) to (y);
		      \draw[->, bend left=1cm] (y.center) to (x);
	      \end{tikzpicture}

	\item  \textit{Antisimétrica}: $(x, y) \en \relacion$, con $x \distinto y$ entonces el par $(y, x) \notin \relacion$, también se puede pensar
	      como $\paratodo x,y \en A, x\relacion y$ e $ y \relacion x \entonces x = y$. Gráficamente \textbf{no} tiene que haber ningún ida y vuelta en el gráfico.
	      Solo en una dirección.
	      \begin{tikzpicture}[baseline=-10, >=Latex, draw=Aquamarine]
		      \node[] (x) {$\bullet$};
		      \node[] at (x.west) {$x$};
		      \node[below right =0.2 of x] (y) {$\bullet$};
		      \node[] at (y.south east) {$y$};
		      \draw[->] (x.center) to (y.center);
	      \end{tikzpicture}
	\item  \textit{Transitiva}: Para toda terna $x, y, z \en A$ tales que $(x, y) \en \relacion$ e $(y,z) \en \relacion$, se tiene que $(x, z) \en \relacion$.
	      Otra manera sería si $\paratodo x, y, z \en A$, $x \relacion y$ e $y \relacion z \entonces x\relacion z$. Gráficamente tiene que haber flecha directa
	      entre las puntas de cualquier camino que vaya por más de dos nodos.
	      \begin{tikzpicture}[baseline=0, >=Latex, draw=Aquamarine]
		      \node[] (x) {$\bullet$};
		      \node[] at (x.north east) {$x$};
		      \node[below right = 0.6cm of x] (y) {$\bullet$};
		      \node[] at (y.north east) {$y$};
		      \node[below right = 0.6cm of y] (z) {$\bullet$};
		      \node[] at (z.north east) {$z$};

		      \draw[->, bend left] (x.center) to (y.center);
		      \draw[->, bend left] (y.center) to (z.center);
		      \draw[->,magenta, bend right]
		      (x.center)
		      to node[midway,below left] {atajo}
		      (z.center);
	      \end{tikzpicture}
\end{enumerate}

\begin{enumerate}[label=\roman*)]

	\item
	      \begin{minipage}{0.37\textwidth}
		      \diecinuevei
	      \end{minipage}
	      \begin{minipage}{0.5\textwidth}
		      \begin{itemize}
			      \item No es reflexiva, porque no hay bucles en todos los vértices, en particular $a \norelacion a$.
			      \item No es simétrica, porque $d \norelacion c$.
			      \item No es antisimétrica, porque $a \relacion b$ y $b \relacion a$ con $a \neq b$.
			      \item No es transitiva, porque $c \relacion h$ y $h \relacion g$, pero $c \norelacion h$.
		      \end{itemize}
	      \end{minipage}

	\item \hacer

	\item \hacer

	\item
	      \begin{minipage}{0.37\textwidth}
		      \diecinueveiv
	      \end{minipage}
	      \begin{minipage}{0.5\textwidth}
		      \begin{itemize}
			      \item Reflexiva, porque hay bucles en todos los elementos de $A$.
			      \item Es simétrica, porque hay ida y vuelta en todos los pares de vértices.
			      \item No es antisimétrica, porque $a \relacion b$ y $b \relacion a$ con $a \neq b$.
			      \item Es transitiva, porque hay \textit{atajos} en todas las relaciones de ternas.
		      \end{itemize}
	      \end{minipage}
\end{enumerate}

%20
\ejercicio Sea $A = \set{1, 2, 3, 4, 5, 6}$. Graficar la relación, $\relacion= {(1,1), (1,3), (3,1), (3,3), (6,4), (4,6), (4,4), (6,6)}$
\begin{minipage}{0.25\textwidth}
	\veinte
\end{minipage}
\begin{minipage}{0.7\textwidth}
	\begin{itemize}
		\item No es reflexiva porque no hay bucles ni en 2 ni en 5.
		\item Es simétrica, porque hay ida y vuelta en todos los pares de vértices.
		\item No es antisimétrica, porque $1 \relacion 3$ y $3 \relacion 1$ con $1 \neq 3$.
		\item Es transitiva. \\
		      \red{Chequear. Caso particula donde no hay ternas de $x,y,z$ distintos}.
		      \blue{Sí, el que $2$ esté ahí solo ni cumple la hipótesis de transitividad.}
	\end{itemize}
\end{minipage}

%21
\ejercicio \hacer

%22
\ejercicio En cada uno de los siguientes casos determinar si la relación $\relacion $ en $A$ es reflexiva, simétrica,
antisimétrica, transitiva, de equivalencia o de orden.

\begin{enumerate}
	\item \textit{Relación de equivalencia}: La relación debe ser reflexiva, simétrica y transitiva.
	\item \textit{Relación de orden}: La relación debe ser reflexiva, antisimétrica y transitiva.
\end{enumerate}

\begin{enumerate}[label=\roman*)]
	\item  $A = \set{1,2,3,4,5}, \relacion = {(1,1), (2,2), (3,3), (4,4), (5,5), (1,2), (1,3), (2,5), (1,5)}$\\
	      \begin{enumerate}
		      \item[R:] Es reflexiva, porque hay bucles en todos los elementos de $A$.
		      \item[S:] No es simétrica, dado que existe $(1, 5)$, pero no $(5, 1)$
		      \item[AS:] Es antisimétrica. No hay ningún par que tenga la vuelta, excepto los casos $x \relacion x$.
		      \item[T:] Es transitiva. La terna 1, 2, 5 es transitiva.
	      \end{enumerate}
	      La relación es R, AS y T, por lo tanto es una \textit{relación de orden}.

	\item \hacer

	\item \hacer
	\item \hacer
	\item \hacer
	\item $A = \partes(\set{n \en \naturales \talque n \leq 30})$, $\relacion$ definida por $X \relacion Y \sisolosi 2 \notin X \inter Y^c$\\
	      $\begin{array}{|c|c|c|c|c|c|}
			      \hline
			      2 \en X     & 2 \en Y     & 2 \en Y^c   & 2 \en X^c   & 2 \notin X \inter Y^c & 2 \notin Y \inter X^c \\ \hline  \hline
			      \magenta{V} & \magenta{V} & \magenta{F} & \magenta{F} & \magenta{V}           & \magenta{V}           \\
			      \cyan{V}    & \cyan{F}    & \cyan{V}    & \cyan{F}    & \cyan{F}              & \cyan{V}              \\
			      \cyan{F}    & \cyan{V}    & \cyan{F}    & \cyan{V}    & \cyan{V}              & \cyan{F}              \\
			      \magenta{F} & \magenta{F} & \magenta{V} & \magenta{V} & \magenta{V}           & \magenta{V}           \\ \hline
		      \end{array}$.\\
	      \begin{enumerate}
		      \item[R:]
		            La relación es reflexiva ya que para que un elemento $X$ esté relacionado con sí mismo debe ocurrir
		            que $X \relacion X \sisolosi 2 \notin X \inter X^c$, es decir $2 \notin \vacio$, lo cual es siempre cierto.

		      \item[S:]
		            La relación no es simétrica. Se puede ver con la \cyan{segunda y tercera} fila de la tabla con un contraejemplo.
		            $X = \set{1}$ y $Y = \set{2},\, X,Y \subseteq A$, $X \relacion Y$, pero $Y \norelacion X$,

		      \item[AS:]
		            La relación no es antisimétrica. Se puede ver con la \magenta{primera o cuarta} fila tabla con un contraejempl
		            con un contraejemplo. Si $X = \set{1,2}$ e $Y = \set{2,3} \entonces X \relacion Y$ y además $Y \relacion X$
		            con  $X \distinto Y$.

		      \item[T:]
		            Es transitiva. Si bien no es lo más fácil de explicar, se puede ver en la tabla que para tener 2 relaciones
		            en una terna $X, Y, Z$ no se puede llegar nunca al caso de la segunda fila de la tabla, donde se lograría que
		            $X \norelacion Z$
	      \end{enumerate}

	\item $A = \naturales \times \naturales,\, \relacion$ definida por $(a,b) \relacion (c,d) \sisolosi bc$ es múltiplo de $ad$.
	      \begin{enumerate}
		      \item[R:] $(a,b) \relacion (a,b) \sisolosi ba = k\cdot ab$ con $k=1$, se concluye que sí es reflexiva.
		      \item[S:]
		            $ \llave{l}{
				            (a,b) \relacion (c,d) \sisolosi bc \llamada{1}= k\cdot ad \\
				            (c,d) \relacion (a,b) \sisolosi ad = h\cdot bc \llamada{1}= h\cdot k \cdot ad = k'ad
			            }$.\\
		            con  $k'=1$ se cumple la igualdad. La relación es simétrica.
		      \item[AS:] Si tomo $(a, b) = (4, 2)$ y $(c,d) = (16, 4)$, tengo que
		            $(a,b) \relacion (c,d)$ con $(a,b) \neq (c,d)$. Por lo tanto
		            la relación no es antisimétrica.
		      \item[T:]$
			            \llave{l}{
				            (a,b) \relacion (c,d) \sisolosi bc \llamada{1}= k\cdot ad \\
				            (c,d) \relacion (e,f) \sisolosi de \llamada{1}= h\cdot cf \\
				            \qvq (a,b) \relacion (e,f) \sisolosi \magenta{be = k'\cdot af}\\
				            \flecha{multiplico}[M.A.M.]
				            \llaves{l}
				            {
					            bc \llamada{1}= k\cdot ad \\
					            de \llamada{1}= h\cdot cf
				            } \flecha{y}[acomodo]
				            be \cdot \cancel{cd} = k \cdot h \cdot af \cdot \cancel{cd} \to
				            \magenta{be \stackrel{\Tilde}= k' \cdot af}.
			            }$\\
		            Se concluye que la relación es transitiva.
	      \end{enumerate}
	      Con esos resultados se puede decir que $\relacion$ en $A$ es de \textit{equivalencia}.

\end{enumerate}
%23
\ejercicio
Sea $A$ un conjunto. Describir todas las relaciones en $A$ que son a la vez
\begin{multicols}{2}
	\begin{enumerate}[label=\roman*)]
		\item simétricas y antisimétricas\\ \red{elementos en bucles sueltos?}
		\item de equivalencia y de orden\\ \red{Idem anterior}
	\end{enumerate}
\end{multicols}

¿Puede una relación en $A$ no ser ni simétrica ni antisimétrica? \red{22 (vi)?}\\

%24
\ejercicio
Sea $A = \set{a, b, c, d, e, f}$. Dada la relación de equivalencia en $A$:
\begin{align*}
	\relacion = \set{(a, a), (b, b), (c, c), (d, d), (e, e), (f, f), (a, b), (b, a), (a, f), (f, a), (b, f), (f, b), (c, e), (e, c)}
\end{align*}
Hallar la clase $\clase{a}$ de $a$, la clase $\clase{b}$ de $b$, la clase $\clase{c}$ de $c$, la clase $\clase{d}$ de $d$, y la partición asociada a $\relacion$ \\
\veinticuatro
$\to
	\llave{rclclcl}{
		\clase{a} & = & \set{a, b, f}  & = & \clase{b} & = & \clase{f} \\
		\clase{c} & = & \set{c, e}     & = & \clase{e} &   & \\
		\clase{d} & = & \set{d}        &   &         &   & \\
	}$ \\
La partición asociada a $\relacion:\,  \set{\set{d}, \set{c,e}, \set{a, b, f}} = \set{\clase{d}, \clase{b}, \clase{a}}$.

%25
\ejercicio
\Hacer
Sea $A = \set{1,2,3,4,5,6,7,8,9,10}$. Hallar y graficar la relación de equivalencia en $A$ asociada a la partición $\set{ \set{1,3}, \set{2,6,7}, \set{4,8,9,10}, \set{5} }$.
¿Cuántas clases de equivalencia distintas tiene? Hallar un representante para cada clase.

%26
\ejercicio
\Hacer
Sean $P = \partes(\set{1,2,3,4,5,6,7,8,9,10})$ el conjunto de partes de $\set{1,\dots, 10}$ y $\relacion$ la relación en $P$ definida por:
\[
	A \relacion B \sisolosi (A \triangle B) \sisolosi \inter \set{1,2,3} = \vacio
\]
\begin{enumerate}[label=\roman*)]
	\item  Probar que $\relacion$ es una relación de equivalencia y decidir si es antisimétrica (\textit{\underline{Sugerencia:}} usar adecuadamente el ejercicio \textbf{14iii})).
	\item Hallar la clase de equivalencia de $A = \set{1,2,3}$.
\end{enumerate}

\separadorCorto

%27
\ejercicio
Sean $A = \set{n \en \naturales \talque n \leq 92}$ y
$\relacion$ la relación en $A$ definida por
$x \relacion y \sisolosi x^2 - y^2 = 93x - 93y$
\begin{enumerate}[label=\roman*)]
	\item Probar que $\relacion$ es una relación de equivalencia. ¿Es antisimétrica?
	\item Hallar la clase de equivalencia de cada $x \en A$. Deducir cuántas clases de equivalencia \textbf{distintas} determina la relación $\relacion$
\end{enumerate}

\separadorCorto

\begin{enumerate}[label=\roman*)]
	\item Primero acomodo la condición de la relación:\\
	      $x^2 - y^2 = 93x - 93y \flecha{acomodo}
		      \llave{rrcl}{
			      \cyan{*^1} & x & = & y  \\
			      & & \o &\\
			      \cyan{*^2} & x + y & = & 93
		      }$\\
	      Para ser relación de equivalencia es necesario que:\\
	      $\relacion \to
		      \llaves{cl}{
			      R: & \text{Reflexiva} \to x \relacion x \sisolosi x \llamada{1} = x  \Tilde\\
			      S: & \text{Simétrica} \to
			      \llave{l}{
				      x \relacion y \sisolosi x + y\llamada{1}I = 93\\
				      y \relacion x \sisolosi y + x \llamada{1}I = 93  \Tilde
			      }\\
			      T: & \text{Transitiva} \to
			      \llave{l}{
				      x \relacion y \sisolosi x \llamada{1}I = 93 - y\\
				      y \relacion z \sisolosi y \llamada{1}I = 93 - z \\
				      \flecha{resto}[m.a.m] x - y = -y + z \to x \llamada{1} = z \sisolosi x \relacion z
			      }\\
		      }$\\
	      La $\relacion$ no es antisimétrica, como contraejemplo se ve que $1 \relacion 92$ y $92 \relacion 1$ con $1 \distinto 92$

	\item
	      \begin{minipage}{0.2\textwidth}
		      \veintisiete
	      \end{minipage}
	      \begin{minipage}{0.7\textwidth}
		      A priori no sé como encontrar las clases de equivalencia, pero solo buscando la relación del $1$ con algún número (excepto el mismo) veo que únicamente
		      se puede relacionar con el $92$ por la condición de la relación. De ahí se pueden inferir que todas las clases van a ser así chiquitas.Las clases de equivalencia :

		      $\llave{ccccc}{
				      \clase{1} & = & \clase{92} & = & \set{1, 92}\\
				      \clase{2} & = & \clase{91} & = & \set{2, 91}\\
				      \vdots  &   & \vdots   &   & \vdots\\
				      \clase{45} & = & \clase{47} & = & \set{45, 47}\\
				      \clase{46} & = & \set{46}
			      }$\\
		      Hay entonces 46 clases. $A = \set{\clase{1},\, \clase{2},\, \dots,\, \clase{45},\, \clase{46}}$
	      \end{minipage}
\end{enumerate}

%28
\ejercicio
\begin{enumerate}[label=\roman*)]
	\item
	      Sea $A = \set{1, 2, 3, 4, 5, 6, 7, 8, 9, 10}$. Consideremos en $\partes(A)$ la relación de equivalencia dada
	      por el cardinal (es decir, la cantidad de elementos): Dos subconjuntos de $A$ están relacionados si y solo si tienen la misma cantidad
	      de elementos ¿Cuántas clases de equivalencia \textbf{distintas} determina la relación? Hallar un representante par acada clase.\\

	      \separadorCorto

	      $\partes(A) = \set{\vacio, \set{1}, \set{1,2}, \cdots, \set{1, 2, 3, 4, 5, 6, 7, 8, 9, 10}}$, el conjunto $\partes(A)$ tiene un total de
	      $2^{10} = 1024$ elementos. La relación determina 11 \textit{clases de equivalencia} distintas.\\
	      $\llave{lcl}{
			      \text{Conjuntos con 0 elementos: } & \clase{0}  & \vacio\\
			      \text{Conjuntos con 1 elemento: } & \clase{1} & \set{3}\\
			      \text{Conjuntos con 2 elementos: }& \clase{2} & \set{5,2}\\
			      \text{Conjuntos con 3 elementos: }& \clase{3} & \set{1,6, 3}\\
			      \text{Conjuntos con 4 elementos: }& \clase{4} & \set{1,8, 10,4}\\
			      \vdots                            &\vdots  & \vdots\\
			      \text{Conjuntos con 10 elementos: } & \clase{10} & \set{1,2,3,4,5,6,7,8,9,10} = A\\
		      }$

	\item
	      En el conjunto de todos los subconjuntos finitos de $\naturales$, consideremos nuevamente la relación de equivalencia dada por el cardinal: Dos subconjuntos finitos
	      de $\naturales$ están relacionados si y solo si tienen la misma cantidad de elementos ¿Cuántas clases de equivalencia \textbf{distintas} determina la relación?
	      Hallar un representante para cada clase.\\

	      \separadorCorto

	      Es parecido al inciso anterior, donde ahora $A = \set{1,2,3, \cdots, N-1, N}$, donde $\partes(\naturales_N)$ tiene $2^N$ elementos.\\
	      La relación determina $N+1$ \textit{clases de equivalencia} distintas.\\
	      $\llave{lcl}{
			      \text{Conjuntos con 0 elementos: } & \clase{0}  & \vacio\\
			      \text{Conjuntos con 1 elemento: } & \clase{1} & \set{3}\\
			      \text{Conjuntos con 2 elementos: }& \clase{2} & \set{5,2}\\
			      \text{Conjuntos con 3 elementos: }& \clase{3} & \set{1,6, 3}\\
			      \text{Conjuntos con 4 elementos: }& \clase{4} & \set{1,8, 10,4}\\
			      \vdots                            &\vdots  & \vdots\\
			      \text{Conjuntos con 10 elementos: } & \clase{N} & \set{1,2,3,4, \cdots, N-1, N} \naturales_N\\
		      }$
\end{enumerate}

\noindent\separador

\noindent\underline{\textit{Funciones}}\\

Sean $A$ y  $B$ conjuntos, y sea $\relacion$ de $A$ en $B$. Se dice que
$\relacion$ es una \textit{función} cuando todo elemento $x \en A$ está relacionado con algún
$y \en B$, y este elemento $y$ es único. Es decir:\\

$\llave{l}{
		\paratodo x \en A, \existe! y \en B \talque x \relacion y\\
		\paratodo x \en A, \existe y \en B \talque x\relacion y, \\
		\text{si } y, z \en B \text{ son tales que } x \relacion y \text{ y } x \relacion z \entonces y = z.
	}
$
\begin{itemize}
	\item Dada $f:A\,(dominio) \to B\,(codominio)$ el conjunto \textit{imagen} es: $\imagen(f)= \set{y \en B : \existe x \en A \talque f(x) = y}$
	\item \begin{itemize}
		      \item \textit{inyectiva:} si $\forall x, x' \en A \text{ tales que } f(x) = f(x')$ se tiene que $ x = x'$
		      \item \textit{sobreyectiva:} si $\forall y \en B, \existe x \en A \text{ tal que } f(x) = y.\: f\text{ es sobreyectiva si } \imagen(f) = B$
		      \item \textit{biyectiva:} Cuando es inyectiva y sobreyectiva.
	      \end{itemize}
	\item $A, B, C$ conjuntos y $f: A \to B \to C,\, g: B \to C$ funciones. Entonces la \textit{composición} de $f$ con $g$, que se nota:\\
	      $g \comp f = g\left(f(x)\right),\, \paratodo x \en A$, resulta ser una función de $A$ en $C$.
	\item $f$ es biyectiva cuando: $f\inv : B \to A$ es la función que satisface que:\\
	      $\paratodo y \en B: f\inv(y) = x \sisolosi f(x) = y$
\end{itemize}

%29
\ejercicio
Determinar si $\relacion$ es una función de $A$ en $B$ en los casos
\begin{enumerate}[label=\roman*)]
	\item $A = \set{1,2,3,4,5},\, B = \set{a,b,c,d},\, \relacion = \set{(1,a), (2,a),(3,a),(4,b),(5,c),(3,d)} $\\
	      No es función, dado que $3 \relacion a$, $3 \relacion d$ y $a \distinto d$

	\item $A = \set{1,2,3,4,5},\, B = \set{a,b,c,d},\, \relacion = \set{(1,a), (2,a),(3,d),(4,b)}$\\
	      No es función, dado que todo elemnto de $A$ tiene que estar relacionado a algún elemento de $B,\, 5 \norelacion y$ para ninún $ y \en B$

	\item $A = \set{1,2,3,4,5},\, B = \set{a,b,c,d},\, \relacion = \set{(1,a), (2,a),(3,d),(4,b),(5,c)} $\\
	      Es función.

	\item $A = \naturales,\, B = \reales, \, \relacion = \set{(a,b) \en \naturales \times \reales \talque a = 2b - 3} $\\
	      Es función.

	\item $A = \reales,\, B = \naturales, \, \relacion = \set{(a,b) \en \reales \times \naturales \talque a = 2b - 3} $\\
	      No es función, $\sqrt{2} \norelacion b$ para ningún $b \en \naturales$
	\item $A = \enteros,\, B = \enteros, \, \relacion = \set{(a,b) \en \enteros \times \enteros \talque a +b \text{ es divisible por }5} $\\
	      No es función, porque $0 \relacion 5$ y $0 \relacion 10$ y necesito que $\paratodo x \en \enteros,\, \existe!\, y \en \enteros$
\end{enumerate}

%30
\ejercicio
Determinar si las siguientes funciones son inyectivas, sobreyectivas o biyectivas. Para las que sean biyectivas hallar la inversa y para la que no sean sobreyectivas hallar la imagen.\\
\begin{enumerate}[label=\roman*)]
	\item $f: \reales \to \reales,\quad f(x) = 12 x^2 - 5$\\
	      No es \textit{inyectiva}, $f(-1) = f(1)$.\\
	      No es \textit{sobreyectiva}, $\imagen(f) = [-5, +\infinito)$.
	\item \hacer

	\item \hacer

	\item $f: \naturales \to \naturales,\quad
		      f(n) = \llave{ll}{
			      \frac{n}{2} & \text{si $n$ es par}\\
			      n + 1 & \text{si $n$ es impar}\\
		      }$\\
	      Es \textit{inyectiva} y \textit{sobreyectiva}. $\paratodo m,m' \en \naturales,
		      \llave{l}{
			      f(2m) = \frac{2m}{2} = m \\
			      f(2m'-1) = 2m' - 1 + 1 = 2m' \\
		      } \to$ Si bien $f(8) = f(3)$ la función es \textit{sobreyectiva} porque
	      genera todo $\naturales$ tan solo con la parte par de la función.\\
	      $f^{-1}(n) =
		      \llave{ll}{
			      2n    & \text{si $n$ es par}\\
			      n - 1 & \text{si $n$ es impar} \\
		      }$\\
	      \red{Qué onda?}

\end{enumerate}

\separador
\separador

\subsection*{Notas random}
\begin{itemize}
	\item Cardinal: cantidad de elementos de un conjunto.
	\item El conjunto $\partes(A)$ tiene $2^n$ el cardinal del conjunto $A$.
	\item Una relación $\relacion$ es un subconjunto de un producto cartesiano.
	\item  Leyes de absorción, obvias pero no tan.
	      $\llave{c}{
			      A \union( A \inter B ) = A\\
			      A \inter ( A \union B ) = A\\
		      }$
	\item probar identidades de conjuntos:
	      \begin{itemize}
		      \item
		            $A = B \to
			            \llave{c}{
				            A \subset B \\
				            B \subset A \\
			            }$
		      \item lógica proposicional
		      \item tabla de verdad
	      \end{itemize}
	\item
\end{itemize}
\end{document}


% Ejemplos:
% text color --> {\color{violet} texto violeta}
% color boxed text --> \colorbox{violet}{texto con fondo violeta}
% color framed text --> \fcolorbox{violet}{white}{texto enmarcado viloleta}
% IMAGENES --> \includegraphics[width=0.5\textwidth]{ARCHIVO}' >> "$nombre_final"
