\textit{\underline{Raíces de un número complejo: }}
\begin{itemize}
	\item Sean $z, w \en \complejos -\set{0}$, $z = re^{\theta i}$ y $w = se^{\varphi i }$ con $r,\, s \en \reales_{>0}$
	      y $\theta,\, \varphi \en \reales$.\\ Entonces $z = w \sisolosi
		      \llave{l}{
			      r = s\\
			      \theta = \varphi + 2 k\pi,\ \text{para algún } k \en \enteros
		      }$
	\item raíces $n$-esimas: $w^n = z
		      \to
		      \llave{l}{
		      s^n = r \\
		      \varphi \cdot n = \theta + 2 k \pi \quad\to \text{para algún $k\en \enteros$}\\
		      \text{$n$ raíces distintas} \to w_k=se^{\varphi_k i}, \text{ donde } s = \sqrt{r} \text{ y }
		      \varphi_k = \frac{\theta}{n} + \frac{2k\pi}{n} = \frac{\theta + 2k\pi}{n}
		      }$
\end{itemize}
\begin{itemize}
	\item $G_n = \set{w \en \complejos / w^n = 1} = \set{e^{\frac{2k \pi}{n} i}\ :\ 0\leq k \leq n-1}$

	\item $(G_n, \cdot)$ es un grupo abeliano, o conmutativo.
	      \begin{itemize}
		      \item $\paratodo w, z \en G_n, w z = z  w \text{ y } z m \en G_n$.

		      \item $1 \en G_n,\ w \cdot 1 = 1 \cdot w = w \qquad \paratodo w \en G_n$.

		      \item $w \en G_n \entonces \existe w^{-1} \en G_n,\ w \cdot w^{-1} = w^{-1}\cdot w = 1$
		            \begin{itemize}
			            \item $\conj w \en G_n,\ w \cdot \conj w = |w|^2 = 1 \entonces \conj w = w^{-1}$
		            \end{itemize}
	      \end{itemize}
	\item \textit{Propiedades: $w \en G_n$}
	      \begin{itemize}
		      \item $m \en \enteros$ y $n \divideA m \entonces w^m = 1$.

		      \item $\congruencia{m}{m'}{n} \entonces w^m = w^{m'}\quad (w^m = w^{r_n(m)})$

		      \item $n \divideA m \sisolosi G_n \subseteq G_m$

		      \item $G_n \inter G_m = G_{(n:m)}$

		      \item Si $(G, \cdot)$ es un grupo y $\#G = n$ decimos que $G$ siempre es cíclico si
		            $\existe w\en G / G = \set{1,w, w^2,\dots, w^{n-1}}$\\
		            \begin{itemize}
			            \item \textit{Observación: } $G_n$ es un grupo cíclico, ej, $w_1 = e^\frac{2\pi i}{n} \to (w_1)^k = w_k$\\
			                  $\to$ las potencias de $w_1$ generan todo $G_n = \set{1, w_1, w_1^2,\dots,w_1^{n-1}}$
		            \end{itemize}

		      \item $w$ es raíz $n-$ésima primitiva de 1 si:
		            $G_n = \set{1,w,w^2,\dots,w^{n-1}} =
			            \set{w^k\ :\ 0\leq k \leq n-1}$\\
		            Ejemplo: $i, -i$ son primitivas de $G_4 = \set{1,i,-1,-i} = \set{i^k\ :\ 0 \leq k \leq 3}$, pero 1 y -1 no son raíces primitivas de $G_4$.
	      \end{itemize}
	\item \textit{Definición: }
	      Sea $w$ una raíz primitiva de orden $n$ (el orden de
	      $w \en G_n,\, \text{ord}(w) = \text{min}\set{k \en \naturales / w^k = 1}$)
	      \begin{itemize}
		      \item $w^m = 1 \sisolosi n \divideA m$
		      \item \textit{Observación: } Si $w \en G_n \entonces \ord(w) \divideA n$
	      \end{itemize}
	\item La suma de las raíces $n$-ésimas de 1 da:
	      $\sumatoria{k=0}{n-1}w_1^k = \frac{w_1^n -1}{w_1 -1} = 0$ pues $w_1 \distinto 1$
	\item El producto de las raíces $n$-ésimas de 1 da:
	      $\productoria{k=0}{n-1} w_1^k = w_1^{0+1+\dots + n-1} =
		      w_1^{\frac{n(n-1)}{2}} =
		      \llave{rl}{
			      1 & \text{si $n$ es impar}\\
			      -1 & \text{si $n$ es par}
		      }$
	\item Sea $w \en G_n$ primitiva. Entonces
	      \begin{itemize}
		      \item $w^k \text{ es primitiva } \sisolosi k \cop n $
		      \item $w_k = e^{\frac{2k\pi}{n}i}$ es primitiva $\sisolosi k \cop n$
		      \item En particular para $n = p$ primo: $w_k$ es primitiva para $1\leq k < p$ o sea si
		            $w \en G_p$ y $w \distinto 1$, entonces $w$ es primitiva
	      \end{itemize}
	\item $w$ es raíz primitiva de $G_n$ y $k \divideA n \entonces w^k$ es primitiva de $G_\frac{n}{k}$
\end{itemize}
