\textit{\underline{Raíces de un número complejo: }}


\begin{itemize}[label= \tiny \faIcon{meh}]
                
        \item hypertarget{teoria-6:tablita}{Tablita de ángulos agradables}:
  \item Sean $z, w \en \complejos -\set{0}$, $z = r_z e^{\theta_z i}$ y $w = r_w e^{\theta_w i }$ con $r_z,\, s_w \en \reales_{>0}$
        y $\theta_z,\, \theta_w \en \reales$.\par
        Entonces $z = w
          \Sii{\red{!!}}
          \llave{l}{
          r_z = r_w \\
          \theta_z = \theta_w + 2 k \pi,\ \text{para algún } k \en \enteros
          }$
  \item raíces $n$-esimas: $w^n = z
          \Sii{\red{!!}}
          \llave{l}{
          (r_w)^n = r_z \\
          \theta_w \cdot n = \theta_z + 2 k \pi \quad \text{para algún $k\en \enteros$}
          }$\par
        De donde se obtendrán $n$ raíces distintas:
        $$
          w_k = z_w e^{\theta_{w_k} i}, \text{ donde } r_w = \sqrt[n]{r_z} \ytext
          \theta_{w_k} = \frac{\theta_z}{n} + \frac{2k\pi}{n} = \frac{\theta_z + 2k\pi}{n}
        $$
        \red{Entender bien como sacar raíces $n$-ésimas es importantísimo para toda la guía de complejos y la próxima de polinomios}.
\end{itemize}

\underline{\textit{Grupos $G_n$:}}

\begin{itemize}[label=\color{gray} \tiny \faIcon{smile}]
\item $G_n = \set{w \en \complejos / w^n = 1} = \set{e^{\frac{2k \pi}{n} i}\ :\ 0\leq k \leq n-1}$
%==========================
% Macro para reducir los gráficos
\newcommand{\unitcircle}[1]{
  \begin{tikzpicture}[baseline=0, scale=1, every node/.style={font=\tiny}]
    \draw[ultra thin,->,gray] (-1.5,0) -- (1.8,0) node[below] {Re};
    \draw[ultra thin,->,gray] (0,-1.5) -- (0,1.5) node[right] {Im};
    \draw[ultra thin] (0,0) circle (1);
    \foreach \x in {0,...,#1} {
        \ifnum \x < #1 {
              \filldraw (\x*360/#1:0.8) node {$\x$};
              \filldraw (\x*360/#1:1) circle (1pt);
              \filldraw (\x*360/#1:1.4) node {$e^{i \frac{2\pi}{#1} \cdot \x}$};
              \draw[thick, Cerulean] (\x*360/#1:1) -- ({(\x+1)*360/#1}:1);
            }
        \fi
      }
  \end{tikzpicture}
}
% Fin macros
%================================

\begin{multicols}{2}
  \begin{enumerate}[label={($n$=\arabic*)}]
    \item $w = 1$
    \item $w = \pm 1$
    \item \unitcircle{3}
    \item \unitcircle{4}
    \item \unitcircle{5}
    \item \unitcircle{6}
    \item \unitcircle{7}
    \item \unitcircle{8}
    \item \unitcircle{9}
    \item \unitcircle{10}
  \end{enumerate}
\end{multicols}
Notar que:
\begin{itemize}
  \item Si $n$ es par el grupo tiene al $-1$.
  \item Toda raíz compleja tiene a su conjugado complejo.
  \item Para ir de un punto a otro, se lo múltiplica por $e^{i \theta}$ eso \textit{rota} al número en $\theta$
        respecto al origen.
\end{itemize}

\item $(G_n, \cdot)$ es un grupo abeliano, o conmutativo.
\begin{itemize}
  \item $\paratodo w, z \en G_n, w z = z  w \text{ y } z m \en G_n$.

  \item $1 \en G_n,\ w \cdot 1 = 1 \cdot w = w \qquad \paratodo w \en G_n$.

  \item $w \en G_n \entonces \existe w^{-1} \en G_n,\ w \cdot w^{-1} = w^{-1}\cdot w = 1$
        \begin{itemize}
          \item $\conj w \en G_n,\ w \cdot \conj w = |w|^2 = 1 \entonces \conj w = w^{-1}$
        \end{itemize}
\end{itemize}

\item \textit{Propiedades: $w \en G_n$}
\begin{itemize}
  \item $m \en \enteros$ y $n \divideA m \entonces w^m = 1$.

  \item $\congruencia{m}{m'}{n} \entonces w^m = w^{m'}\quad (w^m = w^{r_n(m)})$

  \item $n \divideA m \sisolosi G_n \subseteq G_m$

  \item $G_n \inter G_m = G_{(n:m)}$

  \item La suma de una raíz $w$ de $G_n$:
        $\sumatoria{k=0}{n-1}w^k = \frac{w^n -1}{w -1} = 0$ si $w \distinto 1$
\end{itemize}
