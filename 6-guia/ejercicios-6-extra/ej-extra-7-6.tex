\begin{enunciado}{\ejExtra}
  Sea $\omega \en G_{10}$ tal que $\omega^5 \distinto 1$. Encuentre la parte real de
  $$
    \omega + \omega^{-7} + \conj{\omega}^6 + \omega^8 + \sumatoria{k=5}{98}\omega^{5k}.
  $$
\end{enunciado}
\hyperlink{teoria6:propiedadesGn}{Algunas resultados de este tema acá {\tiny $\ot$ click} }

Dado que $\omega \en G_{10}$ ocurre que:
$$
  \omega^5 = \left(e^{i \frac{2k\pi}{10}}\right)^5
  \sii
  \omega^5 =
  \llave{cl}{
    1 & \text{si $k$ es par}\\
    -1 & \text{si $k$ es impar}
  }
  \Entonces{\red{enunciado}}
  \omega^5 = -1
$$
Con ese \textit{resultadillo} ahora podemos reescribir el enunciado como:
$$
  \begin{array}{rcl}
    \omega + \omega^{-7} + \conj{\omega}^6 + \omega^8 + \sumatoria{k = 5}{98}\omega^{5k}
     & \igual{$\llamada1$}[\red{!!}] &
    \omega + \omega^3 + \omega^4 + \omega^8 + \red{0}                                     \\
     & \igual{\red{!}}               & \omega + \omega^3 + \omega^4 + (-1) \cdot \omega^3 \\
     & \igual{\red{!}}               & \omega + \omega^4                                  \\
     & \igual{$\llamada2$}           & \omega + (-1) \cdot \conj{\omega}                  \\
     & \igual{\red{!}}               & \omega - \conj{\omega} = i \cdot 2\im(\omega)
  \end{array}
$$

En $\llamada1$ la sumatoria es una suma onda $1 - 1 + 1 - 1 + \cdots$ donde
se cancela todo.

En $\llamada2$ hago $\omega^4 = \frac{1}{\omega} \cdot \omega^5$ y un poco de acomodar.

Es así que si la expresión es igual a un número imaginario puro se concluye:
$$
  \cajaResultado{
    \re(\omega + \omega^{-7} + \conj{\omega}^6 + \omega^8 + \sumatoria{k = 5}{98}\omega^{5k}) =  0.
  }
$$

\textit{Nota que puede serte útil o no, chupala:}

A varias personas les \textit{tentó} poner en el valor de $\omega = -1$, o quizás $\omega = e^{i\frac{1}{5}\pi}$, porque $\omega^1$, bueh.
$\omega$ es un número cualquiera de los 10 valores que forman $G_{10}$, entonces no es cosa de que uno pueda elegir. Ojo con confundir:
$$
\omega^1 \en G_{10}
\Entonces{pongo}[$\blue{k = 1}$]
e^{i\frac{2\blue{1}}{10}\pi} = e^{i\frac{1}{5}\pi},
$$
onda, no. Nada que ver. Abrazo.

\textit{ Fin de nota que puede serte útil o no, chupala.}

\begin{aportes}
  \item \aporte{\dirRepo}{naD GarRaz \github}
\end{aportes}
