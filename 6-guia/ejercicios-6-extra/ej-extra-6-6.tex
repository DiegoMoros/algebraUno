\begin{enunciado}{\ejExtra}
  Hallar todos los $z \en \complejos$
  tales que
  $(2-z^3)^4 = z^{12}$.
\end{enunciado}

\textit{Enunciado corto, problema grande.}

\bigskip

\underline{Distintas formas de atacarlo}, y cual ves primero depende de vos, o dicho de otra manera de tus poderes, o
de cuanta maná tengas para ejecutar un típico \faIcon{magic} \textit{matemagium incantatum} \faIcon{magic}:

\begin{enumerate}[label=\faIcon{magic}$_{\arabic*}$)]
  \item\label{extra6-6:item1} $$
          (2-z^3)^4 = z^{12}
          \sii
          (2-z^3)^4 = (z^3)^4
          \Sii{\red{!!}}
          \llave{lcl}{
            (2-z^3)^4 = (z^3)^4
            &\sii&
            \llave{l}{
              2-z^3 = z^3 \quad \llamada1
            }\\
            (2-z^3)^4 \igual{\red{!}} (\red{\pm i} \cdot z^3)^4
            &\sii&
            \llave{l}{
              2-z^3 = \red{i} \cdot z^3  \quad \llamada2\\
              2-z^3 = \red{-i} \cdot z^3 \quad\llamada3
            }
          }
        $$
        Boom, 3 ecuaciones para resolver con tus amigos, un domingo de lluvia con mates y tortas fritas o en su defecto,
        \textit{solo, solísimo} en un parcial.

  \item\label{extra6-6:item2}
        $$
          \begin{array}{rcl}
            (2-z^3)^4 = z^{12}
                              & \sii           &
            1 = \parentesis{\frac{z^3}{2-z^3}}^4                                    \\
                              & \Sii{\red{!!}} &
            \frac{z^3}{2-z^3} \en G_4 \quad \text{con} \quad G_4 = \set{1,i,-1, -i} \\
                              & \sii           &
            \llave{rcl}{
            \frac{z^3}{2-z^3} & =              & 1                                  \\
            \frac{z^3}{2-z^3} & =              & -1   \quad$\red{\faIcon{skull}}$   \\
            \frac{z^3}{2-z^3} & =              & i                                  \\
            \frac{z^3}{2-z^3} & =              & -i                                 \\
            }
          \end{array}
        $$
        Boom, 3 igual que antes pero con menos magia aparecieron las ecuaciones de $\llamada1, \llamada2$ y $\llamada3$. La de \red{\faIcon{skull}} no tiene solución.

  \item\label{extra6-6:item3} Como esas en \ref{extra6-6:item1} y \ref{extra6-6:item2} no las vi hasta que hice lo que hay
        a continuación voy a desarrollar esta versión, porque como soy medio fanático de factorizar,
        bueh, me salen estas cosas primero.
        Te aviso que voy a hacer montón de \textit{diferencias de cuadrados\blue{!}}
        $$
          \begin{array}{rcl}
            (2-z^3)^4 = z^{12}
             & \sii           &
            (2-z^3)^4 - (z^3)^4 = 0 \\
             & \Sii{\blue{!}} &
            \magenta{\big(}
            (2-z^3)^2 - (z^3)^2 )
            \magenta{\big)^2}
            \cdot
            \parentesis{(2-z^3)^2 + (z^3)^2)}^2
            = 0                     \\
             & \Sii{\blue{!}} &
            \magenta{\big(}\parentesis{(2-z^3) - z^3}
            \cdot
            \parentesis{(2-z^3) + z^3}
            \magenta{\big)^2}
            \cdot
            \parentesis{(2-z^3)^2 + (z^3)^2}^2
            = 0                     \\
             & \sii           &
            \magenta{\big(}
            2 \cdot (1-z^3)
            \magenta{\big)^2}
            \cdot
            \parentesis{(2-z^3)^2 + (z^3)^2}^2
            = 0                     \\
             & \Sii{\red{!!}} &
            \magenta{\big(}
            1 - z^3
            \magenta{\big)^2}
            \cdot
            \parentesis{(2-z^3)^2 + (z^3)^2}^2
            = 0
          \end{array}
        $$
        \textit{Pocas cosas tan placenteras en la vida como un producto igualado a cero}. Debe ocurrir que:
        $$
          1 - z^3 = 0
          \qquad \text{o bien que} \qquad
          (2-z^3)^2 + (z^3)^2 = 0
        $$
        La primera no es otra cosa que $\llamada1$ con un poco de más de amor {\tiny\rosa{\faIcon{heart}}},
        y la segunda
        $$
          \begin{array}{rcl}
            (2-z^3)^2 + (z^3)^2 = 0
             & \sii           &
            (2 - z^3)^2  = - (z^3)^2    \\
             & \Sii{\red{!!}} &
            (2 - z^3)^2  = (iz^3)^2     \\
             & \sii           &
            (2 - z^3)^2  - (iz^3)^2 = 0 \\
             & \Sii{\blue{!}} &
            \magenta{\big(}
            (2 - z^3) - (iz^3)
            \magenta{\big)}
            \cdot
            \magenta{\big(}
            (2 - z^3) + (iz^3)
            \magenta{\big)}
            = 0                         \\
             & \sii           &
            \magenta{\big(}
            2 - z^3 \cdot (1 + i)
            \magenta{\big)}
            \cdot
            \magenta{\big(}
            2 + z^3 \cdot (-1 + i)
            \magenta{\big)}
            = 0                         \\
          \end{array}
        $$
        Sé lo que estás pensando y la respuesta es \underline{no}, nunca son suficientes las diferencias de cuadrados.

        Este último resultado es nuevamente un producto igualado a cero, poooooor lo tanto:
        $$
          \llave{rcl}{
            2 - z^3 \cdot (1 + i) = 0
            &\sii&
            z^3 = 1 - i  \quad \llamada2\\
            &\text{o bien}&\\
            2 + z^3 \cdot (-1 + i) = 0
            &\sii&
            z^3 = 1 + i  \quad \llamada3\\
          }
        $$
        Llegando así a no otra cosa que a $\llamada2$ y a $\llamada3$ con un poco más de amor nuevamente.
\end{enumerate}

Hayas llegado a las ecuaciones como hayas llegado, poco importa en este momento: Hay que resolver 3 ecuaciones complejas, las cuales
no quiero resolver en detalle, pero como soy un tipazo, ahí dejo las soluciones.
$$
  z^3 = 1
  \Sii{$G_3$}
  \cajaResultado{
    \llave{l}{
      z_1 = 1\\
      z_2 = e^{\frac{2}{3} \pi}\\
      z_3 = e^{\frac{4}{3} \pi}
    }
  }
$$
Ahora con $\llamada2$
$$
  z^3 = 1 - i = \sqrt{2} \cdot e^{i\frac{7}{4} \pi}
  \Sii{\red{!!!}}
  \cajaResultado{
    \llave{l}{
      z_4 = \sqrt[6]{2} \cdot e^{i \frac{7}{12} \pi}\\
      z_5 = \sqrt[6]{2} \cdot e^{i \frac{5}{4} \pi}\\
      z_6 = \sqrt[6]{2} \cdot e^{i \frac{23}{12} \pi}
    }
  }
$$
y por último $\llamada3$:
$$
  z^3 = 1 + i = \sqrt{2} \cdot e^{i\frac{1}{4} \pi}
  \Sii{\red{!!!}}
  \cajaResultado{
    \llave{l}{
      z_7 = \sqrt[6]{2} \cdot e^{i \frac{1}{12} \pi}\\
      z_8 = \sqrt[6]{2} \cdot e^{i \frac{3}{4} \pi}\\
      z_9 = \sqrt[6]{2} \cdot e^{i \frac{17}{12} \pi}
    }
  }
$$

Siendo objetivos, la solución del \ref{extra6-6:item2} es la más elegante lejos, la de \ref{extra6-6:item3} es un delirio,
pero lo importante es llegar al resultado correcto!
Como dijo el \blue{Capitán Planeta}: \textit{¡El poder es tuyo!}

\begin{aportes}
  \item \aporte{\dirRepo}{naD GarRaz \github}
\end{aportes}
