\begin{enunciado}{\ejExtra}
  Sea
  $$
    z = \big(4\sqrt{2} + 4\sqrt{2}i\big)^a \big(8+8\sqrt{3}i\big)^b
  $$
  se pide:
  \begin{enumerate}[label=\alph*)]
    \item Sabiendo que $\arg(z) = \arg(-i)$, hallar el resto de dividir a $3a + 4b$ por 24
    \item Determinar todas las parejas de números enteros $(a,b)$ tales que cumplen lo anterior, y además
          $$
            2^{10} < |z| < 2 ^{25}
          $$
  \end{enumerate}
  \underline{Sugerencia:} Use, sin demostrar, que $2^x < 2^y < 2^z \sii x < y < z.$
\end{enunciado}

\begin{enumerate}[label=\alph*)]
  \item
        Acomodo $z$:
        $$
          \begin{array}{rcl}
            z & =               & \big(4\sqrt{2} + 4\sqrt{2}i\big)^a \big(8+8\sqrt{3}i\big)^b             \\
              & =               & (4\sqrt{2} ( 1 + i ))^a \cdot (8(1 + \sqrt{3}))^b                       \\
              & =               & 8^a 16^b  e^{a \cdot \frac{\pi}{4} i} \cdot e^{b \cdot \frac{\pi}{3} i} \\
              & \igual{\red{!}} & 2^{3a + 4b} \cdot e^{i (\frac{a}{4}  +  \frac{b}{3} ) \pi}
          \end{array}
        $$
        Entonces si $\arg(z) = \arg(-i)$:
        $$
          (\frac{a}{4}  +  \frac{b}{3} ) \pi = \frac{3}{2} \pi + \blue{2k\pi}
          \Sii{\red{!}}
          \cajaResultado{3a  +  4b = 18 + 24k}
        $$
        Esté resultado es literalmente la expresión de un número dividido por 24 con su resto:
        $$
          r_{24}(3a+4b) = 18 \quad \text{con} \quad 0\leq 18 < 24
        $$

  \item
        Condición sobre el $|z|$:
        $$
          |z| = 2^{3a+4b}
          \quad\land\quad
          2^{10} < |z| < 2 ^{25}
          \sii
          10 < 3a+4b < 25 \llamada1
        $$
        Por otro lado tengo:
        $$
          \congruencia{3a + 4b}{18}{24}
          \Sii{def}
          3a + 4b = 24k + 18 \llamada2.
        $$
        Reemplazo $\llamada2$ en $\llamada1$:
        $$
          10 < 24k + 18 < 25
          \sii
          -8 < 24k < 7
          \sii
          k = 0.
        $$
        Por lo tanto tengo:
        $$
          3a + 4b = 18
        $$
        Para encontrar los pares resuelvo la diofántica {\tiny (a ojo en este caso, sino usar euclides)}:
        $$
          (a,b)_{particular} = (2,3) \ytext (a,b)_{homogeneo} = (-4,3)
        $$
        La solución general final con todos los pares queda:
        $$
          \begin{array}{rcl}
            (a,b)_{general} & = & k\cdot (-4,3) +  (2,3)                     \\
                            & = & (-4 + 2k, 3 + 3k) \paratodo k \en \enteros
          \end{array}
        $$
\end{enumerate}

\begin{aportes}
  \item \aporte{https://github.com/nad-garraz}{Nad Garraz \github}
\end{aportes}
