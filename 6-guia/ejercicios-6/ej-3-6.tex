\begin{enunciado}{\ejercicio}
  Hallar todos los número complejos $z$ tales que
  \begin{multicols}{4}
    \begin{enumerate}[label=\roman*)]
      \item $z^2 = -36$
      \item $z^2 = i$
      \item $z^2 = 7+24i$
      \item $z^2 + 15 -8i = 0$
    \end{enumerate}
  \end{multicols}
\end{enunciado}

\begin{enumerate}[label=\roman*)]
  \item\label{ej-3-item-i} A ojo se puede ver el resultado:
        $$
          z_1 = 6i \ytext z_2 = -6i
        $$
        Si no se ve a simple vista:
        \begin{itemize}
          \item Se puede plantear la \textit{ecuación en forma exponencial}, para deducir módulo y argumento.
          \item Cuando la potencia de $z$ es 2, como en este caso, se puede atacar separando para parte real y la imaginaria e igualando.
        \end{itemize}
        Ahora usamos la segunda de esas técnincas:
        $$
          z^2 = (a+bi)^2 = a^2-b^2 + 2abi.
        $$
        Para que 2 números complejos sean iguales, debe ocurrir que:
        \begin{itemize}[label=\tiny\faIcon{calculator}]
          \item Sus partes reales tiene que ser iguales y sus partes imaginarias también.
        \end{itemize}
        $$
          z^2 = -36
        $$
        $$
          a^2-b^2 + 2abi = - 36
          \to
          \llave{cl}{
            a^2 - b^2 = -36 & \quad \text{parte real}       \\
            2ab = 0         & \quad \text{parte imaginaria}
          }
        $$
        De ese sistema queda que:
        $$
          a = 0 \otext b = 0,
        $$
        y dado que $a\ytext b \en \reales$, para que se cumpla la otra ecuación debe suceder que:
        $$
          a = 0 \ytext b = \pm6
        $$
        Por lo tanto se recupera que :

        $$
          \fcolorbox{orange}{white}{
            $z_1 = 0 + 6i = 6i
              \ytext
              z_2 = 0 - 6i = - 6i
            $
          }
        $$

  \item Este no me parece taan obvio. Resuelvo ecuación en forma exponencial:
        $$
          \begin{array}{rcl}
            z         & = & r e^{i \theta} \to \blue{z^2} = r^2 (e^{i \theta})^2 = \blue{r^2 e^{i 2\theta}} \\
            \green{i} & = & \green{e^{i \frac{\pi}{2}}}
          \end{array}
        $$
        La idea es separar la ecuación compleja en \red{2} ecuaciones con números reales. Atento a que $r \en \reales_{>0}$
        y que el argumento $\theta$ es $2\pi$ periódico\red{!}

        Ahora la ecuación queda como:
        $$
          \blue{r^2 e^{i 2\theta}} = \green{e^{i \frac{\pi}{2}}}
          \flecha{módulos por un lado}[argumentos por otro]
          \llave{l}{
            r^2 = 1 \sii r = 1 \\
            2 \theta = \frac{\pi}{2} + 2\magenta{k}\pi \sii \theta = \frac{\pi}{4} + \magenta{k}\pi
          }
        $$
        tengo que $\magenta{k} \en \set{0,1}$ de forma tal que el argumento $\theta \en [0, 2\pi)$.

        Los valores que nos pedían:
        $$
          \fcolorbox{orange}{white}{
            $z_{\scriptscriptstyle \magenta{k}=0} = e^{\frac{\pi}{4}} = \frac{1}{\sqrt{2}} + \frac{1}{\sqrt{2}}i
              \ytext
              z_{\scriptscriptstyle \magenta{k}=1} = e^{\frac{5}{4}\pi} = -\frac{1}{\sqrt{2}} - \frac{1}{\sqrt{2}}i$
          }
        $$

        \textit{Le comentario:}\par
        Sale más fácil por el método del item \ref{ej-3-item-i}? Seguramente, pero pintó hacerlo con exponenciales.

  \item Ataco igual que antes:
        $$
          \begin{array}{rcl}
            z               & =               & r e^{i \theta} \to \blue{z^2} = r^2 (e^{i \theta})^2 = \blue{r^2 e^{i 2\theta}} \\
            \green{7 + 24i} & \igual{\red{!}} & \green{25 e^{i \arctan(\frac{24}{7})} }
          \end{array}
        $$

        Horrible esos valores, probablemente no salga por acá. Pruebo con el método del item \ref{ej-3-item-i}:
        $$
          z^2 = 7 + 24i
        $$

        $$
          a^2-b^2 + 2abi = 7 + 24i
          \to
          \llave{cl}{
            a^2 - b^2 = 7 & \quad \text{parte real}       \\
            2ab = 24      & \quad \text{parte imaginaria}
          }
        $$
        De ese sistema queda que:
        $$
          a\cdot b = 12,
        $$
        meto en la otra ecuación $a = \frac{12}{b}$:
        $$
          \frac{144}{b^2} - b^2 = 7 \Sii{\red{!!}}[$b\en \reales$] b = \pm 3
        $$
        En \red{!!}, bicuadrática.\par
        Con ese resultado los valores quedarían para el sistema:
        $$
          \fcolorbox{orange}{white}{
            $        z_1 = -4 -3i \ytext z_2 = 4 + 3i$
          }
        $$

  \item
        Acomodo para que quede para resolver como el anterior:
        $$
          z^2 + 15 -8i = 0
          \sii
          z^2 = -15 + 8i
        $$

        $$
          a^2-b^2 + 2abi = -15 + 8i
          \to
          \llave{cl}{
            a^2 - b^2 = -15 & \quad \text{parte real}       \\
            2a\cdot b = 8   & \quad \text{parte imaginaria}
          }
        $$
        De ese sistema queda que:
        $$
          a\cdot b = 4,
        $$
        meto en la otra ecuación $a = \frac{4}{b}$:
        $$
          \frac{16}{b^2} - b^2 = -15 \Sii{\red{!!}}[$b\en \reales$] b = \pm 4
        $$

        Con ese resultado los valores quedarían para el sistema:
        $$
          \fcolorbox{orange}{white}{
            $        z_1 = 1 + 4i \ytext z_2 = -1 -4i$
          }
        $$

        % Contribuciones
        \begin{aportes}
          %% iconos : \github, \instagram, \tiktok, \linkedin
          %\aporte{url}{nombre icono}
          \item \aporte{\dirRepo}{Nad Garraz \github}
        \end{aportes}
\end{enumerate}
