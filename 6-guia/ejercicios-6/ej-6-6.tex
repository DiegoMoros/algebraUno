\begin{enunciado}{\ejercicio}
  \begin{enumerate}[label=\alph*)]
    \item Determinar la formar binomial de
          $ \left(\frac{1 + \sqrt{3} i }{1 - i} \right)^{17}$.

    \item Determinar la forma binomial de
          $(-1 + \sqrt{3}i)^n$ para cada $n \en \naturales$.
  \end{enumerate}
\end{enunciado}

\begin{enumerate}[label=\alph*)]
  \item  Multiplico y divido por el conjugado complejo para sacar la parte imaginaria del denominador:
        $$
          z \igual{$\llamada1$} \left(\frac{1 + \sqrt{3} i }{1 - i} \right)^{17} \igual{\red{!}}
          \left(\frac{1 + \sqrt{3} i }{(1 - i)}\cdot \frac{1+i}{1+i} \right)^{17} =
          \left( \frac{(1 + \sqrt{3} i) \cdot (1+i) }{2} \right)^{17} =
          \left( \frac{1 - \sqrt{3}}{2} + \frac{1+\sqrt{3}}{2}i \right)^{17}
        $$
        Ahora paso eso a notación exponencial y acomodo usando propiedades de exponentes:
        $$
          \begin{array}{c}
            \llave{c}{
            1 + \sqrt{3}i = 2 \cdot e^{\frac{\pi}{3}i}     \\
              1 + i = \sqrt{2} \cdot e^{\frac{1}{4}\pi i}
            }                                              \\
            \left( \frac{(1 + \sqrt{3} i) \cdot (1+i) }{2} \right)^{17} =
            \left( \frac{{2 \cdot e^{\frac{\pi}{3}i} } \cdot \sqrt{2} e^{\frac{\pi}{4}i}}{2} \right)^{17} =
            2^{\frac{17}{2}} \cdot  e^{\frac{119}{12}\pi i}  \igual{\red{!}}
            2^{\frac{17}{2}} \cdot  e^{-\frac{1}{12}\pi i} \\
            \llamada1 z =
            2^{\frac{17}{2}}  \cos(\frac{1}{12} \pi) - i 2^{\frac{17}{2}} \sin(\frac{1}{12} \pi)
            \igual{\red!}
            2^{\frac{17}{2}}  \cos(\frac{23}{12} \pi) + i 2^{\frac{17}{2}} \sin(\frac{23}{12} \pi)
          \end{array}
        $$
        Un espanto. Pero bueh, $\frac{1}{12}\pi = 15^\circ \ytext \frac{23}{12} \pi = 345^\circ$.

        \medskip

        \textit{Nota:} Después de hacer el item \ref{ej6:item-b}, \textit{creo} que la idea del ejercicio es hacerlo así:

        Como $ (\frac{1 + \sqrt{3} i }{1 - i} )^{17}$ está compuesto por 2 elementos de nuestro conjunto de números complejos favoritos, lease:

        $$
          \llave{rcl}{
            1 + \sqrt{3} i & = & 2 \cdot e^{i \frac{\pi}{3}}           \\
            1 - i          & = & \sqrt{2} \cdot e^{i \frac{7}{4}\pi}
          }
        $$
        Y esto elevado a la 17 tiene dentro de todo un aspecto, no taaan vomitivo:
        $$
          \llave{rcl}{
            (1 + \sqrt{3} i)^{17} & = & 2^{17} \cdot e^{i \frac{17}{3}\pi} \igual{\red{!}} 2^{17} \cdot e^{i \frac{5}{3}\pi} = 2^{\magenta{16}} \cdot (1 - i\sqrt{3})     \\
            (1 - i)^{17}          & = & (\sqrt{2})^{17} \cdot e^{i \frac{119}{4}\pi} \igual{\red{!}} \cdot (\sqrt{2})^{17} \cdot e^{i \frac{7}{4}\pi}= (\sqrt{2})^{17} \cdot (1 - i)
          }
        $$

        Juntando lo que fue quedando:
        $$
          (\frac{1 + \sqrt{3} i }{1 - i} )^{17} =
          \frac{2^{\magenta{16}} \cdot (1 - i\sqrt{3})}{(\sqrt{2})^{17} \cdot (1 - i)}
          \igual{\red{!!}}
          (\sqrt{2})^{13} \cdot (1 - i \sqrt{3}) \cdot (1 + i) =
          (\sqrt{2})^{13} \cdot \left( (1  + \sqrt{3}) + (1 - \sqrt{3}) i \right)
        $$

        En el \red{!!}, multipliqué y dividí por el conjugado complejo, y simplifiqué el exponente.

        La forma binómica quedaría:
        $$
          \cajaResultado{
            (\frac{1 + \sqrt{3} i }{1 - i} )^{17} =
            (\sqrt{2})^{13} \cdot  (1  + \sqrt{3}) + (\sqrt{2})^{13} \cdot (1 - \sqrt{3}) i
          }
        $$

  \item\label{ej6:item-b} $$
          (-1 + i\sqrt{3})^{\magenta{n}} =
          2^{\magenta{n}} \cdot  e^{i \frac{2\magenta{n}}{3}\pi} =
          2^{\magenta{n}} \cdot \left(\cos(\frac{2\magenta{n}}{3} \pi) + i \sin(\frac{2\magenta{n}}{3} \pi)\right)
        $$
        El módulo va a crecer con $\magenta{n}$, pero la parte exponencial es periódica por inspección:
        $$
          \cajaResultado{
            (-1 + i\sqrt{3})^n
            \igual{\red{!}}
            \llave{lcl}{
              2^n                             & \sii & \congruencia{n}{0}{3} \\
              2^{n-1} \cdot (-1 + i \sqrt{3}) & \sii & \congruencia{n}{1}{3} \\
              2^{n-1} \cdot (-1 - i \sqrt{3}) & \sii & \congruencia{n}{2}{3}
            }
          }
        $$

\end{enumerate}

\begin{aportes}
  \item \aporte{\dirRepo}{naD GarRaz \github}
\end{aportes}
