\begin{enunciado}{\ejercicio}
  \begin{enumerate}[label=\alph*)]
    \item Determinar la formar binomial de
          $ \left(\frac{1 + \sqrt{3} i }{1 - i} \right)^{17}$.

    \item Determinar la forma binomial de
          $(-1 + \sqrt{3}i)^n$ para cada $n \en \naturales$.
  \end{enumerate}
\end{enunciado}

\begin{enumerate}[label=\alph*)]
  \item  Multiplico y divido por el conjugado complejo para sacar la parte imaginaria del denominador:
        $$
          z \igual{$\llamada1$} \left(\frac{1 + \sqrt{3} i }{1 - i} \right)^{17} \igual{\red{!}}
          \left(\frac{1 + \sqrt{3} i }{(1 - i)}\cdot \frac{1+i}{1+i} \right)^{17} =
          \left( \frac{(1 + \sqrt{3} i) \cdot (1+i) }{2} \right)^{17} =             \\
        $$
        Ahora paso eso a notación exponencial y acomodo usando propiedades de exponentes:
        $$
          \begin{array}{c}
            \llave{c}{
            1 + \sqrt{3}i = 2 \cdot e^{\frac{\pi}{3}i}     \\
              1 + i = \sqrt{2} \cdot e^{\frac{1}{4}\pi i}
            }                                              \\
            \left( \frac{(1 + \sqrt{3} i) \cdot (1+i) }{2} \right)^{17} =
            \left( \frac{{2 \cdot e^{\frac{\pi}{3}i} } \cdot \sqrt{2} e^{\frac{\pi}{4}i}}{2} \right)^{17} =
            2^{\frac{17}{2}} \cdot  e^{\frac{119}{12}\pi i}  \igual{\red{!}}
            2^{\frac{17}{2}} \cdot  e^{-\frac{1}{12}\pi i} \\
            \llamada1 z =  2^{\frac{17}{2}}  \cos(\frac{1}{12} \pi) - i 2^{\frac{17}{2}} \sin(\frac{1}{12} \pi)
          \end{array}
        $$
        Un espanto. Pero bueh, $\frac{1}{12}\pi = 15^\circ$

  \item \hacer
\end{enumerate}
