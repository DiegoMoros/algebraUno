\begin{enunciado}{\ejercicio}
  Hallar todos los $n \en \naturales$ tales que
  \begin{enumerate}[label=\roman*)]
    \item $(\sqrt3 -i)^n = 2^{n-1}(-1 + \sqrt3 i)$
    \item $(-\sqrt3 + i)^n \cdot \parentesis{\frac{1}{2} + \frac{\sqrt3}{2}i}$ es un número real negativo.
    \item $\text{arg}((-1+i)^{2n}) = \frac{\pi}{2}$ y $\text{arg}((1-\sqrt3 i)^{n-1}) = \frac{2}{3}\pi$
  \end{enumerate}
\end{enunciado}

\begin{enumerate}[label=\roman*)]

  \item Para resolver las ecuaciones en números complejos con exponentes, en general, es más
        fácil resolver en notación exponencial.
        El miembro izquierdo queda:
        $$
          (\sqrt{3} - i)^n =
          \big( 2 \cdot e^{i \frac{11}{6}\pi} \big)^n =
          2^n \cdot e^{i \frac{11}{6}\pi n}
        $$
        El miembro derecho queda:
        $$
          2^{n-1}(-1 + \sqrt{3} i) =
          2^{n-1} \cdot (2 \cdot e^{\frac{2}{3}})=
          2^n\cdot e^{i\frac{2}{3}\pi}
        $$
        Ahora la igualdad de los números se dará cuando sus módulos y argumentos sean iguales:
        $$
          2^n \cdot e^{i \frac{11}{6}\pi n} = 2^n\cdot e^{i\frac{2}{3}\pi}
          \sii
          \llave{l}{
            2^n = 2^n  \Tilde \\
            \frac{11}{6}\pi n = \frac{2}{3}\pi + \magenta{2k \pi}
            \sii
            11 n = 4 + 12k \llamada1
          }
        $$
        En $\llamada1$ quedó una ecuación para despejar $n$ que es un número entero:
        $$
          \llamada1
          11 n = 4 + 12k
          \Sii{def}
          \congruencia{11 n}{4}{12}
          \sii
          \congruencia{-n}{4}{12}
          \sii
          \congruencia{n}{-4}{12}
          \sii
          \congruencia{n}{8}{12}
        $$
        Finalmente los valores de $n$ buscados para que la ecuación se cumpla son:
        $$
          \congruencia{n}{8}{12}
        $$

  \item
        Un número real $z$ negativo tiene un arg$(z) = \pi $. Ataco el ejercicio parecido al
        anterior en la parte de los exponentes, donde está el argumento:\par

        $$
          \begin{array}{rcl}
            (-\sqrt3 + i)^n                   & = & 2^n \cdot e^{i\frac{5}{6}\pi n} \\
            (\frac{1}{2} + \frac{\sqrt3}{2}i) & = & e^{\frac{\pi}{3}i}
          \end{array}
        $$
        El enunciado queda como:
        $$
          (-\sqrt3 + i)^n \cdot \parentesis{\frac{1}{2} + \frac{\sqrt3}{2}i} = 2^n \cdot e^{i (\frac{5}{6}n + \frac{1}{3})\pi }
        $$
        Ahora, \textit{sin olvidar la periodicidad}, tengo que pedir que el argumento de esa expresión sea $\pi$:
        $$
          (\frac{5}{6}n + \frac{1}{3})\pi = \pi + \magenta{2k\pi}
          \sii
          \frac{5}{6}n + \frac{1}{3} = 1 + 2k
          \sii
          5n = 4 + 12k\llamada1
        $$
        En $\llamada1$ quedo una ecuación para resolver para $n \en \enteros$:
        $$
          \llamada1
          5n = 4 + 12k
          \Sii{def}
          \congruencia{5n}{4}{12}
          \sii
          \congruencia{n}{8}{12}
        $$
        Finalmente los valores de $n$ buscados para que la expresión sea un número negativo:
        $$
          \congruencia{n}{8}{12}
        $$

  \item Arranco pasando las expresiones del enunciado a notación exponencial:
        $$
          \begin{array}{rcl}
            (-1 + i)^{2n}        & = & 2^n \cdot e^{i\frac{3}{2}\pi n} \llamada1          \\
            (1 - \sqrt3 i)^{n-1} & = & 2^{n-1}\cdot e^{i \frac{5}{3}\pi (n-1) } \llamada2
          \end{array}
        $$
        De $\llamada1$ igualando a $\frac{\pi}{2}$, sin olvidar la \textit{periodicidad} del argumento:
        $$
          \frac{3}{2} \pi n = \frac{\pi}{2} + \magenta{2k\pi}
          \sii
          3 n = 1 + 4k
          \Sii{def}
          \congruencia{3n}{1}{4}
          \sii
          \congruencia{n}{3}{4} \llamada3
        $$
        De $\llamada2$ igualando a $\frac{2}{3}\pi$, nuevamente sin olvidar la \textit{periodicidad} del argumento:
        $$
          \frac{5}{3} \pi (n-1) = \frac{2}{3}\pi + \magenta{2k\pi}
          \sii
          5 n - 5 = 2 + 6k
          \sii
          5 n = 7 + 6k
          \Sii{def}
          \congruencia{5n}{7}{6}
          \Sii{\red{!}}
          \congruencia{n}{5}{6} \llamada4
        $$
        Podemos observar que con los resultados de $\llamada3$ y $\llamada4$ esto se convirtió en un ejercicio del \href{\chinito}{TCHR}:
        $$
          \llave{l}{
            \congruencia{n}{3}{4} \\
            \congruencia{n}{5}{6}
          }
          \taa{\red{!}}{}\leftrightsquigarrow
          \llave{l}{
            \congruencia{n}{3}{4} \\
            \congruencia{n}{2}{3}
          }
        $$
        Resolviendo ese sistema, los valores de $n$ buscados:
        $$
          \congruencia{n}{11}{12}
        $$
\end{enumerate}

% Contribuciones
\begin{aportes}
  \item \aporte{https://github.com/nad-garraz}{Nad Garraz \github}
\end{aportes}
