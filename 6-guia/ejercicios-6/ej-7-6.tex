\begin{enunciado}{\ejercicio}
  Hallar todos los $n \en \naturales$ tales que
  \begin{enumerate}[label=\alph*)]
    \item $(\sqrt3 -i)^n = 2^{n-1}(-1 + \sqrt3 i)$
    \item $(-\sqrt3 + i)^n \cdot \parentesis{\frac{1}{2} + \frac{\sqrt3}{2}i}$ es un número real negativo.
    \item $\text{arg}((-1+i)^{2n}) = \frac{\pi}{2}$ y $\text{arg}((1-\sqrt3 i)^{n-1}) = \frac{2}{3}\pi$
  \end{enumerate}
\end{enunciado}

\begin{enumerate}[label=\roman*)]
  \begin{minipage}{0.7\textwidth}
    \item $(\sqrt3 -i)^n = 2^{n-1}(-1 + \sqrt3 i)$ \\
    \separadorCorto
    $(\sqrt3 -i)^n = 2^n e^{i\frac{11}{12}\pi n} = 2^{n+1}\cdot 2e^{i \frac{2}{3} \pi}\\
      \to
      \llave{l}{
        2^n = 2^n \\
        \frac{11}{12}\pi n = \frac{2}{3}\pi + 2k \pi \to 11n = 8+8k \flecha{8(k+1)} \boxed{\congruencia{n}{0}{8}}
      }$
  \end{minipage}

  \item
        Un número real negativo tendrá un arg$(z) = \pi$\\
        $\ub{(-\sqrt3 + i)^n}{2^ne^{i\frac{5}{6}\pi n}} \cdot \ub{\parentesis{\frac{1}{2} + \frac{\sqrt3}{2}i}}{e^{\frac{\pi}{3}i}} =
          2^ne^{i(\frac{5}{6}n + \frac{1}{3}) \pi} \to \theta = (\frac{5}{6}n + \frac{1}{3}) \pi $\\
        $\flecha{$\theta = \pi + 2k\pi$}
          \cancel\pi \frac{5}{6}n + \frac{\cancel\pi}{3} = \cancel\pi + 2k\cancel\pi
          \flecha{acomodo}[congruencia]
          \congruencia{5n}{4}{12}
          \flecha{multiplico}[por 5]
          \boxed{\congruencia{n}{8}{12}} $

  \item $\text{arg}((-1+i)^{2n}) = \frac{\pi}{2}$ y $\text{arg}((1-\sqrt3 i)^{n-1}) = \frac{2}{3}\pi$

\end{enumerate}
