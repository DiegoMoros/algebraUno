\ejExtra
% Macro local
\def\F{\mathcal F}


Sea
$
  \F = \set{h : \set{1,2,3,4} \to \set{1,2,\dots, 50} \ /\  h \text{ es inyectiva}}
$.\\
Definimos en $\F$ la relación $\relacion$ como
$$
  f \relacion g \quad \text{si y sólo si} \quad \# \parentesis{ \im(f) \diferencia \im(g) } = 0 \otext 4.
$$
\begin{enumerate}[label=\alph*)]
  \item Analizar si $\relacion$ es una relación reflexiva, simétrica, antisimétrica y/o transitiva.

  \item Sea
        $f \en \F$
        definida como $f(x) = x$ para $1 \leq x \leq 4$. Calcular cuántas funciones $g \en \F$
        satisfacen $f \relacion g$
\end{enumerate}

\separadorCorto

Observar que $f \en \F$ es una función que tiene un dominio con solo 4 elementos, es decir
$$
  \# \dom(f) = 4 \paratodo f \en \F,
$$
y dado que $f$ es inyectiva, todos los elementos de la imagen deben ser distintos, por lo tanto
$$
  \# \im(f) = 4 \paratodo f \en \F
$$

\begin{enumerate}[label=\alph*)]
  \item
        \textit{Reflexiva: } Quiero ver que si $f \relacion f$.\\
        Esto debe ser cierto, ya que
        $A = \set{\im(f) \diferencia \im(f)} =
          \vacio \ \ytext\  \# \vacio \igual! 0 \ \paratodo f \en \F.
        $
        $\relacion$ es reflexiva \Tilde\\

        \textit{Simétrica: }Quiero ver que si $f \relacion g \entonces g \relacion f$.\\
        Si tengo que $f \relacion g$, sé algo sobre sus conjuntos $\im$ ya que,
        $$
          \llave{ccc}{
            \#\set{\im(f) \diferencia \im(g)} = 0 & \sisolosi & \im(f) \igual{$\llamada1$} \im(g) \\\
            &\otext&\\
            \#\set{\im(f) \diferencia \im(g)} = 4
            &\sisolosi&
            \im(f) \stackrel{\llamada2}\inter \im(g) = \vacio
          }
        $$
        Entonces los conjuntos $\im(f)$ y $\im(g)$ están relacionados por un "$=$" y un "$\inter$",
        dos operadores simétricos por lo tanto $\relacion$ es simétrica. \Tilde\\

        \textit{Antisimétrica: } Quiero ver que si
        $f \relacion g \entonces g \norelacion f$,
        o también a veces está bueno pensarla la antisimetría como si
        $f \relacion g \ytext g \relacion f \entonces f = g$. Bajo la sospecha de que la función no es antisimétrica
        la segunda forma de pensarlo me ayuda a encontrar un \textit{contra}ejemplo.\\
        $$
          f
          \to
          \llave{l}{
            f(1) = 1\\
            f(2) = 2\\
            f(3) = 3\\
            f(4) = 4
          }\quad\ytext\quad
          g
          \to
          \llave{l}{
            g(1) = 4\\
            g(2) = 3\\
            g(3) = 2\\
            g(4) = 1
          }
        $$
        $\llave{l}{
            f \relacion g, \text{ sus imágenes cumplen } \llamada1\\
            g \relacion f, \text{ sus imágenes cumplen } \llamada1
          }
        $, pero por como están definidas las funciones $f \distinto g$. $\relacion$ no es antisimétrica. $\skull$\\


        \textit{Transitiva: } Quiero ver que si
        $f \relacion g
          \ytext
          g \relacion h
          \entonces
          f \relacion h$.\\
        Acá podemos encontrar un \textit{contra}ejemplo para mostrar que no es transitiva, saco de la galera 3 funciones,
        $f, \,g \ytext h \en \F$\\
        $$
          f
          \to
          \llave{l}{
            f(1) = 1\\
            f(2) = 2\\
            f(3) = 3\\
            f(4) = 4
          },\quad
          g
          \to
          \llave{l}{
            g(1) = 5\\
            g(2) = 6\\
            g(3) = 7\\
            g(4) = 8
          } \quad \ytext \quad
          h
          \to
          \llave{l}{
            h(1) = 1\\
            h(2) = 2\\
            h(3) = 9\\
            h(4) = 10
          }
        $$

        $\llave{l}{
            f \relacion g, \text{ sus imágenes cumplen } \llamada2\\
            g \relacion h, \text{ sus imágenes cumplen } \llamada2
          }
        $, pero
        $f \norelacion h$ dado que:
        $$
          \set{\im(f) \diferencia \im(g)} = \set{3,4} \entonces \# \set{\im(f) \diferencia \im(g)} = 2 \distinto 0 \otext 4.
        $$
        $\relacion$ no es transitiva. $\skull$

  \item
        Para que $f$ y $g$ se relacionen se debe cumplir con $\llamada1$ o con $\llamada2$. En otras palabras necesito encontrar funciones
        $g \en \F$ cuya imagen $\im(g) = \set{1,2,3,4}$ o su codominio sea $\ub{\cod = \set{5,6,\dots,49,50}}{\#\cod = 46}$.\\

        \textit{Contar cuando $\im(g) = \set{1,2,3,4}$:}\\
        Hago la \textit{inyección} de los 4 valores que puede tomar la función inyectiva $g$.\\
        $$
          \llave{r c c c c }{
            g \to   &g(1) & g(2) & g(3) & g(4) \\
            &\downarrow &\downarrow &\downarrow & \downarrow\\
            \text{opciones}\to & \#4 & \#3 & \#2 & \#1
          }
        $$
        Hay $4!$ permutaciones \Tilde\\

        \textit{Contar cuando codominio sea $\cod = \set{5,6,\dots,,49,50}$}\\
        Hago la \textit{inyección} de los 46 valores que puede tomar la función inyectiva $g$.\\
        $$
          \llave{r c c c c }{
            g \to   &g(1) & g(2) & g(3) & g(4) \\
            &\downarrow &\downarrow &\downarrow & \downarrow\\
            \text{opciones}\to & \#46 & \#45 & \#44 & \#43
          }
        $$
        Hay $\frac{46!}{42!}$ permutaciones \Tilde\\

        Se concluye que hay un total de $\frac{46!}{42!}  + 4!$ funciones $g \en \F / f \relacion g$ \Tilde
\end{enumerate}
