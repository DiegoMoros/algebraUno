\begin{enunciado}{\ejExtra}
Probar la propiedad distributiva: $X \inter (Y \union Z) = (X \inter Y) \union (X \inter Z)$
\end{enunciado}

Tengo que hacer una doble inclusión:
$\to \begin{cases}
		1) & X \inter (Y \union Z) \subseteq (X \inter Y) \union (X \inter Z) \\
		2) & (X \inter Y) \union (X \inter Z) \subseteq X \inter (Y \union Z)
	\end{cases}$

\begin{enumerate}[label=\arabic*)]
	\item
	      $x \en X \inter (Y \union Z)$ quiere decir que $x \en X$ y
	      $\llaves{c}{
			      x \en Y \\
			      \o      \\
			      x \en Z
		      } $.
	      Por lo tanto $\to
		      \llaves{c}{
			      x \en X \inter Y\\
			      \o \\
			      x \en X \inter Z
		      }$, lo que equivale a $x \en (X \inter Y) \union (X \inter Z)$ \Tilde.\\

	\item
	      Ahora hay que probar la vuelta. Uso razonamiento análogo.
	      $x \en (X \inter Y) \union (X \inter Z)$, por lo que $x \en X$ y
	      $
		      \llaves{c}{
			      x \en X \inter Y \blue{\flecha{dado que}[$Y \subseteq Y \union Z$] x \en X \inter (Y \union Z) }\\
			      \o               \\
			      x \en X \inter Z \blue{\flecha{dado que}[$Z \subseteq Y \union Z$] x \en X \inter (Y \union Z)}
		      }$.
	      Lo que quiere decir \blue{que $x \en X \inter (Y \union Z)$ \Tilde}\\
	      \red{¿Estoy suponiendo cosas que debería demostrar, me estoy salteando pasos?}\\
	      \blue{Para que la solución quede más creíble usé que $S \subseteq S \union T$ fue el dealbreaker. }
\end{enumerate}
