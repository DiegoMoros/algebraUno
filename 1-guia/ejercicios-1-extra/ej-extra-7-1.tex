\begin{enunciado}{\ejExtra}
  Sean $X = \set{n \en \naturales : n \leq 200}$ e $Y = \set{n\en \naturales : n \leq 100}$.\par
  En $\partes(X)$ se define la relación $\relacion$ de la forma:
  $$
    A \relacion B \sisolosi B - A \subseteq Y.
  $$
  \begin{enumerate}[label=\alph*)]
    \item Determinar si $\relacion$ es una relación reflexiva, simétrica, antisimétrica y/o transitiva.
    \item Sea $B = \set{n \en X : n \text{ es par} }$. ¿Cuántos conjuntos $A \en \partes(X)$ satisfacen simultáneamente
          $A \relacion B$ y $\#(A\inter B) = 80$?
  \end{enumerate}
\end{enunciado}

\begin{enumerate}[label=\alph*)]
  \item Para ver esto de qué cosa son \hyperlink{teoria-1:prop-relaciones}{las propiedades de reflexión y eso acá está la teoría}.

        \textit{¿Es $\relacion$ reflexiva?:}
        $$
          A \relacion A \sisolosi A - A = \vacio \subseteq Y
        $$
        Lo cual es cierto, dado que el conjunto vacío, $\vacio$, está  en todo conjunto.

        \medskip

        \textit{¿Es $\relacion$ simétrica?:}\par
        Por hipótesis:
        $$
          A \relacion B \sisolosi B - A \subseteq Y \Tilde
        $$
        Quiero ver que pasa con $B\relacion A$:
        $$
          B \relacion A \Sii{\red{??}} A - B \subseteq Y
        $$
        Propongo que $A = \set{101,200}$ y que $B = \set{1,200}$. Con estos conjuntos se tiene:
        $$
          B - A = \set{1} \subseteq Y
        $$
        peeeeero,
        $$
          A - B = \set{101} \nsubseteq Y
        $$
        Por lo tanto la relación \ul{no} es simétrica.

        \medskip

        \textit{¿Es $\relacion$ antisimétrica?:}\par
        Por hipótesis:
        $$
          A \relacion B \sisolosi B - A \subseteq Y \Tilde
        $$
        De ser antisimétrica debería ocurrir que
        $$
          A \relacion B \entonces B \norelacion A \text{  para  } B \distinto A.
        $$
        Veamos por ejemplo qué pasa con $A = \set{1}$ y $B = \set{2}$, donde $A \distinto B$:
        $$
          B - A = \set{2} \subseteq Y \entonces A \relacion B
        $$
        y también,
        $$
          A - B = \set{1} \subseteq Y \entonces B \relacion A
        $$
        Por lo tanto la relación \ul{no} es antisimétrica.

        \medskip

        \textit{¿Es $\relacion$ transitiva?:}\par
        Por hipótesis:
        $$
          \begin{array}{c}
            \llamada1 A \relacion B \sisolosi B - A \subseteq Y \Tilde \\
            \llamada2 B \relacion C \sisolosi C - B \subseteq Y \Tilde
          \end{array}
        $$
        En diagramas de Venn:
        $$
          \llamada1
          \begin{venndiagram3sets}[shade=blue!30!white, showframe = false,hgap=0, vgap=0, overlap = 1.1cm]
            \fillBCapCNotA
            \fillOnlyB
          \end{venndiagram3sets}
          \ytext
          \llamada2
          \begin{venndiagram3sets}[shade=orange!30!white, showframe = false,hgap=0, vgap=0, overlap = 1.1cm]
            \fillCCapANotB
            \fillOnlyC
          \end{venndiagram3sets}
        $$
        Quiero ver si $A \relacion C$ es decir si $C-A \subseteq Y$:
        $$
          \llamada3
          \begin{venndiagram3sets}[shade=yellow!30!white, showframe = false,hgap=0, vgap=0, overlap = 1.1cm]
            \fillCCapBNotA
            \fillOnlyC
          \end{venndiagram3sets}
        $$
        Esto está lindo porque $\llamada1$ y $\llamada2$ están en $Y$, lo cual equivale a decir que su unión también está en $Y$:
        $$
          \begin{venndiagram3sets}[shade=blue!30!white, showframe = false,hgap=0, vgap=0, overlap = 1.1cm]
            \fillBCapCNotA
            \fillOnlyB
          \end{venndiagram3sets}
          \quad
          \union
          \quad
          \begin{venndiagram3sets}[shade=orange!30!white, showframe = false,hgap=0, vgap=0, overlap = 1.1cm]
            \fillCCapANotB
            \fillOnlyC
          \end{venndiagram3sets}
          \quad
          =
          \quad
          \begin{venndiagram3sets}[shade=magenta!30!white, showframe = false,hgap=0, vgap=0, overlap = 1.1cm]
            \fillBNotA
            \fillCNotB
          \end{venndiagram3sets}
        $$
        En los diagramas se puede ver que $\llamada3$ es un conjunto que está en la unión de $\llamada1$ y $\llamada2$,
        y como, por hipótesis, esa unión está en $Y$, la relación es \textit{transitiva}.

  \item
        $$
          \#B = \#\set{2,4,6,\dots,198,200} = 100
        $$
        \faIcon[regular]{flushed} no se me ocurre, lo voy a pensar...  bye
\end{enumerate}

\begin{aportes}
  \item \aporte{\dirRepo}{Nad Garraz \github}
  \item \aporte{https://github.com/fionamclou}{Fiona M L \github}
\end{aportes}
