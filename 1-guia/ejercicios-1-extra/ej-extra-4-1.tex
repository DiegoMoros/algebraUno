\begin{enunciado}{\ejExtra}
  \textbf{(recuperatorio 1er C. 24)}\par
  Se define en $\enteros$ la relación $\relacion$ dada por
  $$
    n \relacion m \sisolosi 10 \divideA n^2 + 4m^2 + m - 6n.
  $$
  \begin{enumerate}[label=\alph*)]
    \item Probar que
          $n \relacion m \sisolosi 5 \divideA n^2 -m^2 + m -n \ytext \congruencia{n}{m}{2}$.

    \item Probar que $\relacion$ es una relación de equivalencia.
  \end{enumerate}
\end{enunciado}

\begin{enumerate}[label=\alph*)]
  \item
        \begin{itemize}
          \item[\red{$(\entonces)$}]
                $$
                  n \relacion m
                  \Sii{def}
                  \congruencia{\magenta{n^2 + 4m^2 + m - 6n}}{0}{10}
                $$
                Si la expresión es divible por 10, debe ser divisible por 2 y también por 5:
                $$
                  \llave{l}{
                    \magenta{n^2 + 4m^2 + m - 6n} \taa{}{\red{!}}{\conga5}
                    \congruencia{n^2 - m^2 + m - n}{0}{5 }\Tilde\\
                    \magenta{n^2 + 4m^2 + m - 6n} \conga2 n^2 + m \taa{}{\red{!!}}{\conga2}
                    \congruencia{n+m}{0}{2} \sii \congruencia{n}{m}{2} \Tilde
                  }
                $$
                Si no ves lo que pasó en $\red{!!}$ pensá en la paridad de un número y su cuadrado.\par

                Por lo tanto si
                $$
                  n \relacion m
                  \red{\quad\entonces\quad}
                  5 \divideA n^2 -m^2 + m -n \ytext \congruencia{n}{m}{2}
                $$

          \item[\red{$(\Leftarrow)$}]
                $$
                  \congruencia{n^2 -m^2 + m -n}{0}{5}
                  \sii
                  \congruencia{\magenta{n^2 + 4m^2 + m - 6n}}{0}{5}
                  \sii
                  5 \divideA\magenta{n^2 + 4m^2 + m - 6n} \Tilde
                $$
                Ahora uso la información de $\congruencia{n}{m}{2} $
                $$
                  \text{Si }\congruencia{n}{m}{2}
                  \entonces
                  \magenta{n^2 + 4m^2 + m - 6n}
                  \taa{}{\red{!}}{\conga2}
                  \congruencia{ 5 \ub{m (m - 1)}{\text{par\red{!}}}}{0}{2}
                  \sisolosi
                  2 \divideA \magenta{n^2 + 4m^2 + m - 6n}\Tilde
                $$\par
                Por lo tanto si
                $$
                  n \relacion m
                  \red{\quad\Leftarrow\quad}
                  5 \divideA n^2 -m^2 + m -n \ytext \congruencia{n}{m}{2}
                $$
        \end{itemize}

  \item No es casualidad que en el punto anterior tuvieramos una \textit{redefinición} de la
        relación $\relacion$:
        $$
          n \relacion m \sisolosi
          \llave{l}{
            \congruencia{n^2 -m^2 + m -n}{0}{5}\\
            \ytext \\
            \congruencia{n}{m}{2}.
          }
        $$
        En esa forma es mucho más fácil mostrar lo que sigue porque la relación
        queda definida en función de congruencias que \textit{\underline{ya son relaciones de equivalencias}}.
        Para mostrar la relación de equivalencia, hay que probar que es
        reflexiva, simétrica y transitiva.\par

        \textit{Reflexiva: } Si $ n \relacion n \sisolosi
          \llave{l}{
            \congruencia{n^2 -n^2 + n -n = 0}{0}{5} \Tilde\\
            \ytext \\
            \congruencia{n}{n}{2} \Tilde.
          }
        $
        \par
        La relación es \textit{reflexiva}. \par

        \textit{Simétrica: }
        Si $ n \relacion m \entonces m \relacion n$, para algún par $n$, $m$.
        \par
        Si $n \relacion m \entonces
          \llave{l}{
            \congruencia{n^2 -m^2 + m -n}{0}{5}
            \Entonces{$m \relacion n$}
            \congruencia{m^2 -n^2 + n - m = -(n^2 -m^2 + m -n)}{0}{5} \Tilde\\
            \ytext \\
            \congruencia{n}{m}{2}
            \Entonces{$m \relacion n$}
            \congruencia{m}{n}{2}\Tilde
          }
        $\par
        La relación es \textit{simétrica}

        \textit{Transitiva: }
        Quiero ver que si:
        $n \relacion m \ytext m \relacion j \entonces n \relacion j$\par
        Si
        $$n \relacion m \sii
          \llave{l}{
            \congruencia{n^2 -m^2 + m -n}{0}{5}\\
            \ytext \\
            \congruencia{n}{m}{2}
          }
          \ytext
          m \relacion j \sii
          \llave{l}{
            \congruencia{m^2 -j^2 + j -m}{0}{5} \llamada1\\
            \ytext \\
            \congruencia{m}{j}{2} \llamada2
          }
        $$

        entonces

        $$
          \left.
          \begin{array}{c}
            \congruencia{n^2 \magenta{- m^2 + m} - n}{0}{5}
            \Sii{\llamada1}[\red{!}]
            \congruencia{n^2 \magenta{- j^2 + j} - n}{0}{5}
            \\
            \ytext \\
            \congruencia{n}{m}{2}
            \Sii{\llamada2}
            \congruencia{n}{j}{2}
          \end{array}
          \right\}
          \entonces
          \boxed{
            n \relacion j}
        $$\par
        La relación es \textit{transitiva}.\par
        Como la relación resultó ser \textit{reflexiva, simétrica y transitiva,} entonces es de equivalencia. Fin.
\end{enumerate}
