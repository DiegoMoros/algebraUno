\begin{enunciado}{\ejExtra}
  Sea $X = \set{f: \set{ n \en \naturales : n \leq 5} \to \set{n \en \enteros : 0 \leq n \leq 100}}$, es decir, $X$ es el conjunto de todas
  las funciones del conjunto $\set{1, \dots, 5}$ en el conjunto $\set{0,\dots, 100}$. Se define en $X$ la relación $\relacion$ dada por
  $$
    f \relacion g \sisolosi \congruencia{f(4)}{g(4)}{\text{mód } 3}
  $$
  \begin{enumerate}[label=\alph*)]
    \item Decida si la relación es reflexiva, simétrica, transitiva y/o antisimétrica.
    \item Sea $g \en X$ la función difinida $g(n) = 5$ para todo $n \en \set{1, \dots 5}$. Calcule la cantidad de funciones
          inyectivas $f$ tales que $f \relacion g$ y $f(5) = 14$.
  \end{enumerate}
\end{enunciado}

\begin{enumerate}[label=\alph*)]
  \item
        Dado que la relación $\relacion$ está definida en una congruencia algunas de las propiedades se cumplen de forma trivial debido
        a la congruencia.

        \textit{¿Es la relación $\relacion$ reflexiva?}
        $$
          \congruencia{f(4)}{f(4)}{3}
        $$
        Se cumple de manera trivial.
        \cajaResultado{\relacion \text{ es \textit{reflexiva}}}.

        \medskip

        \textit{¿Es la relación $\relacion$ simétrica?}
        $$
          \text{Si }	\congruencia{f(4)}{g(4)}{3}
          \Entonces{?}
          \congruencia{g(4)}{f(4)}{3}
        $$
        Se cumple de manera trivial.
        \cajaResultado{\relacion \text{ es \textit{simétrica}}}.

        \medskip

        \textit{¿Es la relación $\relacion$ transitiva?}
        $$
          \text{Si }
          \congruencia{f(4)}{g(4)}{3}
          \ytext
          \congruencia{g(4)}{h(4)}{3}
          \Entonces{?}
          \congruencia{f(4)}{h(4)}{3}
        $$
        Se cumple de manera trivial.
        $\quad\congruencia{f(4)}{g(4) \conga3 h(4)}{3}\quad$
        \cajaResultado{\relacion \text{ es \textit{transtiva}}}.

        \medskip

        \textit{¿Es la relación $\relacion$ antisimétrica?}
        $$
          \text{Si }
          \congruencia{f(4)}{g(4)}{3}
          \ytext
          \congruencia{g(4)}{f(4)}{3}
          \Entonces{?}
          f = g
        $$
        Y, me parece que no. Un contraejemplo viene al pelo. Busco dos funciones que cumplan la relación pero que sean distintas:
        $$
          \llave{l}{
            f(n) = 6n \\
            g(n) = 3n
          }
          \entonces
          \congruencia{\ub{f(4)}{\conga 3 0}}{\ub{g(4)}{\conga 3 0}}{3}
          \ytext
          \congruencia{\ub{g(4)}{\conga 3 0}}{\ub{f(4)}{\conga 3 0}}{3}
          \quad \text{con} \quad
          f \distinto g
        $$
        El contraejemplo dice que
        \cajaResultado{\relacion \text{ \underline{no} es \textit{antisimétrica}}}.

  \item
        Quiero que ocurra que:
        \begin{itemize}
          \item $f(5) = 14$.
          \item $f$ es inyectiva.
          \item $f \relacion g \entonces \congruencia{f(4) \conga{3} g(4)}{2}{3}$
        \end{itemize}

        Esa última condición me dice que $f(4)$ es de la pinta:
        $$
          f(4) = 3k + 2
          \quad \text{con} \quad k \en \enteros
        $$
        ¿Cuánto posibles valores del conjunto del codominio $\set{0, \dots, 100}$ hay que cumplan que su módulo 3 es 2?
        $$
          0 \leq 3k+2 \leq 100
          \sii
          -\frac{2}{3} \leq 3k+2 \leq \frac{98}{3}
          \sii
          k \en \set{0, \dots, 32}
        $$
        donde $\#\set{0, \dots, 32} = \blue{33}$.

        Ojito que hay una \textit{trampilla} con el $f(5) = 14$, que o sorpresa $\congruencia{14}{\magenta{2}}{3}\red{!!}$

        Por lo tanto si me quiero armar funciones \textit{inyectivas} que cumplan lo pedido:
        $$
          \llave{l}{
            f(4) \to \text{32 opciones}   \\
            f(5) \to \text{ única opción} \\
            f(1) \to \text{99 opciones}   \\
            f(2) \to \text{98 opciones}   \\
            f(3) \to \text{97 opciones}
          }
        $$
        $f(4) \ytext f(5)$ son las que se laburaron, las otras 3 $f(1), f(2) \ytext f(3)$ son el relleno donde solo hay que prestar atención a no romper
        la inyectividad.

        Se concluye que la cantidad de funciones $f$ que cumplen lo pedido son en total:
        $$
          \cajaResultado{
            32 \cdot 1 \cdot 99 \cdot 98 \cdot 97
          }
        $$
\end{enumerate}

\begin{aportes}
  \item \aporte{\dirRepo}{naD GarRaz \github}
  \item \aporte{\neverGonnaGiveYouUp}{Magui \youtube}
\end{aportes}
