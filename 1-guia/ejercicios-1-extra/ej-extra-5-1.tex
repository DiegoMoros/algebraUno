\begin{enunciado}{\ejExtra}
  Sea $X$ el conjunto de todas las funciones de $\set{1,2,3,4,5,6,7,8}$ en $\set{0,1}$. Se define la relación $\relacion$ en $X$ como:
  $$
    f \relacion g \sisolosi f(1) + g(3) = f(3) + g(1).
  $$
  \begin{enumerate}[label=\alph*)]
    \item Probar que $\relacion$ es una relación de equivalencia. ¿Es $\relacion$ antisimétrica?
    \item Calcular la cantidad de clases de equivalencia de $\relacion$ y exhibir un representante de cada una de ellas.
  \end{enumerate}
\end{enunciado}

\begin{enumerate}[label=\alph*)]
  \item
        Para probar que $\relacion$ es una relación de equivalencia, hay que probar que sea \textit{reflexiva}, \textit{simétrica} y \textit{transitiva}.
        \hyperlink{teoria-1:prop-relaciones}{Click acá para la 'teoría' de que son esas cosas}\par

        \bigskip

        Las funciones toman todos los valores que hay en el conjunto del dominio y tienen que mandar ese valor a alguno de los dos valores que están
        en el conjunto del codominio. Podemos observar que la imagen de la función será $\set{0}, \set{1} \otext \set{0,1}$.

        Antes de arrancar a hacer cuentas voy a acomodar la $\relacion$ para que quede más fácil de leer \underline{para mí}.
        No es necesario hacer esto, pero como yo me distraigo hasta con la humedad del ambiente, me resulta más fácil pensarlo. Quedaría así:
        $$
          \fcolorbox{magenta}{white}{
            $f \relacion g \sisolosi f(1) - f(3) = g(1) - g(3).$
          }
        $$

        \textit{Reflexiva:} En este caso se cumple de forma trivial.
        $$
          f \relacion f \sisolosi f(1) - f(3) = f(1) - f(3)
        $$

        \textit{Simétrica:}
        Quiero ver que si $f \relacion g \entonces g \relacion f$.\par Resulta parecido al anterior dado que la igualdad no cambia al conmutar las funciones
        $$
          \begin{array}{l}
            f \relacion g \sisolosi \magenta{f(1) - f(3) = g(1) - g(3)} \\
            g \relacion f \sisolosi g(1) - g(3) = f(1) - f(3) \sisolosi \magenta{f(1) - f(3) = g(1) - g(3)}
          \end{array}
        $$

        \textit{Transitiva:}
        Quiero ver que si
        $$
          f \relacion g \ytext g \relacion h \entonces f \relacion h.
        $$

        Partiendo de las hipótesis de estas relaciones:
        $$
          \begin{array}{l}
            f \relacion g \sisolosi f(1) - f(3) \igual{$\llamada1$} g(1) - g(3) \\
            g \relacion h \sisolosi g(1) - g(3) \igual{$\llamada2$} h(1) - h(3) \\
          \end{array}
        $$

        Despejando de $\llamada2$ y reemplazando en $\llamada1$:
        $$
          g(1) \igual{$\llamada2$} \blue{h(1) - h(3) + g(3)}
          \Entonces{$\llamada1$}
          f(1) - f(3) = (\blue{h(1) - h(3) + g(3) }) - g(3)
          \sii
          \ub{f(1) - f(3) = h(1) - h(3)}{f \relacion h}
        $$

        \textit{Antisimétrica:}
        Puedo armar un ejemplo para ver si es antisimétrica. Cuando se define una función hay que definirla entera y no solo la parte que me interesa!
        Voy a armar un par de funciones que $(f,g)$ con $f \distinto g$ y $f \relacion g$ y que además  $g \relacion f$. Eso sería suficiente para
        mostrar que la función no es antisimétrica
        $$
          \llave{rcl}{
            f(1)   & = & \magenta{0} \\
            f(2)   & = & \blue{0}    \\
            f(3)   & = & \magenta{0} \\
            f(4)   & = & 0           \\
            \vdots & = & \vdots      \\
            f(8)   & = & 0           \\
          }\ytext
          \llave{rcl}{
            g(1)   & = & \magenta{0} \\
            g(2)   & = & \blue{1}    \\
            g(3)   & = & \magenta{0} \\
            g(4)   & = & 0           \\
            \vdots & = & \vdots      \\
            g(8)   & = & 0
          }
        $$
        La función no es antisimétrica.

  \item
        En este ejercicio hay 3 clases. Recuerdo que hago todo el ejercicio escribiendo la relación en esta forma:
        $$
          \fcolorbox{magenta}{white}{
            $f \relacion g \sisolosi f(1) - f(3) = g(1) - g(3).$
          }
        $$

        Solo puedo obtener como resultado de la cuenta, para la expresión del miembro izquierdo (y el derecho):
        $$
          -1, \, 0 \otext 1,
        $$
        Por lo tanto mientras la cuenta de $f(1)-f(3)$ de lo mismo para dos funciones distintas, van a estar relacionas y sino fueran iguales
        las funciones van a estar en clases distintas.

        \textit{Cuando me da 0:}
        $$
          \llave{l}{
            f(1) = \magenta{0} \ytext f(3) = \magenta{0}
            \entonces f(1) - f(3) = \magenta{0} \\
            f(1) = \magenta{1} \ytext f(3) = \magenta{1}
            \entonces f(1) - f(3) = \magenta{0} \\
          }
        $$
        Con esos valores obtengo la clase (y \underline{me invento esta notación, ojo!}) que me da $\magenta{\clase{0}}$.
        Todas las funciones de esa \textit{pinta} van a estar relacionadas. Piden uno pero te doy cuatro elementos de este conjunto a modo de ejemplo,
        porque soy un tipazo, no tengo nada que hacer y con el \textit{copy paste}
        es muy fácil:
        $$
          \llave{rcl}{
            f(1)   & = & \magenta{0} \\
            f(2)   & = & 0           \\
            f(3)   & = & \magenta{0} \\
            f(4)   & = & 0           \\
            \vdots & = & \vdots      \\
            f(8)   & = & 0           \\
          }
          \quad,\quad
          \llave{rcl}{
            g(1)   & = & \magenta{1} \\
            g(2)   & = & 0           \\
            g(3)   & = & \magenta{1} \\
            g(4)   & = & 0           \\
            \vdots & = & \vdots      \\
            g(8)   & = & 0           \\
          }
          \quad,\quad
          \llave{rcl}{
            h(1)   & = & \magenta{1} \\
            h(2)   & = & 0           \\
            h(3)   & = & \magenta{1} \\
            h(4)   & = & 1           \\
            \vdots & = & \vdots      \\
            h(8)   & = & 1           \\
          }
          \ytext
          \llave{rcl}{
            i(1)   & = & \magenta{0} \\
            i(2)   & = & 1           \\
            i(3)   & = & \magenta{0} \\
            i(4)   & = & 0           \\
            \vdots & = & \vdots      \\
            i(8)   & = & 0           \\
          }
        $$

        \bigskip

        \textit{Cuando me da 1:}
        $$
          \llave{l}{
            f(1) = \magenta{1} \ytext f(3) = \magenta{0}
            \entonces f(1) - f(3) = \magenta{1} \\
          }
        $$

        Con esos valores obtengo la clase (\underline{y sigo con la notación inventada, ojo!}) que me da $\magenta{\clase{1}}$.
        Todas las funciones de esa \textit{pinta} van a estar relacionadas. Tres elementos de este conjunto a modo de ejemplo, porque con el \textit{copy paste}
        sigue siendo muy fácil:
        $$
          \llave{rcl}{
            f(1)   & = & \magenta{1} \\
            f(2)   & = & 0           \\
            f(3)   & = & \magenta{3} \\
            f(4)   & = & 0           \\
            \vdots & = & \vdots      \\
            f(8)   & = & 0           \\
          }
          \ytext
          \llave{rcl}{
            g(1)   & = & \magenta{1} \\
            g(2)   & = & 1           \\
            g(3)   & = & \magenta{0} \\
            g(4)   & = & 0           \\
            \vdots & = & \vdots      \\
            g(8)   & = & 0           \\
          }
          \ytext
          \llave{rcl}{
            h(1)   & = & \magenta{1} \\
            h(2)   & = & 0           \\
            h(3)   & = & \magenta{0} \\
            h(4)   & = & 1           \\
            \vdots & = & \vdots      \\
            h(8)   & = & 1           \\
          }
        $$

        \bigskip

        \textit{Cuando me da -1:}
        $$
          \llave{l}{
            f(1) = \magenta{0} \ytext f(3) = \magenta{1}
            \entonces f(1) - f(3) = \magenta{-1} \\
          }
        $$

        Con esos valores obtengo la clase (\underline{y sigo con la notación inventada, ojo!}) que me da $\magenta{\clase{-1}}$.
        Todas las funciones de esa \textit{pinta} van a estar relacionadas. Tres elementos de este conjunto a modo de ejemplo, porque con el \textit{copy paste}
        sigue siendo muy fácil:
        $$
          \llave{rcl}{
            f(1)   & = & \magenta{0} \\
            f(2)   & = & 0           \\
            f(3)   & = & \magenta{1} \\
            f(4)   & = & 0           \\
            \vdots & = & \vdots      \\
            f(8)   & = & 0           \\
          }
          \ytext
          \llave{rcl}{
            g(1)   & = & \magenta{0} \\
            g(2)   & = & 1           \\
            g(3)   & = & \magenta{1} \\
            g(4)   & = & 0           \\
            \vdots & = & \vdots      \\
            g(8)   & = & 0           \\
          }
          \ytext
          \llave{rcl}{
            h(1)   & = & \magenta{0} \\
            h(2)   & = & 0           \\
            h(3)   & = & \magenta{1} \\
            h(4)   & = & 1           \\
            \vdots & = & \vdots      \\
            h(8)   & = & 1           \\
          }
        $$
\end{enumerate}

\begin{aportes}
  \item \aporte{https://github.com/nad-garraz}{Nad Garraz \github}
  \item \aporte{https://github.com/JowinTeran}{Ale Teran \github}
\end{aportes}
