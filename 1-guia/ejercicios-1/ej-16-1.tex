\begin{enunciado}{\ejercicio}
  Sean $A, B$ y $C$ conjuntos. Probar que:
  \begin{enumerate}[label=\roman*)]
    \item $(A \union B) \times C = (A \times C) \union (B \times C)$
    \item $(A \inter B) \times C = (A \times C) \inter (B \times C)$
    \item $(A - B) \times C = (A \times C) - (B \times C)$
    \item $(A \triangle B) \times C = (A \times C) \triangle (B \times C)$
  \end{enumerate}
\end{enunciado}

\begin{enumerate}[label=\roman*)]
  \item
        Para demostrar igualdad de conjuntos habría que probar la doble inclusión, es decir:

        $$
          \begin{array}{l}
            (A \union B) \times C \magenta{\subseteq} (A \times C) \union (B \times C) \\
            (A \times C) \union (B \times C) \magenta{\subseteq} (A \union B) \times C
          \end{array}
        $$
        O bien si podemos conectar los pasos con "$\sisolosi$". En este caso se usa el de los "$\sisolosi$" y mucho
        de las \hyperlink{teoria-1:conjuntos-basicos}{definiciones que podés ver acá en las notas teóricas}:\par
        Sea el par $(x,y)$
        $$
          \begin{array}{c}
            (x,y) \en (A \union B) \times C)
            \Sii{def prod.}[Cartesiano]
            x \en (A \union B) \ytext y \en C
            \Sii{def}[$\union$]
            (x \en A \otext x \en  B  )\ytext x \en C
          \end{array}
        $$
        Si está en $A$ o en $B$ y seguro está en $C$, entonces $x$ tiene que estar en $A \inter C$ o bien en $B \inter C$,
        que no es otra cosa que distribuir el "y" con el "o":
        $$
          \Sii{distribución}
          (x \en A  \ytext x \en C) \otext (x \en  B \ytext x \en C)
          \Sii{\red{!}}
          (x,y) \en (A \times C) \otext (x,y) \en (B\times C)
        $$
        Ese paso del \red{!} es la definición de producto cartesiano como al principio y se concluye que:
        $$
          (A \union B) \times C \igual{\magenta{\Tilde}}
          (A \times C) \union (B\times C)
        $$
  \item \hacer
  \item \hacer
  \item \hacer
\end{enumerate}


% Contribuciones
\begin{aportes}
  %% iconos : \github, \instagram, \tiktok, \linkedin
  %\aporte{url}{nombre icono}
  \item \aporte{https://github.com/nad-garraz}{Nad Garraz \github}
  \item \aporte{https://github.com/koopardo}{Marcos Zea \github}
\end{aportes}
