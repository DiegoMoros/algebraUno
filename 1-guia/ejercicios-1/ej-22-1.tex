\begin{enunciado}{\ejercicio}
  En cada uno de los siguientes casos determinar si la relación $\relacion $ en $A$ es reflexiva, simétrica,
  antisimétrica, transitiva, de equivalencia o de orden.
  \begin{enumerate}[label=\roman*)]
    \item  $A = \set{1,2,3,4,5}, \relacion = {(1,1), (2,2), (3,3), (4,4), (5,5), (1,2), (1,3), (2,5), (1,5)}$\\
    \item 

    \item 
    \item 
    \item 
    \item $A = \partes(\set{n \en \naturales \talque n \leq 30})$, $\relacion$ definida por $X \relacion Y \sisolosi 2 \notin X \inter Y^c$\\
    \item $A = \naturales \times \naturales,\, \relacion$ definida por $(a,b) \relacion (c,d) \sisolosi bc$ es múltiplo de $ad$.
  \end{enumerate}

\end{enunciado}

\begin{enumerate}
  \item \textit{Relación de equivalencia}: La relación debe ser reflexiva, simétrica y transitiva.
  \item \textit{Relación de orden}: La relación debe ser reflexiva, antisimétrica y transitiva.
\end{enumerate}

\begin{enumerate}[label=\roman*)]
  \item  $A = \set{1,2,3,4,5}, \relacion = {(1,1), (2,2), (3,3), (4,4), (5,5), (1,2), (1,3), (2,5), (1,5)}$\\
        \begin{enumerate}
          \item[R:] Es reflexiva, porque hay bucles en todos los elementos de $A$.
          \item[S:] No es simétrica, dado que existe $(1, 5)$, pero no $(5, 1)$
          \item[AS:] Es antisimétrica. No hay ningún par que tenga la vuelta, excepto los casos $x \relacion x$.
          \item[T:] Es transitiva. La terna 1, 2, 5 es transitiva.
        \end{enumerate}
        La relación es R, AS y T, por lo tanto es una \textit{relación de orden}.

  \item \hacer

  \item \hacer
  \item \hacer
  \item \hacer
  \item $A = \partes(\set{n \en \naturales \talque n \leq 30})$, $\relacion$ definida por $X \relacion Y \sisolosi 2 \notin X \inter Y^c$\\
        $\begin{array}{|c|c|c|c|c|c|}
            \hline
            2 \en X     & 2 \en Y     & 2 \en Y^c   & 2 \en X^c   & 2 \notin X \inter Y^c & 2 \notin Y \inter X^c \\ \hline  \hline
            \magenta{V} & \magenta{V} & \magenta{F} & \magenta{F} & \magenta{V}           & \magenta{V}           \\
            \cyan{V}    & \cyan{F}    & \cyan{V}    & \cyan{F}    & \cyan{F}              & \cyan{V}              \\
            \cyan{F}    & \cyan{V}    & \cyan{F}    & \cyan{V}    & \cyan{V}              & \cyan{F}              \\
            \magenta{F} & \magenta{F} & \magenta{V} & \magenta{V} & \magenta{V}           & \magenta{V}           \\ \hline
          \end{array}$.\\
        \begin{enumerate}
          \item[R:]
                La relación es reflexiva ya que para que un elemento $X$ esté relacionado con sí mismo debe ocurrir
                que $X \relacion X \sisolosi 2 \notin X \inter X^c$, es decir $2 \notin \vacio$, lo cual es siempre cierto.

          \item[S:]
                La relación no es simétrica. Se puede ver con la \cyan{segunda y tercera} fila de la tabla con un contraejemplo.
                $X = \set{1}$ y $Y = \set{2},\, X,Y \subseteq A$, $X \relacion Y$, pero $Y \norelacion X$,

          \item[AS:]
                La relación no es antisimétrica. Se puede ver con la \magenta{primera o cuarta} fila tabla con un contraejempl
                con un contraejemplo. Si $X = \set{1,2}$ e $Y = \set{2,3} \entonces X \relacion Y$ y además $Y \relacion X$
                con  $X \distinto Y$.

          \item[T:]
                Es transitiva. Si bien no es lo más fácil de explicar, se puede ver en la tabla que para tener 2 relaciones
                en una terna $X, Y, Z$ no se puede llegar nunca al caso de la segunda fila de la tabla, donde se lograría que
                $X \norelacion Z$
        \end{enumerate}

  \item $A = \naturales \times \naturales,\, \relacion$ definida por $(a,b) \relacion (c,d) \sisolosi bc$ es múltiplo de $ad$.
        \begin{enumerate}
          \item[R:] $(a,b) \relacion (a,b) \sisolosi ba = k\cdot ab$ con $k=1$, se concluye que sí es reflexiva.
          \item[S:]
                $ \llave{l}{
                    (a,b) \relacion (c,d) \sisolosi bc \llamada{1}= k\cdot ad \\
                    (c,d) \relacion (a,b) \sisolosi ad = h\cdot bc \llamada{1}= h\cdot k \cdot ad = k'ad
                  }$.\\
                con  $k'=1$ se cumple la igualdad. La relación es simétrica.
          \item[AS:] Si tomo $(a, b) = (4, 2)$ y $(c,d) = (16, 4)$, tengo que
                $(a,b) \relacion (c,d)$ con $(a,b) \neq (c,d)$. Por lo tanto
                la relación no es antisimétrica.
          \item[T:]$
                  \llave{l}{
                    (a,b) \relacion (c,d) \sisolosi bc \llamada{1}= k\cdot ad \\
                    (c,d) \relacion (e,f) \sisolosi de \llamada{1}= h\cdot cf \\
                    \qvq (a,b) \relacion (e,f) \sisolosi \magenta{be = k'\cdot af}\\
                    \flecha{multiplico}[M.A.M.]
                    \llaves{l}
                    {
                      bc \llamada{1}= k\cdot ad \\
                      de \llamada{1}= h\cdot cf
                    } \flecha{y}[acomodo]
                    be \cdot \cancel{cd} = k \cdot h \cdot af \cdot \cancel{cd} \to
                    \magenta{be \igual{\Tilde} k' \cdot af}.
                  }$\\
                Se concluye que la relación es transitiva.
        \end{enumerate}
        Con esos resultados se puede decir que $\relacion$ en $A$ es de \textit{equivalencia}.

\end{enumerate}
