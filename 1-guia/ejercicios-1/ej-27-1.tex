%Graficos
\def\veintisiete{
  \begin{tikzpicture}[scale=0.5, baseline=0, >=Latex, draw=Aquamarine]

    \node[] (1) {$\bullet$};
    \node[] at (1.west) {$1$};

    \node[right = of 1] (92) {$\bullet$};
    \node[] at (92.east) {$92$};

    \node[below = of 1] (2) {$\bullet$};
    \node[] at (2.west) {$2$};

    \node[right = of 2] (91) {$\bullet$};
    \node[] at (91.east) {$91$};

    \node[below right = .5 of 2] (puntos) {$\vdots$};

    \node[below left = .5 of puntos] (46) {$\bullet$};
    \node[] at (46.west) {$46$};

    \node[right = of 46] (47) {$\bullet$};
    \node[] at (47.east) {$47$};


    % Universo
    \node[shape=ellipse, draw, black, fit={ (1) (92) (46) (47)}] (universo) {};
    \node[above left = 0.1cm of universo] {$A$};

    % Aristas
    \draw[->, loop below] (1) to (1);
    \draw[->, loop below ] (92) to (92);

    \draw[->, loop below] (2) to (2);
    \draw[->, loop below ] (91) to (91);

    \draw[->, loop below] (46) to (46);
    \draw[->, loop below ] (47) to (47);

    \draw[->, bend left] (1.center) to (92);
    \draw[->, bend left] (92.center) to (1);
    \draw[->, bend left] (2.center) to (91);
    \draw[->, bend left] (91.center) to (2);
    \draw[->, bend left] (46.center) to (47);
    \draw[->, bend left] (47.center) to (46);
  \end{tikzpicture}
}

\begin{enunciado}{\ejercicio}

  Sean $A = \set{n \en \naturales \talque n \leq 92}$ y
  $\relacion$ la relación en $A$ definida por
  $x \relacion y \sisolosi x^2 - y^2 = 93x - 93y$
  \begin{enumerate}[label=\alph*)]
    \item Probar que $\relacion$ es una relación de equivalencia. ¿Es antisimétrica?
    \item Hallar la clase de equivalencia de cada $x \en A$.
            Deducir cuántas clases de equivalencia \textbf{distintas} determina la relación $\relacion$.
  \end{enumerate}

\end{enunciado}

\begin{enumerate}[label=\alph*)]
  \item Primero acomodo la condición de la relación:
        $$x^2 - y^2 = 93x - 93y
          \Sii{\red{!!!}}
          \llave{c}{
            x      \igual{$\llamada{1}$}  y \\
            \text{ o bien }                            \\
            x + y  \igual{$\llamada{2}$}  93
          }
        $$
        Hacer este ejercicio sin avivarse de lo que pasa en \red{!!!} es horrible.\par
        Para ser relación de equivalencia es necesario que sea \textit{reflexiva, simétrica} y \textit{transitiva}:\par
        \textit{Reflexiva: }
        $$
          x \relacion x \sisolosi x \igual{$\llamada{1}$} x  \Tilde
        $$
        \textit{Simétrica: }
        $$
          \llave{l}{
            x \relacion y \sisolosi x + y \igual{$\llamada{2}$} 93 \\
            y \relacion x \sisolosi y + x \igual{$\llamada{2}$} 93
          }\Tilde
        $$
        \textit{Transitiva: }
        $$
          \llave{l}{
            x \relacion y \sisolosi x \igual{$\llamada{2}$} 93 - y  \\
            y \relacion z \sisolosi y \igual{$\llamada{2}$}  93 - z \\
          }
          \Entonces{resto}[M.A.M] x - y = -y + z \to x \igual{$\llamada{1}$} z \sisolosi x \relacion z \Tilde
        $$

        \textit{Antisimétrica: }\par
        La $\relacion$ no es antisimétrica, como contraejemplo se ve que
        $1 \relacion 92$ y $92 \relacion 1$ con $1 \distinto 92\quad \skull$.

  \item
        A priori no sé como encontrar las clases de equivalencia, pero solo buscando la relación del $1$
        con algún número (excepto el mismo) veo que únicamente se puede relacionar con el $92$
        por la condición $\llamada2$, dado que $1 + 92 \igual{$\llamada2$} 93$.
        De ahí se pueden inferir que todas las clases van a ser conjuntos \textit{chiquitos}, con los números que sumen
                93.

        \begin{minipage}{0.7\textwidth}
        Las clases de equivalencia :\par
          $\llave{ccccc}{
              \clase{1}  & = & \clase{92} & = & \set{1, 92}  \\
              \clase{2}  & = & \clase{91} & = & \set{2, 91}  \\
              \vdots     &   & \vdots     &   & \vdots       \\
              \clase{46} & = & \clase{47} & = & \set{46, 47} \\
            }$\par
          Hay entonces 46 clases. $A = \set{\clase{1},\, \clase{2},\, \dots,\, \clase{45},\, \clase{46}}$
        \end{minipage}
        \begin{minipage}{0.2\textwidth}
          \veintisiete
        \end{minipage}
\end{enumerate}

% Contribuciones
\begin{aportes}
  %% iconos : \github, \instagram, \tiktok, \linkedin
  %\aporte{url}{nombre icono}
  \item \aporte{https://github.com/nad-garraz}{Nad Garraz \github}
\end{aportes}
