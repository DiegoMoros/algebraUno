%Graficos
\def\veinte{
	\begin{tikzpicture}[scale=0.5, baseline=0, >=Latex, draw=Aquamarine]

		\node[] (1) {$\bullet$};
		\node[] at (1.west) {$1$};

		\node[above right= of 1] (2) {$\bullet$};
		\node[] at (2.east) {$2$};

		\node[below right= of 2] (3) {$\bullet$};
		\node[] at (3.east) {$3$};

		\node[below= of 1] (4) {$\bullet$};
		\node[] at (4.west) {$4$};

		\node[right= of 2] (5) {$\bullet$};
		\node[] at (5.west) {$5$};

		\node[right= of d] (6) {$\bullet$};
		\node[] at (6.east) {$6$};


		% Universo
		\node[shape=ellipse, draw, black, fit={ (1) (2) (3) (4)}] (universo) {};
		\node[above left = 0.1cm of universo] {$A$};

		% Aristas
		\draw[->, loop below] (1) to (1);
		\draw[->, loop above ] (3) to (3);
		\draw[->, loop above] (4) to (4);
		\draw[->, loop below ] (6) to (6);

		\draw[->, bend left] (6.center) to (4);
		\draw[->, bend left] (4.center) to (6);

		\draw[->, bend right] (1.center) to (3);
		\draw[->, bend right] (3.center) to (1);
	\end{tikzpicture}
}
% fin gráficos

\ejercicio 
Sea $A = \set{1, 2, 3, 4, 5, 6}$. Graficar la relación, $\relacion= {(1,1), (1,3), (3,1), (3,3), (6,4), (4,6), (4,4), (6,6)}$

\begin{minipage}{0.25\textwidth}
	\veinte
\end{minipage}
\begin{minipage}{0.7\textwidth}
	\begin{itemize}
		\item No es reflexiva porque no hay bucles ni en 2 ni en 5.
		\item Es simétrica, porque hay ida y vuelta en todos los pares de vértices.
		\item No es antisimétrica, porque $1 \relacion 3$ y $3 \relacion 1$ con $1 \neq 3$.
		\item Es transitiva. \\
		      \red{Chequear. Caso particula donde no hay ternas de $x,y,z$ distintos}.
		      \blue{Sí, el que $2$ esté ahí solo ni cumple la hipótesis de transitividad.}
	\end{itemize}
\end{minipage}
