\begin{enunciado}{\ejercicio}

  \begin{enumerate}[label=\roman*)]
    \item
          Sea $A = \set{1, 2, 3, 4, 5, 6, 7, 8, 9, 10}$. Consideremos en $\partes(A)$ la relación de equivalencia dada
          por el cardinal (es decir, la cantidad de elementos): Dos subconjuntos de $A$ están relacionados si y solo si
          tienen la misma cantidad de elementos ¿Cuántas clases de equivalencia \textbf{distintas}
          determina la relación? Hallar un representante para cada clase.

    \item
          En el conjunto de todos los subconjuntos finitos de $\naturales$, consideremos nuevamente la relación
          de equivalencia dada por el cardinal: Dos subconjuntos finitos de $\naturales$ están relacionados
          si y solo si tienen la misma cantidad de elementos ¿Cuántas clases de equivalencia
          \textbf{distintas} determina la relación?
          Hallar un representante para cada clase.
  \end{enumerate}
\end{enunciado}

\begin{enumerate}[label=\roman*)]
  \item $\partes(A) = \set{\vacio, \set{1}, \set{1,2}, \cdots, \set{1, 2, 3, 4, 5, 6, 7, 8, 9, 10}}$, el conjunto $\partes(A)$ tiene un total de
        $2^{10} = 1024$ elementos. La relación determina 11 \textit{clases de equivalencia} distintas.
        $$
          \begin{array}{ccc}
            \rowcolor{Cerulean!10}
            \text{Característica de la clase} & \text{clase}                       & \text{elemento ejemplo} \\ \hline
            \text{Conjuntos con \#0: }        & \clase{\vacio}                     & \vacio                  \\
            \text{Conjuntos con \#1: }        & \clase{\set{1}}                    & \set{3}                 \\
            \text{Conjuntos con \#2: }        & \clase{\set{1,2}}                  & \set{5,2}               \\
            \text{Conjuntos con \#3: }        & \clase{\set{1,2,3}}                & \set{1,6, 3}            \\
            \text{Conjuntos con \#4: }        & \clase{\set{1,2,3,4}}              & \set{1,8, 10,4}         \\
            \vdots                            & \vdots                             & \vdots                  \\
            \text{Conjuntos con \#10: }       & \clase{\set{1,2,3,4,5,6,7,8,9,10}} & A                       \\
          \end{array}
        $$

        Pobres e incompletos ejemplitos, sino se me va la vida:
        $$
          \begin{tikzpicture}[scale=0.7, baseline=0, >=Latex, draw=Aquamarine]

            \node[] (a) {$\set{2}$};
            %\node[] at (a.north west) {};

            \node[below left=1cm of a] (b) {$\set{3}$};
            %\node[] at (b.south) {};

            \node[below right=1cm of a] (f) {$\set{1}$};
            %\node[] at (f.south) {};

            \node[above right=1cm of a] (d) {$\vacio$};
            %\node[] at (d.west) {};

            \node[right=1.5cm of f] (d2) {$A$};
            %\node[] at (d2.west) {$\vacio$};

            \node[right=1.5cm of a] (c) {$\set{1,2}$};
            %\node[] at (c.south) {$\set{1,2}$};

            \node[right=1cm of c] (e) {$\set{3,10}$};
            %\node[] at (e.south) {$\set{3,10}$};

            % Universo
            \node[shape=ellipse, draw, black, fit={ (a) (b) (d) (f) (e)}] (universo) {};

            % Aristas
            \draw[magenta, ->, loop above] (a) to (a);
            \draw[magenta, ->, loop left ] (b) to (b);
            \draw[->, loop left] (c) to (c);
            \draw[->, loop right ] (e) to (e);
            \draw[magenta, ->, loop right ] (f) to (f);

            \draw[OliveGreen, ->, loop above ] (d) to (d);
            \draw[OliveGreen, ->, loop below ] (d2) to (d2);

            \draw[magenta, ->, bend left] (a.south) to (b);
            \draw[magenta, ->, bend left] (b.north) to (a);
            \draw[magenta, ->, bend left] (a.east) to (f);
            \draw[magenta, ->, bend left] (f.north) to (a);
            \draw[magenta, ->, bend left] (b.east) to (f);
            \draw[magenta, ->, bend left] (f.south) to (b);

            \draw[->, bend right] (c.south) to (e);
            \draw[->, bend right] (e.north) to (c);
          \end{tikzpicture}
        $$

  \item
        Es parecido al inciso anterior, donde ahora $A = \set{1,2,3, \cdots, N-1, N}$, donde $\partes(\naturales_N)$ tiene $2^N$ elementos.\par
        La relación determina $N+1$ \textit{clases de equivalencia} distintas.
        $$
          \begin{array}{cccc}
            \rowcolor{Cerulean!10}
            \text{Característica de la clase} & \text{clase}                   & \text{elemento ejemplo}      & \text{Conjuntos en clase} \\ \hline
            \text{Conjuntos con \#0 }         & \clase{\vacio}                 & \vacio                       & \binom{N}{0}              \\
            \text{Conjuntos con \#1}          & \clase{\set{1}}                & \set{3}                      & \binom{N}{1}              \\
            \text{Conjuntos con \#2 }         & \clase{\set{1,2}}              & \set{5,2}                    & \binom{N}{2}              \\
            \text{Conjuntos con \#3 }         & \clase{\set{1,2,3}}            & \set{1,6, 3}                 & \binom{N}{3}              \\
            \vdots                            & \vdots                         & \vdots                       &                           \\
            \text{Conjuntos con \#$N-1$: }    & \clase{\set{1,2,\ldots,N-1}}   & \set{1,2,3, \ldots,N-2, N-1} & \binom{N}{N-1}            \\
            \text{Conjuntos con \#$N$: }      & \clase{\set{1,2,\ldots,N-1,N}} & A                            & \binom{N}{N}              \\
          \end{array}
        $$
\end{enumerate}

\begin{aportes}
  \item \aporte{\dirRepo}{naD GarRaz \github}
  \item \aporte{https://github.com/franramosfx}{Fran Ramos \github}
\end{aportes}
