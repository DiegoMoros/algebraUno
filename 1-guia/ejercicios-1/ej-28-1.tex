\begin{enunciado}{\ejercicio}

  \begin{enumerate}[label=\roman*)]
    \item
          Sea $A = \set{1, 2, 3, 4, 5, 6, 7, 8, 9, 10}$. Consideremos en $\partes(A)$ la relación de equivalencia dada
          por el cardinal (es decir, la cantidad de elementos): Dos subconjuntos de $A$ están relacionados si y solo si
          tienen la misma cantidad de elementos ¿Cuántas clases de equivalencia \textbf{distintas}
          determina la relación? Hallar un representante par acada clase.

    \item
          En el conjunto de todos los subconjuntos finitos de $\naturales$, consideremos nuevamente la relación
          de equivalencia dada por el cardinal: Dos subconjuntos finitos de $\naturales$ están relacionados
          si y solo si tienen la misma cantidad de elementos ¿Cuántas clases de equivalencia
          \textbf{distintas} determina la relación?
          Hallar un representante para cada clase.
  \end{enumerate}
\end{enunciado}

\begin{enumerate}[label=\roman*)]
  \item $\partes(A) = \set{\vacio, \set{1}, \set{1,2}, \cdots, \set{1, 2, 3, 4, 5, 6, 7, 8, 9, 10}}$, el conjunto $\partes(A)$ tiene un total de
        $2^{10} = 1024$ elementos. La relación determina 11 \textit{clases de equivalencia} distintas.\par
        $\llave{lcl}{
            \text{Conjuntos con 0 elementos: } & \clase{0}  & \vacio\\
            \text{Conjuntos con 1 elemento: } & \clase{1} & \set{3}\\
            \text{Conjuntos con 2 elementos: }& \clase{2} & \set{5,2}\\
            \text{Conjuntos con 3 elementos: }& \clase{3} & \set{1,6, 3}\\
            \text{Conjuntos con 4 elementos: }& \clase{4} & \set{1,8, 10,4}\\
            \vdots                            &\vdots  & \vdots\\
            \text{Conjuntos con 10 elementos: } & \clase{10} & \set{1,2,3,4,5,6,7,8,9,10} = A\\
          }$
  \item
        Es parecido al inciso anterior, donde ahora $A = \set{1,2,3, \cdots, N-1, N}$, donde $\partes(\naturales_N)$ tiene $2^N$ elementos.\par
        La relación determina $N+1$ \textit{clases de equivalencia} distintas.\par
        $\llave{lcl}{
            \text{Conjuntos con 0 elementos: } & \clase{0}  & \vacio\\
            \text{Conjuntos con 1 elemento: } & \clase{1} & \set{3}\\
            \text{Conjuntos con 2 elementos: }& \clase{2} & \set{5,2}\\
            \text{Conjuntos con 3 elementos: }& \clase{3} & \set{1,6, 3}\\
            \text{Conjuntos con 4 elementos: }& \clase{4} & \set{1,8, 10,4}\\
            \vdots                            &\vdots  & \vdots\\
            \text{Conjuntos con 10 elementos: } & \clase{N} & \set{1,2,3,4, \cdots, N-1, N} \naturales_N\\
          }$
\end{enumerate}
