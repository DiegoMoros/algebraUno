\begin{enunciado}{\ejercicio}
  Determinar si las siguientes funciones son inyectivas, sobreyectivas o biyectivas.
  Para las que sean biyectivas hallar la inversa y para la que no sean sobreyectivas hallar la imagen.
  \begin{enumerate}[label=\roman*)]
    \item $f: \reales \to \reales,\quad f(x) = 12x^2 -5.$

    \item $f: \reales^2 \to \reales,\quad f(x,y) = x + y.$

    \item $f: \reales^3 \to \reales^2,\quad f(x,y,z) = (x+y,2z).$

    \item $f: \naturales \to \naturales,\quad
            f(n) =
            \llave{ll}{
              \frac{n}{2} & \text{si $n$ es par}    \\
              n + 1       & \text{si $n$ es impar.} \\
            }$

    \item $f: \enteros \times \enteros \to \enteros,\quad f(a,b) = 3a - 2b.$

    \item $f: \enteros \to \naturales,\quad f(a) =
            \llave{lcl}{
              2a   & \text{si} & a > 0     \\
              1-2a & \text{si} & a \leq 0.
            }$
  \end{enumerate}
\end{enunciado}

\begin{enumerate}[label=\roman*)]
  \item
        $f: \reales \to \reales,\quad f(x) = 12 x^2 - 5$
        No es \textit{inyectiva}, contraejemplo:
        $$
          f(-1) = f(1)
        $$.

        No es \textit{sobreyectiva}:
        $$
          \im(f) = [-5, +\infinito).
        $$

        No es \textit{biyectiva}, no tiene inversa. Habría que restringir dominio para cada rama de la parábola, pero no piden eso \faIcon{hands-wash}

  \item\label{ej30:item-ii}
        $f: \reales^2 \to \reales,\quad f(x,y) = x + y$

        No se \textit{inyectiva}. Contraejemplo:
        $$
          f(1,2) = 3 \ytext f(2,1) = 3
        $$

        Es \textit{sobreyectiva}, dado que $x + y$ genera todo $\reales$.

  \item Sale muy parecido al anterior \ref{ej30:item-ii}

        No es \textit{biyectiva}, no tiene inversa.

  \item $f: \naturales \to \naturales,\quad
          f(n) =
          \llave{ll}{
            \frac{n}{2} & \text{si $n$ es par}   \\
            n + 1       & \text{si $n$ es impar} \\
          }$\par

        No es \textit{inyectiva}. Contraejemplo:
        $$
          f(8) = f(3) \text{ con {\tiny \color{lightgray}{dah!}} } 8 \neq 3
        $$

        \underline{Sí} es \textit{sobreyectiva}.

        Me formo una sucesión de números pares para \textit{evaluar a la función
          en cosas convenientes}:
        $$
          \paratodo m \en \naturales, a_m = 2m
          \quad\entonces
          f(a_m) = f(2m) = \frac{2m}{2} = m
        $$
        Se desprende que tan solo con la parte $\frac{n}{2}$, la imagen de la función genera todo $\naturales$,
        así que como $\im(f) = \naturales$, $f$ es \textit{sobreyectiva}.

        No es \textit{biyectiva}, no tiene inversa.

  \item No es \textit{inyectiva}. Contraejemplo:
        $$
          f(2,1) = 6 - 2 = 4 \ytext f(0,-2) = 4.
        $$

        Para ser \textit{sobreyectiva} la imagen debe ser $\enteros$. Suponiendo que:
        $$
          a = b \entonces f(a,a) = 3a - 2a = a \entonces \im(f) = \enteros
        $$
        Por lo tanto $f$ es \textit{sobreyectiva}.

        No es \textit{biyectiva}, no tiene inversa.

  \item
        La función
        $$
          f: \enteros \to \naturales,\quad f(a) =
          \llave{lcll}{
            2a   & \text{si} & a > 0    & \orange{\to\text{genera los }\naturales_{pares}} \\
            1-2a & \text{si} & a \leq 0 & \orange{\to\text{genera los }\naturales_{impares}}
          }
        $$

        La función es \textit{inyectiva} y \textit{sobreyectiva}. Calculo la inversa:
        $$
          f^{-1}: \naturales \to \enteros,\quad
          \cajaResultado{
            f^{-1}(n) =
            \llave{cl}{
              \frac{n}{2}   & \text{si  $n$ es par}   \\
              \frac{1-n}{2} & \text{si  $n$ es impar}
            }
          }
        $$

\end{enumerate}

\begin{aportes}
  \item \aporte{\dirRepo}{naD GarRaz \github}
  \item \aporte{https://github.com/Nicolasmendez04}{Nico Méndez \github}
\end{aportes}
