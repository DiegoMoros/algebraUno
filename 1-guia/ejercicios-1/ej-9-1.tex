\begin{enunciado}{\ejercicio}
  Sean $A$ y $B$ conjuntos, Probar que $\partes(A) \subseteq \partes(B) \sisolosi A \subseteq B$
\end{enunciado}

Prueba que la hago por absurdo, \hyperlink{teoria-1:absurdo}{mirá la lógica en en apunte.}

\begin{itemize}
  \item[$\magenta{\entonces}$)] Quiero probar que:
        $$
          \ub{\partes(A) \subseteq \partes(B)}{\purple{\text{hipótesis}}}  \magenta{\entonces} \ub{A \subseteq B}{\text{tesis}}
        $$
        Pruebo por absurdo. \underline{Niego} la tesis, la \purple{hipótesis} sigue valiendo.\par

        Supongo que:
        $$
          A \nsubseteq B \Sii{def}[\red{!}] \existe x \en A \text{ tal que } x \taa{\llamada1}{}\notin B.
        $$
        Y lo que intento es llegar a una \red{contradicción}, es decir me gustaría que pase algo que \red{contradiga}
        la \purple{hipótesis}.\par
        Según mi supuesto:
        $$
          x \en A
          \Entonces{def de}[partes]
          \set{x} \en \partes(A).
        $$
        Peeeeeero\red{!!} por \purple{hipótesis}:
        $$
          \partes(A) \subseteq \partes(B)
        $$

        Entonces el conjunto $\set{x}$ también tiene que estar en $\partes(B)$.\par

        \textit{Nota que puede ser útil:}\par
        ¿Cuál es el absurdo? Terminá lo que falta de esta parte de la demostración sin ver
        como sigue y después comparás. \href{\justDoIt}{\faIcon{jedi}\faIcon{hand-sparkles}\faIcon{jedi}}
        Ya está casi terminado, pero juntar los cables con esta info te obliga a entender lo que se está intentando hacer.\par
        \textit{Fin nota que puede ser útil:}\par

        Si el conjunto de un elemento $\set{x}$ está en $\partes(B)$ entonces por la definición
        del conjunto de partes el elemento $x$ tiene que estar en $B$.\par
        Y esto es un absurdo, porque arranqué diciendo en $\llamada1$ que $x \notin B$ y ahora digo que $x \en B$. Absurdo $\skull$.\par
        Como mi supuesto resulto falso, debido a la lógica que está en las \hyperlink{teoria-1:absurdo}{notas teóricas sobre mostrar por absurdo} concluyo que:
        $$
          \boxed{\partes(A) \subseteq \partes(B) \entonces A \subseteq B}\Tilde
        $$

  \item[$\magenta{\Leftarrow}$)] Quiero probar que:
        $$
          \ub{A \subseteq B}{\text{\purple{hipótesis}}} \magenta{\entonces} \partes(A) \subseteq \partes(B)
        $$
        Le pongo nombre $S$ a los elementos de $\partes(A)$. Todo elemento $S \taa{\llamada2}{}\en \partes(A)$
        es un conjunto que cumple que $S \subseteq A$ por la definición del conjunto $\partes(A)$.
        Si todo elemento $S$ cumple que $S \subseteq A$ por \purple{hipótesis} también tiene que estar en $B$.

        \textit{Nota que puede ser útil:} click acá\par
        Terminá lo que falta de esta parte de la demostración sin ver como sigue y después comparás
        .\par
        \textit{Fin nota que puede ser útil:}\par\medskip

        $$
          S \en B \Entonces{def} S \en \partes(B).
        $$

        Entonces en $\llamada2$ dije que los $S$ forman al conjunto $\partes(A)$, y si todos los $S$ están en $\partes(B)$ entonces:

        $$
          \partes(A) \subseteq \partes(B)
        $$

        Queda demostrado que:\par
        $$
          \boxed{ A \subseteq B \entonces \partes(A) \subseteq \partes(B) } \Tilde
        $$
\end{itemize}

% Contribuciones
\begin{aportes}
  %\aporte{url}{nombre icono}
  \item \aporte{https://github.com/nad-garraz}{Nad Garraz \faIcon{github}}
\end{aportes}
