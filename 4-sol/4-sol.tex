\documentclass[12pt,a4paper, spanish]{article}
% Sacar draft para que aparezcan las imagenes.
% Opciones: 10pt, 11pt, landscape, twocolumn, fleqn, leqno...
% Opciones de clase: article, report, letter, beamer...

% Paquetes:
% =========
\usepackage[headheight=110pt, top = 2cm, bottom = 2cm, left=1cm, right=1cm]{geometry} %modifico márgenes
\usepackage[T1]{fontenc} % tildes
\usepackage[utf8]{inputenc} % Para poder escribir con tildes en el editor.
\usepackage[english]{babel} % Para cortar las palabras en silabas, creo.
\usepackage[ddmmyyyy]{datetime}
\usepackage{amsmath} % Soporte de mathmatics
\usepackage{amssymb} % fuentes de mathmatics
\usepackage{array} % Para tablas y eso
\usepackage{caption} % Configuracion de figuras y tablas
\usepackage[dvipsnames]{xcolor} % Para colorear el texto: black, blue, brown, cyan, darkgray, gray, green, lightgray, lime, magenta, olive, orange, pink, purple, red, teal, violet, white, yellow.
\usepackage{graphicx} % Necesario para poner imagenes
\usepackage{enumitem} % Cambiar labels y más flexibilidad para el enumerate
\usepackage{multicol} 
\usepackage{tikz} % para graficar
\usepackage{cancel}
\usepackage{titlesec} % para editar titulos y hacer secciones con formato a medida
\usepackage{ulem}
\usepackage{centernot} % tacha cosas
% \usepackage{lipsum}

% para hacer los graficos tipo grafos
\usetikzlibrary{shapes,arrows.meta, chains, matrix, calc, trees, positioning, fit}
\usetikzlibrary{external}

% Definiciones y nuevos comandos:
% =============
\def\partes{\mathcal P}
\def\relacion{\,\mathcal{R}\,}
\def\norelacion{\,\cancel{\relacion}\,}
\def\universo{\mathcal U}
\def\reales{\mathbb R}
\def\naturales{\mathbb N}
\def\enteros{\mathbb Z}
\def\complejos{\mathbb C}
\def\i{\text{i}}
\def\vacio{\varnothing}
\def\union{\cup}
\def\inter{\cap}
\def\y{\land}
\def\o{\lor}
\def\neg{\sim}
\def\entonces{\Rightarrow}
\def\sisolosi{\iff}
\def\clase{\overline}


\def\existe{\,\exists\,}
\def\noexiste{\,\nexists\,}
\def\paratodo{\forall}
\def\distinto{\neq}
\def\en{\in}
\def\talque{\;|\;}

% =====
\def\qvq{\text{ quiero ver que }}

%funciones
\def\imagen{\text{Im}}
\def\dominio{$\text{Dom}$}
\def\comp{\circ}
\def\inv{^{-1}}
\def\infinito{\infty}

% Llaves, paréntesis, contenedores
\newcommand{\llave}[2]{ \left\{ \begin{array}{#1} #2 \end{array}\right. }
\newcommand{\llaves}[2]{ \left\{ \begin{array}{#1} #2 \end{array} \right\} }
\newcommand{\matriz}[2]{\left( \begin{array}{#1} #2 \end{array} \right)}
\newcommand{\deter}[2]{\left| \begin{array}{#1} #2 \end{array} \right|}
\newcommand{\lista}[2][(1)]{\begin{enumerate}[\bf #1]\setlength\itemsep{-0.6ex} #2 \end{enumerate}}
\newcommand{\listal}[2][-0.6ex]{\begin{enumerate}[\bf(a)]\setlength\itemsep{#1} #2 \end{enumerate}}

% naturales
\newcommand{\sumatoria}[2]{\sum\limits_{#1}^{#2}}
\newcommand{\productoria}[2]{\prod\limits_{#1}^{#2}}
\newcommand{\kmasuno}[1]{\underbrace{#1}_{k+1\text{-ésimo}}}
\newcommand{\HI}[1]{\underbrace{#1}_{\text{HI}}}

% enteros
\def\divide{\,|\,}
\def\congruente{\, \equiv \,}
\newcommand{\congruencia}[3]{#1 \equiv #2 \;(\text{mod}\;#3)}
\newcommand{\divset}[2]{\mathcal{D}(#1) = \set{#2}}



% =====
% Miscelanea
% =====
\newcommand{\estabien}{{\color{blue} Consultado, está bien. \checkmark}}
\newcommand{\hacer}{{\color{black!30!red}Hacer!}}
\newcommand{\Hacer}{{\color{black!30!red}\Large Hacer!}}

\def\llamadaI{\stackrel{\cyan{$*^1$}}}
\def\llamadaII{\stackrel{\cyan{$*^2$}}}
\def\llamadaIII{\stackrel{\cyan{$*^3$}}}

% separador
\def\separador{\noindent\rule{\linewidth}{0.4pt}\\}
\def\separadorCorto{\noindent\rule{0.5\linewidth}{0.4pt}\\}

% sección ejercicio con su respectivo formato y contador
\newcounter{ejercicio}[subsubsection] % contador que se resetea en cada sección
\renewcommand{\theejercicio}{\arabic{ejercicio}} % el contador es un número arabic
\newcommand{\ejercicio}{%
	\stepcounter{ejercicio}% incremento en uno
	\titleformat{\section}[runin]{\normalfont\bfseries}{\theejercicio}{1em}{}%
	\section*{\noindent\theejercicio. \noindent}%
}

% Colores
\newcommand{\red}[1]{ {\color{red} \text{#1}}}
\newcommand{\green}[1]{ {\color{olive} \text{#1}}}
\newcommand{\blue}[1]{ {\color{blue} \text{#1}}}
\newcommand{\cyan}[1]{ {\color{cyan} \text{#1}}}
\newcommand{\magenta}[1]{ {\color{magenta} \text{#1}}}

% Conjuntos entre llaves
\newcommand{\set}[1] { \left\{ #1 \right\} }
\newcommand{\parentesis}[1] { \left( #1 \right) }

% Stackrel text
\newcommand{\stacktext}[2]{ \stackrel{\text{#1}}{#2} }
\def\eq?{\stackrel{\text{?}}}

% Flecha con texto
\NewDocumentCommand{\flecha}{m o}{%
	\IfNoValueTF{#2}{%
		\xrightarrow[]{\text{#1}}
	}{
		\xrightarrow[\text{#2}]{\text{#1}}
	}
}
 % idem con las definiciones

\begin{document}

\pagestyle{empty} % Para que no muestre el número en pie de página

% Info para armar título.
\title{Práctica 4 de álgebra 1} % título
\author{D. Garraz} % autor
\date{last update: \today} % Cambiar de ser necesario

\maketitle  % para que aprezca el título en el documento
\section*{Definiciones y fórmulas útiles}

\begin{itemize}
	\item $d$ divide a $a \to d \divide a \sisolosi \existe k \in \enteros : a = k \cdot d$
	\item $ \divset{-a}{-|a|,\dots,-1,1,\dots,|a|}$.
	\item $d \divide 0 $, dado que $0 = 0\cdot d$. Se desprende que $\divset{0}{\enteros - \set{0}}$
	\item $\llave{l}{
			      d \divide a \sisolosi -d \divide a \text{ (pues }a = k \cdot d \sisolosi a = (-k) \cdot (-d))\\
			      d \divide a \sisolosi d \divide -a \text{ (pues} a = k \cdot d \sisolosi (-a) = (-k) \cdot d)\\
			      \entonces d \divide a \sisolosi |d| \,\divide\, |a|
		      }$

	\item $\llave{l}{
			      d \divide a \text{ y } d \divide b \entonces d \divide a + b\\
			      d \divide a \text{ y } d \divide b \entonces d \divide a - b\\
			      d \divide a \entonces d \divide c \cdot a,\, \paratodo c \en \enteros\\
			      d \divide a \entonces d \divide c \cdot a\\
			      d \divide a \entonces d^2 \divide a^2 \text{  y  } d^n \divide a^n \, \paratodo n \en \naturales\\
			      d \divide a \cdot b \text{ no implica } d \divide a \o d \divide b. \text{ Por ejemplo } 6 \divide 3 \cdot 4
		      }$

	\item
	      $\llave{l}{
			      \textit{$a$ es congruente a $b$ módulo $d$} \text{ si }   d \divide a-b \text{. Se nota } \congruencia{a}{b}{d}\\
			      \congruencia{a}{b}{d} \sisolosi d \divide a-b
		      }$

	\item $
		      \llave{c}{
			      \congruencia{a_1}{b_1}{d}\\
			      \vdots\\
			      \congruencia{a_n}{b_n}{d}
		      }
		      \entonces \congruencia{a_1 + \cdots + a_n}{a_b + \cdots + b_n}{d}
	      $.
	\item $
		      \llave{c}{
			      \congruencia{a_1}{b_1}{d}\\
			      \vdots\\
			      \congruencia{a_n}{b_n}{d}
		      }
		      \entonces \congruencia{a_1 \cdots a_n}{a_b \cdots b_n}{d} \flecha{$a_i = a \y b_i = b$}[$\paratodo i \en \set{1,\dots, n}$] \congruencia{a^n}{b^n}{d}$
\end{itemize}
\textit{\underline{Algoritmo de división:}}\\
Dados $a,d\en \enteros$ con $d \distinto 0$, \textit{\underline{existen}} $k$, $r \en \enteros$ tales que: $a =  k \cdot d + r$, con $0\leq r < |d|$.
Y además estos $k$ y $r$ son \textit{\underline{únicos}}.\\

\noindent Notación: $r_d(a)$ es el resto de dividir a $a$ entre $d$
%
%
%
\section*{Ejercicios dados en clase}
% \ejercicio


\section*{Ejercicios de la guía}
\textit{\underline{Divisibilidad}}
%1
\ejercicio
Decidir si las siguientes afirmaciones son verdaderas $\paratodo a,\, b,\, c \en \enteros$:\\
Calcular
\begin{enumerate}[label=\roman*)]
	\item $a \cdot b \divide c \entonces a \divide c$ \ y \ $b \divide c$ \\
	      \separadorCorto

	      $\llave{l}{
			      c = k\cdot a \cdot b = \underbrace{h}_{k \cdot b} \cdot a \entonces a \divide c \Tilde\\
			      c = k\cdot a \cdot b = \underbrace{i}_{k \cdot a} \cdot b \entonces b \divide c\Tilde
		      }$

	\item $4 \divide a^2 \entonces 2 \divide a $\\
	      \separadorCorto

	      $ a^2 = k \cdot 4 = \underbrace{h}_{k \cdot 2} \cdot 2 \entonces a^2 \divide 2
		      \flecha{si $a\cdot b \divide c$}[$\entonces a \divide c \y b \divide c$]
		      a \divide 2 \Tilde$

	\item $2 \divide a \cdot b \entonces 2 \divide a $ \ o \ $2 \divide b$\\
	      \separadorCorto

	      Si $2 \divide a \cdot b \entonces
		      \llaves{c}{
			      a \text{ tiene que ser } par \\
			      \o \\
			      b \text{ tiene que ser } par \\
		      } \flecha{para que} a \cdot b$ sea par. Por lo tanto si  $2 \divide a \cdot b \entonces 2 \divide a $ \ o \ $2 \divide b$.

	\item $9 \divide a\cdot b \entonces 9 \divide a  $ \ o \ $9 \divide b$\\
	      \separadorCorto

	      Si $a = 3 \y b = 3$, se tiene que $9 \divide 9$, sin embargo $9 \noDivide 3$

	\item $a \divide b + c \entonces a \divide b $ \ o \  $a \divide c$\\
	      \separadorCorto

	      $12 \divide 20 + 4 \entonces 12 \noDivide 20$  \ y\   $ 12\noDivide 4 $

	\item
	      \separadorCorto
	      \hacer
	\item
	      \separadorCorto
	      \hacer
	\item
	      \separadorCorto
	      \hacer
	\item $a \divide b + a^2 \entonces a \divide b$
	      \separadorCorto

	      $  a \divide b + a^2 \entonces b + a^2 = k \cdot a \flecha{acomodo} b = (k - a) \cdot a = h \cdot a \entonces a | b \Tilde$\\
	      $ \flecha{también puedo}[decir si:]
		      \llaves{l}{
			      a \divide a^2 \\
			      a \divide b - a^2
		      } \flecha{por}[propiedad] a \divide (b - a^2) + (a^2) = b \entonces a \divide b \Tilde $


	\item $a \divide b \entonces a^n \divide b^n,\, \paratodo n \en \naturales$
	      \separadorCorto

	      Pruebo por inducción. $p(n)\  :\  a \divide b \entonces a^n \divide b^n $\\
	      $
		      \llave{l}{
			      \text{Caso base: } n = 1 \entonces a|b \entonces a^1 \divide b^1 \checkmark\\
			      \text{Paso inductivo: } \paratodo h \en \naturales, p(h) \ V \entonces p(h+1)\ V?\\
			      \text{Si } a \divide b \entonces a^k \divide b^k  \entonces a^k \cdot c = b^k
			      \flecha{multiplico por}[$b$ M.A.M]
			      b \cdot a^k \cdot c = b^{k+1}
			      \flecha{$a\divide b$}[$a \cdot d = b$]
			      a\cdot d \cdot a^k \cdot c =  a^{k+1}\cdot(cd) = b^{k+1}\\
			      \flecha{concluyendo}[que] a^{k+1} \divide b^{k+1} \text{como quería mostrarse.}
		      }
	      $ Como $p(1) \y p(k) \y p(k+1)$ resultaron verdaderas, por el principio de inducción $p(n)$
	      es verdadera $\paratodo n \en \naturales$

\end{enumerate}

\ejercicio
Hallar todos los $n \en \naturales$ tales que:
\begin{enumerate}[label=\roman*)]
	\item $3n - 1 \divide n+7$\\
	      \separadorCorto

	      Busco eliminar la $n$ del \textit{miembro} derecho.\\
	      $\llaves{l}{
			      3n - 1 \divide n + 7
			      \flecha{$a|c \entonces$}[$a \divide k\cdot c$]
			      3n - 1 \divide \red{3} \cdot (n+7) = 3n + 21\\
			      \flecha{$a \divide b \y a \divide c$}[$\entonces a \divide b \pm c$]
			      3n-1 \divide 3n + 21 - (3n -1) = 22
		      } \to 3n - 1 \divide 22\\
		      \flecha{busco $n$}[para que] \frac{22}{3n-1} \en \divset{22}{1\pm1, \pm2, \pm11, \pm22} \flecha{probando} n \en \set{1,4} \Tilde$
	\item
	\item

	\item $n-2 \divide n^3 - 8$

	      \separadorCorto

	      $n-2 \divide n^3 - 8 = (n-2) \cdot (n^2 + 2n + 4)$ Esto va a dividir para todo $n \distinto 2$\\
	      \red{corroborar. ¿Otra forma de llegar a eso?}

\end{enumerate}

%3
\ejercicio Sean $a.\,b \en \enteros$.
\begin{enumerate}[label=\roman*)]
	\item Probar que $a-b \divide a^n - b^n$ para todo $n \en \naturales \y a \distinto b \en \enteros$
	\item Probar que si $n$ es un número natural par y $a \distinto -b$, entonces $a+b \divide a^n - b^n$.
	\item Probar que si $n$ es un número natural impar y $a \distinto -b$, entonces  $a+b \divide a^n + b^n$.
\end{enumerate}
\separadorCorto

\begin{enumerate}[label=\roman*)]
	\item  Si $a-b \divide a^n - b^n \stackrel{def}\sisolosi \congruencia{a^n}{b^n}{a-b} \sisolosi
		      \llaves{l}{
		      \congruencia{a}{b}{a-b} \\
		      \congruencia{a^2}{\underbrace{a}_{\stackrel{(a-b)}\congruente b}\cdot b}{a-b} \to \congruencia{a^2}{b^2}{a-b} \\
		      \quad\vdots\\
		      \congruencia{a^n}{b^n}{a-b} \\
		      } \to\\ a-b \divide a^n - b^n$

	\item  Sé que $\congruencia{a}{-b}{a+b} \sisolosi
		      \llaves{l}{
		      \congruencia{a^2}{\underbrace{a}_{\stackrel{(a+b)}\congruente -b}\cdot b}{a+b} \flecha{propiedad}[congruencia] \congruencia{a^2}{(-1)^2 \cdot b^2}{a+b} \\
		      \quad\vdots\\
		      \congruencia{a^n}{(-1)^n \cdot b^n}{a+b} \to
		      \llave{ll}{
			      \congruencia{a^n}{ b^n}{a+b} & \text{con n par}  \\
			      \congruencia{a^n}{(-1)^n \cdot b^n}{a+b}   & \text{con n impar} \\
		      }\\
		      }$\\
	      \red{Vale esto com prueba de algo?}

	\item Sería lo mismo que en el item anterior.


	      %4
	      \ejercicio
	      Sea $a \in \enteros$ impar. Probar que $2^{n+2} \divide a^{2^n} - 1$ para todo $n \en \naturales$\\
	      \separadorCorto

	      Pruebo por inducción: $p(n): 2^{n+2} \divide a^{2^n} - 1$\\
	      $\llaves{l}{
			      \textit{Caso bade: } p(1)\ :\ 2^3 \divide a^2 - 1 = (a - 1) \cdot (a + 1)
			      \flecha{$a$ es eimpar}[$a = 2m -1$] (2m - 2)\cdot(2m) =\\
			      4 \cdot \ub{m \cdot (m-1)}{par} = 4 \cdot (2\cdot h) = 8 * h \flecha{por lo}[tanto] 8 \divide 8 \cdot h \text{ con $h$ entero.}\checkmark\\

			      \textit{Paso inductivo: } p(k) \ V  \entonces p(k+1) \ V? \flecha{es}[decir]
			      2^{k+2} \divide a^{2^k} - 1 \entonces  2^{k+3} \divide a^{2^{k+1}} - 1\ V? \\

			      \textit{Hipótesis inductiva: }\\
			      \llave{l}{
				      2^{k+3} \divide a^{2^{k+1}} - 1 \flecha{acomodar}[diferencia cuadrados] 2\cdot 2^k \divide (a^{2^k})^2 - 1 =\\
				      \ub{ (a^{2^k} - 1) }{\text{par}} \cdot  \ub{(a^{2^k} + 1)}{\text{par}}  \\
				      \red{Necesito probar eso?$\to$}\flecha{Si $a \stackrel{\tiny HI}\divide b$ y $c \divide d$}[$\entonces a\cdot c \divide b\cdot d$]
				      \ub{2^{k+2}}{a} \cdot \ub{2}{c}   \divide  \ub{ (a^{2^k} - 1)}{b} \cdot \ub{(a^{2^k} + 1)}{d} \entonces 2^{k+3} \divide a^{2^{k+1}} - 1 \quad V \checkmark
			      }
		      } $\\
	      Como $p(1) \y p(k) \y p(k+1)$ resultaron verdaderas, por el principio de inducción $p(n)$
	      es verdadera $\paratodo n \en \naturales$

	      %5
	      \ejercicio
	      \separadorCorto
	      %6
	      \ejercicio
	      \separadorCorto


	      %7
	      \ejercicio
	      \begin{enumerate}[label=\roman*)]
		      \item $99 \divide 10^{2n} + 197$
		      \item $9 \divide 7 \cdot 5^{2n} + 2^{4n+1}$
		      \item $56 \divide 13^{2n} + 28n^2 - 84n -1$
		      \item $256 \divide 7^{2n} + 208n - 1$
	      \end{enumerate}

	      \separadorCorto
	      % definicion local
	      \def\cong99{\stackrel{(99)}\congruente}
	      % fin definicion local
	      \begin{enumerate}[label=\roman*)]
		      \item $99 \divide 10^{2n} + 197 \stackrel{def}{\sisolosi}
			            \congruencia{10^{2n} + 197}{0}{99} \to
			            \congruencia{10^{2n} + 198}{1}{99} \to
			            \congruencia{10^{2n} + \ub{198}{\cong99 0}}{1}{99} \to \congruencia{100^n}{1}{99}\to \\
			            \llave{l}{
				            \flecha{sé}[que] \congruencia{100}{1}{99} \sisolosi \congruencia{100^2}{\ub{100}{ \cong99 1 }}{99} \to
				            \congruencia{100^2}{1}{99} \sisolosi \dots \sisolosi \congruencia{100^n}{1}{99}\\
				            \red{¿Tengo que demostrar ese renglón por inducción o con "propiedad de congruencia" funciona?}
			            }
		            $\\
		            Se concluye que  $99 \divide 10^{2n} + 197 \sisolosi 99 \divide \ub{100 - 1}{99}$


		      \item
		            % definicion local
		            \def\cong9{\stackrel{(9)}\congruente}
		            % fin definicion local

		            $9 \divide 7 \cdot 5^{2n} + 2^{4n+1}
			            \stackrel{def}{\sisolosi}
			            \congruencia{7\cdot5^{2n} + 2^{4n+1}}{0}{9}
			            \flecha{sumo $2\cdot 5^{2n}$}[M.A.M]
			            \congruencia{\ub{9 \cdot 5^{2n}}{\cong9 0} + 2\cdot 2^{4n}}{2 \cdot 5^{2n}}{9}\\
			            \flecha{simplifico}[y acomodo]
			            \congruencia{2^{4n}}{5^{2n}}{9} \to
			            \congruencia{16^n}{25^n}{9}
			            \flecha{simetría}[congruencia]
			            \congruencia{25^n}{16^n}{9}
			            \flecha{$25\cong9 16$}
			            \congruencia{25}{16}{9} =
			            \congruencia{9}{0}{9}\\
		            $
		            Se concluye que $9 \divide 7 \cdot 5^{2n} + 2^{4n+1} \sisolosi 9 \divide 9$ \red{$\leftarrow$ ¿Se concluye esto...?}

		      \item \hacer
		      \item \hacer
	      \end{enumerate}

	      \textit{\underline{Algoritmo de División}}:\\
	      %8
	      \ejercicio

	      % definiciones locales
	      \begin{enumerate}[label=\roman*)]
		      \begin{minipage}{0.5\linewidth}
			      \item $a =133,\quad b = -14.$
			      \item $a =13,\quad b = 111.$
			      \item $a =3b+7,\quad b \distinto 0.$
		      \end{minipage}
		      \begin{minipage}{0.5\linewidth}
			      \item $a = b^2 - 6,\quad b \distinto 0.$
			      \item $a = n^2 + 5,\quad b = n+2 \ (n \en \naturales).$
			      \item $a = n + 3,\quad = n^2 + 1 \ (n \en \naturales).$
		      \end{minipage}
	      \end{enumerate}
	      \separadorCorto
	      \begin{enumerate}[label=\roman*)]
		      \item $133 : (-14) \entonces 133 = \blue{(-9)} \cdot (-14) + \magenta{7}  $
		      \item $ $
		      \item $a = 3b + 7 \to
			            \text{me interesa: }\to
			            \llaves{l}{
				            |b| \leq |a| \checkmark\\
				            0 \leq \magenta{r} < |b| \checkmark \\
			            } \to\\ $


                  $ \to\llave{l}{
				            \text{Si: } |b| > 7 \to (\blue{$q$},\magenta{$r$}) = (3,7)\\
				            \text{Si: } |b| \leq 7 \to (\blue{$q$},\magenta{$r$}) = (3,7)\\
                    \footnotesize
				            \begin{array}{|l|l|l|l|l|l|l|l|}
					            \hline
                      (a,b) & (-14,-7) & (-11,-6) & (-8,-5) & (-5,-4) & (4, -1) & \dots \\ \hline
                      (\blue{$q$},\magenta{$r$}) & (2,0) & (2,1) & (2,2) & (2,3) & (4, 0) & \dots \\ \hline
                      \hline
					            % (\blue{q},\magenta{r})&
				            \end{array}
                  }
            $
	      \end{enumerate}
	\item


	      %9
	      \ejercicio
	      \separadorCorto

	      %10
	      \ejercicio
	      \separadorCorto


\end{enumerate}




\end{document}
