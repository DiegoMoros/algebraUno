\documentclass[12pt,a4paper, spanish]{article}
% Sacar draft para que aparezcan las imagenes.
% Opciones: 10pt, 11pt, landscape, twocolumn, fleqn, leqno...
% Opciones de clase: article, report, letter, beamer...

% Paquetes:
% =========
\usepackage[headheight=110pt, top = 2cm, bottom = 2cm, left=1cm, right=1cm]{geometry} %modifico márgenes
\usepackage[T1]{fontenc} % tildes
\usepackage[utf8]{inputenc} % Para poder escribir con tildes en el editor.
\usepackage[english]{babel} % Para cortar las palabras en silabas, creo.
\usepackage[ddmmyyyy]{datetime}
\usepackage{amsmath} % Soporte de mathmatics
\usepackage{amssymb} % fuentes de mathmatics
\usepackage{array} % Para tablas y eso
\usepackage{caption} % Configuracion de figuras y tablas
\usepackage[dvipsnames]{xcolor} % Para colorear el texto: black, blue, brown, cyan, darkgray, gray, green, lightgray, lime, magenta, olive, orange, pink, purple, red, teal, violet, white, yellow.
\usepackage{graphicx} % Necesario para poner imagenes
\usepackage{enumitem} % Cambiar labels y más flexibilidad para el enumerate
\usepackage{multicol} 
\usepackage{tikz} % para graficar
\usepackage{cancel}
\usepackage{titlesec} % para editar titulos y hacer secciones con formato a medida
\usepackage{ulem}
\usepackage{centernot} % tacha cosas
% \usepackage{lipsum}

% para hacer los graficos tipo grafos
\usetikzlibrary{shapes,arrows.meta, chains, matrix, calc, trees, positioning, fit}
\usetikzlibrary{external}

% Definiciones y nuevos comandos:
% =============
\def\partes{\mathcal P}
\def\relacion{\,\mathcal{R}\,}
\def\norelacion{\,\cancel{\relacion}\,}
\def\universo{\mathcal U}
\def\reales{\mathbb R}
\def\naturales{\mathbb N}
\def\enteros{\mathbb Z}
\def\complejos{\mathbb C}
\def\i{\text{i}}
\def\vacio{\varnothing}
\def\union{\cup}
\def\inter{\cap}
\def\y{\land}
\def\o{\lor}
\def\neg{\sim}
\def\entonces{\Rightarrow}
\def\sisolosi{\iff}
\def\clase{\overline}


\def\existe{\,\exists\,}
\def\noexiste{\,\nexists\,}
\def\paratodo{\forall}
\def\distinto{\neq}
\def\en{\in}
\def\talque{\;|\;}

% =====
\def\qvq{\text{ quiero ver que }}

%funciones
\def\imagen{\text{Im}}
\def\dominio{$\text{Dom}$}
\def\comp{\circ}
\def\inv{^{-1}}
\def\infinito{\infty}

% Llaves, paréntesis, contenedores
\newcommand{\llave}[2]{ \left\{ \begin{array}{#1} #2 \end{array}\right. }
\newcommand{\llaves}[2]{ \left\{ \begin{array}{#1} #2 \end{array} \right\} }
\newcommand{\matriz}[2]{\left( \begin{array}{#1} #2 \end{array} \right)}
\newcommand{\deter}[2]{\left| \begin{array}{#1} #2 \end{array} \right|}
\newcommand{\lista}[2][(1)]{\begin{enumerate}[\bf #1]\setlength\itemsep{-0.6ex} #2 \end{enumerate}}
\newcommand{\listal}[2][-0.6ex]{\begin{enumerate}[\bf(a)]\setlength\itemsep{#1} #2 \end{enumerate}}

% naturales
\newcommand{\sumatoria}[2]{\sum\limits_{#1}^{#2}}
\newcommand{\productoria}[2]{\prod\limits_{#1}^{#2}}
\newcommand{\kmasuno}[1]{\underbrace{#1}_{k+1\text{-ésimo}}}
\newcommand{\HI}[1]{\underbrace{#1}_{\text{HI}}}

% enteros
\def\divide{\,|\,}
\def\congruente{\, \equiv \,}
\newcommand{\congruencia}[3]{#1 \equiv #2 \;(\text{mod}\;#3)}
\newcommand{\divset}[2]{\mathcal{D}(#1) = \set{#2}}



% =====
% Miscelanea
% =====
\newcommand{\estabien}{{\color{blue} Consultado, está bien. \checkmark}}
\newcommand{\hacer}{{\color{black!30!red}Hacer!}}
\newcommand{\Hacer}{{\color{black!30!red}\Large Hacer!}}

\def\llamadaI{\stackrel{\cyan{$*^1$}}}
\def\llamadaII{\stackrel{\cyan{$*^2$}}}
\def\llamadaIII{\stackrel{\cyan{$*^3$}}}

% separador
\def\separador{\noindent\rule{\linewidth}{0.4pt}\\}
\def\separadorCorto{\noindent\rule{0.5\linewidth}{0.4pt}\\}

% sección ejercicio con su respectivo formato y contador
\newcounter{ejercicio}[subsubsection] % contador que se resetea en cada sección
\renewcommand{\theejercicio}{\arabic{ejercicio}} % el contador es un número arabic
\newcommand{\ejercicio}{%
	\stepcounter{ejercicio}% incremento en uno
	\titleformat{\section}[runin]{\normalfont\bfseries}{\theejercicio}{1em}{}%
	\section*{\noindent\theejercicio. \noindent}%
}

% Colores
\newcommand{\red}[1]{ {\color{red} \text{#1}}}
\newcommand{\green}[1]{ {\color{olive} \text{#1}}}
\newcommand{\blue}[1]{ {\color{blue} \text{#1}}}
\newcommand{\cyan}[1]{ {\color{cyan} \text{#1}}}
\newcommand{\magenta}[1]{ {\color{magenta} \text{#1}}}

% Conjuntos entre llaves
\newcommand{\set}[1] { \left\{ #1 \right\} }
\newcommand{\parentesis}[1] { \left( #1 \right) }

% Stackrel text
\newcommand{\stacktext}[2]{ \stackrel{\text{#1}}{#2} }
\def\eq?{\stackrel{\text{?}}}

% Flecha con texto
\NewDocumentCommand{\flecha}{m o}{%
	\IfNoValueTF{#2}{%
		\xrightarrow[]{\text{#1}}
	}{
		\xrightarrow[\text{#2}]{\text{#1}}
	}
}
 % idem con las definiciones

\begin{document}

\pagestyle{empty} % Para que no muestre el número en pie de página

% Info para armar título.
\title{Práctica 4 de álgebra 1} % título
\author{D. Garraz} % autor
\date{last update: \today} % Cambiar de ser necesario

% \maketitle  % Para que aprezca el título en el documento
\section*{Definiciones y fórmulas útiles}

\begin{itemize}
	\item $d$ divide a $a \to d \divide a \sisolosi \existe k \in \enteros : a = k \cdot d$
	\item $ \divset{-a}{-|a|,\dots,-1,1,\dots,|a|}$.
	\item $d \divide 0 $, dado que $0 = 0\cdot d$. Se desprende que $\divset{0}{\enteros - \set{0}}$
	\item $\llave{l}{
			      d \divide a \sisolosi -d \divide a \text{ (pues }a = k \cdot d \sisolosi a = (-k) \cdot (-d))\\
			      d \divide a \sisolosi d \divide -a \text{ (pues} a = k \cdot d \sisolosi (-a) = (-k) \cdot d)\\
			      \entonces d \divide a \sisolosi |d| \,\divide\, |a|
		      }$

	\item $\llave{l}{
			      d \divide a \text{ y } d \divide b \entonces d \divide a + b\\
			      d \divide a \text{ y } d \divide b \entonces d \divide a - b\\
			      d \divide a \entonces d \divide c \cdot a,\, \paratodo c \en \enteros\\
			      d \divide a \entonces d \divide c \cdot a\\
			      d \divide a \entonces d^2 \divide a^2 \text{  y  } d^n \divide a^n \, \paratodo n \en \naturales\\
			      d \divide a \cdot b \text{ no implica } d \divide a \o d \divide b. \text{ Por ejemplo } 6 \divide 3 \cdot 4
		      }$

	\item
	      $\llave{l}{
			      \textit{$a$ es congruente a $b$ módulo $d$} \text{ si }   d \divide a-b \text{. Se nota } \congruencia{a}{b}{d}\\
			      \congruencia{a}{b}{d} \sisolosi d \divide a-b
		      }$

	\item $
		      \llave{c}{
			      \congruencia{a_1}{b_1}{d}\\
			      \vdots\\
			      \congruencia{a_n}{b_n}{d}
		      }
		      \entonces \congruencia{a_1 + \cdots + a_n}{a_b + \cdots + b_n}{d}
	      $.
	\item $
		      \llave{c}{
			      \congruencia{a_1}{b_1}{d}\\
			      \vdots\\
			      \congruencia{a_n}{b_n}{d}
		      }
		      \entonces \congruencia{a_1 \cdots a_n}{a_b \cdots b_n}{d} \flecha{$a_i = a \y b_i = b$}[$\paratodo i \en \set{1,\dots, n}$] \congruencia{a^n}{b^n}{d}$
\end{itemize}

\section*{Ejercicios dados en clase}
\ejercicio


\section{Ejercicios de la guía}

\textit{\underline{Divisibilidad}}
%1
\ejercicio
Decidir si las siguientes afirmaciones son verdaderas $\paratodo a,\, b,\, c \en \enteros$:\\
Calcular
\begin{enumerate}[label=\roman*)]
	\item $a \cdot b \divide c \entonces a \divide c$ \ y \ $b \divide c$ \\
	      \separadorCorto

	      $\llave{l}{
			      c = k\cdot a \cdot b = \underbrace{h}_{k \cdot b} \cdot a \entonces a \divide c \Tilde\\
			      c = k\cdot a \cdot b = \underbrace{i}_{k \cdot a} \cdot b \entonces b \divide c\Tilde
		      }$

	\item $4 \divide a^2 \entonces 2 \divide a $\\
	      \separadorCorto

	      $ a^2 = k \cdot 4 = \underbrace{h}_{k \cdot 2} \cdot 2 \entonces a^2 \divide 2
		      \flecha{si $a\cdot b \divide c$}[$\entonces a \divide c \y b \divide c$]
		      a \divide 2 \Tilde$

	\item $2 \divide a \cdot b \entonces 2 \divide a $ \ o \ $2 \divide b$
	      \separadorCorto
	      \hacer

	\item $9 \divide a\cdot b \entonces 9 \divide a  $ \ o \ $9 \divide b$
	      \separadorCorto
	      \hacer

	\item $a \divide b + c \entonces a \divide b $ \ o \  $9 \divide b$
	      \separadorCorto
	      $12 \divide 20 + 4 \entonces 12 \noDivide 20$  \ y\   $ 12\noDivide 4 $

	\item
	      \separadorCorto
	      \hacer
	\item
	      \separadorCorto
	      \hacer
	\item
	      \separadorCorto
	      \hacer
	\item $a \divide b + a^2 \entonces a \divide b$
	      \separadorCorto

	      $  a \divide b + a^2 \entonces b + a^2 = k \cdot a \flecha{acomodo} b = (k - a) \cdot a = h \cdot a \entonces a | b \Tilde$\\
	      $ \flecha{también puedo}[decir si:]
		      \llaves{l}{
			      a \divide a^2 \\
			      a \divide b - a^2
		      } \flecha{por}[propiedad] a \divide (b - a^2) + (a^2) = b \entonces a \divide b \Tilde $


	\item
	      \separadorCorto
	      \hacer
\end{enumerate}

\ejercicio
Hallar todos los $n \en \naturales$ tales que:
	\begin{enumerate}[label=\roman*)]
		\item $3n - 1 \divide n+7$\\
      \separadorCorto

      Busco eliminar la $n$ del \textit{miembro} derecho.\\
      $\llaves{l}{
        3n - 1 \divide n + 7
        \flecha{$a|c \entonces$}[$a \divide k\cdot c$]
        3n - 1 \divide \red{3} \cdot (n+7) = 3n + 21\\
        \flecha{$a \divide b \y a \divide c$}[$\entonces a \divide b \pm c$]
        3n-1 \divide 3n + 21 - (3n -1) = 22
      } \to 3n - 1 \divide 22\\
      \flecha{busco $n$}[para que] \frac{22}{3n-1} \en \divset{22}{1\pm1, \pm2, \pm11, \pm22} \flecha{probando} n \en \set{1,4} \Tilde$
    \item 
    \item

    \item $n-2 \divide n^3 - 8$

      \separadorCorto

      $n-2 \divide n^3 - 8 = (n-2) \cdot (n^2 + 2n + 4)$ Esto va a dividir para todo $n \distinto 2$\\
      \red{corroborar. ¿Otra forma de llegar a eso?} 


	\end{enumerate}

\end{document}
