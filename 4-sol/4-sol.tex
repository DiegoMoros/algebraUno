\documentclass[12pt,a4paper, spanish]{article}
% Sacar draft para que aparezcan las imagenes.
% Opciones: 12pt, 10pt, 11pt, landscape, twocolumn, fleqn, leqno...
% Opciones de clase: article, report, letter, beamer...

% Paquetes:
% =========
\usepackage[headheight=110pt, top = 2cm, bottom = 2cm, left=1cm, right=1cm]{geometry} %modifico márgenes
\usepackage[T1]{fontenc} % tildes
\usepackage[utf8]{inputenc} % Para poder escribir con tildes en el editor.
\usepackage[english]{babel} % Para cortar las palabras en silabas, creo.
\usepackage[ddmmyyyy]{datetime}
\usepackage{amsmath} % Soporte de mathmatics
\usepackage{amssymb} % fuentes de mathmatics
\usepackage{array} % Para tablas y eso
\usepackage{caption} % Configuracion de figuras y tablas
\usepackage[dvipsnames]{xcolor} % Para colorear el texto: black, blue, brown, cyan, darkgray, gray, green, lightgray, lime, magenta, olive, orange, pink, purple, red, teal, violet, white, yellow.
\usepackage{graphicx} % Necesario para poner imagenes
\usepackage{enumitem} % Cambiar labels y más flexibilidad para el enumerate
\usepackage{multicol} 
\usepackage{tikz} % para graficar
\usepackage{cancel}
\usepackage{titlesec} % para editar titulos y hacer secciones con formato a medida
\usepackage{ulem}
\usepackage{centernot} % tacha cosas
\usepackage{bbding} % símbolos de donde uso FiveStar
% \usepackage{lipsum}

% para hacer los graficos tipo grafos
\usetikzlibrary{shapes,arrows.meta, chains, matrix, calc, trees, positioning, fit}
\usetikzlibrary{external}

% Definiciones y nuevos comandos:
% =============
\def\partes{\mathcal P}
\def\relacion{\,\mathcal{R}\,}
\def\norelacion{\,\cancel{\relacion}\,}
\def\universo{\mathcal U}
\def\reales{\mathbb R}
\def\naturales{\mathbb N}
\def\enteros{\mathbb Z}
\def\complejos{\mathbb C}
\def\i{\text{i}}
\def\vacio{\varnothing}
\def\union{\cup}
\def\inter{\cap}
\def\y{\land}
\def\o{\lor}
\def\neg{\sim}
\def\entonces{\Rightarrow}
\def\sisolosi{\iff}
\def\clase{\overline}


\def\existe{\,\exists\,}
\def\noexiste{\,\nexists\,}
\def\paratodo{\forall}
\def\distinto{\neq}
\def\en{\in}
\def\talque{\;|\;}

% =====
\def\qvq{\text{ quiero ver que }}

%funciones
\def\imagen{\text{Im}}
\def\dominio{$\text{Dom}$}
\def\comp{\circ}
\def\inv{^{-1}}
\def\infinito{\infty}

% Llaves, paréntesis, contenedores
\newcommand{\llave}[2]{ \left\{ \begin{array}{#1} #2 \end{array}\right. }
\newcommand{\llaves}[2]{ \left\{ \begin{array}{#1} #2 \end{array} \right\} }
\newcommand{\matriz}[2]{\left( \begin{array}{#1} #2 \end{array} \right)}
\newcommand{\deter}[2]{\left| \begin{array}{#1} #2 \end{array} \right|}
\newcommand{\lista}[2][(1)]{\begin{enumerate}[\bf #1]\setlength\itemsep{-0.6ex} #2 \end{enumerate}}
\newcommand{\listal}[2][-0.6ex]{\begin{enumerate}[\bf(a)]\setlength\itemsep{#1} #2 \end{enumerate}}

% naturales
\newcommand{\sumatoria}[2]{\sum\limits_{#1}^{#2}}
\newcommand{\productoria}[2]{\prod\limits_{#1}^{#2}}
\newcommand{\kmasuno}[1]{\underbrace{#1}_{k+1\text{-ésimo}}}
\newcommand{\HI}[1]{\underbrace{#1}_{\text{HI}}}

% enteros
\def\divide{\,|\,}
\def\congruente{\, \equiv \,}
\newcommand{\congruencia}[3]{#1 \equiv #2 \;(\text{mod}\;#3)}
\newcommand{\divset}[2]{\mathcal{D}(#1) = \set{#2}}



% =====
% Miscelanea
% =====
\newcommand{\estabien}{{\color{blue} Consultado, está bien. \checkmark}}
\newcommand{\hacer}{{\color{black!30!red}Hacer!}}
\newcommand{\Hacer}{{\color{black!30!red}\Large Hacer!}}

\def\llamadaI{\stackrel{\cyan{$*^1$}}}
\def\llamadaII{\stackrel{\cyan{$*^2$}}}
\def\llamadaIII{\stackrel{\cyan{$*^3$}}}

% separador
\def\separador{\noindent\rule{\linewidth}{0.4pt}\\}
\def\separadorCorto{\noindent\rule{0.5\linewidth}{0.4pt}\\}

% sección ejercicio con su respectivo formato y contador
\newcounter{ejercicio}[subsubsection] % contador que se resetea en cada sección
\renewcommand{\theejercicio}{\arabic{ejercicio}} % el contador es un número arabic
\newcommand{\ejercicio}{%
	\stepcounter{ejercicio}% incremento en uno
	\titleformat{\section}[runin]{\normalfont\bfseries}{\theejercicio}{1em}{}%
	\section*{\noindent\theejercicio. \noindent}%
}

% Colores
\newcommand{\red}[1]{ {\color{red} \text{#1}}}
\newcommand{\green}[1]{ {\color{olive} \text{#1}}}
\newcommand{\blue}[1]{ {\color{blue} \text{#1}}}
\newcommand{\cyan}[1]{ {\color{cyan} \text{#1}}}
\newcommand{\magenta}[1]{ {\color{magenta} \text{#1}}}

% Conjuntos entre llaves
\newcommand{\set}[1] { \left\{ #1 \right\} }
\newcommand{\parentesis}[1] { \left( #1 \right) }

% Stackrel text
\newcommand{\stacktext}[2]{ \stackrel{\text{#1}}{#2} }
\def\eq?{\stackrel{\text{?}}}

% Flecha con texto
\NewDocumentCommand{\flecha}{m o}{%
	\IfNoValueTF{#2}{%
		\xrightarrow[]{\text{#1}}
	}{
		\xrightarrow[\text{#2}]{\text{#1}}
	}
}
 % idem con las definiciones

\begin{document}

\pagestyle{empty} % Para que no muestre el número en pie de página

% Info para armar título.
\title{Práctica 4 de álgebra 1} % título
\author{D. Garraz} % autor
\date{last update: \today} % Cambiar de ser necesario

\maketitle  % para que aprezca el título en el documento
\section{Definiciones y fórmulas útiles}

\begin{itemize}
	\item $d$ divide a $a \to d \divide a \sisolosi \existe k \in \enteros : a = k \cdot d$
	\item $ \divset{-a}{-|a|,\dots,-1,1,\dots,|a|}$.
	\item $d \divide 0 $, dado que $0 = 0\cdot d$. Se desprende que $\divset{0}{\enteros - \set{0}}$
	\item $\llave{l}{
			      d \divide a \sisolosi -d \divide a \text{ (pues }a = k \cdot d \sisolosi a = (-k) \cdot (-d))\\
			      d \divide a \sisolosi d \divide -a \text{ (pues} a = k \cdot d \sisolosi (-a) = (-k) \cdot d)\\
			      \entonces d \divide a \sisolosi |d| \,\divide\, |a|
		      }$

	\item $\llave{l}{
			      d \divide a \ytext d \divide b \entonces d \divide a + b\\
			      d \divide a \ytext d \divide b \entonces d \divide a - b\\
			      d \divide a \entonces d \divide c \cdot a, \paratodo c \en \enteros\\
			      d \divide a \entonces d \divide c \cdot a\\
			      d \divide a \entonces d^2 \divide a^2 \ytext d^n \divide a^n  \paratodo n \en \naturales\\
			      d \divide a \cdot b \text{ no implica } d \divide a \o d \divide b. \text{ Por ejemplo } 6 \divide 3 \cdot 4
		      }$

	\item
	      $\llave{l}{
			      \textit{$a$ es congruente a $b$ módulo $d$} \text{ si }   d \divide a-b \text{. Se nota } \congruencia{a}{b}{d}\\
			      \congruencia{a}{b}{d} \sisolosi d \divide a-b
		      }$

	\item $
		      \llave{c}{
			      \congruencia{a_1}{b_1}{d}\\
			      \vdots\\
			      \congruencia{a_n}{b_n}{d}
		      }
		      \entonces \congruencia{a_1 + \cdots + a_n}{a_b + \cdots + b_n}{d}
	      $.
	\item $
		      \llave{c}{
			      \congruencia{a_1}{b_1}{d}\\
			      \vdots\\
			      \congruencia{a_n}{b_n}{d}
		      }
		      \entonces \congruencia{a_1 \cdots a_n}{a_b \cdots b_n}{d} \flecha{$a_i = a \y b_i = b$}[$\paratodo i \en \set{1,\dots, n}$] \congruencia{a^n}{b^n}{d}$
\end{itemize}

\textit{\underline{Algoritmo de división:}}\\
% macro local
\newcommand{\condicionResto}[1]{\ub{0 \leq #1 < |d|}{\text{\tiny cumple condición de resto}}}
\begin{itemize}
	\item
	      Dados $a, \, d \en \enteros$ con $d \distinto 0$,
	      \textit{\underline{existen}} $k$ (cociente),
	      $r \text{(resto)} \en \enteros$ tales que:\\
	      \[
		      \llaves{c}{
			      a =  k \cdot d + r,\\
			      \text{con } 0\leq r < |d|.
		      }
	      \]
	      Y además estos $k$ y $r$ son \textit{\underline{únicos}}.\\

	\item \textit{Notación: } $r_d(a)$ es el resto de dividir a $a$ entre $d$

	\item $\condicionResto{r} \entonces r = r_d(r)$. Un número que cumple condición de resto, es su resto.

	\item $r_d(a) = 0 \sisolosi d \divide a \sisolosi \congruencia{a}{0}{d}$

	\item $\congruencia{a}{r_d(a)}{d}$. Tiene mucho sentido.

	\item $\congruencia{a}{r}{d}$ con $\condicionResto{r} \entonces r = r_d(a)$

	\item $\congruencia{r_1}{r_2}{d}$ con $\condicionResto{r_1,r_2} \entonces r_1 = r_2$

	\item $\congruencia{a}{b}{d} \sisolosi r_d(a) = r_d(b)$. Dos números que son congruentes, tienen igual resto.

	\item $r_d(a+b) = r_d(r_d(a) + r_d(b))$ ya que si
	      $ \llaves{c}{
			      \congruencia{a}{r_d(a)}{d}\\
			      \congruencia{b}{r_d(b)}{d}
		      } \to \congruencia{a + b}{r_d(a) + r_d(b)}{d}
	      $

	\item $r_d(a \cdot b) = r_d(r_d(a) \cdot r_d(b))$ ya que si
	      $ \llaves{c}{
			      \congruencia{a}{r_d(a)}{d}\\
			      \congruencia{b}{r_d(b)}{d}
		      } \to \congruencia{a \cdot b}{r_d(a) \cdot r_d(b)}{d}$
\end{itemize}

\textit{\underline{Sistema de numeración: }}
\begin{itemize}
	\item Sea $d \en \naturales, d \geq 2$. Entonces $\paratodo a \en \naturales_0$ se puede
	      escribir en la forma
	      \[
		      a = r_n d^n + r_{n-1} d^{n-1} + \cdots + r_1 d^1 + r_0
	      \]
	      con $0 \leq r_i < d$ para $0 \leq i \leq n$ con $r_n,\dots, r_0$ son únicos
	      en esas condiciones.

	\item \textit{Notación:} $a = (r_n r_{n-1}\cdots r_1 r_0)_d =
		      \llave{l}{
			      2020 = (2020)_{10} \\
			      2020 = (7E4)_{16} \\
			      2020 = (31040)_{5}
		      }$
	\item $d^n = (1 \ub{0\cdots 0}{n})$

	\item ¿Cuál es el número más grande que puedo escribir usando n cifras en base $d$?\\
	      $(\ub{d-1\quad d-1\quad \cdots\quad d-1}{n \text{ cifras}})_d =  \sumatoria{i=0}{n-1}(d-1)d^i = d^n -1$

	\item ¿Cuántos números hay con $\leq n$ cifras?\\
	      Hay del $0$ hasta el $d^n -1$, es decir $d^n$.

	\item ¿Cuál es la forma más rápida de calcular $2^{16}$

\end{itemize}
\textit{\underline{Máximo común divisor: }}
%%%%%% Macro local
\def\mcd{(a:b)}
\def\D{\mathcal D}
\def\cz{s\cdot a + t \cdot b}
%%%%%% fin Macro local
\begin{itemize}
	\item Sean $a,b \en \enteros$, \underline{no ambos nulos}. El MCD entre $a$ y $b$ es el mayor de los divisores
	      común entre $a$ y $b$ y se nota $\mcd$

	\item $\mcd \en \naturales$ (pues $\mcd \geq 1$) siempre existe.
	      $\D com_+(a,b) = \D_+(a) \inter \D_+(b) \distinto \vacio
		      \text{ pues } 1 \en \D com_+(a,b)$.
	      Se ve también que está acotado por el menor entre $a$ y $b$, pues si
	      $d \divide a \y d \divide b \entonces d \leq |a| \y d \leq|b|$ y es \underline{único}.

	\item Sean $a$ y $b \en \enteros$, no ambos nulos.\\
	      \begin{itemize}
		      \item $\mcd = (\pm a : \pm b)$
		      \item $\mcd = (b:a)$
		      \item $(a:1) = 1$
		      \item $(a:0) = |a|,\ \paratodo a \en \enteros -\set{0}$
		      \item si $b \divide a \entonces \mcd = |b|$ si $b \en \enteros - \set{0}$
		      \item $\mcd = (a: b+na)$ con $n \en \enteros$
		      \item $\mcd = (a: r_a(b))$ con $n \en \enteros$
	      \end{itemize}

	\item \textit{Algoritmo de Euclides}:
	      Sean $a, b \en \enteros$ con $b \distinto 0$, entonces, $\paratodo k \en \enteros$, se tiene:
	      $\mcd = (b:a-kb)$. En particular, como $r_b(a) = a-kb$, con $k$ el cociente (para $b\distinto 0$), se tiene
	      $\mcd = (b: r_b(a))$

	\item \textit{Combinacion Entera}:
	      Sean $a,b \en \enteros$ no ambos nulos, entonces $\existe s,\ t \en \enteros$ tal que $\mcd = \cz$.
	      \begin{itemize}
		      \item Sean $a,b \en \enteros$ no ambos nulos, $d \en \enteros - \set{0}$. Entonces:
		            \[
			            d \divide a \ytext d \divide b \sisolosi d \divide \ub{\mcd}{\cz}
		            \].
		      \item Sea $c \en \enteros$ entonces $\existe s', t' \en \enteros$ con $c = s'a + t'b \sisolosi \mcd \divide c$.
		      \item Todos los números múltiplos del MCD se escriben como combinación entera de $a$ y $b$.
		      \item Si un número es una combinación entera de $a$ y $b$ entonces es un múltiplo del MCD.
		      \item Sean $a,\ b \en \enteros$ no ambos nulos, y sea $k \en \naturales$
		            \[
			            (ka:kb) = k(a:b)
		            \]
	      \end{itemize}
	\item \textit{Coprimos: }
	      \begin{itemize}
		      \item
		            Dados $a,b \en \enteros$, no ambos nulos, se dice que son \textit{coprimos} si $\mcd = 1$
		            \[
			            \begin{array}{c}
				            a \cop b \sisolosi \mcd = 1 \\
				            \qquad a\cop b \sisolosi \existe s,\ t \en \enteros \text{ tal que } 1 = \cz
			            \end{array}
		            \]
		      \item
		            Sean $a,b \en \enteros$, no ambos nulos. Entonces $\frac{a}{\mcd}\cop \frac{b}{\mcd}$.
		      \item Coprimizar es :
		            $\llaves{l}{
				            a = \mcd \cdot a'\\
				            b = \mcd \cdot b'
			            }\to a' \ytext b'$ son coprimos.

		      \item Sean $a, c, d \en \enteros$ con $c,d$ no nulos. Entonces:
		            \[
			            c \divide a \ytext d \divide a \ytext c \cop d \red{$\sisolosi$} c\cdot d \divide a
		            \]
		      \item Sean $a, b, d \en \enteros$ con $d \distinto 0$. Entonces:
		            \[
			            d \divide a \cdot b \ytext d \cop a   \entonces d \divide b
		            \]

	      \end{itemize}

	\item \textit{\underline{Primos y Factorización:}}
	      \begin{itemize}
		      \item Sea $p$ primo y sean $a,b \en \enteros$. Entonces:
		            \[
			            p \divide a\cdot b \entonces p \divide a \o p \divide b
		            \]
		      \item \textit{Si $p$ divide a algún producto de números, tiene que dividir a alguno de los factores $\to$}\\
		            Sean $a_1,\dots, a_n \en \enteros$:\\
		            \begin{center}
			            $
				            \llave{l}{
					            p \divide a_1 \cdot a_2 \cdots a_n \entonces p \divide a_i \text{ para algún } i \text{ con } 1 \leq i \leq n.\\
					            p \divide a^n \entonces p \divide a.
				            }$
		            \end{center}
		      \item Si $a \en \enteros$, $p$ primo:\\
		            \begin{center}
			            $\llave{l}{
					            (a:p) = 1 \sisolosi p \noDivide a\\
					            (a:p) = p \sisolosi p \divide a
				            }$
		            \end{center}
		      \item Sea $n \en \enteros - \set{0},\,
			            n = \ub{s }{\set{-1,1}} \cdot \productoria{i=1}{k} p_i^{\alpha_i} =
			            p_1^{\alpha_1} \cdots p_k^{\alpha_k}$
		            su factorización en primos. Entonces todo divisor $m$ positivo de $n$ se escribe como:\\
		            \[
			            \llave{c}{
			            \text{Si } m \divide n \to  m = p_1^{\beta_1} \cdots p_k^{\beta_k}
			            \text{ con } 0 \leq \beta_i \leq \alpha_i,\, \paratodo i\, 1\leq i \leq k\\
			            \text{ y hay } \\
			            (\alpha_1 + 1) \cdot (\alpha_2 + 1)\cdots (\alpha_k + 1) = \productoria{i=1}{k} \alpha_i +1 \\
			            \text{divisores positivos de } n.
			            }\]
	      \end{itemize}

\end{itemize}
\subsubsection*{Ejercicios dados en clase:}
\ejercicio  4400 ¿Cuántos divisores distintos tiene? ¿Cuánto vale la suma de sus divisores.\\
\separadorCorto
$
	4400
	\flecha{factorizo} 4400 = 2^4 \cdot 5^2 \cdot 11
	\flecha{los divisores $m \divide 4400$}[tendrán la forma]
	m = \pm 2^{\alpha} \cdot 2^{\beta} \cdot 2^{\gamma},\,
	\text{ con }
	\llaves{c}{
		0 \leq \alpha\leq 4\\
		0 \leq \beta \leq 2\\
		0 \leq \gamma \leq 1
	}
$\\
Hay entonces un total de $5 \cdot 3 \cdot 2 = 30$ divisores positivos y $60$ enteros.\\
Ahora busco la suma de esos divisores:
$
	\sumatoria{i=0}{4} \sumatoria{j=0}{2}\sumatoria{k=0}{1} 2^i \cdot 5^j \cdot 11^k =
	\parentesis{\sumatoria{i=0}{4} 2^i } \cdot \parentesis{ \sumatoria{j=0}{2} 5^j } \cdot \parentesis{ \sumatoria{k=0}{1} 11^k}\\
	\flecha{sumas}[geométricas]
	\ub{\frac{2^{4+1} - 1}{2 - 1}}{31} \cdot \ub{\frac{5^{2+1} - 1}{5 - 1}}{31} \cdot \ub{\frac{11^{1+1} - 1}{11 - 1}}{12} = 11532.
$\\

\ejercicio
Hallar el menor $n \en \naturales$ tal que:
\begin{enumerate}[label=\roman*)]
	\item $(n:2528) = 316 $
	\item $n$ tiene exáctamente 48 divisores positivos
	\item $27 \noDivide n$
\end{enumerate}
\separadorCorto
$
	\llave{l}{
		\flecha{factorizo}[2528] 2528 = 2^5 \cdot 79 \Tilde\\
		\flecha{factorizo}[316] 316 = 2^2 \cdot 79 \Tilde\\
		\flecha{reescribo}[condición] (n:2^5 \cdot 79) = 2^2 \cdot 79
	}\\
	\flecha{quiero}[encontrar] n = 2^{\alpha_2} \cdot 3^{\alpha_3} \cdot 5^{\alpha_5} \cdot 7^{\alpha_7} \cdots 79^{\alpha_79} \cdots .\\
	\flecha{como} (n:2^5 \cdot 79) = 2^2 \cdot 79
	\flecha{tengo}[que]
	\llave{ll}{
		\alpha_2 = 2, & \text{dado que $2^2 \cdot 79 \divide n$. \blue{Recordar que busco el menor $n$!}.}\\
		\alpha_{79} \geq 1, & \text{Al igual que antes.} \\
		\flecha{notar}[que] \alpha_3 < 3 &  \text{ si no } 3^3 = 27 \divide n
	}\\
	\flecha{la estrategia sigue con}[el primo más chico que haya]
	\llave{l}{
		48 = \ub{(\alpha_2 + 1)}{2 + 1} \cdot (\alpha_3 + 1) \cdots\\
		48 = 3 \cdot (\alpha_3 + 1) \cdots\\
		16 = (\alpha_3 + 1) \cdot (\alpha_5 + 1) \cdot (\alpha_7 + 1) \cdots \ub{(\alpha_{79} + 1)}{=2\text{ quiero el menor}} \cdots \\
		8 = (\alpha_3 + 1) \cdot (\alpha_5 + 1) \cdot (\alpha_7 + 1) \cdots \\
		8 = \ub{(\alpha_3 + 1)}{=2} \cdot \ub{(\alpha_5 + 1)}{=2} \cdot \ub{(\alpha_7 + 1)}{=2} \cdot 1 \cdots 1 \\
	}
$
El $n$ que cumple lo pedido sería $n = 2^2 \cdot 3^1 \cdot 5^1 \cdot 7^1 \cdot 79^1$

\ejercicio
Sabiendo que $(a:b) = 5$. Probar que $(3ab: a^2 + b^2) = 25$

\separadorCorto

Coprimizar:
$
	\llave{l}{
		c = \frac{a}{5}\\
		d = \frac{b}{5}\\
	}
	\to
	(a:b) = 5 \cdot \ub{(c:d)}{1} = 5\\
	\to
	\llave{l}{
		\flecha{según}[enunciado]
		25  = (3ab : a^2 + b^2) \flecha{reemplazo} 25 = 25 \cdot \ub{(3 cd : c^2 + d^2)}{1}
	}\\
	\flecha{Voy a probar}[que] (3cd : c^2 + d^) = 1.\\
	\flecha{Supongo que}[no lo fuera] \existe p \to
	\llave{l}{
		p \divide (3cd : c^2 + d^2) \to
		\llave{l}{
			p \divide 3cd \to
			\llave{l}{
				p \divide 3  \to p = 3
				\flecha{tabla}[$r_3(c^2 + d^2)$]
				\llave{lll}{
					0 & \text{si} & \ub{\congruencia{c,d}{0}{3}}{\magenta{noup! }(c:d) = 1} \\
					\neq 0 & \text{si} & \text{ otro caso}
				}\\
				p \divide c \to
				\llave{l}{
					\flecha{como} p \divide c^2 + d^2 \entonces p \divide d^2 \to \ub{p \divide d}{\magenta{noup! } (c:d) = 1} \\
				}\\
				p \divide d \to \magenta{noup! idem}\\
			}
			\\
			p \divide c^2 + d^2
		}
	}
$\\
Si ningún primo $p$ divide a $(3cd : c^2 + d^2) \entonces (3cd : c^2 + d^2) = 1$

\ejercicio
Ejercicio parcial
\begin{enumerate}[label=\roman*)]
	\item Calcular los posibles valores de: $(7^{n-1} + 5^{n+2} : 5\cdot 7^n - 5^{n+1})$.
	\item Encontrar $n$ tales que el mcd para ese $n$ tome 3 valores distintos.
\end{enumerate}

\separadorCorto
Busco independencia de $n$ en algún lado del $\mcd$. Si $ d = (7^{n-1} + 5^{n+2} : 5\cdot 7^n - 5^{n+1})\to\\
	\llave{l}{
		d \divide 7^{n-1} + 5^{n+2} \\
		d \divide 5\cdot 7^n - 5^{n+1}
	}
	\to
	\llave{l}{
		d \divide \ub{7^{n-1} + 5^{n+2}}{\congruente 2^n\ (5)} \\
		d \divide 5 \cdot ( 7^n - 5^n)
	}
	\flecha{$p \noDivide d \y d \divide p \cdot k$}[$\entonces p \divide k$]
	\llave{l}{
		d \divide 7^{n-1} + 5^{n+2} \\
		d \divide 7^n - 5^n
	}\\
	\to
	\llave{l}{
		d \divide 176 \cdot 5^n \\
		d \divide 7^n - 5^n
	}
	\flecha{$p \noDivide d \y d \divide p \cdot k$}[$\entonces p \divide k$]
	\llave{l}{
		d \divide 176 \\
		d \divide 7^n - 5^n
	}
	\to d = (176 : 7^n - 5^n) \Tilde\\
$
Factorizo: $176 = 2^4 \cdot 11 \to \divsetP{176}{1,2,4,8,11,16,22,44,88,176}$.\\
Descarto
$\to
	\llave{lclcl}{
		1           & \to &\congruencia{7^n - 5^n}{2^n}{5} &\to & d \text{ tiene que ser par y } 2 > 1\\
		11          & \to &\congruencia{7^n - 5^n}{2^n}{5} &\to & d \text{ tiene que ser par}\\
	}\\
	\divsetP{d}{2,4,8,16,22,44,88,176}\\
$
Estudio congruencia de los pares e impares:
$
	\llave{l}{
		\congruencia{7^{2k} - 5^{2k}}{1^k -25^k}{8} \to \congruencia{1 - \ub{1}{\stackrel{(8)}\congruente 25}}{0}{8}\\
		\congruencia{7^{2k+1} - 5^{2k+1}}{3 - 1}{4} \to \congruente 2\,(4)\\
	}\\
$ Puedo descartar a los múltiplos de 4 que no sean múltiplos de 8.
$	\to \divsetP{d}{2,8,16,22,88,176}$\\
\red{No lo terminé, no entiendo bien este paso y como descartar algún otro.}


\ejercicio
Parcial del primer cuatrimestre 2024\\
Estudiar los valores parar \textbf{todos} los $a \en \enteros$ de $(a^3 + 1 : a^2 - a + 1)$.\\
\separadorCorto
Primero hay que notar que el lado $a^2-a+1$ es siempre impar ya que:\\
$\llaves{l}{
		(2k-1)^2 - (2k -1) + 1 \stackrel{(2)}\congruente (-1)^2 -1 + 1 \stackrel{(2)}\congruente 1\\
		(2k)^2 - (2k) +1 \stackrel{(2)}\congruente (0)^2 - 0 + 1\stackrel{(2)}\congruente 1.
	}$ Por lo tanto 2 no puede ser un divisor de ambas expresiones  y si $2 \noDivide A \entonces 2 \cdot k \noDivide A$ tampoco.\\
Se ve fácil contrarecíproco: $\ub{2k}{par} \divide A \entonces 2 \divide A$. Porque existe un $k$ tal que
$2 \cdot c \cdot k = A \entonces 2 \cdot (c\cdot k) = A.$\\
Ahora cuentas para simplificar la expresión y encontrar número del lado derecho.\\
$
	\llave{l}{
		d \divide  a^3 + 1\\
		d \divide  a^2 -a +1
	}
	\to d \divide 30
	\to\divsetP{d}{1,2,3,5,6,10,15,30}
	\flecha{por lo de antes}[no hay divisores pares] \divsetP{d}{1,3,5,15}\\
	\flecha{hacer tabla de restos}[empezar por los números chicos]
	\llaves{ccc}{
		r_3(a^3 + 1) = 0 & \text{ si }& \congruencia{a}{2}{3}\\
		\y& &\\
		r_3(a^2 -a + 1) = 0 & \text{ si } & \congruencia{a}{2}{3}
	}
	\to
	\llaves{cc}{
		r_5(a^3 + 1) \distinto 0 & \paratodo a \en \enteros\\
	}
$.\\
Luego si  $5 \noDivide (a^3 + 1 : a^2 - a + 1) \entonces \ub{15}{5\cdot 3} \noDivide (a^3 + 1 : a^2 - a + 1)
	\flecha{se achica el}[conjunto de divisores] \divsetP{d}{1,3}\\
	d = \llave{ccc}{
		3 & \text{ si } & \congruencia{a}{2}{3}\\
		1 & \text{ si } & \congruencia{a}{1 \o 2}{3}\\
	}
$


\ejercicio
Sean $a$, $b \en \enteros$ tal que $(a:b) = 6$. Hallar todos los $d = (2a + b : 3a - 2b)$ y dar un ejemplo en cada caso.\\
\separadorCorto

Conviene \textit{coprimizar}: $(a:b) = 6 \sisolosi
	\llaves{l}{
		a = 6A\\
		b = 6B
	}$ con $(A:B) \llamada{1}= 1$\\
$d =
	(2\cdot 6A + 6B : 3\cdot 6A - 2\cdot 6B) =
	(6\cdot( 2 \cdot A + B) : 6\cdot (3\cdot A - 2\cdot B)) =
	6 \cdot \ub{(2A + B : 3A - 2B)}{D}\\
	\to d \llamada{2}= 6 D \flecha{busco divisores}[comunes]
	\llave{l}{
		D \divide 2A + B \\
		D \divide 3A - 2B \\
	} \flecha{operaciones}[$\dots$]
	\llave{l}{
		D \divide 7B \\
		D \divide 7A \\
	}\entonces
	D = (7A:7B) =7 \cdot (A:B) \llamada{1}= 7$  \\
Por lo tanto $D \en \divsetP{7}{1,7}$, pero yo quiero encontrar ejemplos de $a$ y $b$:\\
$\llamada{2}\to
	\llave{c}{
		d = 6 \cdot 7 = 42
		\llave{ll}{
			\text{Si:} & A = 2 \to a = 12\\
			& B = 3 \to b = 18\\
			\multicolumn{2}{l}{(7:0)\entonces D = 7 \to d = (42:0) = \ub{42}{6\cdot D}}
		}\\

		\magenta{$\o$} \\
		d = 6 \cdot 1 = 6
		\llave{ll}{
			\text{Si:} & A = 0 \to a = 0\\
			& B = 1 \to b = 6\\
			\multicolumn{2}{l}{(1:-2) \entonces D = 1 \to d = (6:-12) = \ub{6}{6\cdot D}}
		}
	}$



\newpage

%=========================
% Ejercicio guia
%=========================

\section*{Ejercicios de la guía:}
\setcounter{ejercicio}{0} % Reset the custom counter
\noindent\textit{\underline{Divisibilidad}}
%1
\ejercicio
Decidir si las siguientes afirmaciones son verdaderas $\paratodo a,\, b,\, c \en \enteros$:\\
Calcular
\begin{enumerate}[label=\roman*)]
	\item $a \cdot b \divide c \entonces a \divide c$ \ y \ $b \divide c$ \\
	      \separadorCorto

	      $\llave{l}{
			      c = k\cdot a \cdot b = \underbrace{h}_{k \cdot b} \cdot a \entonces a \divide c \Tilde\\
			      c = k\cdot a \cdot b = \underbrace{i}_{k \cdot a} \cdot b \entonces b \divide c\Tilde
		      }$

	\item $4 \divide a^2 \entonces 2 \divide a $\\
	      \separadorCorto

	      $ a^2 = k \cdot 4 = \underbrace{h}_{k \cdot 2} \cdot 2 \entonces a^2 \divide 2
		      \flecha{si $a\cdot b \divide c$}[$\entonces a \divide c \y b \divide c$]
		      a \divide 2 \Tilde$

	\item $2 \divide a \cdot b \entonces 2 \divide a $ \ o \ $2 \divide b$\\
	      \separadorCorto

	      Si $2 \divide a \cdot b \entonces
		      \llaves{c}{
			      a \text{ tiene que ser } par \\
			      \o \\
			      b \text{ tiene que ser } par \\
		      } \flecha{para que} a \cdot b$ sea par. Por lo tanto si  $2 \divide a \cdot b \entonces 2 \divide a $ \ o \ $2 \divide b$.

	\item $9 \divide a\cdot b \entonces 9 \divide a  $ \ o \ $9 \divide b$\\
	      \separadorCorto

	      Si $a = 3 \y b = 3$, se tiene que $9 \divide 9$, sin embargo $9 \noDivide 3$

	\item $a \divide b + c \entonces a \divide b $ \ o \  $a \divide c$\\
	      \separadorCorto

	      $12 \divide 20 + 4 \entonces 12 \noDivide 20$  \ y\   $ 12\noDivide 4 $

	\item
	      \separadorCorto
	      \hacer
	\item
	      \separadorCorto
	      \hacer
	\item
	      \separadorCorto
	      \hacer
	\item $a \divide b + a^2 \entonces a \divide b$\\
	      \separadorCorto

	      $  a \divide b + a^2 \entonces b + a^2 = k \cdot a \flecha{acomodo} b = (k - a) \cdot a = h \cdot a \entonces a | b \Tilde$\\
	      $ \flecha{también puedo}[decir si:]
		      \llaves{l}{
			      a \divide a^2 \\
			      a \divide b - a^2
		      } \flecha{por}[propiedad] a \divide (b - a^2) + (a^2) = b \entonces a \divide b \Tilde $


	\item $a \divide b \entonces a^n \divide b^n, \paratodo n \en \naturales$\\
	      \separadorCorto

	      Pruebo por inducción. $p(n)\  :\  a \divide b \entonces a^n \divide b^n $\\
	      $
		      \llave{l}{
			      \text{Caso base: } n = 1 \entonces a|b \entonces a^1 \divide b^1 \Tilde\\
			      \text{Paso inductivo: } \paratodo h \en \naturales, p(h) \ V \entonces p(h+1)\ V?\\
			      \text{Si } a \divide b \entonces a^k \divide b^k  \entonces a^k \cdot c = b^k
			      \flecha{multiplico por}[$b$ M.A.M]
			      b \cdot a^k \cdot c = b^{k+1}
			      \flecha{$a\divide b$}[$a \cdot d = b$]
			      a\cdot d \cdot a^k \cdot c =  a^{k+1}\cdot(cd) = b^{k+1}\\
			      \flecha{concluyendo}[que] a^{k+1} \divide b^{k+1} \text{como quería mostrarse.}
		      }
	      $ Como $p(1) \y p(k) \y p(k+1)$ resultaron verdaderas, por el principio de inducción $p(n)$
	      es verdadera $\paratodo n \en \naturales$\\
	      \textit{Este resultado es importante y se va a ver en muchos ejercicios.\\
		      $a \divide b \entonces a^n \divide b^n \sisolosi
			      \congruencia{b}{0}{a} \entonces \congruencia{b^n}{\ub{0}{\stackrel{(a^n)}\congruente a^n}}{a^n} \sisolosi \congruencia{b^n}{a^n}{a^n}$}


\end{enumerate}

%2
\ejercicio
Hallar todos los $n \en \naturales$ tales que:
\begin{enumerate}[label=\roman*)]
	\item $3n - 1 \divide n+7$\\
	      \separadorCorto

	      Busco eliminar la $n$ del \textit{miembro} derecho.\\
	      $\llaves{l}{
			      3n - 1 \divide n + 7
			      \flecha{$a|c \entonces$}[$a \divide k\cdot c$]
			      3n - 1 \divide \red{3} \cdot (n+7) = 3n + 21\\
			      \flecha{$a \divide b \y a \divide c$}[$\entonces a \divide b \pm c$]
			      3n-1 \divide 3n + 21 - (3n -1) = 22
		      } \to 3n - 1 \divide 22\\
		      \flecha{busco $n$}[para que] \frac{22}{3n-1} \en \divset{22}{1\pm1, \pm2, \pm11, \pm22} \flecha{probando} n \en \set{1,4} \Tilde$
	\item
	\item

	\item $n-2 \divide n^3 - 8$ \\
	      \separadorCorto

	      $\flecha{$a \divide b$}[$\entonces a \divide k \cdot b$]
		      n-2 \divide \ub{(n-2) \cdot (n^2 + 2n + 4)}{n^3 - 8} $ Esto va a dividir para todo $n \distinto 2$\\

\end{enumerate}

%3
\ejercicio Sean $a.\,b \en \enteros$.
\begin{enumerate}[label=\roman*)]
	\item Probar que $a-b \divide a^n - b^n$ para todo $n \en \naturales \y a \distinto b \en \enteros$
	\item Probar que si $n$ es un número natural par y $a \distinto -b$, entonces $a+b \divide a^n - b^n$.
	\item Probar que si $n$ es un número natural impar y $a \distinto -b$, entonces  $a+b \divide a^n + b^n$.
\end{enumerate}
\separadorCorto
%=============
%Macro local
\def\aMenosB{\stackrel{(a-b)}\congruente}
\def\aMasB{\stackrel{(a-b)}\congruente}
%FinMacro local
%=============
\begin{enumerate}[label=\roman*)]
	\item  Si $a-b \divide a^n - b^n \stackrel{def}\sisolosi \congruencia{a^n}{b^n}{a-b} \sisolosi
		      \llaves{l}{
			      \congruencia{a}{b}{a-b} \\
			      \congruencia{a^2}{\ub{a}{\aMenosB b}\cdot b}{a-b} \to \congruencia{a^2}{b^2}{a-b} \\
			      \quad\vdots\\
			      \congruencia{a^n}{b^n}{a-b}
		      }\\
		      \to \congruencia{a^n}{b^n}{a-b} \sisolosi a-b \divide a^n - b^n$

	\item  Sé que $\congruencia{a}{-b}{a+b} \sisolosi
		      \llaves{l}{
			      \congruencia{a^2}{\ub{a}{\aMasB -b} \cdot b}{a+b} \flecha{propiedad}[congruencia] \congruencia{a^2}{(-1)^2 \cdot b^2}{a+b} \\
			      \quad \vdots \quad \llamada{1}{\ot} \\
			      \congruencia{a^n}{(-1)^n \cdot b^n}{a+b} \to
			      \llaves{ll}{
				      \congruencia{a^n}{ b^n}{a+b} & \text{con n par}  \\
				      \congruencia{a^n}{(-1)^n \cdot b^n}{a+b}   & \text{con n impar} \\
			      }\llamada{2}{}\\
		      }\\
		      \llamada{2}{}
		      \boxed{
			      \llave{l}{
				      \text{Con $n$ par: }  \congruencia{a^n}{ b^n}{a+b} \entonces a+b \divide a^n - b^n  \\
				      \text{Con $n$ impar: }  \congruencia{a^n}{ - b^n}{a+b} \entonces a+b \divide a^n + b^n
			      }
		      }
	      $

	      $\llamada{1}{}
		      \textit{Inducción:}\\
		      \llave{l}{
			      p(n)\ :\ \magenta{$\congruencia{a}{-b}{a+b}$} \entonces \congruencia{a^n}{(-1)^n \cdot b^n}{a+b} \paratodo n \en \naturales.\\
			      \textit{Caso base: }\\
			      p(1)\ :\
			      \magenta{$\congruencia{a}{-b}{a+b}$} \entonces \congruencia{a^1}{(-1)^1\cdot b^1}{a+b}
			      \entonces
			      \magenta{$\congruencia{a}{-b}{a+b}$} \text{ Verdadero}.\\
			      \textit{Hipótesis inductiva: }\\
			      p(k)\ V \entonces p(k+1)\ V? \\
			      \magenta{$\congruencia{a}{-b}{a+b}$} \entonces \congruencia{a^k}{(-1)^k\cdot b^k}{a+b}
			      \entonces
			      \magenta{$\congruencia{a}{-b}{a+b}$} \entonces \congruencia{a^{k+1}}{(-1)^{k+1}\cdot b^{k+1}}{a+b} \\
			      \text{Parto de } p(k):\\
			      \llave{l}{
				      \congruencia{a^k}{(-1)^k\cdot b^k}{a+b}\\
				      \flecha{multiplico}[por $a$]
				      \congruencia{a\cdot a^k}{(-1)^k\cdot \ub{a}{\aMasB -b}\cdot b^k}{a+b}\\
				      \flecha{y acomodo}
				      \congruencia{a^{k+1}}{(-1)^{k+1} \cdot b^{k+1}}{a+b} \Tilde
			      }
		      }
	      $
	      \\
	      Como $p(1),\ p(k),\ p(k+1)$ son verdaderas por principio de inducción lo es también $p(n) \paratodo n \en \naturales$


	\item hecho en el anterior.


	      %4
	      \ejercicio
	      Sea $a \in \enteros$ impar. Probar que $2^{n+2} \divide a^{2^n} - 1$ para todo $n \en \naturales$\\
	      \separadorCorto

	      Pruebo por inducción: $p(n): 2^{n+2} \divide a^{2^n} - 1$\\
	      $\llaves{l}{
			      \textit{Caso base: } p(1)\ :\ 2^3 \divide a^2 - 1 = (a - 1) \cdot (a + 1)
			      \flecha{$a$ es impar}[$a = 2m -1$] (2m - 2)\cdot(2m) =\\
			      4 \cdot \ub{m \cdot (m-1)}{par} = 4 \cdot (2\cdot h) = 8 * h \flecha{por lo}[tanto] 8 \divide 8 \cdot h \text{ con $h$ entero.}\Tilde\\

			      \textit{Paso inductivo: } p(k) \ V  \entonces p(k+1) \ V? \flecha{es}[decir]
			      2^{k+2} \divide a^{2^k} - 1 \entonces  2^{k+3} \divide a^{2^{k+1}} - 1\ V? \\

			      \textit{Hipótesis inductiva: }\\
			      \llave{l}{
				      2^{k+3} \divide a^{2^{k+1}} - 1 \flecha{acomodar}[diferencia cuadrados] 2\cdot 2^k \divide (a^{2^k})^2 - 1 =\\
				      \ub{ (a^{2^k} - 1) }{\text{par}} \cdot  \ub{(a^{2^k} + 1)}{\text{par}}  \\
				      \tiny\llaves{lll}{
					      a \divide b                 & \sisolosi   & a\cdot k_1 = b\\
					      c \divide d                 & \sisolosi   & c\cdot k_2 = d\\ \hline
					      a\cdot c \divide b \cdot d  & \sisolosi   & a\cdot c \cdot \ub{k_3}{k_1\cdot k_2} =  b\cdot d
				      }
				      \flecha{Si $a \stackrel{\tiny HI}\divide b$ y $c \divide d$}[$\entonces a\cdot c \divide b\cdot d$]
				      \ub{2^{k+2}}{a} \cdot \ub{2}{c}   \divide  \ub{ (a^{2^k} - 1)}{b} \cdot \ub{(a^{2^k} + 1)}{d} \entonces 2^{k+3} \divide a^{2^{k+1}} - 1 \quad V \Tilde
			      }
		      } $\\
	      Como $p(1) \y p(k) \y p(k+1)$ resultaron verdaderas, por el principio de inducción $p(n)$
	      es verdadera $\paratodo n \en \naturales$

	      %5
	      \ejercicio
	      \separadorCorto
	      %6
	      \ejercicio
	      \separadorCorto


	      %7
	      \ejercicio
	      \begin{enumerate}[label=\roman*)]
		      \item $99 \divide 10^{2n} + 197$
		      \item $9 \divide 7 \cdot 5^{2n} + 2^{4n+1}$
		      \item $56 \divide 13^{2n} + 28n^2 - 84n -1$
		      \item $256 \divide 7^{2n} + 208n - 1$
	      \end{enumerate}

	      \separadorCorto
	      % definicion local
	      \def\cong99{\stackrel{(99)}\congruente}
	      % fin definicion local
	      \begin{enumerate}[label=\roman*)]
		      \item $99 \divide 10^{2n} + 197 \stackrel{def}{\sisolosi}
			            \congruencia{10^{2n} + 197}{0}{99} \to
			            \congruencia{10^{2n} + 198}{1}{99} \to
			            \congruencia{10^{2n} + \ub{198}{\cong99 0}}{1}{99} \to \congruencia{100^n}{1}{99}\to \\
			            \llave{l}{
				            \flecha{sé}[que] \congruencia{100}{1}{99} \sisolosi \congruencia{100^2}{\ub{100}{ \cong99 1 }}{99} \to
				            \congruencia{100^2}{1}{99} \sisolosi \dots \sisolosi \congruencia{100^n}{1}{99}\\
				            \red{¿Tengo que demostrar ese renglón por inducción o con "propiedad de congruencia" funciona?}
			            }
		            $\\
		            Se concluye que  $99 \divide 10^{2n} + 197 \sisolosi 99 \divide \ub{100 - 1}{99}$


		      \item
		            % definicion local
		            \def\cong9{\stackrel{(9)}\congruente}
		            % fin definicion local

		            $9 \divide 7 \cdot 5^{2n} + 2^{4n+1}
			            \stackrel{def}{\sisolosi}
			            \congruencia{7\cdot5^{2n} + 2^{4n+1}}{0}{9}
			            \flecha{sumo $2\cdot 5^{2n}$}[M.A.M]
			            \congruencia{\ub{9 \cdot 5^{2n}}{\cong9 0} + 2\cdot 2^{4n}}{2 \cdot 5^{2n}}{9}\\
			            \flecha{simplifico}[y acomodo]
			            \congruencia{2^{4n}}{5^{2n}}{9} \to
			            \congruencia{16^n}{25^n}{9}
			            \flecha{simetría}[congruencia]
			            \congruencia{25^n}{16^n}{9}
			            \flecha{$25\cong9 16$}
			            \congruencia{25}{16}{9} =
			            \congruencia{9}{0}{9}\\
		            $
		            Se concluye que $9 \divide 7 \cdot 5^{2n} + 2^{4n+1} \sisolosi 9 \divide 9$ \red{$\leftarrow$ ¿Se concluye esto...?}

		      \item \hacer
		      \item \hacer
	      \end{enumerate}

	      \textit{\underline{Algoritmo de División}}:\\
	      %8
	      \ejercicio
	      Calcular el cociente y el resto de la división de $a$ por $b$ en los casos:\\

	      % definiciones locales
	      \begin{enumerate}[label=\roman*)]
		      \begin{minipage}{0.5\linewidth}
			      \item $a =133,\quad b = -14.$
			      \item $a =13,\quad b = 111.$
			      \item $a =3b+7,\quad b \distinto 0.$
		      \end{minipage}
		      \begin{minipage}{0.5\linewidth}
			      \item $a = b^2 - 6,\quad b \distinto 0.$
			      \item $a = n^2 + 5,\quad b = n+2 \ (n \en \naturales).$
			      \item $a = n + 3,\quad = n^2 + 1 \ (n \en \naturales).$
		      \end{minipage}
	      \end{enumerate}
	      \separadorCorto
	      \begin{enumerate}[label=\roman*)]
		      \item $133 : (-14) \entonces 133 = \blue{(-9)} \cdot (-14) + \magenta{7}  $
		      \item $ $
		      \item $a = 3b + 7 \to
			            \text{me interesa: }\to
			            \llaves{l}{
				            |b| \leq |a| \Tilde\\
				            0 \leq \magenta{r} < |b| \Tilde \\
			            } \to\\ $


		            $ \to\llave{l}{
				            \text{Si: } |b| > 7 \to (\blue{$q$},\magenta{$r$}) = (3,7)\\
				            \text{Si: } |b| \leq 7 \to (\blue{$q$},\magenta{$r$}) = (3,7)\\
				            \footnotesize
				            \begin{array}{|l|l|l|l|l|l|l|l|}
					            \hline
					            (a,b)                      & (-14,-7) & (-11,-6) & (-8,-5) & (-5,-4) & (4, -1) & \dots \\ \hline
					            (\blue{$q$},\magenta{$r$}) & (2,0)    & (2,1)    & (2,2)   & (2,3)   & (4, 0)  & \dots \\ \hline
					            \hline
				            \end{array}
			            }
		            $

		      \item $a = b^2 - 6,\quad b \distinto 0.$\\
		            \separadorCorto

	      \end{enumerate}
	\item


	      %9
	      \ejercicio Sabiendo que el resto de la división de un entero $a$ por 18 es 5, calcular el resto de:
	      \begin{enumerate}[label=\roman*)]
		      \item la división de $a^2 -3a +11$ por 18.
		      \item la división de $a$ por 3.
		      \item la división de $4a+1$ por 9.
		      \item la división de $7a^2 + 12$ por 28.
	      \end{enumerate}
	      \separadorCorto
	      \begin{enumerate}[label=\roman*)]
		      \item $r_{18}(a) = r_{18}( \ub{r_{18}(a)^2}{5^2} - \ub{r_{18}(3)}{3} \cdot \ub{r_{18}(a)}{5} + \ub{r_{18}(11)}{11} ) = r_{18}(21) = 3 $

		            \separadorCorto

		      \item $
			            \llaves{l}{
				            a = 3 \cdot q + r_3(a)\\
				            6 \cdot a = 18 \cdot q + \ub{\green{$6 \cdot r_3(a)$}}{r_{18}(6a)}\\
			            } \to
			            r_{18}(6a) = r_{18}( r_{18}(6) \cdot r_{18}(a) ) = r_{18}(30) = 12\\
			            \entonces \green{$6 \cdot r_3(a)$} = r_{18}(6a) \to  r_3(a) = 2
		            $

		            \separadorCorto

		      \item $r_9(4a+1) = \ub{r_9(4 \cdot r_9(a) + 1)}{\blue{*1}} \to\\
			            a = 18 \cdot q + 5 = 9 \cdot \ub{( 9 \cdot q)}{q'} + \ub{5}{r_9(a)}
			            \flecha{\blue{*1}}
			            r_9(a) = r_9(21) = 3
		            $

		      \item
		            $r_{28}(7a^2 + 12) = r_{28}(7 \cdot r_{28}(a)^2 + 12) \flecha{¿qué es} r_{28}(a)$\\
		            $\llave{l}{
				            a = 18 \cdot q + 5 \flecha{busco algo}[para el 28]\\
				            14 \cdot a = \ub{252 \cdot q}{28 \cdot 9\cdot q } + 70
				            \flecha{corrijo según}[condición resto]
				            28 \cdot 9\cdot q + \ub{2\cdot28 +14}{70} = 28\cdot (9\cdot q + 2) + 14  \Tilde\\
				            \flecha{por lo}[tanto] 14a = 28\cdot q' + 14 \entonces \congruencia{14\cdot a}{14}{28} \sisolosi  \congruencia{a}{1}{28}
			            }$\\
		            Ahora que sé que $r_{28}(a) = 1$ sale que $r_{28}(7a^2 + 12) = r_{28}(7 \cdot \ub{r_{28}(a)^2}{1} + 12) = r_{28}(19)=19 \Tilde$
	      \end{enumerate}

	      %10
	      \ejercicio
	      \begin{enumerate}[label=\roman*)]
		      \item
		            Si $\congruencia{a}{22}{14}$, hallar el resto de dividir a $a$ por 14, por 2 y por 7.
		      \item
		            Si $\congruencia{a}{13}{5}$, hallar el resto de dividir a $33a^3 + 3a^2 -197a +2$ por 5.
		      \item Hallar, para cada $n \en \naturales$, el resto de la división de $\sumatoria{i=1}{n} (-1)^i \cdot i!$ por 12
	      \end{enumerate}

	      \separadorCorto


	      \begin{enumerate}[label=\roman*)]
		      \item $\llave{l}{
				            \congruencia{a}{22}{14} \to a = 14 \cdot q + \ub{22}{14 + 8} = 14 \cdot (q + 1) + 8 \flecha{el resto}[es] r_{14}(a) = 8 \Tilde\\
				            \congruencia{a}{22}{14} \to a = \ub{14 \cdot q}{2 \cdot (7 \cdot q)} + \ub{22}{2 \cdot 11} = 2 \cdot (7q + 11) + 0 \flecha{el resto}[es] r_{2}(a) = 0 \Tilde\\
				            \congruencia{a}{22}{14} \to a = \ub{14 \cdot q}{7 \cdot (2 \cdot q)} + \ub{22}{1 + 7 \cdot 3} = 7 \cdot (2q + 3) + 1 \flecha{el resto}[es] r_{7}(a) = 1 \Tilde\\
			            }$

		      \item  Dos números congruentes tienen el mismo resto. $\congruencia{a}{13}{5}  \sisolosi \congruencia{a}{3}{5}$
		            $r_5(33a^3 + 3a^2 -197a +2) = r_5( 3 \cdot r_5(a)^3 + 3 \cdot r_5(a)^2 - 2\cdot r_5(a) + 2 )\\
			            \flecha{como $\congruencia{a}{13}{5}$}[$r_5(a) = 3$] r_5(33a^3 + 3a^2 -197a +2) = 4$
		      \item \hacer
	      \end{enumerate}

\end{enumerate}

%11
\ejercicio
\begin{enumerate}[label=\roman*)]
	\item Probar que $\congruencia{a^2}{-1}{5} \sisolosi \congruencia{a}{2}{5}  \o \congruencia{a}{3}{5}$
	\item Probar que no existe ningún entero $a$ tal que $\congruencia{a^3}{-3}{7}$
	\item Probar que $\congruencia{a^7}{a}{7} \paratodo a \en \enteros$
	\item Probar que $7 \divide a^2 + b^2 \sisolosi 7 \divide a \y 7 \divide b.$
	\item Probar que $5 \divide a^2 + b^2 + 1 \sisolosi 5 \divide a \o 5 \divide b$. ¿Vale la recíproca?
\end{enumerate}

\separadorCorto

\begin{enumerate}[label=\roman*)]
	\item Me piden que pruebe una congruencia es válida solo para ciertos $a \en \enteros$. Pensado en términos de \textit{restos}
	      quiero que el resto al poner los $a$ en cuestión cumplan la congruencia.\\
	      $
		      \llave{l}{
			      \congruencia{a^2}{-1}{5} \sisolosi
			      \congruencia{a^2}{4}{5}  \sisolosi
			      \congruencia{a^2 - 4}{0}{5} \sisolosi
			      \congruencia{(a-2)\cdot (a+2)}{0}{5}\\
			      \flecha{quiero que el }[resto sea 0]
			      r_5(a^2 + 1) =
			      r_5(a^2 - 4) =
			      r_5(r_5(a-2) \cdot r_5(a+2)) =
			      \ub{r_5(( r_5(a) - 2) \cdot (r_5(a) + 2) )}{\llamada{1}{}} = 0\\
			      \flecha{el resto será}[ 0 cuando] r_5(a^2 + 1) = 0 \llamada{1}\sisolosi
			      r_5(( r_5(a) - 2) \cdot (r_5(a) + 2) ) = 0
			      \llave{l}{
				      r_5(a) = 2 \Leftrightarrow  \congruencia{a}{2}{5} \Tilde\\
				      r_5(a) = -2 \Leftrightarrow  \congruencia{a}{\ub{3}{\stackrel{(5)}\congruente -2}}{5} \Tilde\\
			      }
		      }$\\
	      \textit{Más aún}:\\
	      Para una congruencia módulo 5 habrá solo 5 posibles restos, por lo tanto se pueden ver todos los casos
	      haciendo una \textit{table de restos}.\\
	      $
		      \begin{array}{|c|c|c|c|c|c|}
			      \hline
			      a        & 0 & 1 & 2 & 3 & 4 \\ \hline\hline
			      r_5(a)   & 0 & 1 & 2 & 3 & 4 \\ \hline
			      r_5(a^2) & 0 & 1 & 4 & 4 & 1 \\ \hline
		      \end{array}
		      \to   $ La tabla muestra que para un dado $a\\ \to
		      r_5(a) =
		      \llaves{l}{
			      2 \sisolosi \congruencia{a}{2}{5} \sisolosi \congruencia{a^2}{4}{5} \sisolosi \congruencia{a^2}{-1}{5} \\
			      3 \sisolosi \congruencia{a}{3}{5} \sisolosi \congruencia{a^2}{4}{5} \sisolosi \congruencia{a^2}{-1}{5} \\
		      }$

	\item \hacer

	\item Me piden que exista una dada congruencia para todo $a \en \enteros$. Eso equivale a probar a que al dividir el \textit{lado izquierdo}
	      entre el \textit{divisor}, el \textit{resto} sea lo que está en el \textit{lado derecho} de la congruencia.\\
	      $\congruencia{a^7 - a}{0}{7}
		      \sisolosi \congruencia{a \cdot \ub{(a^6 - 1)}{(a^3 - 1) \cdot (a^3 + 1)}}{0}{7}
		      \sisolosi \congruencia{a \cdot (a^3 - 1) \cdot (a^3 + 1)}{0}{7}
		      \flecha{tabla de restos con}[sus propiedades lineales]\\
		      \begin{array}{|c|c|c|c|c|c|c|c|}
			      \hline
			      a            & 0 & 1 & 2 & 3 & 4 & 5 & 6 \\ \hline\hline
			      r_7(a)       & 0 & 1 & 2 & 3 & 4 & 5 & 6 \\ \hline
			      r_7(a^3 - 1) & 6 & 0 & 0 & 5 & 0 & 5 & 5 \\ \hline
			      r_7(a^3 + 1) & 1 & 2 & 2 & 0 & 2 & 0 & 0 \\ \hline
		      \end{array}
		      \to $ Cómo para todos los $a$, alguno de los factores del resto siempre se anula, es decir:\\
	      $r_7(a^7 - a) = r_7(r_7(a) \cdot  r_7(a^3 - 1) \cdot r_7(a^3 + 1)) = 0 \paratodo a \en \enteros$
	\item
	\item

\end{enumerate}

%12
\ejercicio
\separadorCorto

%13
\ejercicio
%%%%%% macro local
\def\cong7{\stackrel{(7)}\congruente }
%%%%%% finmacro local
Se define por recurrencia la sucesión $(a_n)_{n\en\naturales}$:\\
\[
	a_1 = 3,\, a_2 = -5 \ytext a_{n+2} = a_{n+1} - 6^{2n} \cdot a_n + 21^n \cdot n^{21} \text{, para todo } n \en \naturales.
\]
Probar que $\congruencia{a_n}{3^n}{\text{mod } 7} $ para todo $n\en\naturales$.\\
\separadorCorto
La infumabilidad de esos números me obliga a atacar a esto con el resto e inducción.\\
$\flecha{acomodo}[enunciado feo]
	r_7(a_{n+2}) = r_7( r_7(a_{n+1}) - \ub{r_7(36)^n}{\cong7 1} \cdot r_7(a_n) + \ub{r_7(21)^n}{\cong7 0} \cdot r_7(n)^{21}  ) =
	\ub{r_7(a_{n+2}) = r_7(a_{n+1}) - r_7(a_n)}{\llamada{1}\ } $\Tilde\\
Puesto de otra forma $ \congruencia{a_{n+2}}{a_{n+1} - a_n}{7}
	\to
	\llave{l}{
		\congruencia{a_1}{3^1}{7} \sisolosi \congruencia{a_1}{3}{7}   \\
		\congruencia{a_2}{3^2}{7} \sisolosi \congruencia{a_2}{2}{7}  \\
		\congruencia{a_3}{3^3}{7} \sisolosi \congruencia{a_3}{6}{7}  \\
	}$\\
Quiero probar que  $\congruencia{a_n}{3^n}{\text{mod } 7} \to$  inducción completa:\\

$ p(n)\ :\ \congruencia{a_n}{3^n}{\text{mod } 7} \paratodo n \en \naturales\\
	\llave{ll}{
		\text{Casos base: } &
		\llave{l}{
			p(n = 1)\ :\ \congruencia{a_1}{3^1}{7} \text{Verdadera}\\
			p(n = 2)\ :\ \congruencia{a_2}{3^2}{7}\cong7 2 \cong7 -5 \text{Verdadera}\\
		} \\

		\text{Paso Inductivo: } &
		\llave{l}{
			p(k)\ : \  \congruencia{a_k}{3^k}{\text{mod } 7} \text{ Verdadera }\\
			\y\\
			p(k+1)\ : \  \congruencia{a_{k+1}}{3^{k+1}}{\text{mod } 7} \text{ Verdadera }\\
			\entonces p(k+1)\ : \  \congruencia{a_{k+2}}{3^{k+2}}{\text{mod } 7} \text{ Verdadera?}\\
		} \\

		\text{Hipótesis inductiva: } &
		\llave{l}{
			\congruencia{a_k}{3^k}{\text{mod } 7} \\
			\congruencia{a_{k+1}}{3^{k+1}}{\text{mod } 7} \\
			\flecha{sumo}[pensando en $\llamada{1}{}$]
			\congruencia{\ub{a_{k+1} - a_k}{a_{k+2}}}{\ub{3^{k+1} - 3^k}{2\cdot 3^k}}{\text{mod } 7}\\
			\flecha{paso en}[limpio]  \congruencia{a_{k+2}}{\ub{9}{\cong7 2} \cdot 3^k}{7} \cong7 3^{k+2}
			\to  \boxed{\congruencia{a_{k+2}}{3^{k+2}}{7} }
		}
	}
$\\
Concluyendo como $p(1), p(2), p(k), p(k+1) \y p(k+2)$ resultaron verdaderas por el principio de inducción
$p(n)$ es verdadera $\paratodo n \en \naturales$.

%14
\ejercicio
\begin{enumerate}[label=\roman*)]
	\item Hallar el desarrollo en base 2 de
	      \begin{enumerate}[label=(\alph*)]
		      \item 1365
		      \item 2800
		      \item $3\cdot 2^{12}$
		      \item $13 \cdot 2^n + 5 \cdot 2^{n-1}$
	      \end{enumerate}
\end{enumerate}
\separadorCorto

%15
\ejercicio
\separadorCorto
%16
\ejercicio
\separadorCorto
%17
\ejercicio
\separadorCorto

\textit{\underline{Máximo común divisor:}}

%18
\ejercicio
En cada uno de los siguientes casos calcular el máximo común divisor entre $a$ y $b$ y escribirlo como combinación lineal entera de $a$ y $b$:
\begin{enumerate}[label=\roman*)]
	\item $a = 2532,\ b = 63$.
	\item $a = 131,\ b = 23$.
	\item $a = n^4 - 3,\ b = n^2 + 2\ (n \en \naturales)$.
\end{enumerate}

\separadorCorto

%19
\ejercicio
\separadorCorto
%20
\ejercicio
Sea $a \en \enteros$.
\begin{enumerate}[label=\roman*)]
	\item Probar que $(5a + 8 : 7a + 2) = 1 \otext 41.$ Exhibir un valor de $a$ para el cual da 1, y verificar
	      que efectivamente para $a = 23$ da $43$

	\item Probar que $(2\cdot a ^2 +3a: 5a +6) = 1 \otext 43.$ Exhibir un valor de $a$ para el cual da 1, y verificar
	      que efectivamente para $a = 16$ da $43$

	\item Probar que $(a^2-3a+2 : 3a^3 -5a^2) = 2 \otext 4.$ y exhibir un valor de $a$ para cada caso.\\
	      (Para este item es \textbf{indispensable} mostrar que el máximo común divisor nunca puede ser 1).
	      que efectivamente para $a = 16$ da $43$
\end{enumerate}
\separadorCorto
\begin{enumerate}[label=\roman*)]
	\item \red{Pasar!}
	\item \hacer
	\item
	      $(a^2-3a+2 : 3a^3 -5a^2) \flecha{Euclides} (\ub{a^2 - 3a + 2}{\llamada{1}{} par} : \ub{6a -8}{\llamada{1}{} par})\\
		      \flecha{busco}[divisor]
		      \llaves{l}{
			      d \divide a^2 - 3a + 2\\
			      d \divide 6a - 8\\
		      }
		      \flecha{$\times 6$}[$\times a$]
		      \llaves{l}{
			      d \divide 10a -12\\
			      d \divide 6a - 8\\
		      }
		      \flecha{$\times 6$}[$\times 10$]
		      \llaves{l}{
			      d \divide 8
		      }\to
		      \divsetP{8}{1,2,4,8} \llamada{1}= \set{2,4,8}\\
		      \llave{ll}{
			      a = 1 & (0:-2) = 2\\
			      a = 2 & (0:4) = 4
		      }$\\
	      \red{Parecido al hecho en clase.}\\
	      \red{¿Qué onda el 8? Hice mal cuentas? Si no, cómo lo descarto?}
\end{enumerate}


%21
\ejercicio Sean $a,b \en \enteros$ coprimos. Probar que $7a - 3b \ytext 2a -b$ son coprimos.

\separadorCorto

$\llaves{ccccc}{
		d \divide 7a-3b & \flecha{$\cdot 2$} &d \divide b  &\to&d \divide b\\
		d \divide 2a-b &\flecha{$\cdot 7$} & d \divide 2a-b & \to &d \divide a\\
	} \flecha{propiedad}[divisor y \mcd] d \divide (a:b) \flecha{$(a:b)$}[coprimos] d \divide 1\\
$ Por lo tanto $(7a - 3b  : 2a -b) = 1$ son coprimos como se quería mostrar.
%22
\ejercicio
\separadorCorto

%23
\ejercicio
\begin{enumerate}[label=\roman*)]
	\item Determinar todos los $a,b \en \enteros$ coprimos tales que $\frac{b+4}{a} + \frac{5}{b} \en \enteros$.
	\item Determinar todos los $a,b \en \enteros$ coprimos tales que $\frac{9a}{b} + \frac{7a^2}{b^2} \en \enteros$.
	\item Determinar todos los $a,b \en \enteros$ tales que $\frac{2a + 3}{a + 1} + \frac{a + 2}{4} \en \enteros$.
\end{enumerate}

\separadorCorto

\begin{enumerate}[label=\roman*)]
	\item
	      $\frac{b+4}{a} + \frac{5}{b} =
		      \frac{b^2 + 4b + 5a}{ab}
		      \flecha{quiero que}
		      ab \divide \frac{b^2 + 4b + 5a}{ab}\\
		      \flecha{por}[coprimitusibilidad]
		      \llave{l}{
			      a | b^2 + 4b +5a \\
			      b | b^2 + 4b +5a \\
		      } \to
		      \llave{l}{
			      a | b^2 + 4b  \\
			      b | 5a \\
		      } \flecha{es seguro que $b\noDivide a$}[debe dividr a 5]
		      \llave{l}{
			      a | b\cdot(b+4)  \\
			      b | 5 \\
		      }
	      $\\
	      Seguro tengo que $b = \set{\pm1, \pm5} \to$ pruebo valores de $b$ y veo que valor de $a$ queda:\\
	      $\llave{l}{
			      b = 1 \to (a\divide 5 ,1) \to \set{(\pm 1, 1).(\pm5,1)} \\
			      b = -1 \to (a \divide -3,1) \to \set{(\pm 1, -1).(\pm 3,1)} \\
			      b = 5 \to (a \divide 45,5) \flecha{\magenta{atención que}}[$(a:b) = 1$] \set{(\pm 1, 5), (\pm 3, 5).(\pm 9,5)} \\
			      b = -5 \to (a \divide 5,-5) \flecha{\magenta{atención que}}[$(a:b) = 1$] \set{(\pm 1, -5)}
		      }$

	\item \Hacer
	\item \hacer
\end{enumerate}

\textit{\underline{Primos y factorización: }}

%24
\ejercicio
\separadorCorto

%25
\ejercicio
Sea $p$ primo positivo.
\begin{enumerate}[label=\roman*)]
	\item Probar que si $0 < k < p \divide \binom{p}{k}$.
	\item Probar que si $a, b \en \enteros$, entonces $\congruencia{(a+b)^p}{a^p + b^p}{p}$.
\end{enumerate}
\separadorCorto



\end{document}
