\documentclass[12pt,a4paper, spanish]{article}
% Sacar draft para que aparezcan las imagenes.
% Opciones: 12pt, 10pt, 11pt, landscape, twocolumn, fleqn, leqno...
% Opciones de clase: article, report, letter, beamer...

% Paquetes:
% =========
\usepackage[headheight=110pt, top = 2cm, bottom = 2cm, left=1cm, right=1cm]{geometry} %modifico márgenes
\usepackage[T1]{fontenc} % tildes
\usepackage[utf8]{inputenc} % Para poder escribir con tildes en el editor.
\usepackage[english]{babel} % Para cortar las palabras en silabas, creo.
\usepackage[ddmmyyyy]{datetime}
\usepackage{amsmath} % Soporte de mathmatics
\usepackage{amssymb} % fuentes de mathmatics
\usepackage{array} % Para tablas y eso
\usepackage{caption} % Configuracion de figuras y tablas
\usepackage[dvipsnames]{xcolor} % Para colorear el texto: black, blue, brown, cyan, darkgray, gray, green, lightgray, lime, magenta, olive, orange, pink, purple, red, teal, violet, white, yellow.
\usepackage{graphicx} % Necesario para poner imagenes
\usepackage{enumitem} % Cambiar labels y más flexibilidad para el enumerate
\usepackage{tikz} % para graficar
\usepackage{cancel}
\usepackage{titlesec} % para editar titulos y hacer secciones con formato a medida
\usepackage{ulem}
\usepackage{centernot} % tacha cosas
\usepackage{bbding} % símbolos de donde uso FiveStar
\usepackage{skull} % símbolos de donde uso Skull
% \usepackage{lipsum}
\usepackage{soul} % para tachar en mathmode -> \hbox{\sout{$x+1$}}
\usepackage{polynom} % para división de polinomios

% para hacer los graficos tipo grafos
\usetikzlibrary{shapes,arrows.meta, chains, matrix, calc, trees, positioning, fit}
\usetikzlibrary{external}


\begin{document}
% Definiciones y nuevos comandos:
% =============
\def\partes{\mathcal P}
\def\relacion{\,\mathcal{R}\,}
\def\norelacion{\,\cancel{\relacion}\,}
\def\universo{\mathcal U}
\def\reales{\mathbb R}
\def\naturales{\mathbb N}
\def\enteros{\mathbb Z}
\def\complejos{\mathbb C}
\def\i{\text{i}}
\def\vacio{\varnothing}
\def\union{\cup}
\def\inter{\cap}
\def\y{\land}
\def\o{\lor}
\def\neg{\sim}
\def\entonces{\Rightarrow}
\def\sisolosi{\iff}
\def\clase{\overline}


\def\existe{\,\exists\,}
\def\noexiste{\,\nexists\,}
\def\paratodo{\forall}
\def\distinto{\neq}
\def\en{\in}
\def\talque{\;|\;}

% =====
\def\qvq{\text{ quiero ver que }}

%funciones
\def\imagen{\text{Im}}
\def\dominio{$\text{Dom}$}
\def\comp{\circ}
\def\inv{^{-1}}
\def\infinito{\infty}

% Llaves, paréntesis, contenedores
\newcommand{\llave}[2]{ \left\{ \begin{array}{#1} #2 \end{array}\right. }
\newcommand{\llaves}[2]{ \left\{ \begin{array}{#1} #2 \end{array} \right\} }
\newcommand{\matriz}[2]{\left( \begin{array}{#1} #2 \end{array} \right)}
\newcommand{\deter}[2]{\left| \begin{array}{#1} #2 \end{array} \right|}
\newcommand{\lista}[2][(1)]{\begin{enumerate}[\bf #1]\setlength\itemsep{-0.6ex} #2 \end{enumerate}}
\newcommand{\listal}[2][-0.6ex]{\begin{enumerate}[\bf(a)]\setlength\itemsep{#1} #2 \end{enumerate}}

% naturales
\newcommand{\sumatoria}[2]{\sum\limits_{#1}^{#2}}
\newcommand{\productoria}[2]{\prod\limits_{#1}^{#2}}
\newcommand{\kmasuno}[1]{\underbrace{#1}_{k+1\text{-ésimo}}}
\newcommand{\HI}[1]{\underbrace{#1}_{\text{HI}}}

% enteros
\def\divide{\,|\,}
\def\congruente{\, \equiv \,}
\newcommand{\congruencia}[3]{#1 \equiv #2 \;(\text{mod}\;#3)}
\newcommand{\divset}[2]{\mathcal{D}(#1) = \set{#2}}



% =====
% Miscelanea
% =====
\newcommand{\estabien}{{\color{blue} Consultado, está bien. \checkmark}}
\newcommand{\hacer}{{\color{black!30!red}Hacer!}}
\newcommand{\Hacer}{{\color{black!30!red}\Large Hacer!}}

\def\llamadaI{\stackrel{\cyan{$*^1$}}}
\def\llamadaII{\stackrel{\cyan{$*^2$}}}
\def\llamadaIII{\stackrel{\cyan{$*^3$}}}

% separador
\def\separador{\noindent\rule{\linewidth}{0.4pt}\\}
\def\separadorCorto{\noindent\rule{0.5\linewidth}{0.4pt}\\}

% sección ejercicio con su respectivo formato y contador
\newcounter{ejercicio}[subsubsection] % contador que se resetea en cada sección
\renewcommand{\theejercicio}{\arabic{ejercicio}} % el contador es un número arabic
\newcommand{\ejercicio}{%
	\stepcounter{ejercicio}% incremento en uno
	\titleformat{\section}[runin]{\normalfont\bfseries}{\theejercicio}{1em}{}%
	\section*{\noindent\theejercicio. \noindent}%
}

% Colores
\newcommand{\red}[1]{ {\color{red} \text{#1}}}
\newcommand{\green}[1]{ {\color{olive} \text{#1}}}
\newcommand{\blue}[1]{ {\color{blue} \text{#1}}}
\newcommand{\cyan}[1]{ {\color{cyan} \text{#1}}}
\newcommand{\magenta}[1]{ {\color{magenta} \text{#1}}}

% Conjuntos entre llaves
\newcommand{\set}[1] { \left\{ #1 \right\} }
\newcommand{\parentesis}[1] { \left( #1 \right) }

% Stackrel text
\newcommand{\stacktext}[2]{ \stackrel{\text{#1}}{#2} }
\def\eq?{\stackrel{\text{?}}}

% Flecha con texto
\NewDocumentCommand{\flecha}{m o}{%
	\IfNoValueTF{#2}{%
		\xrightarrow[]{\text{#1}}
	}{
		\xrightarrow[\text{#2}]{\text{#1}}
	}
}
 % idem con las definiciones

\pagestyle{empty} % Para que no muestre el número en pie de página

% Info para armar título.
\title{Práctica 7 de álgebra 1} % título
\author{D. Garraz} % autor
\date{last update: \today} % Cambiar de ser necesario

\maketitle  % para que aprezca el título en el documento

% Macro Local
\newcommand{\polGen}[1]{\sumatoria{i=0}{n} #1_i X^i}

\section{Definiciones y fórmulas útiles}

\begin{itemize}
	\item \textit{Operaciones: }

	      \begin{itemize}
		      \item[$+:$] Sean $f,g \en \K[X]$ con $f = \polGen{a}$ y $g = \polGen{b}$\\
		            $\entonces f + g = \sumatoria{i=0}{n} (a_i + b_i) X^i \en \K[X]$
		      \item[$\cdot:$] Sean $f,g \en \K[X]$ con $f = \polGen{a}$ y $g = \sumatoria{j=0}{m} b_j X^j$\\
                $\entonces f \cdot g = \sumatoria{k=0}{n+m} (\sumatoria{i+j=k}{} a_i \cdot b_j) X^k \en \K[X]$
	      \end{itemize}
        \item $(\K[X],+,\cdot) \text{ es un anillo conmutativo } \to f\cdot(g+h) = f\cdot g + f\cdot h,\, \paratodo f,\,g,\,h \en \K[X]$
	\item \textit{Algoritmo de división}: $f,g \en \K[X]$ no nulos, existen únicos $q$ y $R \en \K[X]$ tal que $f = q\cdot g + R$
	      con $gr(R) < gr(f)$ o $R = 0$

	\item $\alpha$ es raíz de $f \sisolosi X - \alpha \dividea f \sisolosi f = q \cdot ( X - \alpha)$

	\item \textit{Máximo común divisor: } Polinomio mónico de mayor grado que divide a ambos polinomios en $\K[X]$
	      y vale el algoritmo de Euclides.
	      \begin{itemize}
		      % \item $(f:g) \dividea f$ y $(f:g) \dividea g$

		      \item $f = (f:g)\cdot k_f$ y $g = (f:g)\cdot k_g$ con $k_f$ y $k_g$ en $\K[X]$

		      \item Dos polinomios son coprimos si $(f:g) = 1 \sisolosi f \distinto g$
	      \end{itemize}

	\item \textit{Raíces múltiples: } $f \en \K[x], \alpha \en \K$ es raíz de $f$ de multiplicidad
	      $m \en \naturales_0$ si $(X - \alpha)^m \dividea f$ y $(X - \alpha)^{m+1} \noDivide f$.
	      O sea, $f = (X - \alpha)^m \cdot q(\alpha) \distinto 0$

	\item Vale que $\alpha$ es raíz múltiple de $f \sisolosi f(\alpha) = 0$ y $f(\alpha) = 0 \sisolosi \alpha$
      es raíz de $(f:f'),\, X - \alpha \dividea (f:f')$

\end{itemize}

\subsubsection*{Ejercicios dados en clase:}
\ejercicio

\ejercicio



\newpage

%=========================
%=========================
%=========================
%=========================
%=========================
%=========================
% Ejercicios guia
%=========================

\section*{Ejercicios de la guía:}
\setcounter{ejercicio}{0} % Reset the custom counter

%1
\ejercicio

\setcounter{ejercicio}{8}
%9
\ejercicio
Calcular el máximo común divisor entre $f$ y $g$ en $\racionales[X]$ y escribirlo como combinación
polinomial de $f$ y $g$ siendo:
%Macro Local
\def\f1{X^5 + X^3 - 6X^2 + 2X +2}
\def\g1{X^4 - X^3 - X^2 + 1}
\begin{enumerate}[label=\roman*)]
	\item $f = \f1$, $g = \g1$,

	\item $f = X^6 + X^4 + X^2  +1$, $g = X^3 + X$,

	\item $f = 2X^6 - 4X^5 + X^4 + 4X^3 - 6X^2 + 4X + 1$, $g = X^5 - 2X^4 + 2X^2 - 3X + 1$,
\end{enumerate}

\separadorCorto



\begin{enumerate}[label=\roman*)]
	\item
	      \polylongdiv[style=D]{X^5 + X^3 - 6X^2 + 2X +2}{X^4 - X^3 - X^2 + 1}

	      $\flecha{Euclides} (f:g) = (g : 3X^3 -55X^2 +X + 1)\\
		      \flecha{escribo a $f$}[en función de $g$]
		      f = (X+1) \cdot g + 3X^3 -55X^2 +X + 1$

	      \polylongdiv[style=D]{X^4 - X^3 - X^2 + 1}{3X^3 -5X^2 +X + 1}

	      \polylongdiv[style=D]{3X^3 -5X^2 +X + 1}{- \frac{2}{9}X^2-\frac{5}{9}X+\frac{7}{9}}

	      \polylongdiv[style=D]{- \frac{2}{9}X^2-\frac{5}{9}X+\frac{7}{9}}{\frac{171}{4}X - \frac{171}{4}}

	      \polylonggcd{X^5 + X^3 - 6X^2 + 2X +2}{X^4 - X^3 - X^2 + 1}

	      El MCD será el último resto no nulo y mónico \boxed{\to (f : g) = X-1}\\

	      \red{Y ahora tengo que escribir $X-1 = F\cdot f + G\cdot g$?}\\
	      \red{Algún truco para no lidiar con esas fracciones?}\\

	\item \polylonggcd{X^6 + X^4 + X^2  +1}{X^3 + X}\\

	      El MCD será el último resto no nulo y mónico $\to$\boxed{ (f : g) = X^2+1}\\
	      El MCD escrito como combinación polinomial de $f$ y $g \to$ \boxed{ X^2 + 1 = f \cdot 1 + g \cdot (-X^3)}

	\item
	      $\flecha{Haciendo}[Euclides]$

	      \polylonggcd{2X^6 - 4X^5 + X^4 + 4X^3 - 6X^2 + 4X + 1}{X^5 - 2X^4 + 2X^2 - 3X + 1}\\
	      El MCD será el último resto no nulo y \textit{mónico} $\to$ \boxed{(f : g) = 1}\\
	      El MCD escrito como combinación polinomial de $f$ y $g \to$ \boxed{1 = \frac{1}{3} g \cdot(2X^2-4X+1) - \frac{1}{3} f \cdot (X-2)}\\

\end{enumerate}
%
%
\end{document}
