% Macro Local
\newcommand{\polGen}[1]{\sumatoria{i=0}{n} #1_i X^i}

\begin{itemize}
  \item \textit{Operaciones: }
        \begin{itemize}
          \item[$+:$] Sean $f,g \en \K[X]$ con $f = \polGen{a}$ y $g = \polGen{b}$\par
                $$
                  \entonces f + g = \sumatoria{i=0}{n} (a_i + b_i) X^i \en \K[X]
                $$
          \item[$\cdot:$] Sean $f,g \en \K[X]$ con $f = \polGen{a}$ y $g = \sumatoria{j=0}{m} b_j X^j$\\
                $\entonces f \cdot g = \sumatoria{k=0}{n+m} (\sumatoria{i+j=k}{} a_i \cdot b_j) X^k \en \K[X]$
        \end{itemize}

  \item \textit{Algoritmo de división}:\par
        $f, g \en \K[X]$ no nulos, existen únicos $q$ y $R \en \K[X]$ tal que
        $$
          f = q \cdot g + R
        $$
        con $\gr(R) < \gr(f)$ o $R = 0$.

  \item \textit{Raíz de un Polinomio:}\par
        $$
          \alpha \text{ es raíz de } f \sisolosi X - \alpha \divideA f \sisolosi f = q \cdot ( X - \alpha)
        $$

  \item \hypertarget{teoria-7:mcd}{\textit{Máximo común divisor:}}\par

        Polinomio, $(f:g) \en \K[X]$, \textit{mónico} de mayor grado que divide a ambos polinomios en $\K[X]$
        y vale el algoritmo de Euclides.
        \begin{itemize}[label = \tiny \faIcon{meh}]
          \item $(f:g) \divideA f \ytext (f:g) \divideA g$

          \item $f = (f:g) \cdot k_f  \ytext  g = (f:g) \cdot k_g \quad \text{con} \quad k_f  \ytext   k_g$ en $\K[X]$

          \item Dos polinomios son coprimos si $(f:g) = 1 \sisolosi f \distinto g$
        \end{itemize}

  \item \hypertarget{teoria-7:raicesMultiples}{\textit{Raíces múltiples:}}\par
        Sea $f \en \K[x]$ no nulo, y sea $\alpha \en \K$. Se dice que:
        \begin{itemize}[label=\tiny \faIcon{meh}]
          \item Cuando $f$ tiene una raíz múltiple:
                $$
                  \begin{array}{c}
                    \alpha\text{ es raíz \textit{múltiple} de } f \sisolosi f = (X - \alpha)^2 q \\
                    f = (X - \alpha)\cdot q,\ \text{ con } q \en \K[X]  \ytext  \magenta{ q(\alpha) = 0}.
                  \end{array}
                $$

          \item Cuando la raíz \textit{no} es múltiple, es \textit{simple} cuando:
                $$
                  \begin{array}{c}
                    \alpha \text{ es raíz \textit{simple} de } f \sisolosi (X - \alpha) \divideA f
                    \ytext (X - \alpha)^2 \taa{\red{!!}}{}\noDivide f \\
                    f = (X - \alpha)\cdot q,\ \text{ con } q \en \K[X]  \ytext  \magenta{ q(\alpha) \distinto 0}.
                  \end{array}
                $$
                Prestale atención a los \red{!} porque sino la vas a cagar.

          \item Sea $m \en \naturales_0$. Se dice que $\alpha$ es raíz de multiplicidad (exactamente)
                $m$ de $f$, y se nota:
                $$
                  \mult(\alpha;f) = m \sisolosi (X - \alpha)^m \divideA f,
                $$
                y también
                $$
                  (X - \alpha)^{m+1} \noDivide f.
                $$

                O equivalentemente,
                $$
                  f = (X - \alpha)^m q \quad \text{con} \quad q \en \K[X],  \ytext  \magenta{q (\alpha) \distinto 0}.
                $$

          \item \textit{Raíces y MCD:}\par
                Sean $f$, $g \en \K[X]$ no ambos nulos, y $\alpha \en \K$:
                \red{Esta se usa bastante}.
                $$
                  \entonces f(\alpha) = g(\alpha) = 0 \sii (f:g)(\alpha) = 0
                $$

          \item $\alpha$ es raíz múltiple de $f$ si y solo si:
                $$
                  f(\alpha) = 0  \ytext  f'(\alpha) =
                  0 \sisolosi \alpha \text{ es raíz de } (f:f') \sisolosi X - \alpha \divideA (f:f')
                $$
          \item La multiplicidad $m$ de una raíz, será $m-1$ en la derivada:
                $$
                  \mult(\alpha,f) = m \sisolosi f(\alpha) = 0  \ytext  \mult(\alpha;f') = m-1
                $$

          \item Relación entre la multiplicidad de una raíz de $f$ y sus derivadas: \par
                $\mult(\alpha;f) = m \sisolosi
                  \llave{rl}{
                    \mult(\alpha;f)\geq m &
                    \llaveInv{c}{
                      f(\alpha) = 0\\
                      \vdots\\
                      f^{(m-1)}(\alpha) = 0
                    } \\
                    \hline
                    \mult(\alpha;f) = m   &
                    \llaveInv{r}{
                      \  f^{(m)}(\alpha) \distinto 0 \magenta{\text{ la $m$-ésima derivada no se anula.}}
                    }
                  }$
                Todo ese quilombo de cosas lo que dice es por ejemplo, que si tenés una raíz $\alpha$ de $f$
                \textbf{triple} entonces la \textbf{tercera derivada} \red{NO PUEDE SER 0}, $f^{'''}(\alpha) \taa{\red{!!}}{}\distinto 0$.\par
                Pero tanto la función, su primera y segunda derivada DEBEN SER 0,
                $
                  f(\alpha) \igual{\red{!!}}
                  f'(\alpha) \igual{\red{!!}}
                  f^{''}(\alpha) \igual{\red{!!}} 0
                $

          \item \hypertarget{teoria-7:lema-gauss}{\textit{Lema de Gauss:}}\par
                Sea $f = a_nX^n + \cdots + a_0 \en \enteros[X]$ con $a_0 \distinto 0.$ \red{Si $\frac{\alpha}{\beta} \en \racionales[X]$ es una raíz
                  racional de $f$}, con $\alpha \ytext \beta \en \enteros$ coprimos, entonces $\alpha \divideA a_0 \ytext \beta \divideA a_n$.\par

                El \textit{Lema de Gauss} implica que en el conjunto de fracciones irreducibles $\frac{\alpha}{\beta}$ están \red{todas} las raíces
                racionales de $f$.

        \end{itemize}

  \item \hypertarget{teoria-7:irreducibles}{Polinomios irreducibles}:\par
        Sea $f \en K[X]$
        \begin{itemize}
          \item
                Se dice que f es \textit{irreducible} en $K[X]$ cuando $f \not\en K$ y los únicos divisores de $f$ son de la forma
                $g = c$ o $g = cf$ para algún $c \en K^\times$. O sea $f$ tiene únicamente dos divisores mónicos (distintos), que son
                1 y $\frac{f}{\cp(f)}$

          \item
                Se dice que $f$ es \textit{reducible} en $K[X]$ cuando $f \not\en K$ y $f$ tiene algún divisor $g \en K[X]$ con
                $g \distinto c$ y $g \distinto cf,\, \paratodo c \en K^{\times}$, es decir $f$ tiene algún divisor $g \en K[X]$ (no nulo
                por definición) con $0 < \gr(g) < \gr(f).$
        \end{itemize}
\end{itemize}
