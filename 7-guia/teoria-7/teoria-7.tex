% Macro Local
\newcommand{\polGen}[1]{\sumatoria{i=0}{n} #1_i X^i}
\newcommand{\mult}[1]{\text{mult}(#1)}


\subsubsection*{Un poco de teoría}

\begin{itemize}
	\item \textit{Operaciones: }
	      \begin{itemize}
		      \item[$+:$] Sean $f,g \en \K[X]$ con $f = \polGen{a}$ y $g = \polGen{b}$\\
		            $\entonces f + g = \sumatoria{i=0}{n} (a_i + b_i) X^i \en \K[X]$
		      \item[$\cdot:$] Sean $f,g \en \K[X]$ con $f = \polGen{a}$ y $g = \sumatoria{j=0}{m} b_j X^j$\\
		            $\entonces f \cdot g = \sumatoria{k=0}{n+m} (\sumatoria{i+j=k}{} a_i \cdot b_j) X^k \en \K[X]$
	      \end{itemize}
	\item $(\K[X],+,\cdot) \text{ es un anillo conmutativo } \to f\cdot(g+h) = f\cdot g + f\cdot h,\, \paratodo f,\,g,\,h \en \K[X]$

	\item \textit{Algoritmo de división}: $f,g \en \K[X]$ no nulos, existen únicos $q$ y $R \en \K[X]$ tal que $f = q\cdot g + R$
	      con $\gr(R) < \gr(f)$ o $R = 0$

	\item $\alpha$ es raíz de $f \sisolosi X - \alpha \divideA f \sisolosi f = q \cdot ( X - \alpha)$

	\item \textit{Máximo común divisor: } Polinomio mónico de mayor grado que divide a ambos polinomios en $\K[X]$
	      y vale el algoritmo de Euclides.
	      \begin{itemize}
		      \item $(f:g) \divideA f$ y $(f:g) \divideA g$

		      \item $f = (f:g)\cdot k_f$ y $g = (f:g)\cdot k_g$ con $k_f$ y $k_g$ en $\K[X]$

		      \item Dos polinomios son coprimos si $(f:g) = 1 \sisolosi f \distinto g$
	      \end{itemize}

	\item \textit{Raíces múltiples: }\\
	      Sea $f \en \K[x]$ no nulo, y sea $\alpha \en \K$. Se dice que:
	      \begin{itemize}
		      \item $\alpha$ es raíz \underline{múltiple} de $f \sii f = (x - \alpha)^2 q$ para algún $q \en \K[X]$

		      \item $\alpha$ es raíz \underline{simple} de $f \sii x - \alpha \divideA f$ en $\K[X]$,
		            pero $(X - \alpha)^2 \noDivide f$ en $\K[X] \sii f= (X - \alpha) q$
		            para algún $q \en \K[X]$ tal que $q(\alpha) \distinto 0$.

		      \item Sea $m \en \naturales_0$. Se dice que $\alpha$ es raíz de multiplicidad (exactamente)
		            $m$ de $f$, y se nota $\mult{\alpha;f} = m \sisolosi (X - \alpha)^m \divideA f$,
		            pero $(x - \alpha)^{m+1} \noDivide f$.\\
		            O equivalentemente, $f = (X - \alpha)^m q$ con $q \en \K[X]$,
		            pero $q(\alpha) \distinto 0$

		      \item Sea $f \en \K[X]$ no nulo $\mult{\alpha; f} \leq \gr{f}$:
	      \end{itemize}

	\item Vale que $\alpha$ es raíz múltiple de $f \sisolosi f(\alpha) = 0$ y $f'(\alpha) = 0 \sisolosi \alpha$
	      es raíz de $(f:f'),\, X - \alpha \divideA (f:f')$
	      \begin{itemize}
            \item $
              \mult{\alpha,f} = m 
              \sisolosi 
              f(\alpha) = 0$ y $\mult{\alpha;f'} = m-1$

		      \item $\mult{\alpha;f} = m \sisolosi
			            \llave{rl}{
				            \mult{\alpha;f}\geq m     &
				            \llaveInv{c}{
					            f(\alpha) = 0\\
					            \vdots\\
					            f^{(m-1)(\alpha) = 0}
				            }\\
				            \mult{\alpha;f} = m     &
				            \llaveInv{c}{
					            f^{m)(\alpha) \distinto 0}
				            }
			            }$
	      \end{itemize}
\end{itemize}
