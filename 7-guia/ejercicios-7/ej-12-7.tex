%!TEX root = ../7-sol.tex
\begin{enunciado}{\ejercicio}
Hallar la forma binomial de cada una de las raíces complejas del polinomio $f(X) = X^6 + X^3 - 2$.
\end{enunciado}

Primera raíz: $f(\alpha_1 = 1) = 0 \to f(X) = q(X) \cdot (X - 1)$. Busco $q(X)$ con algoritmo de división.\\
\polylongdiv[style=D]{X^6 + X^3 - 2 }{X - 1}\\

El cociente $q(X) = X^5 + X^4 + X^3 + 2X^2 + 2X + 2$ se puede factorizar en grupos como\\
$q(X) = (X^2+X+1) \cdot (X^3 + 2)$. Entonces las 5 raíces que me faltan para tener las 6 que debe tener $f \en \complejos[X]$
salen de esos dos polinomios.\\

$
	X^2 + X +1 = 0 \entonces
	\llave{l}{
		\alpha_2 = -\frac{1}{2} + \frac{\sqrt{3}}{2}\\
		\alpha_3 = -\frac{1}{2} - \frac{\sqrt{3}}{2}
	}
$

$
	X^3 + 2 = 0 \flecha{exponencial}[$X = re^{i\theta}$]
	\llaves{l}{
		r^3 = 2 \to r = \sqrt[3]{2}\\
		3\theta = \pi + \magenta{2k\pi} \to \theta = \frac{\pi}{3} + \frac{2k\pi}{3} \text{ con } k = 0,\, 1,\, 2.
	} \to
	\llave{l}{
		\alpha_4 = \sqrt[3]{2} e^{i \frac{\pi}{3}} = \sqrt[3]{2} (\frac{1}{2} + i \frac{\sqrt{3}}{2})    \\
		\alpha_5 = \sqrt[3]{2} e^{i \pi}  = -\sqrt[3]{2}\\
		\alpha_6 = \sqrt[3]{2} e^{i \frac{5\pi}{3}} = \sqrt[3]{2} (\frac{1}{2} - i \frac{\sqrt{3}}{2})
	}
$

