\begin{enunciado}{\ejercicio}
  Sea $n \en \naturales,\, n\geq 3$. Hallar el resto de la división de $X^{2n} + 3X^{n+1} + 3X^n - 5X^2 +2X + 1$
  por $X^3 - X$ en $\racionales[X]$.
\end{enunciado}

Es parecido al ejercicio \ref{guia7-ej10}? Creo que sí:
$$
  \llave{l}{
    f(X) = X^{2n} + 3X^{n+1} + 3X^n - 5X^2 +2X + 1 \\
    g(X) \igual{\red{!!}} X \cdot (X-1) \cdot (X+1)
  }
  \entonces
  f = q(X) \cdot g(X) + r(X)
  \text{ con }
  \gr(\ub{aX^2 + bX + c}{r(X)} ) \menorIgual{\red{!}} 2
$$
Evalúo para armar un sistema:
$$
  \llave{l}{
    f(0) = q(0) \cdot \ub{g(0)}{=0} + r(0) = 1\\
    f(1) = q(1) \cdot \ub{g(1)}{=0} + r(1) = 3\\
    f(-1) = q(-1) \cdot \ub{g(-1)}{=0} + r(-1) = 1 + 3(-1)^{n+1} + 3(-1)^n - 5 -2 + 1 =
    \llave{rl}{
      2 & n \text{ impar }\\
      1 & n \text{ par }
    }
  }
$$
Habemos sistemus de ecuaciunus para encontrar a $r(X)$:
$$
  \llave{l}{
    r(0) = c = 1 \\
    r(1) = a + b + 1 = 3 \to a+b = 2 \\
    r(-1) = a - b + 1 =
    \llave{rl}{
      2 \to a - b = 1 & n \text{ impar }\\
      1 \to a - b = 0 & n \text{ par }
    }
  }
$$
Nuevamente el uso de matrices es totalmente opcional. Entonces resuelvo dos sistemitas según la paridad de $n$:
$$
  \llave{cccc}{
    \flecha{$n$}[impar] &
    \matriz{cc|c}{
      1 & 1 & 2  \\
      1 & -1 & 1 \\
    }
    \sii
    \matriz{cc|c}{
      1 & 0 & \frac{3}{2}  \\
      0 & 1 & \frac{1}{2} \\
    }
    &\to&
    \cajaResultado{
      r_{impar}(X) = \frac{3}{2}X^2 + \frac{1}{2}X + 1
    }
    \\
    \flecha{$n$}[par] &
    \matriz{cc|c}{
      1 & 1 & 2  \\
      1 & -1 & 0 \\
    }
    \sii
    \matriz{cc|c}{
      1 & 0 & 1  \\
      0 & 1 & 1 \\
    }
    &\to&
    \cajaResultado{
      r_{par}(X) = X^2 + X + 1
    }
  }
$$

\begin{aportes}
  \item \aporte{\dirRepo}{naD GarRaz \github}
\end{aportes}
