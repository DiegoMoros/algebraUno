\ejercicio
Determinar la multiplicidad de $a$ como raíz de $f$  en los casos
\begin{enumerate}[label=\roman*)] 
 \item $f = X^5 - 2X^3 + X$,\quad $a=1$,\\ 
 \item $f = X^6 - 3X^4 + 4$,\quad $a=i$,\\ 
 \item $f = (X-2)^2(X^2-4) + (X-2)^3(X-1)$, \quad $a=2$,\\ 
 \item $f = (X-2)^2(X^2-4) - 4(X-2)^3$, \quad $a=2$,\\ 
 \end{enumerate}

 \separadorCorto

\begin{enumerate}[label=\roman*)] 
 \item $f = X^5 - 2X^3 + X$,\quad $a=1$,\\ 
\red{Pasar}

 \item $f = X^6 - 3X^4 + 4$,\quad $a=i$,\\ 
   \divPol{X^6 - 3X^4 + 4}{X^2 + 1}
\red{Pasar}

   \hacer
 \item $f = (X-2)^2(X^2-4) + (X-2)^3(X-1)$, \quad $a=2$,\\ 
\red{Pasar}

 \item $f = (X-2)^2(X^2-4) - 4(X-2)^3$, \quad $a=2$,\\ 
\red{Pasar}
 \end{enumerate}
