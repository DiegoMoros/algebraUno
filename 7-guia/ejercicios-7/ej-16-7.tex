\ejercicio
Determinar la multiplicidad de $a$ como raíz de $f$  en los casos
\begin{enumerate}[label=\roman*)] 
 \item $f = X^5 - 2X^3 + X$,\quad $a=1$,
 \item $f = X^6 - 3X^4 + 4$,\quad $a=i$, 
 \item $f = (X-2)^2(X^2-4) + (X-2)^3(X-1)$, \quad $a=2$, 
 \item $f = (X-2)^2(X^2-4) - 4(X-2)^3$, \quad $a=2$.
 \end{enumerate}

 \separadorCorto

\begin{enumerate}[label=\roman*)] 
 \item $f = X^5 - 2X^3 + X$,\quad $a=1$,\\ 
   Todos casos de factoreo:\\
    $f = X^5 - 2X^3 + X =
    X(X^4-2X^2 + 1) = 
    X(X^2 - 1)^2 =
    X (X - 1)^{\red{2}} (X + 1)^2 =
    $
    \\
    \boxed{\text{ La multiplicidad de $a=1$ como raíz es 2.}}

 \item $f = X^6 - 3X^4 + 4$,\quad $a=i$,\\ 
   Si $a=i$ es raíz, entonces $-i$ también lo es en un polinomio $\reales[X]$\\
   \divPol{X^6 - 3X^4 + 4}{X^2 + 1}\\
    $f =(X^2 + 1)(X^4 - 4X^2 + 4) =
    (X^2 + 1)(X^2-2)^2 =
    (X^2 + 1)(X - \sqrt2)^2 (X + \sqrt2)^2 =\\
    (X - i)^{\red{1}}(X + i)(X - \sqrt2)^2 (X + \sqrt2)^2 =
    $
    \\
    \boxed{\text{ La multiplicidad de $a=i$ como raíz de $f$ es 1.}}

 \item $f = (X-2)^2(X^2-4) + (X-2)^3(X-1)$, \quad $a=2$,\\ 
   $
    f =(X-2)^3 ( (X+2) + (X+1)) =
    (X-2)^3 (2X+3)
    $
    \\
    \boxed{\text{La multiplicidad de $a=2$ como raíz de $f$ es 3.}}

 \item $f = (X-2)^2(X^2-4) - 4(X-2)^3$, \quad $a=2$,\\ 
$
 f = 
    (X-2)^2(X^2-4) - 4(X-2)^3 =
    (X-2)^2(X-2)(X+2) - 4(X-2)^3 =\\
    (X-2)^3(X+2 - 4) =
    (X-2)^4
    $
    \\
    \boxed{\text{La multiplicidad de $a=2$ como raíz de $f$ es 4.}}

 \end{enumerate}
