\ejercicio

Sea $f = X^{20} + 8X^{10} + 2a$. Determinar todos los valores de
$a \en \complejos$ para los cuales $f$ admite una raíz múltiple en
$\complejos$. Para cada valor hallado determinar cuántas raíces
distintas tiene $f$ y la multiplicidad de cada una de ellas.

\separadorCorto

Si $f$ tiene raíces múltiples
$
	\alpha_k
	\sii
	f(\alpha_k) = f'(\alpha_k) =  0$,
por lo tanto  tanto comienzo buscando las raíces de $f'$ para sacarme ese
$a$ de en medio.\\

$
	f' = 20X^{19} + 80 X^9 =
	20 X^9 (X^{10} + 4)
	\entonces
	f' = 0
	\sii
	\llave{l}{
		X = 0\\
		X^{10} = -4
		\sii
		X = \sqrt[10]{4} e^{i\frac{2k+1}{10}\pi}\ k\en \enteros_{[0,9]}
	}
$\\
Hay de momento 11 raíces de $f'$. Me interesa saber si son raíces de $f$:\\
$
	f(0) = 2a
	\entonces
	f(0) = 0
	\sii
	a = 0\\
	f =
	(X^{10})^2 + 8X^{10} + 2a
	\entonces
	f(\alpha = \magenta{X^{10} = -4}) =
	(\magenta{-4})^2 + 8(\magenta{-4}) + 2a =
	-16 + 2a = 0
	\sii
	a = 8
$\\

Entonces:\\
Si
$
	a = 0
	\entonces
	f = X^{10}(X^{10} + 8)\\
	\entonces
	f = 0
	\sii
	X = 0 \otext X^{10} = -8
$, donde
\boxed{\mu(0;f) = 10} y
\boxed{\mu(\sqrt[10]{8}e^{i\frac{2k+1}{10}\pi});f) = 1\ k \en \enteros_{[0-9]}}.\\
11 raíces distintas.\\

Si
$
	a = 8
	\entonces
	f = X^{20} + 8X^{10} + 16 = (X^{10} + 4)^2\\
	\entonces
	f = 0
	\sii
	X^{10} = -4
$, donde
\boxed{\mu(\sqrt[10]{4}e^{i\frac{2k+1}{10}\pi});f) = 2\ k \en \enteros_{[0-9]}}.\\
10 raíces distintas.
