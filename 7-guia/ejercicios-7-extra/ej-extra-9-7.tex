\begin{enunciado}{\ejExtra}
  Hallar $f \en \racionales[X]$ de grado mínimo tal que cumpla las siguientes condiciones
  \begin{itemize}
    \item $f$ comparte una raíz con $X^3 - 3X^2 + 7X -5$
    \item $X+3-\sqrt{2} \divideA f$,
    \item $1-2 i$ es raíz de $f$ y $f'(1-2i) = 0$
  \end{itemize}
\end{enunciado}

Vamos con la primera. Si dos polinomios , $f$ y $g = X^3 - 3X^2 + 7X -5$, comparten raíz
buscamos raíces de $g$ con el \hyperlink{teoria-7:lema-gauss}{\textit{lema de Gauss}} de donde
tomaremos las raíces que nos sirvan para construir nuestro $f$ \textit{mónico y de grado mínimo}:
$A = \set{\pm 1, \pm 5}$, con $\alpha = 1 \entonces g(1) = 0 \Tilde$.\par
Como $\alpha = 1$ es raíz, entonces $X-1 \divideA g$:

$$
  \divPol{X^3 - 3X^2 + 7X -5}{X-1}
$$

$g = (X - 1) \cdot (X^2 -2X + 5)$, busco raíces del cociente $X^2 - 2X + 5$, usando resolvente
$$
  r_{+,-} = \frac{2 \pm w}{2}, \text{ con } w^2 = -16 \to
  \llave{l}{
    r_+ = 1 + 2i \\
    r_- = 1 - 2i.
  }
$$

Finalmente,

$$
g
\igual{$\llamada1$}
(X - 1) \cdot \ub{(X - (1+2i)) \cdot (X - (1-2i))}{X^2 -2X + 5} \Tilde,
$$

antes de elegir cuales de estas raíces serán comunes a $f$

es recomendable estudiar las otras condiciones del enunciado.\par\medskip

$X + 3 - \sqrt{2} = X - (-3 + \sqrt{2}) \divideA f$, por lo que $(-3 + \sqrt{2})$ es una raíz de $f$ y dado que
$f \en \racionales[X]$ también \red{debe estar} el conjugado irracional $-3 - \sqrt{2}$.

$$
  \llave{c}{
    X - (-3 + \sqrt{2}) \divideA f \\
    \ytext                         \\
    X - (-3 - \sqrt{2}) \divideA f
  }
  \Sii{\red{!}}
  X^2 + 6X + 7 \divideA f \Tilde.
$$

La tercera condición tiene \textit{mucha data}. Nos da una raíz compleja de $f$, por lo cual también tendremos
su conjugado complejo porque $f \en \racionales[X]$.

Esa raíz es una de las que está en  $ g \llamada1$.\par

El dato de $f'$, también nos indica que la multiplicidad de $1 - 2i$ como 
raíz es por lo menos 2, ya que $f'(1 - 2i) = 0$, y por lo tanto $\mult(1 + 2i;f)$ también será por lo menos 2. \par\medskip

Tenemos todo para armar a $f$:

$$
f =  (X^2 - 2X + 5)^2 \cdot (X^2 + 6X + 7) \Tilde
$$


