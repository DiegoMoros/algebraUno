\begin{enunciado}{\ejExtra}
	Factorice en irreducibles de $\racionales[X], \reales[X], \text{ y } \complejos[X]$ el polinomio
	$$
		f = x^5 - x^4 -6x^3 + 12 x^2 + 40 x + 32,
	$$
	sabiendo que tiene alguna raíz en común con $g = x^4 - x^3 - 9x^2 - 16x - 10$.
\end{enunciado}

Como los polinomios comparten una raíz, sé que $(f : g) \distinto 1$. Usando al crack, titán de Euclides busco:
$$
	(f:g) \text{ dado que } (f:g) \divideA f  \ytext (f:g) \divideA g
$$
y de ahí voy a sacar las raíces hermosas esas que tanto necesito.
$$
	\divPol{x^5 - 4x^4 -6x^3 + 12 x^2 + 40 x + 32}{x^4 - x^3 - 9x^2 - 16x - 10}
$$

$(f : g ) = (x^4 - x^3 - 9x^2 - 16x - 10 : x^2 + 2x + 2)$, sigo con Euclides:

$$
	\divPol{x^4 - x^3 - 9x^2 - 16x - 10}{x^2 + 2x +2}
$$

Este último resultado confirma que:
$$
	(f:g) = x^2 + 2x + 2 \igual{\red{!!}} (-1 + i)\cdot(-1 - i)
$$
Reduzco a $f$ para buscar más raíces:
$$
	\divPol{x^5 - 4x^4 -6x^3 + 12 x^2 + 40 x + 32}{x^2 + 2x + 2}
$$
De esta manera puedo escribir:
$$
	f = (x^2 + 2x + 2) \cdot (x^3 - 6x^2 + 4x + 16)
$$
\rollingEyes con el \textit{lema de Gauss} posibles raíces de:
$$
	x^3 - 6x^2 + 4x + 16 \to \set{\pm1, \pm 2, \pm 4, \pm 8, \pm 16}.
$$
De las cuales funciona el 4 \rollingEyes.

Vuelvo a dividir \rollingEyes:
$$
	\divPol{x^3 - 6x^2 + 4x + 16}{x-4}
$$
Podemos reescribir {\LARGE \rollingEyes}:
$$
	f = (x^2 + 2x + 2) \cdot (x-4) \cdot (x^2 -2x - 4)
$$
{\huge \rollingEyes} el último factor tiene raíces $1 - \sqrt{5}$ y $1 + \sqrt{5}$ y ya escribo $f$ en la factorizaciones pedidas:
$$
	\cajaResultado{
		\begin{array}{rcll}
			f & = & (x^2 + 2x + 2) \cdot (x - 4) \cdot (x^2 -2x - 4)                               & \en \racionales[X] \\
                f & = & (x^2 + 2x + 2) \cdot (x - 4) \cdot (x - (1 - \sqrt{5})) \cdot (x - (1 + \sqrt{5}))         & \en \reales[X]     \\
                f & = & (x - (-1 + i)) \cdot (x - (-1 - i)) \cdot (x - 4) \cdot (x - (1 - \sqrt{5})) \cdot (x - (1 + \sqrt{5})) & \en \complejos[X]
		\end{array}
	}
$$

\begin{aportes}
	\item \aporte{\dirRepo}{naD GarRaz \github}
\end{aportes}
