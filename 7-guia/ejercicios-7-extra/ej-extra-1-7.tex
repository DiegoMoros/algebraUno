\ejExtra

\def\c{\textbf{c}}
\begin{enumerate}[label=\alph*)]
	\item Hallar todos los posibles $\c \en \reales,\, \c > 0$ tales que:
	      $$ f = X^6 - 4X^5 - X^4 + 4X^3 + 4X^2 + 48X + \c$$ tenga una raíz de argumento $\frac{3\pi}{2}$

	\item Para cada valor de $\c$ hallado, factorizar $f$ en $\racionales[X],\, \reales[X]$ y $\complejos[X]$, sabiendo que tiene al menos una raíz doble.
\end{enumerate}

\separadorCorto
\begin{enumerate}[label=\alph*)]
	\item
	      Si la raíz $\alpha = r e^{i\frac{3\pi}{2}} = r(-i) \entonces f(r(-i)) = 0$\\
	      Voy a usar que:
	      $\llamada1
		      \llave{rcr}{
			      (-i)^2 & = &  -1\\
			      (-i)^3 & = &  i\\
			      (-i)^4 & = &  1\\
			      (-i)^5 & = &  -i\\
			      (-i)^6 & = &  -1
		      }$

	      $
		      f(r(-i)) =
		      (r(-i))^6 - 4(r(-i))^5 - (r(-i))^4 + 4^3 + 4(r(-i))^2 + 48(r(-i)) + \c \stackrel{\llamada1}=\\
		      -r^6 + 4r^5i - r^4 - 4 r^3i - 4r^2 - 48ri + \c = 0
		      \sisolosi
		      \llave{l}{
			      \re : -r^6  - r^4  - 4r^2  + \c = 0 \entonces \c = r^6  + r^4  + 4r^2\\
			      \im : r(4r^4- 4 r^2- 48) = 0
			      \flecha{bicuadrática}[$r^2 = y$ y $r\en\reales_{> 0}$]
			      r^2 = 3
		      }
	      $

	      Por lo tanto si
	      $\c =
		      r^6  + r^4  + 4r^2 =
		      (r^2)^3  + (r^2)^2  + 4r^2
		      \entonces
		      \boxed{\c = 48}\Tilde
	      $\\
	      con raíces $\pm \sqrt{3}i$ dado que $f\en \racionales[X]$


	\item

	      Debe ocurrir que $(X - \sqrt{3}i))(X + \sqrt{3}i) = X^2 + 3 \divideA f$\\
	      \divPol{X^6 - 4X^5 - X^4 + 4X^3 + 4X^2 + 48X + 48}{X^2 + 3}\\

	      $f = (X^2 + 3)\ub{(X^4 - 4X^3 - 4X^2 + 16X + 16)}{q}$ como $f$
	      tiene al menos una raíz doble puedo ver las raíces de la derivada de $q$:\\
	      $
		      q' = (4X^3 - 12X^2 - 8 X + 16)' =
		      4(X^3 - 3X^2 - 2X + 4) = 0
		      \flecha{Posibles raíces, Gauss :(}[$\pm1,\pm2,\pm4$]
		      q'(1) = 0 \text{, pero }
              g(\magenta{1})\distinto 0 \entonces f(1) \distinto 0 \\
		      \flecha{divido para}[bajar grado]
	      $
	      \divPol{X^3 - 3X^2 - 2X + 4}{X - 1}
	      $\\
		      g' = 4(X-1)\ub{(X^2 - 2X -4)}{ = h}
		      \flecha{busco raíces}[de $h$]
		      X^2 - 2X - 4 = 0
		      \sisolosi
              \alpha_{1,2} = 1 \pm \sqrt{5}\\
		      h =
              (X - (1 + \sqrt{5})) \cdot (X - (1 - \sqrt{5}) =
              X^2 - 2X - 4
	      $
          Para calcular que $f(\alpha_1)= g(\alpha_1) = 0$ y comprobar que es una raíz doble, puedo hacer:
          \divPol{X^4 - 4X^3 - 4X^2 + 16X + 16}{X^2 - 2X - 4} \checkmark
          $g = h^2 =  (X^2 - 2X - 4)^2 \to \text{no la vi venir}$ \\


          \textit{factorizaciones: }\\
          \boxed{
          \llave{rcl}{
            \racionales[X] & \to & f = \parentesis{X^2 + 3} (X^2 + 3)(X^2 - 2X - 4)^2\\
            \reales[X] & \to & f = (X - (1 + \sqrt{5})) (X - (1 - \sqrt{5})) (X^2 - 2X - 4)^2\\
            \complejos[X] & \to & f = (X - (1 + \sqrt{5})) (X - (1 - \sqrt{5})) (X-3i)^2 (X+3i)^2
          }
        }\Tilde
          
\end{enumerate}
