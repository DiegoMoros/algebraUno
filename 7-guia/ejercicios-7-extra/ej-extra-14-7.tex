\begin{enunciado}{\ejExtra}
  Sea $f = X^5 - \big(\frac{1}{2} + \frac{\sqrt{3}}{2}i \big)$.
  \begin{enumerate}[label=\alph*)]
    \item Probar que $ f \divideA X^{30} - 1$
    \item Hallar el polinomio $g \en \reales[X]$ mónico de grado mínimo tal que $f \divideA g$.
  \end{enumerate}
\end{enunciado}

\begin{enumerate}[label=\alph*)]
  \item
        Las raíces de $f$:
        $$
          X^5 - (\frac{1}{2} + \frac{\sqrt{3}}{2}i ) = 0
          \sii X^5 = e^{i \frac{\pi}{3}}
          \Sii{\red{!!}}
          X \en
          \set{
            e^{i \frac{1}{15} \pi},
            e^{i \frac{7}{15} \pi},
            e^{i \frac{13}{15} \pi},
            e^{i \frac{19}{15} \pi},
            e^{i \frac{5}{3} \pi}
          }
        $$

        Si $X^5 - (\frac{1}{2} + \frac{\sqrt{3}}{2}i )  \divideA X^{30} - 1$,
        entonces  las raíces de $f$ también deben ser raíces de $X^{30} - 1$.

        Observad que:
        $$
          X^{30} - 1 =
          (X^{15} - 1) \cdot (X^{15} + 1) =
        $$
        Y ya veo tu cara así {\Large \surprise}! Sí, todos los elementos de $G_{15}$ son raíces de $G_{30}$

        \bigskip
        \begin{center}
          En general,:
          $$
            \text{Si se tienen 2 conjuntos }  G_n \ytext G_m \quad \text{con} \quad m \divideA n
            \entonces
            G_m \subseteq G_n
          $$
        \end{center}

        \bigskip
        Todas la raíces de $f$ están en $G_{15}$, así que $f \divideA X^{30} - 1$

        \bigskip

        \textit{Versión de galera y bastón:}
        La versión más elegante, pero que no se me ocurre primero ni a palos (mirá el ejercicio \ref{ej:8} {\tiny$\ot$ click}):
        Sabemos que:
        $$
          \begin{array}{rcl}
            x - \Big(\frac{1}{2} + \frac{\sqrt{3}}{2}i \Big) \divideA x^6 - \Big(\frac{1}{2} + \frac{\sqrt{3}}{2}i\Big)^6
             & \Entonces{$x \to X^5$}[$a \to \alpha$] &
            X^5 - \Big(\frac{1}{2} + \frac{\sqrt{3}}{2}i \Big) \divideA (X^{5})^6 - \Big(\frac{1}{2} + \frac{\sqrt{3}}{2}i \Big)^6 \\
             & \Sii{\red{!!}}                         &
            X^5 - \Big(\frac{1}{2} + \frac{\sqrt{3}}{2}i \Big) \divideA X^{30} - 1
          \end{array}
        $$
        Por ende el polinio $f$ divide a $X^{30} - 1$.

  \item $f \divideA g$
        Las raíces de $f$ son todas complejas, así que voy a necesitar los conjugados para tener un $g \en \reales[X]$.
        Esto va a ser útil:
        $$
          \boxed{
            (X - z)(X- \conj{z})
            =
            X^2 - 2\re(z)X + |z|^2
          }\llamada1
        $$
        Las raíces son:
        $$
          \set{
            e^{i \frac{1}{15} \pi},
            e^{i \frac{7}{15} \pi},
            e^{i \frac{13}{15} \pi},
            e^{i \frac{19}{15} \pi},
            e^{i \frac{25}{15} \pi}
          }
          \Entonces{agrego los}[conjugados]
          \set{
            e^{i \frac{1}{15} \pi},
            e^{i \frac{1}{3} \pi},
            e^{i \frac{7}{15} \pi},
            e^{i \frac{11}{15} \pi},
            e^{i \frac{13}{15} \pi},
            e^{i \frac{17}{15} \pi},
            e^{i \frac{19}{15} \pi},
            e^{i \frac{23}{15} \pi},
            e^{i \frac{5}{3} \pi},
            e^{i \frac{29}{15} \pi}
          }
        $$
        Armo el polinomio con esta bosta usando la expresión en $\llamada1$:
        $$
          \scriptstyle
          g =
          (X^2 -2\cos(\frac{1}{15}\pi) + 1) \cdot
          (X^2 -2\cos(\frac{1}{3}\pi) + 1) \cdot
          (X^2 -2\cos(\frac{7}{15}\pi) + 1) \cdot
          (X^2 -2\cos(\frac{11}{15}\pi) + 1) \cdot
          (X^2 -2\cos(\frac{13}{15}\pi) + 1)
        $$

        Listo? Esto es la respuesta? \textit{Tengo miedo, estoy cansado, jefe}.

\end{enumerate}

\begin{aportes}
  \item \aporte{\dirRepo}{naD GarRaz \github}
\end{aportes}
