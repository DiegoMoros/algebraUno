\begin{enunciado}{\ejExtra}
  Determinar todos los primos $p$  positivos tales que el polinomio
  $$
    f = p X^3 - X^2 + 13X - 1
  $$
  tenga al menos una raíz racional positiva. Para cada valor de $p$ hallado, factorizar $f$ como producto de polinomios irreducibles en
  $\racionales[X],\,\reales[X] \ytext \complejos[X]$.
\end{enunciado}
El \hyperlink{teoria-7:lema-gauss}{lema de Gauss} dice que las raíces racionales que el polinomio puede tener, tienen que estar en el
conjunto:
$$
  \set{\pm 1, \pm \frac{1}{13}}
$$

Si $p$ tiene que ser primo y positivo entonces $p = 13$:
$$
  \begin{array}{rcl}
    f = 13 X^3 - X^2 + 13X - 1 & = & 13 X \cdot (X^2 - 1) - (X^2 - 1)                      \\
                               & = & (13 X - 1) \cdot (X^2 - 1)                            \\
                               & = & 13\cdot(X - \frac{1}{13}) \cdot (X - 1) \cdot (X + 1)
  \end{array}
$$
Todo lindo las raíces:
$$
  f = 13\cdot(X - \frac{1}{13}) \cdot (X - 1) \cdot (X + 1) = 0
  \sii
  \llave{rcl}{
    X_1 & = & \frac{1}{13} \\
    X_2 & = & 1            \\
    X_3 & = & -1           \\
  }
$$
Y factorizado en $\racionales[X],\, \reales[X] \ytext \complejos[X]$ queda.
$$
  f = 13\cdot(X - \frac{1}{13}) \cdot (X - 1) \cdot (X + 1)
$$

\begin{aportes}
  \item \aporte{https://github.com/nad-garraz}{Nad Garraz \github}
\end{aportes}
