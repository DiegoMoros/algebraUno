\begin{enunciado}{\ejExtra}
  Determinar todos los primos $p$  positivos tales que el polinomio
  $$
    f = p X^3 - X^2 + 13X - 1
  $$
  tenga al menos una raíz racional positiva. Para cada valor de $p$ hallado, factorizar $f$ como producto de polinomios irreducibles en
  $\racionales[X],\,\reales[X] \ytext \complejos[X]$.
\end{enunciado}
El \hyperlink{teoria-7:lema-gauss}{lema de Gauss} dice que las raíces racionales que el polinomio puede tener, tienen que estar en el
conjunto de los divisores del \textit{coeficiente principal} $p$ y el \textit{termino independiente} $-1$:
$$
  \set{\pm 1,\pm \frac{1}{2},\pm \frac{1}{3},\pm \frac{1}{5},\pm \frac{1}{7}, \dots, \pm \frac{1}{p}}
$$
Ahora hay que hacer cuentas para todos los primos y ver cuál funciona, \textit{nah, mentira}.
Si $\frac{1}{p}$ es raíz entonces hay que dividir {\tiny ($p^{-1} = \frac{1}{p}$, boludeces, no!)}:
$$
  \divPol{pX^3 - X^2 + 13X -1}{X-\frac{1}{p}}
$$
Y a esto hay que pedirle que el \magenta{resto sea 0}, porque $\frac{1}{p}$ es raíz racional:
$$
  -1 + \frac{13}{p} = 0 \sii p = 13
$$

Si $p$ tiene que ser primo y positivo entonces $p = 13$, usando el resultado de la división:
$$
  \begin{array}{rcl}
    f = 13 X^3 - X^2 + 13 X - 1 & = & 13 (X - \frac{1}{13}) \cdot (X^2 + 1)             \\
                                & = & 13 (X - \frac{1}{13}) \cdot (X - i) \cdot (X + i) \\
  \end{array}
$$
Todo lindo las raíces:
$$
  \llave{rcl}{
    X_1 & = & \frac{1}{13} \\
    X_2 & = & i            \\
    X_3 & = & -i           \\
  }
$$
Y factorizado en $\racionales[X],\, \reales[X] \ytext \complejos[X]$ queda.
\begin{center}
  \fcolorbox{orange}{white}{
    $
      \begin{array}{rcl}
        \racionales[X]:\quad f & = & 13\cdot(X - \frac{1}{13}) \cdot (X^2 + 1)             \\
        \reales[X]:\quad f     & = & 13\cdot(X - \frac{1}{13}) \cdot (X^2 + 1)             \\
        \complejos[X]:\quad f  & = & 13\cdot(X - \frac{1}{13}) \cdot (X - i) \cdot (X + i)
      \end{array}
    $
  }
\end{center}

\begin{aportes}
  \item \aporte{https://github.com/nad-garraz}{Nad Garraz \github}
\end{aportes}
