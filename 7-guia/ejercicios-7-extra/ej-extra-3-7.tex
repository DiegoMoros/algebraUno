\begin{enunciado}{\ejExtra}
  Hallar \textbf{todos} los polinomios \textbf{mónicos} $f \en \racionales[X]$
  de grado mínimo que cumplan simultáneamente las siguientes condiciones:
  \begin{enumerate}[label=\roman*)]
    \item $1 - \sqrt{2}$ es raíz de $f$;
    \item $X(X-2)^2 \divideA (f:f')$;
    \item $(f:X^3 - 1) \distinto 1$;
    \item $f(-1) = 27$;
  \end{enumerate}

\end{enunciado}

\begin{enumerate}[label=\roman*)]
  \item Como $f \en \racionales[X]$ si $\alpha_1 = 1 - \sqrt{2}$ es raíz entonces  $\alpha_2 = 1 + \sqrt{2}$
        para que no haya coeficientes irracionales en el polinomio.\\
        $$
          (X - (1 - \sqrt{2}))
          \cdot
          (X - (1 + \sqrt{2}))
          =
          X^2 - 2X - 1
        $$
        Por lo tanto:
        $$
          X^2 - 2X - 1
        $$
        será un factor de $f \en \racionales[X]$.

  \item Para el requerimiento $X(X-2)^2 \divideA (f:f')$:
        $$
          X(X-2)^2 \divideA (f:f')
          \Sii{def}
          (f:f') = X(X-2)^2 \cdot q,
        $$
        de donde se deduce que \underline{por lo menos} {\tiny(dado que no conoce $q$ y tampoco importa ahora)}:
        $$
          \llave{l}{
            \alpha_3 = 0
            \text{ es por lo menos raíz simple de $f'$}
            \entonces
            \text{ es por lo menos raíz doble de $f$} \\
            \alpha_4 = 2
            \text{ es por lo menos raíz doble de $f'$}
            \entonces
            \text{ es por lo menos raíz triple de $f$}
          }.
        $$

        Por lo tanto como en los ejercicios estos piden \textit{menor grado}:
        $$
          X^2(X-2)^3
        $$
        también serán factores de $f \en \racionales[X]$.

  \item Si $(f:X^3 - 1) \distinto 1$ quiere decir que \underline{por lo menos} alguna de las 3 raíces de:\par
        $$
          X^3 - 1
          \igual{\red{!}}
          (X-1)\cdot (X - (-\frac{1}{2} + \frac{\sqrt{3}}{2} i))\cdot (X - (-\frac{1}{2} - \frac{\sqrt{3}}{2} i))
        $$
        tiene que aparecer en la factorización de $f$.\par

        Parecido al item i) si tengo una raíz compleja, también necesito el conjugado complejo de la raíz, para que no me queden
        coeficientes de $f$ con componente imaginaria:
        $$
          X^3 - 1 = (X-1) \cdot (X^2 + X +1),
        $$
        a priori me quedaría con el \textit{factor de menor grado} siempre que eso no \textit{rompa} otras condiciones, pero todavía no tomo
        la decisión \faIcon[regular]{meh}.

        Por lo tanto:
        $$
          (X-1) \otext (X^2 + X +1)
        $$
        ya veremos cual, aparecerá en la factorización de $f \en \racionales[X]$.

  \item $f(-1) = 27$. Hasta el momento juntando los resultados tengo 2 candidatos $\magenta{f_1}$ y $\blue{f_2}$:
        $$
          \begin{array}{lcl}
            \magenta{f_1} = (X^2 - 2X - 1)\cdot (X-2)^3 \cdot X^2 \cdot (X^2 + X + 1) & \to & \magenta{f_1}(-1) = 2 \cdot (-27) \cdot 1 \cdot 1 = -54   \\
            \blue{f_2} = (X^2 - 2X - 1)\cdot (X-2)^3 \cdot X^2 \cdot (X-1)            & \to & \blue{f_2}(-1) =  2 \cdot (-27) \cdot 1 \cdot (-2) = 108,
          \end{array}
        $$
        ninguno es el 27 que quiero, así que hay que hacer algo más.
\end{enumerate}

Para encontrar \textit{un} polinomio \textbf{mónico} que cumpla lo pedido tomaría el $\blue{f_2}$ que tiene \red{menor grado} de los dos y lo multiplicaría por:
$$
  f = \blue{f_2} \cdot \yellow{(X - a)} \quad \text{con} \quad a \en \racionales
$$

de manera que pueda elegir el $a$ para cumplir lo que quiero:
$$
  f(-1) = \blue{f_2}(-1) \cdot \yellow{(X - a)}  = 108 \cdot (-1 - a) = 27 \sii a = -\frac{5}{4}
$$
$$
  f =  (X^2 - 2X - 1)\cdot (X-2)^3 \cdot X^2 \cdot (X-1) \cdot (X + \frac{5}{4})
$$
así cumpliendo todas las condiciones.

% Contribuciones
\begin{aportes}
  %% iconos : \github, \instagram, \tiktok, \linkedin
  %\aporte{url}{nombre icono}
  \item \aporte{https://github.com/nad-garraz}{Nad Garraz \github}
  \item \aporte{https://github.com/JowinTeran}{Ale Teran \github}
\end{aportes}
