\begin{enunciado}{\ejExtra}
  \begin{enumerate}[label=\alph*)]
    \item
          Determinar todos los valores de $n\en\naturales$ \textbf{(positivo)} para los cuales el polinomio
          $$
            f=X^5 + \frac{n}{3}X^4 - \frac{8}{3}X^3 + \frac{11}{3}X^2 - X
          $$
          tiene una raíz \textbf{entera} no nula.
    \item
          Para el o los valores hallados en el ítem (a), factorizar el polinomio $f$ obtenido como producto de irreducibles en $\racionales[X],\reales[X]$ y $\complejos[X]$
  \end{enumerate}
\end{enunciado}

\begin{enumerate}[label=\alph*)]
  \item
        Determinar todos los valores de $n\en\naturales$ \textbf{(positivo)} para los cuales el polinomio
        $$
          f=X^5 + \frac{n}{3}X^4 - \frac{8}{3}X^3 + \frac{11}{3}X^2 - X
        $$
        tiene una raíz \textbf{entera} no nula.\medskip

        \textbf{Solución:}\par
        Limpiando los denominadores de $f$ se obtiene el polinomio $g$ con las mismas raíces:
        $$
          g=3X^5 + nX^4 - 8X^3 + 11X^2 - 3X= X(\ub{3X^4 + nX^3 - 8X^2 + 11X - 3}{h})
        $$
        Por enunciado ignoramos la raiz nula y utilizando el Lema de Gauss buscamos las raíces racionales de
        $$
          h=3X^4 + nX^3 - 8X^2 + 11X - 3
        $$
        Aquí, $a_0=-3$ y $a_n=3$\par
        $$
          \text{Div}(a_0)=\text{Div}(a_n)=\{\pm1,\pm3\}
        $$

        Como busco raíces enteras, las busco en el conjunto:

        $$
          \{ \pm1,\pm3 \}
        $$

        Chequeo:
        $$
          \begin{array}{rcl}
            h(-1)=0 & \iff & n=-19 \notin \naturales          \\
            h(1)=0  & \iff & n=-3 \notin \naturales           \\
            h(-3)=0 & \iff & \fbox{n=5} \en \naturales        \\
            h(3)=0  & \iff & n=\frac{67}{9} \notin \naturales
          \end{array}
        $$
        \textbf{Rta:} $n=5$ es el único valor de $n\en\naturales$ para los cuales el polinomio $f$ tiene una raíz entera no nula.\par
  \item
        Para el o los valores hallados en el ítem (a), factorizar el polinomio $f$
        obtenido como producto de irreducibles en $\racionales[X],\reales[X]$ y $\complejos[X]$ \par

        \textbf{Solución:}\par

        Primero factorizo la raiz nula de de $f$ \par
        $$
          f=X^5 + \frac{5}{3}X^4 - \frac{8}{3}X^3 + \frac{11}{3}X^2 - X =
          X(X^4 + \frac{5}{3}X^3 - \frac{8}{3}X^2 + \frac{11}{3}X - 1)
        $$
        Se, por el item (a), que $-3$ es una de las raíces racionales de $f$.
        Busco otras posibles raiíces racionales en el polinomio $h$ (con $n=5$)
        obtenido en el item (a) en el conjunto $\{ \pm\frac{1}{3} \}$ \par
        $h(-\frac{1}{3})=-\frac{208}{27}$ \par
        $h(\frac{1}{3})=0 \implies \frac{1}{3}$ es una raiz racional de f. \par

        Factorizo el polinomio $f$ diviendolo por el producto de las dos raíces
        encontradas $(X+3)\cdot(X-\frac{1}{3}) = X^2 + \frac{8}{3} - 1$
        $$
          \divPol{X^4 + \frac{5}{3}X^3 - \frac{8}{3}X^2 + \frac{11}{3}X - 1}{X^2 + \frac{8}{3}X - 1}
        $$
        Factorizo el polinomio cuadrático $X^2 + \frac{8}{3}X - 1$
        $$ \Delta = (-1)^2 - 4 \cdot 1 \cdot 1=-3 $$
        $$ x_+ = \frac{1 + \sqrt 3 i}{2} \en\complejos \text{ y } x_- = \frac{1 - \sqrt 3 i}{2} \en\complejos $$

        \textbf{Rta:}\par
        $\therefore f = X(X+3)( X - \frac{1}{3} )( X - (\frac{1}{2} + \frac{\sqrt 3}{2}i) )( X - (\frac{1}{2} - \frac{\sqrt 3}{2}i) ) \en \complejos$
        con todos sus factores de multiplicidad 1 y por lo tanto \hyperlink{7-teoria:irreducibles}{irreducibles}. \par
        $\therefore f = X(X+3)( X - \frac{1}{3} )(X^2-X+1) \en \reales$
        con 3 factores de multiplicidad 1 y 1 de multiplicidad 2 pero de raíces complejas
        y por lo tanto \hyperlink{7-teoria:irreducibles}{irreducibles} en $\reales$. \par
        $\therefore f = X(X+3)( X - \frac{1}{3} )(X^2-X+1) \en \racionales$
        con 3 factores de multiplicidad 1 y 1 de multiplicidad 2 pero de raíces complejas
        y por lo tanto \hyperlink{7-teoria:irreducibles}{irreducibles} en $\racionales$.
\end{enumerate}











