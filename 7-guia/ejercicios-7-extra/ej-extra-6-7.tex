\ejercicio
a) Determinar todos los valores de $n\en\naturales$ \textbf{(positivo)} para los cuales el polinomio
$$
f=X^5 + \frac{n}{3}X^4 - \frac{8}{3}X^3 + \frac{11}{3}X^2 - X
$$
tiene una raíz \textbf{entera} no nula. 
\\
\\
\textbf{Solución:}
\\
Limpiando los denominadores de $f$ se obtiene el polinomio $g$ con las mismas raíces:
$$
g=3X^5 + nX^4 - 8X^3 + 11X^2 - 3X= X(\ub{3X^4 + nX^3 - 8X^2 + 11X - 3}{h})
$$
Por enunciado ignoramos la raiz nula y utilizando el Lema de Gauss buscamos las raices racionales de
$$
h=3X^4 + nX^3 - 8X^2 + 11X - 3
$$
Aquí, $a_0=-3$ y $a_n=3$
\\
$$
\text{Div}(a_0)=\text{Div}(a_n)=\{\pm1,\pm3\}
$$

Como busco raíces enteras, las busco en el conjunto:

$$
\{ \pm1,\pm3 \}
$$

Chequeo:\\
$h(-1)=0 \iff n=-19\notin\naturales$ \\
$h(1)=0 \iff n=-3\notin\naturales$ \\
$h(-3)=0 \iff \fbox{n=5}\in\naturales$ \\
$h(3)=0 \iff n=\frac{67}{9}\notin\naturales$ \\
\textbf{Rta:} $n=5$ es el unico valor de $n\in\naturales$ para los cuales el polinomio $f$ tiene una raíz entera no nula.\\
\\
b) Para el o los valores hallados en el ítem (a), factorizar el polinomio $f$ obtenido como producto de irreducibles en $\racionales[X],\reales[X]$ y $\complejos[X]$ \\
\\
\textbf{Solución:}\\
Primero factorizo la raiz nula de de $f$ \\
$$
f=X^5 + \frac{5}{3}X^4 - \frac{8}{3}X^3 + \frac{11}{3}X^2 - X = 
X(X^4 + \frac{5}{3}X^3 - \frac{8}{3}X^2 + \frac{11}{3}X - 1)
$$
Se, por el item (a), que $-3$ es una de las raices racionales de $f$. Busco otras posibles raices racionales en el polinomio $h$ (con $n=5$) obtenido en el item (a) en el conjunto $\{ \pm\frac{1}{3} \}$ \\
$h(-\frac{1}{3})=-\frac{208}{27}$ \\
$h(\frac{1}{3})=0 \implies \frac{1}{3}$ es una raiz racional de f. \\
\\
Factorizo el polinomio $f$ diviendolo por el producto de las dos raíces encontradas $(X+3)\cdot(X-\frac{1}{3}) = X^2 + \frac{8}{3} - 1$
$$
\polylongdiv[style=D]{X^4 + \frac{5}{3}X^3 - \frac{8}{3}X^2 + \frac{11}{3}X - 1}{X^2 + \frac{8}{3}X - 1}
$$
Factorizo el polinomio cuadratico $X^2 + \frac{8}{3}X - 1$
$$ \Delta = \sqrt{(-1)^2 - 4 \cdot 1 \cdot 1}=\sqrt{3}\cdot i $$
$$ x_+ = \frac{1 + \sqrt 3 i}{2} \in\complejos \text{ y } x_- = \frac{1 - \sqrt 3 i}{2} \in\complejos $$

\textbf{Rta:}\\
$\therefore f = X(X+3)( X - \frac{1}{3} )( X - (\frac{1}{2} + \frac{\sqrt 3}{2}i) )( X - (\frac{1}{2} - \frac{\sqrt 3}{2}i) ) \in \complejos$ con todos sus factores de multiplicidad 1 y por lo tanto irreducibles. \\
$\therefore f = X(X+3)( X - \frac{1}{3} )(X^2-X+1) \in \reales$ con 3 factores de multiplicidad 1 y 1 de multiplicidad 2 pero de raices complejas y por lo tanto irreducibles en $\reales$. \\
$\therefore f = X(X+3)( X - \frac{1}{3} )(X^2-X+1) \in \racionales$ con 3 factores de multiplicidad 1 y 1 de multiplicidad 2 pero de raices complejas y por lo tanto irreducibles en $\racionales$. \\












