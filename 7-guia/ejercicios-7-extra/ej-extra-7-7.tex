\begin{enunciado}{\ejExtra}
  Determinar un polinomio $f \en \racionales[X]$ de grado mínimo que satisfaga
  simultáneamente:\par

  \begin{enumerate}[label=$\scriptscriptstyle\blacksquare$]
    \item $f$ es mónico,
    \item $\gr(f:2X^3-5X^2-20X +11) = 2$
    \item $f$ tiene una raíz $z \en G_3$ con $z \distinto 1$, que es doble,
    \item $f(0) = 33$;
  \end{enumerate}
\end{enunciado}

El dato de
$\gr\ob{(f:\ub{2X^3-5X^2-20X +11}{g})}{d} = 2$ indica que hay un polinomio,
$d$, con $\gr(d) = 2$ que cumple que
$
  \llave{l}{
    d \divideA f \\
    d \divideA 2X^3-5X^2-20X +11
  } \text{ entonces,}\, f
$ tiene 2 raíces en común con $g$. Estás raíces pueden ser una doble o dos simples. Dado que nos piden que sea de grado mínimo
habrá que tener \textit{cuidado} cual elegir para no violar ninguna condición.\par

\medskip

Calculemos las posibles raíces de $g$ usando \textit{lema de gauss}:\par
Posibles raíces serán los cocientes de los divisores de 11 y los de 2.
$$
  \divset{11}{\pm1,\pm11}, \divset{2}{\pm1,\pm2}:
$$
$$
  \set{\pm 1,\pm\frac{1}{2},\pm 11, \pm\frac{11}{2}}.
$$

Probando esos valores encuentro que $g(\frac{1}{2}) = 0$ y ninguna de las
otras funcionó. Le bajamos el grado con el algoritmo de división a $g$.
$$
  \divPol{2X^3 - 5X^2 - 20X + 11}{X - \frac{1}{2}}
$$

Hasta el momento:
$$
  g = (X-\frac{1}{2}) \cdot (2X^2 - 4X -22) + 0,
$$
buscamos raíces de $2X^2 - 4X -22$:\par

$$\alpha_{+,-} =
  \frac{4 \pm 8\sqrt{3}}{4} =
  1 \pm 2\sqrt{3} =
  \llave{l}{
    1 + 2\sqrt{3} \\
    1 - 2\sqrt{3} \\
  }
$$

Entonces:
$f$ tiene 2 raíces en común con
$g = (X - \frac{1}{2})(X - (1+2\sqrt{3})) (X - (1-2\sqrt{3})) $. Dado que
$f \en \racionales[X]$  voy a seleccionar las raíces que tienen número irracionales por
la condición de grado mínimo. Recordar que si $f \en \racionales[X]$ tiene una raíz con números irracionales,
también debe estar su conjugado irracional.

\bigskip

Con la condición que dice que $f$ tiene una raíz $z \en G_3$ con $z \distinto 1$,
que es doble, no nos dejan muchas opciones. $G_3$ tiene tres raíces, solución
de $w^3 = 1$, dado que por enunciado no puede ser 1, entonces solo quedan:
$$
  -\frac{1}{2} \pm \frac{\sqrt{3}}{2}i
$$
(\blue{Si no te acordás como encontrar raíces de la
  familia $G_n$ te dejo el ejercicio \refEjercicio{ej:12}) que se hacen las cuentas.}

\bigskip

Ok, tengo esas dos raíces: $-\frac{1}{2} \pm \frac{\sqrt{3}}{2}i$ ¿Cuál elijo? ¡Cualquiera
sirve! Porque, \textit{nuevamente {\tiny \faIcon[regular]{meh-rolling-eyes}}}, como $f en \racionales[X]$ si agarro una raíz compleja
también necesito su conjugado complejo, lo mismo que antes.\par

Hasta el momento tenemos:
$$
  \begin{array}{c}
    f =
    \ob{(X - (1+2\sqrt{3})) (X - (1-2\sqrt{3}))}{X^2 - 2X - 11}
    \ub{(X - (-\frac{1}{2} + \frac{\sqrt{3}}{2}i))^{2^{\llamada1}}
    (X - (-\frac{1}{2} - \frac{\sqrt{3}}{2}i))^{2^{\llamada1}}}{(X^2 + X + 1)^2} = \\
    =(X^2 - 2X - 11)(X^2 + X + 1)^2
  \end{array}
$$

$\llamada1$ Si es doble una de las complejas, también debe serlo su conjugado, porque
$f\en \racionales[X]$.\bigskip

Nos queda cumplir que $f(0) = 33$, si bien ahora $f(0) = -11$. Acá tenemos que tener en cuenta
la primera condición. $f$ es \textit{mónico}, así que no podemos corregir el valor poniendo un coeficiente principal.

Hay que proponer otro factor en $\racionales[X]$, que al evaluar de el número que al multiplicarse con $-11$ nos dé
33. El candidato es $(X-3)$, dado que en 0 vale $-3$ y así $f(0) = (-11) \cdot (-3) = 33$ como queremos.\par

\bigskip

El $f \en\racionales[X]$ que cumple lo pedido:
$$
  \cajaResultado{f = (X^2 - 2X - 11)(X^2 + X + 1)^2(X-3)}
$$

% Contribuciones
\begin{aportes}
  %% iconos : \github, \instagram, \tiktok, \linkedin
  %\aporte{url}{nombre icono}
  \item \aporte{https://github.com/nad-garraz}{Nad Garraz \github}
\end{aportes}
