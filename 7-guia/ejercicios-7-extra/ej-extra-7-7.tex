\begin{enunciado}{\ejExtra}
        Determinar un polinomio $f \en \racionales[X]$ de grado mínimo que satisfaga 
        simultáneamente:\par

\begin{enumerate}[label=$\scriptscriptstyle\blacksquare$]
	\item $f$ es mónico,
    \item $\gr(f:2X^3-5X^2-20X +11) = 2$
    \item $f$ tiene una raíz $z \en G_3$ con $z \distinto 1$, que es doble,
	\item $f(0) = 33$;
\end{enumerate}
\end{enunciado}

El dato de
    $\gr(f:\ub{2X^3-5X^2-20X +11}{g}) = 2$ indica que $f$ tiene 2 raíces en común con $g$, 
    puede ser una doble o dos simples. Dado que nos piden que sea de grado mínimo
    habrá que tener cuidado cual elegir.\par
    Calculemos las posibles raíces de $g$ usando \textit{lema de gauss}:
    Posibles raíces serán los cocientes de los divisores de 11 y los de 2.
    $\divset{11}{\pm1,\pm11}, \divset{2}{\pm1,\pm2}$:
    $\set{\pm 1,\pm\frac{1}{2},\pm 11, \pm\frac{11}{2}}$.
    Probando esos valores encuentro que $g(\frac{1}{2}) = 0$ y ninguna de las
    otras funcionó. Le bajamos el grado con el algoritmo de división a $g$.
$$
    \divPol{2X^3 - 5X^2 - 20X + 11}{X - \frac{1}{2}}
$$
    $g = (X-\frac{1}{2})\cdot(\ub{2X^2 - 4X -22}{h}) + 0 $, buscamos raíces de $h$:\par

    $$\alpha_{+,-} =
    \frac{4 \pm 8\sqrt{3}}{4} =
    1 \pm 2\sqrt{3} =
    \llave{l}{
            1 + 2\sqrt{3} \\
            1 - 2\sqrt{3} \\
    }
    $$

    Entonces: 
    $f$ tiene 2 raíces en común con 
    $g = (X - \frac{1}{2})(X - (1+2\sqrt{3})) (X - (1-2\sqrt{3})) $. Dado que 
    $f \en \racionales[X]$  voy a seleccionar las raíces $ \en \irracionales$ por
    la condiciónde grado mínimo.\bigskip

    Con la condición que dice que $f$ tiene una raíz $z \en G_3$ con $z \distinto 1$,
    que es doble, no nos dejan muchas opciones. $G_3$ tiene tres raíces, solución
    de $w^3 = 1$, dado que por enunciado no puede ser 1, entonces solo quedan.
    $-\frac{1}{2} \pm \frac{\sqrt{3}}{2}i$ \green{(si no te acordás como encontrar raíces de la
    familia $G_n$ te dejo el ejercicio \refEjercicio{ej:12}) que se hacen las cuentas.}
    Ok, tengo esas dos $-\frac{1}{2} \pm \frac{\sqrt{3}}{2}i$ ¿Cuál elijo? cualquiera nos
    sirve, porque, \textit{nuevamente}, como $f en \racionales[X]$ si agarro una raíz compleja
    también necesito su conjugado complejo, lo mismo que antes.\par

Hasta el momento tenemos:\par
$$
\begin{array}{c}
f =
\ob{(X - (1+2\sqrt{3})) (X - (1-2\sqrt{3}))}{X^2 - 2X - 11}
\ub{(X - (-\frac{1}{2} + \frac{\sqrt{3}}{2}))^{2^{\llamada1}}
(X - (-\frac{1}{2} - \frac{\sqrt{3}}{2}))^{2^{\llamada1}}}{(X^2 + X + 1)^2} =\\
=(X^2 - 2X - 11)(X^2 + X + 1)^2
\end{array}
$$

$\llamada1$ Si es doble una de las complejas, también debe serlo su conjugado, porque
$f\en \racionales[X]$.\bigskip

Nos queda cumplir que $f(0) = 33$, si bien ahora $f(0) = -11$. Acá tenemos que tener en cuenta
la primera condición. $f$ es \textit{mónico}, así que no podemos corregir el valor con coeficiente independiente.
Hay que proponer otro factor en $\racionales[X]$, que al evaluar de el número que al multiplicarse con $-11$ nos dé
33. El candidato es $(X-3)$, dado que en 0 vale $-3$ y así $f(0) = (-11) \cdot (-3) = 33$ como queremos.\par

El $f \en\racionales[X]$ que cumple lo pedido:
$$
\boxed{f = (X^2 - 2X - 11)(X^2 + X + 1)^2(X-3)}
$$



