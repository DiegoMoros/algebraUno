\begin{enunciado}{\ejExtra}
  Factorizar el polinomio
  $
    P = X^6 - X^5 - 13X^4 + 14X^3 + 35X^2 -49X + 49
  $
  como producto de irreducibles en $\complejos[X], \reales[X] \ytext \racionales[X]$ sabiendo que $\sqrt7$ es una
  raíz múltiple.

\end{enunciado}

Un polinomio con coeficientes racionales, y una raíz irracional $\alpha = \sqrt7$,
tendrá también al \textit{conjugado irracional}
\footnote{Estoy usando la misma notación para \textit{conjugado racional} y
  \textit{conjugado complejo}. ¿Está bien? No sé, no me importa mientras se entienda.}
, $\conj{\alpha} = -\sqrt{7}$\par

Si agregamos la información de que  $\sqrt7$ es \textit{por lo menos} raíz doble, obtenemos que:\par

$
  \llave{l}{
    \sqrt7 \text{ es raíz de } f
    \entonces
    -\sqrt7 \text{ es raíz de } f
    \entonces
    (X^2 - 7) \divideA f\\
    \sqrt7 \text{ es raíz doble de } f
    \entonces
    -\sqrt7 \text{ es raíz doble de } f
    \entonces
    (X^2 - 7)^2 = X^4 - 14X^2 + 49 \divideA f \Tilde\\
  }
$\par

\divPol{X^6-X^5-13X^4+14X^3+35X^2-49X+49}{X^4-14X^2+49}\par

$
  f = (X^4-14X^2+49) \cdot (X^2 - X + 1)
  \flecha{resolvente}
  \llave{l}{
    \alpha_{+,-} = \frac{1 \pm w}{2}\\
    w^2 = -3 \\
  }\\
  \to
  f = (X^4-14X^2+49) \cdot (X - (\frac{1}{2} + i\frac{\sqrt3 }{2})) (X - (\frac{1}{2} -i \frac{\sqrt3 }{2}))
$\par

\boxed{
  \llave{rcl}{
    \racionales[X] & \to & f = (X^2 + 7)^2  (X^2 - X + 1) \\
    \reales[X] & \to & f = (X + \sqrt7)^2(X - \sqrt7)^2  (X^2 - X + 1)\\
    \complejos[X] & \to & f = (X + \sqrt7)^2(X - \sqrt7)^2  (X - (\frac{1}{2} + i\frac{\sqrt3 }{2})) (X - (\frac{1}{2} - i\frac{\sqrt3 }{2}))
  }}
