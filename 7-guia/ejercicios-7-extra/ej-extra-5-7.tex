\begin{enunciado}{\ejExtra}
  Sea $(f_n)_{(n\geq 1)}$ la sucesión de poliniomios en $\reales[X]$ definida como:\par
  $$
    f_1 = X^5 + 3X^4 + 5X^3 + 11X^2 - 20 \ytext f_{n+1} = (X + 2)^2 f'_n + 3 f_n, \text{ para cada } n \en \naturales.
  $$\par
  Probar que $-2$ es raíz doble de $f_n$ para todo $n \en \naturales$.
\end{enunciado}

No caer en la \red{\textit{trampilla}$^{\skull}$} de olvidar que para que una raíz de $f$ sea doble, i.e.  $\mult(-2;f) \red{\igual{!}} 2$
debe ocurrir lo \textit{"obvio"}, $f(-2) = f'(-2) = 0$ y también que $\red{f^{''}(-2) \distinto 0}$. Si olvidamos esto último
solo probaríamos que la $\mult(-1;f) \geq 2$ y tendríamos el ejercicio mal \red{$\skull$}.\par

\textit{Por inducción en $n$: } $q(n): $ "$-2$ es raíz doble de $f_n,\, \paratodo n \en \naturales$"

\textit{\underline{Caso base:}} ¿$q(1)$ es V?\par
$
  \llave{lcl}{
    f_1 = X^5 + 3X^4 + 5X^3 + 11X^2 - 20 &
    \flecha{evaluar}[en $-2$]            &
    f_1(-2) = 0                            \\

    f'_1 = 5X^4 + 12X^3 + 15X^2 + 22X    &
    \flecha{evaluar}[en $-2$]            &
    f'_1(-2) = 0                           \\

    f^{''}_1 = 20X^3 + 36X^2 + 30X + 22  &
    \flecha{evaluar}[en $-2$]            &
    f^{''}_1(-2) = -54 \distinto 0
  }
$

$\therefore \mult(-2; f_1) = 2 \entonces -2$ es raíz doble de $f_1 \entonces q(1)$ es V \Tilde\par

\underline{\textit{Paso inductivo:}} ¿Si $q(k) \text{ verdadera } \entonces q(k+1) \text{ también lo es}$, $\paratodo k \en \naturales$?

\textit{HI:} $-2$ es raíz doble de
$f_k
  \sii
  \llave{l}{
    f_k(-2) = 0 \llamada1  \\
    f'_k(-2) = 0 \llamada2 \\
    f''_k(-2) \distinto 0 \llamada3
  }
$

QPQ dado $k \en \naturales,\, q(k+1):$ $-2$ es raíz doble de $f_{k+1} \igual{def} (X + 2)^2 f'_k + 3f_k$:\par

\textit{Derivar:}\par
$$
  \llave{l}{
    f_{k+1} \igual{def} (X + 2)^2 f'_k + 3 \cdot f_k \\
    f'_{k+1}  = 2(X+2)f'_k + (X+2)^2 f^{''}_k + 3\cdot f'_k\\
  f^{''}_{k+1} =
  2f'_k + (2X+4)f^{''}_k + 2(X+2) f^{''}_k + (X+2)^2 f^{'''}_k + 3 \cdot f^{''}_k
  }
$$

\medskip

\textit{Evaluar en $-2$:}\par
$
  \begin{array}{lll}
    f_{k+1}(-2) \igual{?} 0
     & \sii &
    f_{k+1}(-2) = \cancel{(-2 + 2)}^2 f'_k(-2) + 3f_k(-2) =
    0^2 f'_k(-2) + 3 f_k(-2)=
    3 f_k(-2) \igual{$\llamada1$}
    0 \Tilde \vspace{10pt}                             \\
    f'_{k+1}(-2) \igual{?} 0
     & \sii &
    f_{k+1}'(-2) =
    2\cancel{(-2 + 2)}f'_k(-2) + \cancel{(-2 + 2)}^2 f^{''} + f'_k(-2) =
    f'_k(-2) \igual{$\llamada2$} 0 \Tilde\vspace{10pt} \\
    f^{''}_{k+1}(-2) \taa{?}{}\distinto 0
     & \sii &
    \llave{l}{
      {f^{''}_{k+1}(-2) = 2f'_k(-2) +
      2\cancel{(-2+2)}f^{''}_k(-2) +
    2\cancel{(-2+2)} f^{''}_k(-2) } +                  \\
    + \cancel{(-2+2)}^2 f^{'''}_k(-2) +
    f^{''}_k(-2) = 2\ob{f'_k(-2)}{= 0 \llamada2} + \ob{f^{''}_k(-2)}{\distinto 0 \llamada3}
    \distinto 0 \Tilde
    }
  \end{array}
$

$\therefore \mult(-2; f_{k+1}) = 2 \entonces -2$ es raíz doble de $f_{k+1} \entonces q(k+1)$ es V \Tilde

Como $q(1),\, q(k)$ y $q(k+1)$ resultaron verdaderas, por principio de inducción $q(n)$ también lo es $\paratodo n \en \naturales$.

% Contribuciones
\begin{aportes}
  %% iconos : \github, \instagram, \tiktok, \linkedin
  %\aporte{url}{nombre icono}
  \item \aporte{https://github.com/nad-garraz}{Nad Garraz \github}
  \item \aporte{https://github.com/daniTadd}{Dani Tadd \github}
  \item \aporte{\neverGonnaGiveYouUp}{No me acuerdo el autor original \youtube}
\end{aportes}
