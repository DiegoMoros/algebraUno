\ejercicio Hallar, cuando existan, todos los enteros $a$ que satisfacen simultáneamente:\\

\begin{enumerate}[label=\roman*)]
	\item
	      $
		      \llave{l}{
			      \llamada1 \congruencia{a}{3}{10} \\
			      \llamada2 \congruencia{a}{2}{7} \\
			      \llamada3 \congruencia{a}{5}{9} \\
		      }
	      $\\
	      El sistema tiene solución dado que 10, 7 y 9 son coprimos dos a dos. Resuelvo:\\
	      $\flecha{Arranco}[ en $\llamada1$]
		      a = 10k + 3 \conga{7}
		      3k + 3 \conga{\llamada2}
		      2\ (7)
		      \flecha{usando que}[$3\cop 7$] \congruencia{k}{2}{7}
		      \to k = 7q + 2.\\
		      \flecha{actualizo}[$a$]
		      a = 10 \cdot \ub{(7q + 2)}{k} + 3 = 70 q + 23 \conga9
		      7q \conga{\llamada3}
		      5\ (9)
		      \flecha{usando que}[$7\cop 9$] \congruencia{q}{0}{9}
		      \to q = 9j\\
		      \flecha{actualizo}[$a$]
		      a = 70 \ub{(9j)}{q} + 23 = 680j + 23 \to \boxed{\congruencia{a}{23}{630}} \Tilde
	      $\\
	      La solución hallada es la que el Teorema chino del Resto me garantiza que tengo en el
	      intervalo $[0, 10\cdot 7 \cdot 9)$

	\item

	\item $
		      \llave{l}{
			      \llamada1 \congruencia{a}{1}{12} \\
			      \llamada2 \congruencia{a}{7}{10} \\
			      \llamada3 \congruencia{a}{4}{9}
		      }
	      $


\end{enumerate}
