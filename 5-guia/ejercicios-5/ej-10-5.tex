\begin{enunciado}{\ejercicio}
  Hallar, cuando existan, todos los enteros $a$ que satisfacen simultáneamente:\\

  \begin{enumerate}[label=\roman*)]
    \begin{multicols}{3}
      \item
      $
        \llave{ll}{
          \congruencia{a}{3}{10} \\
          \congruencia{a}{2}{7}  \\
          \congruencia{a}{5}{9}
        }
      $
      \item
      $
        \llave{ll}{
          \congruencia{a}{1}{6}  \\
          \congruencia{a}{2}{20} \\
          \congruencia{a}{3}{9}
        }
      $

      \item $
        \llave{l}{
          \congruencia{a}{1}{12} \\
          \congruencia{a}{7}{10} \\
          \congruencia{a}{4}{9}
        }
      $
    \end{multicols}
  \end{enumerate}

\end{enunciado}

\begin{enumerate}[label=\roman*)]
  \item
        $
          \llave{ll}{
            \congruencia{a}{3}{10} & \llamada1 \\
            \congruencia{a}{2}{7}  & \llamada2 \\
            \congruencia{a}{5}{9}  & \llamada3 \\
          }
        $\\
        El sistema tiene solución dado que 10, 7 y 9 son coprimos dos a dos. Resuelvo:\\
        $\flecha{Arranco}[ en $\llamada1$]
          a = 10k + 3 \conga{7}
          3k + 3 \conga{\llamada2}
          2\ (7)
          \flecha{usando que}[$3\cop 7$] \congruencia{k}{2}{7}
          \to k = 7q + 2.\\
          \flecha{actualizo}[$a$]
          a = 10 \cdot \ub{(7q + 2)}{k} + 3 = 70 q + 23 \conga9
          7q \conga{\llamada3}
          5\ (9)
          \flecha{usando que}[$7\cop 9$] \congruencia{q}{0}{9}
          \to q = 9j\\
          \flecha{actualizo}[$a$]
          a = 70 \ub{(9j)}{q} + 23 = 680j + 23 \to \boxed{\congruencia{a}{23}{630}} \Tilde
        $\\
        La solución hallada es la que el Teorema \href{\chinito}{chino} del Resto me garantiza que tengo en el
        intervalo $[0, 10\cdot 7 \cdot 9)$

  \item \hacer

  \item \hacer
\end{enumerate}
