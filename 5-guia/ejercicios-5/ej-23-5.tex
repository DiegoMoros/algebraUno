\begin{enunciado}{\ejercicio}
  Hallar todos los divisores positivos de $5^{140} = 25^{70}$ que sean congruentes
  a 2 módulo 9 y 3 módulo 11.
\end{enunciado}

Escribiendo el enunciado en ecuaciones queda algo así:

$$
  \llave{l}{
    \congruencia{25^{70}}{0}{d} \sii \congruencia{5^{140}}{0}{d} \\
    \congruencia{d}{2}{9}                                                                                    \\
    \congruencia{d}{3}{11}.
  }
$$
De la primera ecuación queda que el divisor $d = \magenta{5^\alpha}$ con $\alpha$ compatible
con las otras ecuaciones.
$$
  \llave{l}{
    \congruencia{5^{140}}{0}{ \magenta{5^\alpha}} \\
    \congruencia{\magenta{5^\alpha}}{2}{9} \llamada1                   \\
    \congruencia{\magenta{5^\alpha}}{3}{11} \llamada2
  }
$$

Estudio la periodicidad que aparece al calcular los restos de las exponenciales. ¿Para qué valor de $\alpha$ tendré $5^{\alpha} \congruente 1$?.
Empiezo con $\llamada1$. Notar que:
$$
  \congruencia{5^3}{-1}{9}
  \Sii{\red{!}}[$(\Leftarrow)\ 5^3\perp9$]
  \congruencia{5^6}{1}{9}
$$
Este último resultado me dice que como mucho hay 6 posibles valores distintos para $r_9(5^\alpha)$ y eso se ve fácil escribiendo \textit{genérica, pero convenientemente},
el exponente:
$$
  5^{6k+r_6(\alpha)} = (5^6)^k \cdot 5^{r_6(\alpha)} \conga9 5^{r_6(\alpha)}                                \\
$$
Los valores del conjunto $r_6(\alpha)$ son solo $\set{0,1,2,3,4,5}$, calculo a mano los posibles resultados de $\llamada1$:
$$
  \begin{array}{|l|l|l|l|l|l|l|}
    \hline
    r_{\red{6}}(\alpha) & 0 & 1 & 2 & 3 & 4 & 5       \\ \hline
    r_9(5^\alpha)       & 1 & 5 & 7 & 8 & 4 & \red{2} \\ \hline
  \end{array}
$$
Concluyo que:
$$
  \congruencia{\magenta{5^\alpha}}{2}{9}  \sii \congruencia{\alpha}{5}{6}                     \\
$$

\bigskip

El estudio de $\llamada2$ es un poco más feliz, porque 11 es primo y podemos usar PTF (\hyperlink{teoria-5:PTF}{teoría acá}), entonces se encuentra la periodicidad de los restos de
la exponencial más rápido:
$$
  \congruencia{5^\alpha}{3}{11}
  \Entonces{$11 \noDivide 5$}[PTF, con $\alpha=\green{10}$]
  \congruencia{5^{\green{10}} \taa{\red{!}}{}{\congruente} 5^{r_{10}(10)}}{1}{11}
$$
Con ese resultado uno se tienta a repetir lo que se hizo para $\llamada1$, pero ahí está \textit{la trampa del ejercicio}.
El PTF nos da un resultado, pero no quiere decir que no haya otro de valor menor. Es decir que puede haber un $\alpha < 10$ tal que
cumpla que $5^\alpha \conga{11} 1$. Veamos si eso es así:
$$
  \begin{array}{|l|l|l|l|l|l|l|l|l|l|l|l|}
    \hline
    r_{10}(\alpha)   & 0 & 1 & 2       & 3 & 4 & 5        & \dots \\ \hline
    r_{11}(5^\alpha) & 1 & 5 & \red{3} & 4 & 9 & \cyan{1} & \dots \\ \hline
  \end{array}                                       \\
$$
y sí, estaba ese número ahí para molestar. De no haber encontrado ese \cyan{1} ahí, se \ul{hubieran perdido soluciones}.
Para que se cumpla $\llamada2$:
$$
  \congruencia{\magenta{5^\alpha}}{3}{11}  \sii \congruencia{\alpha}{2}{5}                     \\
$$

Los valores de $\alpha$ deben cumplir el sistema:

$$
  \llave{l}{
    \congruencia{\alpha}{5}{6} \\
    \congruencia{\alpha}{2}{5},
  }
$$

con 6 y 5 coprimos hay solución por \href{\chinito}{TCH}. La solución: $\congruencia{\alpha}{17}{30}$ y además $0 < \alpha \leq 140$ (si no vuela todo por los aires)
lo que se cumple para:
$$
  \alpha = 30k + 17 =
  \llave{lcr}{
    17  & \text{ si } & k = 0 \\
    47  & \text{ si } & k = 1 \\
    77  & \text{ si } & k = 2 \\
    107 & \text{ si } & k = 3 \\
    137 & \text{ si } & k = 4
  }
$$
Finalmente los divisores porsitivos pedidos del enunciado son:
$$
  \boxed{\divsetP{25^{70}}{ 5^{17}, 5^{47}, 5^{77}, 5^{107}, 5^{137} }}
$$

\begin{aportes}
  \item \aporte{https://github.com/nad-garraz}{Nad Garraz \github}
\end{aportes}
