\begin{enunciado}{\ejercicio}
	Hallar todas las soluciones $(x,y) \en \enteros^2$ de la ecuación
	$$
		110x + 250y = 100
	$$
	que satisfacen simultáneamente que $37^2 \divideA (x - y)^{4321}$.
\end{enunciado}

La solución de la diofántica:
$$
	(x,y) = k\cdot (25, - 11) + (410, -180) \flecha{$\llamada1$}
	\llave{rcl}{
		x &=& 25k + 410 \\
		y &=& -11k - 180
	}
$$

Entonces hay que ver para que valores de $k$ se cumple que:

$$
	37^2 \divideA (\magenta{x - y})^{4321} \flecha{$\llamada1$}
	37^2 \divideA (\magenta{ 590 + 36k })^{4321}
$$

Buscamos posibles valores:
$$
	\llamada2
	37^2 \divideA ( 590 + 36k )^{4321}
	\Entonces{transitividad}[$\llamada3$]
	37 \divideA ( 590 + 36k )^{4321}
	\Sii{$p \divideA a^n \sii p \divideA a$}[$p$ primo]
	37 \divideA 590 + 36k
$$
Que como ecuación de congruencia queda:
$$
	\congruencia{-36k}{590}{37} \sii
	\congruencia{k}{35}{37}
$$

Por lo tanto de $\llamada2$ solo faltaría probar la vuelta ($\Leftarrow$) en $\llamada3$ se tiene que para los $k$:
$$
	\congruencia{k}{35}{37}
	\sii
	37^{\yellow{1}} \divideA (590 + 36k)^{4321} \Sii{\red{??}}
	37^{\blue{2}} \divideA (590 + 36k)^{4321}
$$
Veamos:
$$
	\begin{array}{c}
		37^{\yellow{1}} \divideA (590 + 36k)^{4321}
		\Entonces{37 es}[ primo]
		37 \divideA 590 + 36k
		\Entonces{\red{!}}
		37^2 \divideA (590 + 36k)^2 \\
		\entonces
		37^2 \divideA (590 + 36k)^2 \cdot (50 + 36k)^{4319}
		\sii
		37^{\blue{2}} \divideA (590 + 36k)^{4321}
	\end{array}
$$

De esa manera queda demostrado que

$$
	37^2 \divideA ( 590 + 36k )^{4321}
	\sii
	37 \divideA ( 590 + 36k )^{4321}
	\sii
	37 \divideA 590 + 36k
$$
como pedía el ejercicio.





% Contribuciones
\begin{aportes}
	%% iconos : \github, \instagram, \tiktok, \linkedin
	%\aporte{url}{nombre icono}
	\item \aporte{https://github.com/JowinTeran}{Ale Teran \github}
	\item \aporte{https://github.com/nad-garraz}{Nad Garraz \github}
\end{aportes}
