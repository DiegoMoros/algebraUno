\begin{enunciado}{\ejercicio}
  Hallar todos los $a\en \enteros$ para los cuales $(2a-3:4a^{2}+10a-10) \neq 1$
\end{enunciado}

Sea $d=(2a-3:4a^{2}+10a-10)$. Entonces:

$$
  \llave{l}{
    d \divideA 2a-3 \Entonces{$\times 2a$} d \divideA 4a^{2} - 6a \\
    d \divideA 4a^{2}+10a-10
  }
  \llave{l}{
    d \divideA 4a^{2}-6a \\
    d \divideA 4a^{2}+10a-10
  }
  \Entonces{$F_2- F_1$}[]
  d \divideA 16a-10
$$

Luego, tenemos

$$
  \llave{l}{
    d \divideA 16a-10 \\
    d \divideA 2a-3 \Entonces{$\times (-8)$} d \divideA -16a+24
  }
  \Entonces{$F_1 + F_2$}
  d \divideA 14
  \entonces d \en \set{1, 2, 7, 14}
$$

Veo tabla de restos con $d=2$ con la expresión $2a-3$

$$
  \begin{array}{|r|c|c|}
    \hline
    r_2(a)    & 0 & 1 \\ \hline
    r_2(2a-3) & 1 & 1 \\ \hline
  \end{array}
$$

Entonces, como $2 \noDivide 2a-3 \paratodo a \en \enteros$, tenemos que $d \neq 2 \ytext d \neq 14 \paratodo a \en \enteros$. \\
De modo que queremos hallar los valores de $a$ para los cuales $d=7$.

Veo la tabla de restos con $d=7$ con ambas expresiones.

$$
  \begin{array}{|r|c|c|c|c|c|c|c|}
    \hline
    r_7(a)              & 0 & 1 & 2 & 3 & 4 & 5       & 6 \\ \hline
    r_7(2a-3)           & 4 & 6 & 1 & 3 & 5 & \red{0} & 2 \\ \hline
    r_7( 4a^{2}+10a-10) & 4 & 4 & 5 & 0 & 3 & \red{0} & 5 \\ \hline
  \end{array}
$$

Entonces, $d=7 \sisolosi \congruencia{a}{5}{7}$.
Particularmente, $d \neq 1 \sisolosi \boxed{\congruencia{a}{5}{7}}$.

\begin{aportes}
  \item \aporte{https://github.com/Nunezca}{Nunezca \github}
\end{aportes}
