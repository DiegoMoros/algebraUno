\begin{enunciado}{\ejercicio}
  Probar que para todo $a\en \enteros$ vale
  \begin{multicols}{2}
    \begin{enumerate}[label=\alph*)]
      \item $728 \divideA a^{27} - a^3$
      \item $\frac{2a^7}{35} + \frac{a}{7} - \frac{a^3}{5}  \en \enteros$
    \end{enumerate}
  \end{multicols}
\end{enunciado}

\begin{enumerate}[label=\alph*)]
  \item
        Escribiendo la consigna como ecuación de congruencia:
        $$
          728 \divideA a^{27} - a^3
          \Sii{def}
          \congruencia{a^{27} - a^3}{0}{728}
        $$
        Reescribo al divisor como: $728 = 2^3 \cdot 7 \cdot 13$. Los primos son nuestros aliados en estos
        ejercicios para poder usar PTF. Reescribo la ecuación a un sistema equivalente:
        $$
          \congruencia{a^{27} - a^3}{0}{728}
          \leftrightsquigarrow
          \llave{l}{
            \congruencia{a^{27} - a^3}{0}{8} \llamada1 \\
            \congruencia{a^{27} - a^3}{0}{7} \llamada2 \\
            \congruencia{a^{27} - a^3}{0}{13} \llamada3
          }
        $$
        Empiezo analizando $\llamada1$. Como el 8 no es primo, no se puede usar el PTF, así que lo
        encaramos \textit{old style} con propiedades de exponentes y pensando que en congruencia módulo 8,
        $r_8(a) \en \set{0,1,2,3,4,5,6,7}$:
        A ver que pasa si $r_8(a)$ es par:
        $$
          \congruencia{
            a^{27} - a^3
            \igual{\red{!}}
            (2k)^3 \cdot ( (2k)^{24} - 1 ) = 8k^3\cdot ( (2k)^{24} - 1 )
          }{0}{8}
        $$
        Por lo que cuando $a$ es par, $\llamada1$ es válida. Ahora estudio los casos con $r_8(a)$ impar, que por
        suerte no son muchos:
        $$
          \congruencia{a^{27} - a^3 = a^3\cdot (a^{24} - 1)}{0}{8} \Sii{\red{!!!}}
          \llave{rc}{
            \congruencia{1^3 \cdot (1^{24} - 1)}{0}{8} & \quad      \text{si } a = 1 \\
            \congruencia{3^3\cdot(\magenta{9}^{12} - 1)}{0}{8}  & \quad  \text{si } a = 3     \\
            \congruencia{5^3\cdot(\magenta{25}^{12} - 1)}{0}{8}  & \quad   \text{si } a = 5    \\
            \congruencia{7^3\cdot(\magenta{49}^{12} - 1)}{0}{8}  & \quad   \text{si } a = 7
          }
        $$
        Por lo que si $r_8(a)$ es impar, también mostramos que $\llamada1$ es válida:
        $$
          \congruencia{a^{27} - a^3}{0}{8} \paratodo a \en \enteros
        $$

        Ahora vienen casos más agradables (espero) con divisores primos que nos permiten usar el PTF.\par
        Analizo $\llamada2$
        $$
          \congruencia{a^3 \cdot (a^{24} - 1)}{0}{7}
          \sii
          \llave{rc}{
            \congruencia{0^3 \cdot (0^{24} - 1)}{0}{7}                            & \quad      \text{si } 7 \divideA a  \\
            \Sii{PTF}[\red{!}]
            \congruencia{a^3 \cdot (a^{\magenta{0}} - 1)}{0}{7} \sii \congruencia{0}{0}{7} & \quad      \text{si } 7 \noDivide a
          }
        $$
        Por lo que $\llamada2$ se cumple para todo valor de $a$.

        Analizo $\llamada3$ muuuy parecido:
        $$
          \congruencia{a^3 \cdot (a^{24} - 1)}{0}{13}
          \sii
          \llave{rc}{
            \congruencia{0^3 \cdot (0^{24} - 1)}{0}{13}                             & \quad      \text{si } 13 \divideA a \\
            \Sii{PTF}[\red{!}]
            \congruencia{a^3 \cdot (a^{\magenta{0}} - 1)}{0}{13} \sii \congruencia{0}{0}{13} & \quad      \text{si } 7 \noDivide a
          }
        $$
        Por lo que $\llamada3$ se cumple para todo valor de $a$.

        Queda así demostrado que:
        $$
          \llave{l}{
            \congruencia{a^{27} - a^3}{0}{8} \\
            \congruencia{a^{27} - a^3}{0}{7} \\
            \congruencia{a^{27} - a^3}{0}{13}
          }
          \leftrightsquigarrow
          \congruencia{a^{27} - a^3}{0}{728}
          \Sii{def}
          728 \divideA a^{27} - a^3 \quad \paratodo a \en \enteros
        $$

  \item El enunciado puede pensarse como una ecuación de congruencia:
        $$
          \frac{2a^7}{35} + \frac{a}{7} - \frac{a^3}{5}  \en \enteros
          \Sii{\red{!!}}
          \congruencia{2a^7 + 5a - 7 a^3}{0}{35}
        $$
        Quizás conviene usar un sistema \textit{equivalente}, con divisored primos que nos permitan
        usar el PTF:
        $$
          \congruencia{2a^7 + 5a - 7 a^3}{0}{35}
          \leftrightsquigarrow
          \llave{l}{
            \congruencia{2a^7 + 5a - 7 a^3}{0}{7} \llamada1 \\
            \congruencia{2a^7 + 5a - 7 a^3}{0}{5} \llamada2
          }
        $$
        Empiezo por $\llamada1$:
        $$
          \congruencia{2a^7 + 5a - 7 a^3}{0}{7}
          \sii
          \congruencia{2a^7 + 5a}{0}{7} \sii
          \llave{l}{
            \text{si } 7 \divideA a  \entonces \congruencia{0}{0}{7} \\
            \text{si } 7 \noDivide a  \Entonces{PTF}[\red{!}]
            \congruencia{7a}{0}{7} \sii \congruencia{0}{0}{7}        \\
          }
        $$
        De este último resultado, concluímos que no importa el valor de $a$, es decir:
        $$
          \congruencia{2a^7 + 5a - 7 a^3}{0}{7} \quad \paratodo a \en \enteros
        $$
        Ahora analizo $\llamada2$:
        $$
          \congruencia{2a^7 + 5a - 7 a^3}{0}{5}
          \sii
          \congruencia{2a^7 - 2a^3}{0}{5} \sii
          \llave{l}{
            \text{si } 5 \divideA a  \entonces \congruencia{0}{0}{5}   \\
            \text{si } 5 \noDivide a  \Entonces{PTF}[\red{!}]
            \congruencia{2a^3 - 2a^3}{0}{5} \sii \congruencia{0}{0}{5} \\
          }
        $$
        Al igual que en el caso anterior, concluímos que no importa el valor de $a$, es decir:
        $$
          \congruencia{2a^7 + 5a - 7 a^3}{0}{5} \quad \paratodo a \en \enteros
        $$

        Se concluye entonces que la expresión:
        $$
          \congruencia{2a^7 + 5a - 7 a^3}{0}{35} \sii
          \frac{2a^7}{35} + \frac{a}{7} - \frac{a^3}{5}  \en \enteros \paratodo a \en \enteros
        $$
\end{enumerate}

\begin{aportes}
  \item \aporte{https://github.com/nad-garraz}{Nad Garraz \github}
\end{aportes}
