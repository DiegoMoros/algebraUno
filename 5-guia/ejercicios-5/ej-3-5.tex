\begin{enunciado}{\ejercicio}
  Si se sabe que cada unidad de un cierto producto $A$ cuesta $39$ pesos y que cada unidad de un cierto
  producto $B$ cuesta 48 pesos, ¿cuántas unidades de cada producto se pueden comprar gastando exactamente
  135 pesos?
\end{enunciado}

Armo diofántica con enunciado, tengo en cuenta que
$A \geq 0 \ytext B \geq 0$, dado que son productos {\color{pink}{\faIcon{brain}}}.
$$
  \llave{c}{
    39A + 28B = 135\\
    \Sii{coprimizar}[$(A:B) = 3$]\\
    13\blue{A} + 16\green{B} = 45,\,\\
    \text{tiene solución, ya que $\ub{(13:16)}{ = 1} \divideA 45$}\\
    \flecha{sale a ojo}\\
    \boxed{(\blue{A},\green{B}) = (\blue{1},\green{2})}
  }
$$
