\begin{enunciado}{\ejercicio}
  Si se sabe que cada unidad de un cierto producto $A$ cuesta $39$ pesos y que cada unidad de un cierto
  producto $B$ cuesta 48 pesos, ¿cuántas unidades de cada producto se pueden comprar gastando exactamente
  135 pesos?
\end{enunciado}

Armo diofántica con enunciado, tengo en cuenta que
$A \geq 0 \ytext B \geq 0$, dado que son productos físicos {\color{pink}{\faIcon{brain}}}.

La ecuación \textit{presupuestaria} queda:
$$
  39A + 28B = 135
$$
Siempre que puedo coprimizar lo hago:
$$
  (A:B) = 3
  \entonces
  13 \blue{A} + 16\green{B} = 45,
$$
Veo que hay solución dado que,
$$
  (13:16) \divideA 45
$$
Resuelvo a ojo. Si no lo ves hacé Euclides combinación entera y zarasa:
$$
  \cajaResultado{
    (\blue{A},\green{B}) = (\blue{1},\green{2})
  }
$$

\begin{aportes}
  \item \aporte{\dirRepo}{naD GarRaz}
\end{aportes}
