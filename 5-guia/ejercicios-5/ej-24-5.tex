\begin{enunciado}{\ejercicio}
  Hallar todos los $p \en \naturales$ que satisfacen:
  \begin{multicols}{2}
    \begin{enumerate}[label=\alph*)]
      \item  $2p \divideA 38^{2p^2 -p -1} + 3p + 171$
      \item  $3p \divideA 5^{p-1} + 3^{p^2+2} + 833$
    \end{enumerate}
  \end{multicols}
\end{enunciado}

Voy a buscar primos $p$ que cumplan lo pedido, usando PTF:

\bigskip
\begin{enumerate}[label=\alph*)]

  \item Para poder usar PTF tengo que tener un primo en el divisor, quiebro:
        $$
          \begin{array}{c}
            \congruencia{38^{2p^2 -p -1} + 3p + 171}{\blue{0}}{2p}
            \equivalente
            \llave{l}{
            \congruencia{p}{1}{2} \llamada1 \\
              \congruencia{38^{2p^2 -p -1} + 3p + 171}{\blue{0}}{p} \llamada2
            }
          \end{array}
        $$
        Quiero que ocurra que:
        $$
          \congruencia{38^{2p^2 -p -1} + 3p + 171}{\blue{0}}{p}
          \sii
          \llave{l}{
            \Sii{si $p \divideA 38$}[$\entonces p = \purple{19}$]
            \congruencia{ 38^{2 \cdot \purple{19}^2 - \purple{19} -1} + 3 \cdot \purple{19} + 171 \conga{19} 0}{\blue{0}}{\purple{19}} \llamada3 \\
            \Sii{PTF}[$p \noDivide 38$ \red{!!}]
            \congruencia{38^{\red{0}} + 0 + 171 = \ub{172}{2^2 \cdot 43}}{\blue{0}}{p} \llamada4
          }
        $$

        Cálculo usado hacer el PTF con $p$ genérico:
        $$
          \polyset{vars=p}
          \divPol{2p^2 - p - 1}{p-1}
        $$

        Después de hacer todo eso, sacamos de $\llamada1$ que $p$ es impar. De $\llamada3$ obtenemos que un posible valor sería:
        $$
          \cajaResultado{
            p = 19
          }
        $$
        y luego del caso $p \noDivide 38$ sale que debe ocurrir que $172 \conga{p} 0$, y dado que $p$ \underline{tiene que ser impar}, entonces
        queda que:
        $$
          \cajaResultado{
            p = 43
          }
        $$.

  \item Parecido al anterior. Voy a ver que pasa con $p = 3$ y con $p = 5$ así descarto esos valores y después le manto PTF:
        $$
          p = 3
          \entonces
          9 \divideA 5^2 + 3^{11} + 833
          \Sii{def}
          \noCongruencia{7 + 0 + 5 \conga{9} 3}{\blue{0}}{9}
        $$
        La expresión no se divide por 9, entonces descarto $p = 3$. A ver con $p = 5$:
        $$
          p = 5
          \entonces
          3 \cdot 5 \divideA 5^4 + 3^{27} + 833
          \equivalente
          \llave{l}{
            \congruencia{5^2 \cdot 5^2 + 3^{27} \cdot 3^{3} + 833 \conga3 1 + 0 + 2}{\blue{0}}{3} \\
            \congruencia{5^4 + (3^4)^{6} \cdot 3^{3} + 833 \conga5 0 + 2 + 3 }{\blue{0}}{5}
          }
        $$
        Con $p = 5$ funciona el sistema con divisores coprimos, así que $p = 5$ es un número primo que cumple el enunciado.

        \medskip

        Para un primo $p$ genérico con $p \distinto 3 \ytext p \distinto 5$:
        $$
          3p \divideA 5^{p-1} + 3^{p^2+2} + 833
          \equivalente
          \llave{l}{
            \congruencia{5^{p-1} + 3^{p^2+2}+833}{\blue{0}}{p} \llamada1 \\
            \congruencia{5^{p-1} + 3^{p^2+2}+833}{\blue{0}}{3} \llamada2
          }
        $$
        Laburo en $\llamada1$ con PTF:
        $$
          \llamada1
          \entonces
          \congruencia{5^0 + 3^1 + 833 = 837 = 3^3 \cdot 31}{\blue{0}}{p}
        $$
        Como el $p = 3$ está descartado a este punto tengo que pedir que $p = \purple{31}$ para que se cumpla la congruencia.
        Solo me falta chequear que $\llamada2$ funciones bien para este valor:
        $$
          \llamada2
          \entonces
          \congruencia{5^{\purple{31}-1} + 3^{\purple{31}^2+2}+833 \conga{3} 834 }{\blue{0}}{3}
        $$

        \medskip

        Concluyendo de esta manera que el número primo $31$ también es solución:
        $$
          \cajaResultado{
            p \en \set{5, 31}
          }
        $$
\end{enumerate}

\begin{aportes}
  \item \aporte{\dirRepo}{naD GarRaz \github}
\end{aportes}
