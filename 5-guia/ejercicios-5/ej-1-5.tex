\begin{enunciado}{\ejercicio}
    Determinar, cuando existan, todos los $(a,b) \en \enteros^2$ que satisfacen
    \begin{multicols}{4}
      \begin{enumerate}[label=\roman*)]
        \item $7a+11b=10$
        \item $20a+16b=36$
        \item $39a-24b=6$
        \item $1555a-300b=11$
      \end{enumerate}
    \end{multicols}
  \end{enunciado}


\begin{enumerate}[label=\roman*)]

  \item 
  Tiene solución, pues $(7:11)=1 \divideA 10$ \\
  Una solución particular es $(a_0,b_0)=(3,-1)$. Luego, la solución general es \\
  $$\boxed{(a,b)=(-11k+3,7k-1), k \en \enteros}$$


  \item
  Tiene solución, pues $(20:16)=4 \divideA 36$ \\
  Coprimizando la ecuación:
  $$
  20a+16b=36 
  \leftrightsquigarrow
  5a+4b=9
  $$
  \\
  Una solución particular de la ecuación equivalente es $(a_0,b_0)=(1,1)$. Luego, la solución general es \\
  $$\boxed{(a,b)=(4k+1,-5k+1), k \en \enteros}$$

  \item
  Tiene solución, pues $(39:-24)=3 \divideA 6$ \\
  Coprimizando la ecuación:
  $$
  39a-24b=6
  \leftrightsquigarrow
  13a-8b=2
  $$
  \\
  Una solución particular de la ecuación equivalente es $(a_0,b_0)= (2,3)$. Luego, la solución general es \\
  $$\boxed{(a,b)=(8k+2,13k+3), k \en \enteros}$$

  \item
  No tiene solución, pues $(1555:-300)=5 \noDivide 11$

  
\begin{aportes}
	\item \aporte{https://github.com/Nunezca}{Nunezca \github}
\end{aportes}

