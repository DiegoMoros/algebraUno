\begin{enunciado}{\ejercicio}
  Resolver en $\enteros$ la ecuación de congruencia $\congruencia{7X^{45}}{1}{46}$.

\end{enunciado}
Acomodo un poco la ecuación que esta fea:
$$
  \congruencia{7X^{45}}{1}{46}
  \Sii{$13 \cop 46$}
  \congruencia{91X^{45}}{13}{46}
  \Sii{\red{!}}
  \congruencia{X^{45}}{33}{46}
$$

La idea es quebrar para poder el PTF:

$$
  \flecha{quiebro}[\red{!}]
  \llave{l}{
    \congruencia{X^{45}}{10}{23}
    \Sii{$23\noDivide X$}[\red{!!!}]
    X^{22} X^{22} X^1 \taa{(23)}{\text{PTF}}{\congruente} \congruencia{X}{10}{23} \\

    \congruencia{X^{45}}{1}{2} \Sii{$X^{45}$ es impar}[entonces X también]
    \congruencia{X}{1}{2}                                                                    \\
  }
$$

En el \red{!!!} acomodo $X^{45}$ para poder usar el PTF y $\congruencia{X^{22}}{X^0}{23}$

Se tiene hasta el momento:
$$
  \congruencia{7X^{45}}{1}{46}
  \leftrightsquigarrow
  \llave{l}{
    \congruencia{X}{10}{23} \\
    \congruencia{X}{1}{2}
  }
$$

Sacar de acá, meter allá y coso:

$$
  X = 23\magenta{k} + 10 \conga2 \magenta{k} \congruente 1\ (2) \Sii{def} \magenta{k} = 2\yellow{j} + 1
$$
Por lo tanto:
$$
  X = 23(2\yellow{j} +1 ) + 10 = 46 \yellow{j} + 33 \Sii{def} \congruencia{X}{33}{46}
$$

% Contribuciones
\begin{aportes}
  %% iconos : \github, \instagram, \tiktok, \linkedin
  %\aporte{url}{nombre icono}
  \item \aporte{https://github.com/nad-garraz}{Nad Garraz \github}
\end{aportes}
