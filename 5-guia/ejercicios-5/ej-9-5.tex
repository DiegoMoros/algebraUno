\begin{enunciado}{\ejercicio}
    Describir los valores de $(5a+8:7a+3)$ en funcion de los valores de $a \en \enteros$
 \end{enunciado}
 
 Sea $d=(5a+8:7a+3)$. Entonces:
 $$
 \llave{l}{
     d \divideA 5a+8 \Entonces{$\times 7$} d \divideA 35a+56 \\
     d \divideA 7a+3 \Entonces{$\times 5$} d \divideA 35a+15
   }
 \Entonces{$F_1-F_2$}
 d \divideA 41
 \entonces 
 d \en \set{1,41}
 $$
 \\
 Ahora debemos ver cuando $41 \divideA 5a+8 \y 41 \divideA 7a+3$ simultáneamente. 
 Veamos primero cuando $41 \divideA 7a+3$ :
 \\
 $$
 41 \divideA 7a+3
 \sisolosi
 \congruencia{7a+3}{0}{41}
 \sisolosi
 \congruencia{7a}{-3}{41}
 \Sii{$(6:41)=1$}[]
 \congruencia{6\cdot7a}{6\cdot(-3)}{41}
 \sisolosi
 \congruencia{42a}{-18}{41}
 \sisolosi
 $$
 \\
 $$
 \sisolosi
 \congruencia{a}{23}{41}
 $$
 \\
 Ahora debemos ver si cuando $\congruencia{a}{23}{41}$ se verifica que $41 \divideA 5a+8$. Veamoslo:
 \\
 $$
 41 \divideA 5a+8 
 \sisolosi
 \congruencia{5a+8}{0}{41}
 \Sii{$\congruencia{a}{23}{41}$}[]
 \congruencia{5\cdot23+8}{0}{41}
 \sisolosi
 \congruencia{123}{0}{41}
 \sisolosi
 41 \divideA 123 \Tilde
 $$
 
 
 Luego, $41 \divideA 5a+8 \y 41 \divideA 7a+3 \sisolosi \congruencia{a}{23}{41}$
 \\
 De modo que
 $$
 \llave{ll}{
   \boxed{d=41} ~ si ~ \congruencia{a}{23}{41} \\
   \boxed{d=1} ~ si ~ \noCongruencia{a}{23}{41}
 }
 $$
 
 
 \begin{aportes}
     \item \aporte{https://github.com/Nunezca}{Nunezca \github}
 \end{aportes}
