\ejercicio

%Macro local
\def\sumLocal{\sumatoria{i=1}{1759}}
% fin macro local

Hallar el resto de la división de:
\begin{enumerate}[label=\roman*)]
	\item $43 \cdot 7^{135} + 24^{78} + 11^{222}$ por 70
	\item $\sumLocal i^{42}$ por 56
\end{enumerate}

\separadorCorto

\begin{enumerate}[label=\roman*)]
	\item \hacer

	\item Calcular el resto pedido equivale a resolver la ecuaición de equivalencia:\\
	      $ \congruencia{X}{\sumLocal i^{42}}{56} $ que será aún más simple en la forma:
	      $\llave{l}{
			      \congruencia{X}{\sumLocal i^{42}}{7}\\
			      \congruencia{X}{\sumLocal i^{42}}{8}
		      }\\
		      \text{Primerlo estudio la ecuación de módulo 7: }\\
		      \llave{l}{
                \congruencia{\sumLocal i^{42}}{X}{7} \llamada{1}

			      \flecha{7 es primo, uso \magenta{Fermat}}[si $p \noDivide i \to i^{42} = \congruencia{(i^6)^7}{1}{7}$]
			      \sumLocal i^{42} = \sumLocal (i^6)^7
			      \flecha{$251 \cdot 7 + 2 = 1759$}\\
			      \sumLocal (i^6)^7 \conga7 251 \cdot \parentesis{(1^6)^7 + (2^6)^7 + (3^6)^7 + (4^6)^7 + (5^6)^7 + (6^6)^7 + (7^6)^7} + \parentesis{(1^6)^7 + (2^6)^7 + (3^6)^7 + (4^6)^7}\\
			      \sumLocal (i^6)^7 \magenta{\conga7 } 251 \cdot \parentesis{1 + 1 + 1 + 1 + 1 + 1 + 0} + \parentesis{1 + 1 + 1 + 1} =
			      251 \cdot  6 + 4 \conga7 3\\
                  \flecha{$\llamada{1}$ }
                  \boxed{\congruencia{X}{3}{7}}\\
		      }$\\
	      Ahora se labura el módulo 8.\\

	      $\llave{l}
		      {
			      \congruencia{\sumLocal i^{42}}{X}{8}
			      \flecha{8 no es primo}[no uso Fermat]
			      \text{Analizo a mano}
			      \flecha{$219 \cdot 8 + 7 = 1759$}
			      \congruencia{X}{\sumLocal i^{42}}{8} \conga8
			      \\
			      \conga8 219 \cdot \ub{(1^{42} + 2^{42} + 3^{42} + 4^{42} + 5^{42} + 6^{42} + 7^{42} + 0^{42})}{\text{8 términos: } r_8(i^{42}) = (r_8(i))^{42} }
			      + (1^{42} + 2^{42} + 3^{42} + 4^{42} + 5^{42} + 6^{42} + 7^{42}) \\
			      \to
			      \llaves{l}{
				      2^{42} = (2^3)^{14}\conga8 0\\
				      4^{42} = (2^3)^{14} \cdot (2^3)^{14}\conga8 0\\
				      6^{42} = (2^3)^{14} \cdot 3^{42} \conga8 0\\
				      1^{42} = 1\\
				      3^{42} = (3^2)^{21} \conga8 1^{21} = 1\\
				      5^{42} = (5^2)^{21} \conga8 1^{21} = 1\\
				      7^{42} = (7^2)^{21} \conga8 1^{21} = 1
			      }\\
			      \flecha{reemplazo}[esa en]
			      \sumLocal i^{42} \conga8 219 \cdot 4 + 4  = 880 \conga8 0 \to \boxed{\congruencia{X}{0}{8}}\\
		      }$\\
	      El sistema
	      $\llave{l}{
			      \congruencia{X}{3}{7} \\
			      \congruencia{X}{0}{8} \\
		      }$ tiene solución $\congruencia{X}{24}{56}$, por lo tanto el \textit{resto pedido}: \boxed{ r_{56}\parentesis{\sumLocal i^{42}} = 24}
\end{enumerate}
