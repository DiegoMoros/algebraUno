\begin{enunciado}{\ejercicio}
  Hallar el resto de la división de $a$ por $p$ en los casos.
  \begin{enumerate}[label=\alph*)]
    \item $a = 33^{1427}, \, p =5$
    \item $a = 71^{22283},\, p=11$
    \item $a = 5 \cdot 7^{2451} + 3 \cdot 65^{2345} - 23 \cdot 8^{138}, \, p = 13$
  \end{enumerate}
\end{enunciado}

\begin{enumerate}[label=\alph*)]
  \item Escribo como ecuación de congruencia:
        $$
          \congruencia{33^{1427}}{3^{1427}}{5}
        $$
        Dado que 5 es primo puedo usar el PTF, notar que $r_4(1427) = 3 $
        $$
          \congruencia{33^{1427}}{3^{1427}}{5}
          \Sii{PTF}
          \congruencia{3^{1427}}{3^3}{5} \ytext 3^3 = 27 \conga5 2
        $$
                Por lo tanto:
                $$
                r_5(33^{1427}) = 2
                $$
  \item
        Rescribo: $22283 = 22280 + 3$ y notar que el $r_{10}(22280) = 0$
        $$
          a = 71^{22283} =
          71^{22280 + 3} =
          71^{22280} \cdot 71^3
          \taa{PTF}{\red{!}}\congruente 71^3 \ (11) \sii \congruencia{a}{5^3 \congruente 4}{11}
        $$
        Por lo tanto:
        $$
          r_{11}(a) = 4
        $$

  \item
        Acomodo un poco la expresión, pensando en el PTF. Los exponentes tienen que quedar lindos para encontrar los restos de $p-1$
        $$
          \begin{array}{l}
            \congruencia{a}{5 \cdot 7^{2448 + 3} + 0 - 10 \cdot 8^{132 + 6}}{13}
            \Sii{PTF}[\red{!}]
            \congruencia{a}{5 \cdot 7^3 - 10 \cdot 8^6}{13}
            \Sii{\red{!!}}
            \congruencia{a}{5 \cdot 5 - 23 \cdot 12}{13}
          \end{array}
        $$
        \textit{\faIcon{calculator} Nota que puede ser de interés:}\par
        Con la calculadora salen fácil los cálculos, pero está bueno poder calcularlos a mano masajeando las potencias, onda
        $8^6 = 64^3 \conga{13} (-1)^3 = -1 \conga{13} 12$\par
        \textit{\faIcon{calculator} Fin Nota que puede ser de interés.}

        Un par de cuentas y calcular congruencia y queda:
        $$
          \congruencia{a}{9}{13}
        $$
\end{enumerate}

% Contribuciones
\begin{aportes}
  %% iconos : \github, \instagram, \tiktok, \linkedin
  %\aporte{url}{nombre icono}
  \item \aporte{https://github.com/nad-garraz}{Nad Garraz \github}
\end{aportes}
