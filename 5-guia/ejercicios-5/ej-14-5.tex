\begin{enunciado}{\ejercicio}
  Hallar los posibles restos de dividir a un entero $a$ por 238 sabiendo que $\congruencia{a^2}{21}{238}$.
\end{enunciado}

$$
  \congruencia{a^2}{21}{238}
  \equivalente
  \llave{l}{
    \congruencia{a^2}{1}{2}  \Sii{\red{!}} \congruencia{a}{1}{2}\\
    \congruencia{a^2}{0}{7} \Sii{\red{!}} \congruencia{a}{0}{7}\\
    \congruencia{a^2}{4}{17} \Sii{$\llamada1$}
    \llave{l}{
      \congruencia{a}{2}{17} \\
      \otext \\
      \congruencia{a}{15}{17}
    }
  }
$$

Donde $\llamada1$:
$$
  \begin{array}{|c|c|c|c|c|c|c|c|c|c|c|c|c|c|c|c|c|c|}
    \hline
    r_{17}(a)   & 0 & 1 & \cyan{2} & 3 & 4  & 5 & 6 & 7  & 8  & 9  & 10 & 11 & 12 & 13 & 14 & \cyan{15} & 16 \\ \hline
    r_{17}(a^2) & 0 & 1 & \cyan{4} & 9 & 16 & 8 & 2 & 15 & 13 & 13 & 15 & 2  & 2  & 16 & 6  & \cyan{4}  & 1  \\ \hline
  \end{array}
$$

Y el resto es historia. Dos hermosos sistemas para resolver:
$$
  \llave{l}{
    \congruencia{a}{1}{2} \llamada2\\
    \congruencia{a}{0}{7}\llamada3\\
    \congruencia{a}{2}{17} \llamada4
  }
  \ytext
  \llave{l}{
    \congruencia{a}{1}{2} \llamada2\\
    \congruencia{a}{0}{7} \llamada3\\
    \congruencia{a}{15}{17} \llamada5
  }
$$
$$
  \llamada2 \congruencia{a}{1}{2}
  \Sii{def}
  a = 2\blue{k} + 1
  \Entonces{meto}[en $\llamada3$]
  \congruencia{2\blue{k} + 1}{0}{7}
  \sii
  \congruencia{\blue{k}}{3}{7}
  \Entonces{reemplazo}[en $a$]
  a = 2(7 \green{h} + 3) + 1 = 14\green{h} + 7 \llamada6
$$
Ese valor de $\llamada6$ lo uso en las otras dos ecuaciones:
$$
  \begin{array}{c}
    \flecha{meto en}[$\llamada4$]
    \congruencia{14\green{h} + 7}{2}{17}
    \Sii{$\times 11$}[$11\cop 17$]
    \congruencia{\green{h}}{13}{17}
    \Entonces{reemplazo}[en $a$]
    a = 14(17\magenta{j} + 13) + 7 = 238\magenta{j} + 189 \\
    \flecha{meto en}[$\llamada5$]
    \congruencia{14\green{h} + 7}{15}{17}
    \Sii{$\times 11$}[$11\cop 17$]
    \congruencia{\green{h}}{3}{17}
    \Entonces{reemplazo}[en $a$]
    a = 14(17\magenta{j} + 3) + 7 = 238\magenta{j} + 49   \\
  \end{array}
$$

Se termina el ejercicio. Los valores que cumplen lo pedido:
$$
  \cajaResultado{
    \congruencia{a}{189}{238}
    \ytext
    \congruencia{a}{49}{238}
  }
$$

\begin{aportes}
  \item \aporte{\dirRepo}{naD GarRaz \github}
\end{aportes}
