\def\MCD{(a:b)}

\begin{itemize}
	\item Sea $aX + bY = c \text{ con } a,\, b,\, c \en \enteros,\, a \distinto 0 \y b \distinto 0$ y sea
	      $S = \set{ (x,y) \en \enteros^2 : aX + bY = C }$.\\
	      Entonces $S \distinto \vacio \sisolosi (a:b) \divideA c$
	      % \sisolosi
	      %  \ub{A}{\frac{a}{\MCD}} x + \ub{B}{\frac{b}{\MCD}} y = \ub{C}{\frac{c}{(a:b)}} \en \enteros$,
	      % \text{también tiene solución para $x$ y $y$.}

	\item Las soluciones al sistema: $S = \set{(x,y) \en \enteros^2 \text{ con }
			      \llaves{l}{
				      x = x_0 + kb'\\
				      y = y_0 + kb'
			      }
			      , k \en \enteros}
	      $

	\item $\congruencia{aX}{c}{b}$  con  $ a,\, b \distinto 0$ tiene solución $\sisolosi \MCD \divideA c$ tiene solución $\sisolosi \MCD \divideA c$. En ese caso, coprimizando:
\end{itemize}

\textit\underline{Ecuaciones de congruencia}

\begin{itemize}
	\item Algoritmo de solución:
	      \begin{enumerate}[label=\arabic*)]
		      \item reducir $a,\, c$ módulo $m$. Podemos suponer $0 \leq a, c < m$
		      \item tiene solución $\sisolosi (a:m) \divideA c$. Y en ese caso coprimizo:
		            \[
			            \congruencia{aX}{c}{m} \sisolosi \congruencia{a'X}{c'}{m},\ \text{ con } a' = \frac{a}{(a:m)},\, m' = \frac{m}{(a:m)} \text{ y } c' = \frac{c}{(a:m)}
		            \]
		      \item   Ahora que $a' \cop m'$, puedo limpiar los factores comunes entre
                $a'$ y $c'$ (los puedo simplificar)
		            \[
                      \congruencia{a'X}{c'}{m'}
                      \sisolosi
                      \congruencia{a''X}{c''}{m'}
                      \text{ con }
                      a'' = \frac{a'}{(a':c')} \text{ y } c'' = \frac{c'}{(a':c')}
		            \]
		      \item Encuentro una solución particular $X_0$ con $ 0\leq X_0 < m'$ y tenemos
		            \[
			            \congruencia{aX}{c}{m} \sisolosi \congruencia{X}{X_0}{m'}
		            \]
	      \end{enumerate}

\end{itemize}

\textit\underline{Ecuaciones de congruencia}
Sean $m_1,\dots m_n \en \enteros$ \underline{coprimos dos a dos} ($\paratodo i \distinto j$, se tiene $m_i \cop m_j$). \\
Entonces, dados $c_1,\dots, c_n \en \enteros$ cualesquiera, el sistema de ecuaciones de congruencia.
\[
	\llave{c}{
		\congruencia{X}{c_1}{m_1}\\
		\congruencia{X}{c_2}{m_2}\\
		\vdots\\
		\congruencia{X}{c_n}{m_n}
	}
\]

es equivalente al sistema (tienen misma soluciones)

\[
	\congruencia{X}{x_0}{m_1\cdot m_2 \cdots m_n}
\]
para algún $x_0$ con $0\leq x_0 < m_1\cdot m_2 \cdots m_n$

\textit{\underline{Pequeño teorema de Fermat}}
\begin{itemize}
	\item Sea $p$ primo, y sea $a \en \enteros$. Entonces:
	      \begin{enumerate}[label=\arabic*.)]
		      \item $ \congruencia{a^p}{a}{p} $
		      \item $ p \noDivide a \entonces \congruencia{a^{p-1}}{1}{p} $
	      \end{enumerate}
	\item Sea $p$ primo, entonces $ \paratodo a \en \enteros$ tal que $ p \noDivide a$ se tiene:
	      \[
		      \congruencia{a^n}{a^{r_{p-1}(n)}}{p} ,\, \paratodo n \en \naturales
	      \]
	\item Sea $a \en \enteros$ y $p > 0$ primo tal que $\ub{(a:p) = 1}{a \cop p}$, y sea  $d \en \naturales$ con $d \leq p-1$
	      el mínimo tal que:
	      \[
		      \congruencia{a^d}{1}{p} \entonces d \divideA (p-1)
	      \]
\end{itemize}


\textit{\underline{Aritmética modular:}}

% Macro local
\newcommand{\moduloN}[1]{\enteros/_{#1\enteros}}
% Fin macro local

\begin{itemize}
	\item Sea $n \en \naturales, n\geq2$\\
	      $
		      \moduloN{n} = \set{\clase{0},\clase{1},\cdots, \clase{n-1} }\\
		      \clase{a},\clase{b} \en \moduloN n:
		      \llave{l}{
			      \clase{a} \clase{+} \clase{b} := \overline{r_n(a+b)} \\
			      \clase{a} \clase{\cdot} \clase{b} := \overline{r_n(a\cdot b)}
		      }
	      $
	\item Sea $p$ primo, en $\moduloN{p}$ todo elemento no nulo tiene inverso multiplicativo,
	      análogamente a $\enteros$.\\
	      Si $m \en \naturales$ es compuesto,

	      \begin{itemize}
		      \item \textit{No todo} $\clase{a} \en \moduloN{m}$ con $\clase{a} \distinto \clase{0}$ es inversible.

		      \item $\existe \clase{a}, \clase{b} \en \moduloN{m}$ con $\clase{a}, \clase{b} \distinto 0$ tal que
		            $\clase{a}\cdot \clase{b} = \clase{0}$

		      \item Inv$(\moduloN m) = \set{\clase{a} \en \set{\clase{0},\clase{1},\dots, \clase{m-1}}}$ tales que
		            $a \cop m$
	      \end{itemize}
	\item Si $m = p$, con $p$ primo, todo elemento no nulo de $\moduloN p$ tiene inverso:
	      \begin{itemize}
		      \item Inv$(\moduloN p) = \set{\clase{1},\dots, \clase{p-1}}$.
		      \item  \underline{$p$ primo $\entonces \moduloN p$ es un cuerpo.}
              \item en $\moduloN{p}:\ (\clase{a} + \clase{b})^p = \clase{a}^p + \clase{b}^p $
	      \end{itemize}

\end{itemize}
