\begin{enunciado}{\ejExtra}
  Sea $n\en \naturales$ tal que $(n^{109} + 37:52) = 26$ y $(n^{63} -21:39) = 39$.
  Calcular el resto de dividir a $n$ por 156.
\end{enunciado}

Masajeando un poco el enunciado:

$$
  (n^{109} + 37: 13 \cdot 2^2) =  13 \cdot 2
  \ytext
  (n^{63} -21 : 13 \cdot 3) = 13 \cdot 3
$$

\textit{¿Qué información obtenemos de los máximos común múltiplos?: }

Para que $(n^{109} + 37 : 52) = 26$ debe ocurrir que:

$$
  \llamada1
  \llave{rcl}{
    n^{109} + 37 \divideA 13 & \Sii{def} & \congruencia{n^{109}}{2}{13} \Sii{$13$ primo}[$13 \noDivide n$] \congruencia{n^{\blue1}}{2}{13} \\
    n^{109} + 37 \divideA 2  & \Sii{def} & \congruencia{n^{109}}{1}{2}  \Sii{\red{!!}} \congruencia{n}{1}{2}                                     \\
    n^{109} + 37 \noDivide 4 & \Sii{def} & \noCongruencia{n^{109}}{3}{4} \Sii{$\llamada3$}
    \llave{l}{
      \congruencia{n^{109}}{0}{4}                                            \\
      \otext                                                                                        \\
      \boxed{\congruencia{n^{109}}{1}{4}} \Sii{\red{!!!}}[$n \cop 4$] \congruencia{n}{1}{4} \\
      \otext                                                                                        \\
      \congruencia{n^{109}}{2}{4}
    }
  }
$$
En $\red{!!}$ pienso en la paridad que tiene que tener $n$. Dado que $n^{109}$ tiene que ser impar, concluyo que $n$ es impar. Como $n$ debe ser impar en $\llamada3$, y me dicen que
$\noCongruencia{n^{109}}{3}{4}$, solo puede ocurrir que $\congruencia{n^{109}}{1}{4}$.

Si eso último te parece medio fantasma \purple{\faIcon[regular]{ghost}}, podés encarar $\llamada3$, quizás más mecánicamente, armando 3 sistemas, uno para
cada condición del divisor 4. De esa forma vas a eliminar \textit{incompatibilidades}
y vas a terminar llegando al único sistema posible {\tiny (me contó un pajarito \blue{\faIcon{twitter}})}.

\medskip

El sistema de $\llamada1$ queda como:

\medskip

Para que $(n^{63} -21 : 39) = 39$ debe ocurrir que:

$$
  \llamada2
  \llave{rcl}{
    13 \divideA  n^{63} -21 & \Sii{def} &
    \congruencia{n^{63}}{8}{13}
    \Sii{$13$ primo}[$13 \noDivide n$]
    \congruencia{n^{\blue{3}}}{2^3}{13}
    \Sii{\red{!!!}}[$n \cop 13$]
    \congruencia{n}{2}{13}                \\
    3 \divideA  n^{63} -21  & \Sii{def} &
    \congruencia{n^{63}}{0}{3}
    \Sii{\red{!}}
    \congruencia{n}{0}{3}                 \\
  }
$$

\bigskip

Junto info que salió de los máximos común múltiplos. Quedan dos sistemas de $\llamada1$ y $\llamada2$:

$$
  \llave{l}{
    \congruencia{n}{2}{13} \\
    \congruencia{n}{1}{2}  \\
    \congruencia{n}{1}{4}
    \scriptscriptstyle
    \leftrightsquigarrow
    \scriptscriptstyle{
      \llave{l}{
        \scriptscriptstyle
        \congruencia{n}{1}{2} \\
        \scriptscriptstyle
        \congruencia{n}{1}{2}
      }
    }
  }
  \ytext
  \llave{l}{
    \congruencia{n}{2}{13} \\
    \congruencia{n}{0}{3}
  }
$$

Juntando la info para que sea compatible. Me quedo con $\congruencia{n}{1}{4} $ para no agregar soluciones de más:

$$
  \llave{l}{
    \congruencia{n}{2}{13} \\
    \congruencia{n}{1}{4}  \\
    \congruencia{n}{0}{3}
  }
$$
Los divisores son coprimos dos a dos, es decir que por \href{\chinito}{TCHR} tenemos solución. La cual queda si no hago cagadas en las cuentas:
$$
  \congruencia{n}{93}{156}
$$
Por lo tanto el resto que nos pedían es:
$$
  \cajaResultado{r_{156}(n) = 93}
$$
\begin{aportes}
  \item \aporte{\dirRepo}{naD GarRaz \github}
\end{aportes}
