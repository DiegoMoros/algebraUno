\ejExtra
Hallar los posibles restos de dividir a $a$ por 70, sabiendo que
$(a^{1081}+ 3a + 17 : 105) = 35$\\

\separadorCorto

$ (\ub{a^{1081}+ 3a + 17}{m} : \ub{105}{3\cdot 5 \cdot 7})  = \ub{35}{5 \cdot 7}
	\flecha{debe ocurrir}[que]
	\llave{c}{
		5 \divideA m\\
        \text{y}\\
		7 \divideA m\\
        \text{y}\\
		3 \noDivide m
	}\\
	5 \divideA m
	\to \congruencia{a^{1081}+ 3a + \ub{17}{\conga{5} 2}}{0}{5}
	\to
	\llave{cl}{
		\text{si} & 5 \divideA a \to \congruencia{2}{0}{5}
		\entonces \noCongruencia{a}{0}{5} \\
        \text{o}\\
		\text{si} & 5 \noDivide a
		\flecha{$a^{1081} = a (a^4)^{270}$}[5 primo y $5 \noDivide a$]
		\congruencia{a + 3a + 2}{0}{5}
        \entonces
        \boxed{\congruencia{a}{2}{5}}\\
	}
	\\
	7 \divideA m
	\to \congruencia{a^{1081}+ 3a + \ub{17}{\conga{7} 3}}{0}{7}
	\to
	\llave{cl}{
		\text{si} & 7 \divideA a \to \congruencia{3}{0}{7} \entonces \noCongruencia{a}{0}{7}\\
        \text{o}\\
		\text{si} & 7 \noDivide a
		\flecha{$a^{1081} = a (a^6)^{180}$}[7 primo y $7 \noDivide a$]
		\congruencia{a + 3a + 3}{0}{7} \to \congruencia{4a}{-3}{7}
        \entonces
        \boxed{\congruencia{a}{1}{7}} \\
	}
	\\
	3 \noDivide m
	\to \noCongruencia{a^{1081} + \ub{3}{=0}a + \ub{17}{\conga{3} 2}}{0}{3}
	\to
	\llave{ll}{
		\text{si} & 3 \divideA a 
        \to
        \noCongruencia{2}{0}{3} \entonces \congruencia{a}{0}{3} \\
        \text{o}\\
		\text{si} & 3 \noDivide a
		\flecha{$a^{1081} = a (a^2)^{540}$}[3 primo y $3 \noDivide a$]
		\noCongruencia{a + 2}{0}{3}
        \entonces
        \llave{l}{
        \noCongruencia{a}{1}{3}\\
        \noCongruencia{a}{0}{3}\\
        }
        \entonces
        \boxed{\congruencia{a}{2}{3}}\\
	}
$\\
Las condiciones marcan 2 sistemas:\\
\begin{minipage}{0.5\textwidth}
	\centering
	$
		\llave{l}{
			\congruencia{a}{2}{5} \\
			\congruencia{a}{1}{7} \\
			\congruencia{a}{0}{3}
		}
		\to \boxed{\congruencia{a}{22}{105} }
	$
\end{minipage}
\begin{minipage}{0.5\textwidth}
	\centering
	$\llave{l}{
			\congruencia{a}{2}{5} \\
			\congruencia{a}{1}{7} \\
			\congruencia{a}{2}{3}
		}
		\to \boxed{\congruencia{a}{92}{105} }
	$
\end{minipage}
Veo que para el conjunto de posibles $a$
$\llaves{c}{
		a = 105k_1 + 22 \\
		o \\
		a = 105k_2 + 92
	}\flecha{calculo}[$\conga{70}$]$
$\congruencia{a}{22}{35}\\
\flecha{quiero los restos}[pedidos del enunciado]
r_{70}(a) = \set{22, 57}$, 
valores de $a$ que cumplan condición de $r_{70}(a)$ \Tilde
