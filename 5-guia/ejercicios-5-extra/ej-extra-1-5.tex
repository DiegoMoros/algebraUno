\begin{enunciado}{\ejExtra}
  Hallar los posibles restos de dividir a $a$ por 70, sabiendo que
  $(a^{1081}+ 3a + 17 : 105) = 35$
\end{enunciado}
Como $105 = 3\cdot5\cdot7$ quiero que ocurra:
$$
  \llave{l}{
    \congruencia{a^{1081}+ 3a + 17}{0}{5} \\
    \congruencia{a^{1081}+ 3a + 17}{0}{7} \\
    \noCongruencia{a^{1081}+ 3a + 17}{0}{3}
  }
  \Sii{\red{!}}
  \llave{l}{
    \congruencia{a^{1081}+ 3a + 2}{0}{5} \llamada1 \\
    \congruencia{a^{1081}+ 3a + 3}{0}{7} \llamada2 \\
    \noCongruencia{a^{1081} + 2}{0}{3} \llamada3.
  }
$$
De esa forma me aseguro que el MCD sea 35. Ahora empiezo a estudiar $\llamada1$:

$$
  \congruencia{a^{1081}+ 3a + 2}{0}{5}
  \sii
  \llave{cl}{
    \flecha{caso si}[$\congruencia{a}{0}{5}$]             & \congruencia{2}{0}{5}
    \Entonces{concluyo}[que]
    \noCongruencia{a}{0}{5}                                                       \\
                                                          & \text{o}              \\
    \flecha{PTF: $5$ es primo}[y $\noCongruencia{a}{0}{5}$] &
    \congruencia{a + 3a + 2}{0}{5}
    \sii
    \congruencia{a}{2}{5}}                     \\
$$
\textit{Hasta el momento} tengo que para que se cumpla lo pedido:
$$
  \congruencia{a}{2}{5} \llamada4
$$

Busco más condiciones en $\llamada2$:
$$
  \congruencia{a^{1081}+ 3a + 1}{0}{7}
  \sii
  \llave{cl}{
    \flecha{caso si}[$\congruencia{a}{0}{7}$]             & \congruencia{1}{0}{7}
    \Entonces{concluyo}[que]
    \noCongruencia{a}{0}{7}                                                        \\
                                                           & \text{o}              \\
    \flecha{PTF: $7$ es primo}[y $\noCongruencia{a}{0}{7}$] &
    \congruencia{4a + 3}{0}{7}
    \sii
    \congruencia{a}{1}{7}}                    . \\
$$
Se concluye de este último resultado que $a$ también puede tomar los valores:
$$
  \congruencia{a}{1}{7} \llamada5
$$
Por último falta estudiar los valores de $a$ en $\llamada3$:
$$
  \noCongruencia{a^{1081} + 2}{0}{3}
  \sii
  \llave{cl}{
    \flecha{caso si}[$\congruencia{a}{0}{3}$]             & \noCongruencia{2}{0}{3}
    \Entonces{concluyo}[que]
    \congruencia{a}{0}{3} \llamada6                                                  \\
                                                           & \text{o}                \\
    \flecha{PTF: $3$ es primo}[y $\noCongruencia{a}{0}{3}$] &
    \noCongruencia{a + 2}{0}{3}
    \sii
    \noCongruencia{a}{1}{3}
    \Entonces{concluyo}[que \red{!!}] \congruencia{a}{2}{3} \llamada6.
  }
$$
En el \red{!!} estoy en la rama donde $\noCongruencia{a}{0}{3}$ y si $\congruencia{a}{1}{3}$ no se cumpliría $\llamada3$, por eso el único valor que sale
de esa rama es $\congruencia{a}{2}{3}$.

Se concluye de este último resultado que $a$ también puede tomar los valores:
$$
  \congruencia{a}{0}{3}
  \otext
  \congruencia{a}{2}{3}
$$

El resultado del estudio de $\llamada1, \llamada2 \ytext \llamada3$,
dio los resultados de $\llamada4, \llamada5, \llamada6$ con los cuales se puede formar los 2 sistemas:
$$
  \llave{l}{
    \congruencia{a}{2}{5} \\
    \congruencia{a}{1}{7} \\
    \congruencia{a}{0}{3}
  }
  \flecha{divisores coprimos}[\href{\chinito}{TCH}] \boxed{\congruencia{a}{57}{105} }
  \ytext
  \llave{l}{
    \congruencia{a}{2}{5} \\
    \congruencia{a}{1}{7} \\
    \congruencia{a}{2}{3}
  }
  \flecha{divisores coprimos}[\href{\chinito}{TCH}]
  \boxed{\congruencia{a}{92}{105} }
$$

El conjunto de posibles $a$:

$$
  \llave{c}{
    a = 105k_1 + 57 \\
    \otext          \\
    a = 105k_2 + 92
  }
$$
Solo falta calcular el $r_{70}(a)$ para estos valores de $a$. Calcular el resto es ver la congruencia 70, $r_{70}(a)$:
$$
  \llave{l}{
    a \conga{70} 35 k_1 + 57 \Sii{def}  \cajaResultado{\congruencia{a}{57}{70}} \\
    a \conga{70} 35 k_2 + 22 \Sii{def}  \cajaResultado{\congruencia{a}{22}{70}}
  }
$$

Los restos pedidos de dividir $a$ por 70:
$$
  \cajaResultado{
    r_{70}(a) \en \set{22, 57}
  }
$$

% Contribuciones
\begin{aportes}
  %% iconos : \github, \instagram, \tiktok, \linkedin
  %\aporte{url}{nombre icono}
  \item \aporte{\dirRepo}{naD GarRaz \github}
  \item \aporte{\neverGonnaGiveYouUp}{Nacho \youtube}
\end{aportes}

