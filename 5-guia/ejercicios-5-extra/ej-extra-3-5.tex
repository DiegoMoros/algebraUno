\ejercicio
Determinar todos los $n \en \enteros$ tales que
$$(n^{433} + 7n + 91 : 931) = 133.$$
Expresar las soluciones mediante una única ecuación.

\separadorCorto

%Macro
\def\expresion{n^{433} + 7n + 91}

Para que se cumpla que
$(n^{433} + 7n + 91 : \ub{931}{7^2\cdot 19}) = \ub{133}{7\cdot 19}$
deben ocurrir las siguientes condiciones:\\
$
	\llave{l}{
		7 \divideA \expresion\\
		19 \divideA \expresion\\
		7^2 \noDivide \expresion\\
	}
$\\

\textit{Estudio la divisibilidad $7$: }\\
Si
$
	7 \divideA \expresion
	\sisolosi
	\congruencia{\expresion}{0}{7}
	\sisolosi
	\congruencia{n^{433}}{0}{7}
	\flecha{analizo casos}[$7 \divideA n$ o $7 \noDivide n$]\\
	\congruencia{n^{433}}{0}{7}
	\sii
	\llave{l}{
		\text{(\magenta{rama  $7\noDivide n$})\ }
		\flecha{7 es primo}[$7\noDivide n$]
		\congruencia{(\ub{n^6}{\conga7 1})^{72}\cdot n}{0}{7}
		\sii
		\congruencia{n}{0}{7},
		\text{ pero esta rama } 7 \noDivide n \to \skull\\
		% --
		\text{(\magenta{rama  $7\divideA n$})\ }
		\flecha{7 es primo}[$7 \divideA n$]
		\congruencia{0}{0}{7}
		\text{ y como esta rama } 7 \divideA n
		\to
		\boxed{\congruencia{n}{0}{7} }\Tilde\llamada1\\
	}$\\

\textit{Conclusión divisibilidad $7$:}
$$7 \divideA \expresion \sii \congruencia{n}{0}{7}$$
\\

\textit{Estudio la divisibilidad $7^2 = 49$: }\\
Si
$
	7^2 \noDivide \expresion
	\sisolosi
	\noCongruencia{\expresion}{0}{49}
	\sisolosi
	\noCongruencia{n^{433}+ 7n + 42}{0}{49}\\
	\flecha{de $\llamada1$ tengo que}[$\congruencia{n}{0}{7} \sii n = \blue{7k}$]
	\noCongruencia{(\blue{7k})^{433} + 7\cdot\blue{7k} + 42}{0}{49}
	\sii
	\noCongruencia{7 \cdot (49)^{216}\cdot k^{433} + 49k + 42}{0}{49}
	\sii
	\noCongruencia{42}{0}{49}
$\\
\textit{Conclusión divisibilidad $49$:}
$$49 \noDivide \expresion\  \paratodo n \en \enteros$$\\

\textit{Estudio la divisibilidad $19$: }\\
Si
$
	19 \divideA \expresion
	\sisolosi
	\congruencia{\expresion}{0}{19}
	\sisolosi
	\congruencia{n^{433} + 7n +15}{0}{19}
	\flecha{analizo casos}[$19 \divideA n$ o $19 \noDivide n$]\\
	\congruencia{n^{433}+ 7n +15}{0}{19}
	\sii
	\llave{l}{
		\text{(\magenta{rama  $19\noDivide n$})\ }
		\flecha{19 es primo}[$19\noDivide n$]
		\congruencia{(\ob{n^{18}}{\conga{19} 1})^{24}\cdot n + 7n + 15}{0}{19}
		\sii
		\congruencia{8n}{-15}{19}
		\sii\\
		\stackrel{\times 7}\sisolosi
        \boxed{\congruencia{n}{10}{19}}\Tilde\llamada2\\
		% --
		\text{(\magenta{rama  $19\divideA n$})\ }
		\flecha{19 es primo}[$19 \divideA n$]
		\congruencia{15}{0}{19} \to \text{ningún } n
	}$\\

\textit{Conclusión divisibilidad $19$:}
$$19 \divideA \expresion \sii \congruencia{n}{10}{19}$$
\\

$$
	\llave{l}{
		\llamada1\congruencia{n}{0}{7}\\
		\llamada2\congruencia{n}{10}{19}
	}
	\flecha{$7 \cop 19$}[hay solución por TCH]
	\llave{l}{
		\flecha{$\llamada2$}[en $\llamada1$]
		n = 7 (19k + 10)= 133k + 70
		\to
		\boxed{\congruencia{n}{70}{133}}\Tilde
	}
$$
