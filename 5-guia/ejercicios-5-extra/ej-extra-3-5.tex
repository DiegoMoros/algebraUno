%Macro
%================
\def\expresion{n^{433} + 7n + 91 \xspace}
%================

\begin{enunciado}{\ejExtra}
  Determinar todos los $n \en \enteros$ tales que
  $$
    (n^{433} + 7n + 91 : 931) = 133.
  $$
  Expresar las soluciones mediante una única ecuación.
\end{enunciado}

Para que se cumpla que
$(n^{433} + 7n + 91 : \ub{931}{7^2\cdot 19}) = \ub{133}{7\cdot 19}$
deben ocurrir las siguientes condiciones:
$$
  \llave{rcl}{
    \vspace{5pt}
    7   & \divideA  & \expresion \\
    \vspace{5pt}
    19  & \divideA  & \expresion \\
    7^2 & \noDivide & \expresion
  }
$$

\textit{Estudio la divisibilidad $7$: }

Si
$$
  7 \divideA \expresion
  \sisolosi
  \congruencia{\expresion}{0}{7}
  \sisolosi
  \congruencia{n^{433}}{0}{7}
$$
Analizo esa expresión en los casos cuando $7 \divideA n$ y cuando $7 \noDivide n$ o lo que es lo mismo cuando
$\congruencia{n}{0}{7}$ y cuando $\noCongruencia{n}{0}{7}$:
$$
  \congruencia{n^{433}}{0}{7}
  \sii
  \llave{l}{
          \Sii{$\congruencia{n}{0}{7}$}
    \congruencia{0}{0}{7}
    \entonces
    \boxed{\congruencia{n}{0}{7} }\Tilde\llamada1\\
    % --
    \Sii{PTF, 7 es primo}[$\noCongruencia{n}{0}{7}$]
    \congruencia{n}{0}{7} \to \text{incompatible } \red{\skull}
  }
$$

\textit{Conclusión divisibilidad $7$:}
$$
  7 \divideA \expresion \sii \congruencia{n}{0}{7}
$$

\textit{Estudio la divisibilidad $7^2 = 49$: }

Si,
$$
  7^2 \noDivide \expresion
  \Sii{def}
  \noCongruencia{\expresion}{0}{49}
  \sisolosi
  \noCongruencia{n^{433}+ 7n + 42}{0}{49},
$$
de $\llamada1$ tengo que $\congruencia{n}{0}{7} \Sii{def} n = \blue{7k}$, por lo tanto:
$$
  \noCongruencia{(\blue{7k})^{433} + 7\cdot\blue{7k} + 42}{0}{49}
  \sii
  \noCongruencia{7 \cdot (49)^{216}\cdot k^{433} + 49k + 42}{0}{49}
  \sii
  \noCongruencia{42}{0}{49}
$$

\textit{Conclusión divisibilidad $49$:}
$$
  49 \noDivide \expresion  \paratodo n \en \enteros
$$

\textit{Estudio la divisibilidad $19$: }

Si
$$
  19 \divideA \expresion
  \sisolosi
  \congruencia{\expresion}{0}{19}
  \sisolosi
  \congruencia{n^{433} + 7n +15}{0}{19}
$$
Analizo esa expresión en los casos cuando $19 \divideA n$ y cuando $19 \noDivide n$ o lo que es lo mismo cuando
$\congruencia{n}{0}{19}$ y cuando $\noCongruencia{n}{0}{19}$:
$$
  \congruencia{n^{433}+ 7n +15}{0}{19}
  \sii
  \llave{l}{
          \Sii{$\congruencia{n}{0}{19}$}
    \congruencia{15}{0}{19} \to \text{ningún } n\\
    % --
    \Sii{PTF, 19 es primo}[$\noCongruencia{n}{0}{19}$]
    \congruencia{8n}{4}{19}
    \Sii{$\red{(\Leftarrow)}$}[$7 \cop 19$]
    \congruencia{56n}{28}{19}
    \sii
    \boxed{\congruencia{n}{10}{19}}\Tilde\llamada2
  }
$$

\textit{Conclusión divisibilidad $19$:}
$$19 \divideA \expresion \sii \congruencia{n}{10}{19}$$

$$
  \llave{l}{
    \llamada1\congruencia{n}{0}{7} \\
    \llamada2\congruencia{n}{10}{19}
  }
  \flecha{$7 \cop 19$ hay solución}[por \href{\chinito}{T chino R}]
  \llave{l}{
    \flecha{$\llamada2$}[en $\llamada1$]
    n = 19\blue{k} + 10 \conga7 5\blue{k} + 3 \conga7 0
    \sii
    \congruencia{\blue{k}}{5}{7}}\Tilde
$$
Finalmente:
$$
  n = 19 \cdot (7\magenta{q} + 5) + 10
  \sii
  \cajaResultado{\congruencia{n}{105}{133}}
$$

% Contribuciones
\begin{aportes}
  %% iconos : \github, \instagram, \tiktok, \linkedin
  %\aporte{url}{nombre icono}
  \item \aporte{https://github.com/nad-garraz}{Nad Garraz \github}
  \item \aporte{https://github.com/daniTadd}{Dani Tadd \github}
\end{aportes}
