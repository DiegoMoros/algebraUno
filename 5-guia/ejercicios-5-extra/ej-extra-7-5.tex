\ejercicio

\def\PTF{\ ^{\yellow{\text{\FiveStar}}}}
\newcommand{\congPTF}[1]{\taa{(#1)}{\PTF}{\congruente}}


Hallar todos los primos $p \en \naturales$ tales que
$$
\congruencia{3^{p^2 + 3}}{-84}{p} \ytext \congruencia{(7p + 8)^{2024}}{4}{p}.
$$
\separadorCorto

A lo largo del ejercicio se va a usar fuerte el colorario del pequeño teorema de Fermat, $\PTF$
$$
\text{si $p$ primo y $p\noDivide a$, con $a \en \enteros$} \entonces \congruencia{a^n}{a^{r_{p-1}}}{p}
$$
$
\congruencia{3^{p^2 + 3}}{-84}{p}
\llave{l}{
	\flecha{caso}[$p \noDivide 3$ ]
	\llave{l}{
	3^{p^2 + 3}
	\congPTF{p}
	3^{r_{\magenta{(p-1)}}(p^2 + 3)}\\
	\flecha{división}[polinomio]
	p^2 + 3 = \magenta{(p-1)}(p+1) + \ob{4}{\llamada1 r_{ \magenta{(p-1)}}(p^2 + 3)}
	\entonces
	3^{p^2 + 3}
	\taa{(p)}{\llamada1}\congruente
	\ub{3^4}{81} \llamada2\\
	\congruencia{3^{p^2 + 3}}{-84}{p}
	\stackrel{\llamada2}\sii 
	\congruencia{81}{-84}{p}
	\sii 
	\congruencia{\ub{165}{5\cdot3\cdot11}}{0}{p}
	\stackrel{p\noDivide 3}\sisolosi
	\boxed{p = 5} \otext \boxed{ p =11}
	}\\
	\flecha{caso}[$p \divideA 3$ ]
	\llave{l}{
	p \divideA 3
	\sii 
	p = 3
	\entonces
	3^{p^2 + 3}
	\conga3 
	0
	\congruente
	\ub{-84}{\conga3 0} \ (3)
	\entonces
	\boxed{p = 3 }
	}
}
$
Tengo entonces 3 posibles valores para $p \en \set{ 3,5,11}$. Los uso para ver cuál o cuáles
verifican la segunda condición $\congruencia{(7\cdot p + 8)^{2024}}{4}{p}$.\\

\underline{\textit{Con }  $p = \blue{3}$}:\\

$
(7\cdot \blue{3} + 8)^{2024} \conga3
2^{2024} 
	\congPTF{3}
2^{r_2(2024)} \conga3
2^0 \conga3
1
\entonces
\boxed{p = 3} \Tilde
$\\

\underline{\textit{Con }  $p = \green{5}$}:\\

$
(7\cdot \green{5} + 8)^{2024} \conga5
3^{2024} \congPTF{5}
3^{r_4(2024)} \conga5
3^0 \conga5
1 \not\congruente
4\ (5) \skull
$

\underline{\textit{Con }  $p = \yellow{11}$}:\\

$
(7\cdot \yellow{11} + 8)^{2024} \conga{11}
8^{2024} \congPTF{11}
8^{r_{10}(2024)} \conga{11}
8^4 =
\ub{4096}{r_{11}(4096) = 4} \congruente
4\ (11) \Tilde
$

Por lo tanto los valores de $p$ que cumplen lo pedido son:
\boxed{
	\begin{array}{c}
	p = 3\\
	\ytext \\
	p = 11
\end{array}
}\Tilde

