\ejercicio

\red{Pulir ejercicio, tiene cálculos redundantes. Se hace más complicado de lo necesario.}

Sea $a \en \enteros$ tal que $(a^{197} - 26 : 15) = 1$. Hallar los posibles valores de
$(a^{97} - 36 : 135)$\\

\separadorCorto

Tengo que dar posible valores para $(a^{97} - 36 : 135)$. Como $135 = 3^3\cdot 5$,
los posibles valores serán de la forma $3^\alpha \cdot 5^\beta$ con
$ \llaves{l}{
		0 \leq \alpha \leq 3\\
		0 \leq \beta \leq 1\\
	}$ potencialmente $\ub{8}{(3+1)\cdot (1+1)} $posibles valores distintos $\set{1, 3, 9, 27, 5, 15, 45, 135}$\\
Como condición mínima para que no sea siempre $(a^{97} - 36 : 135) = 1$ es que
$\llaves{c}{
		3 \divideA a^{97} - 36\\
		\text{o bien,}\\
		5 \divideA a^{97} - 36
	}$ si no ocurre ninguna de éstas el MCD será 1.\\

$\llave{l}{
		5 \divideA a^{97} - 36 \to a^{97} - 36 \conga{5} a^{97} -1 \conga5 0
		\to
		\llave{ll}{
			\flecha{si}[$5 \divideA a$] & \congruencia{-1}{0}{5}
			\flecha{ningún $a$ tal que}[$5 \divideA a$ logra que] 5 \divideA a^{97} - 36 \\

			\flecha{si}[$5 \noDivide a$] & \congruencia{(a^4)^{24} \cdot a -1 }{0}{5}
			\flecha{5 primo, $5 \noDivide a$}[$\congruencia{a^4}{1}{5}$]
			\boxed{\congruencia{a}{1}{5}}
		}\\
		\text{Se concluye de esta rama que si } 5 \divideA a^{97} - 36
		\entonces
		\boxed{\congruencia{a}{1}{5}}\llamada3 \Tilde \\

		\separadorCorto

		3 \divideA a^{97} - 36 \to a^{97} - 36 \conga{3} a^{97} \conga3 0
		\to
		\llave{ll}{
			\flecha{si}[$3 \divideA a$] & \congruencia{0}{0}{3}
			\flecha{dado que en}[esta rama $a \divideA 3$] \boxed{ \congruencia{a}{0}{3} } \\
			\flecha{si}[$3 \noDivide a$] & \congruencia{(a^2)^{48} \cdot a}{0}{3}
			\flecha{3 primo, $3 \noDivide a$}[$\congruencia{a^2}{1}{3} $]
			\boxed{\congruencia{a}{0}{3}}
		}\\
		\text{Se concluye de esta rama que si } 3 \divideA a^{97} - 36
		\entonces
		\boxed{\congruencia{a}{0}{3}}\llamada4 \Tilde
	}$\\

Todo muy lindo pero los valores de $a$ están condicionados por  $(a^{197} - 26 : 15) = 1$
una condición que "nada" tiene que ver con el MCD, pero que condiciona los valores que puede
tomar $a$. Como $a^{197} - 26 $ y $15 = 3 \cdot 5$ son coprimos sus factorizaciones en primos
no pueden tener ningún número factor en común, dicho de otra forma:
$\llaves{c}{
		5 \noDivide  a^{197} - 26 \\
		\text{ y } \\
		3 \noDivide  a^{197} - 26
	}$ estudiar estas condiciones me va a restringir los valores de $a$ que puedo usar para construir
los posibles MCDs.\\

$\llave{l}{
		\flecha{\magenta{supongo} $5\divideA a^{197} - 26$ y me}[\magenta{quedo con el complemento}]
		a^{197} - 1 \conga{5} 0
		\to
		\llave{l}{
			\flecha{si}[$5 \divideA a$]
			\congruencia{-1}{0}{5}
			\flecha{\green{ningún} $a$ tal que}[$5 \divideA a$ logra que] 5 \divideA a^{197} - 26 \\
			\flecha{el \magenta{complemento de "\green{ningún}" es }}[\magenta{todo $a$} pero como $5\divideA a$]
			\congruencia{a}{0}{5}\\
			\separadorCorto
			\flecha{si}[$5 \noDivide a$]
			\congruencia{(a^4)^{49} \cdot a - 1}{0}{5}
			\flecha{5 primo, $5\noDivide a$}[$\congruencia{a^4}{1}{5}$]
			\congruencia{a}{1}{5}\llamada1\\
			\flecha{agarro el}[complemento de $\llamada1$]
			\llave{l}{
				\hbox{\sout{$\congruencia{a}{0}{5}$}} \to \text{rama $5\noDivide a$} \\
				\congruencia{a}{2}{5} \\
				\congruencia{a}{3}{5} \\
				\congruencia{a}{4}{5} \\
			}
		}\\

		\separadorCorto

		\flecha{\magenta{supongo} $3\divideA a^{197} - 26$ y me}[\magenta{quedo con el complemento}]
		a^{197} - 2 \conga{3} 0
		\to
		\llave{l}{
			\flecha{si}[$3 \divideA a$]
			\congruencia{-2}{0}{3}
			\flecha{\green{ningún} $a$ tal que}[$3 \divideA a$ logra que] 3 \divideA a^{197} - 26 \\
			\flecha{el \magenta{complemento de "\green{ningún}" es }}[\magenta{todo $a$} pero como $3\divideA a$]
			\congruencia{a}{0}{3}\\
			\separadorCorto
			\flecha{si}[$3 \noDivide a$]
			\congruencia{(a^2)^{93} \cdot a - 2}{0}{3}
			\flecha{3 primo, $3\noDivide a$}[$\congruencia{a^2}{2}{3}$]
			\congruencia{a}{2}{3}\llamada2\\
			\flecha{agarro el}[complemento de $\llamada2$]
			\llave{l}{
				\hbox{\sout{$\congruencia{a}{0}{3}$}} \to \text{rama $3\noDivide a$} \\
				\congruencia{a}{1}{3}
			}
		}\\

	}$\\
\text{Se concluye del estudio que si } $5 \noDivide a^{197} - 26$ y
$3 \noDivide a^{197} - 26 \to $
\boxed{
	\llave{cr}{
		\congruencia{a}{0}{5} & \text{ o }\\
		\congruencia{a}{2}{5} & \text{ o }\\
		\congruencia{a}{3}{5} & \text{ o }\\
		\congruencia{a}{4}{5}\\
		\text{ y }&\\
		\congruencia{a}{0}{3} & \text{ o }\\
		\congruencia{a}{1}{3}
	}
}, 8 sistemas $\skull$ \\

Para que el MCD \textit{no sea 1}, se deben satisfacer $\llamada3$ o $\llamada4$, lo cual no ocurre
nunca con $\llamada3$. Eso acota los valores de $(a^{97} - 36 : 135)$ a $\set{1,3,9,27}$\\
De los 4 sistemas útiles:\\
$
	\llave{l}{
		\congruencia{a}{0}{5}\\
		\congruencia{a}{0}{3}
	}
	\flecha{solución}
    \congruencia{a}{0}{15}
    \flecha{con $a = 0$}
	\llave{l}{
		0^{97} - 36 \conga{3} 0\\
		0^{97} - 36 \conga{9} 0 \Tilde\\
		0^{97} - 36 \conga{27} - 9 \not\conga{27} 0 \\
	}
	\\
	\llave{l}{
		\congruencia{a}{2}{5}\\
		\congruencia{a}{0}{3}
	}
	\flecha{solución} \congruencia{a}{12}{15}
	\flecha{con $a = 12$}
	\llave{l}{
		12^{97} - 36 \conga{3} 0\\
		12^{97} - 36 \conga{9} 4^{97} \cdot (3^2)^{48} \cdot 3 \conga9 0 \Tilde\\
		12^{97} - 36 \conga{27} 4^{97} \cdot (3^3)^{32} \cdot 3^1 - 9 \conga{27} -9 \not\conga{27} 0 \\
	}
	\\
	\llave{l}{
		\congruencia{a}{3}{5}\\
		\congruencia{a}{0}{3}
	}
	\flecha{solución} \congruencia{a}{3}{15}
	\\
	\llave{l}{
		\congruencia{a}{4}{5}\\
		\congruencia{a}{0}{3}
	}
	\flecha{solución} \congruencia{a}{9}{15}
	\flecha{con $a = 9$}
	\llave{l}{
		9^{97} - 36 \conga{3} 0\\
		9^{97} - 36 \conga{9} 0 \to (9^{97} - 36:135) = 9 \Tilde\\
		9^{97} -36 \conga{27} (3^3)^{64} \cdot 3^2 - 9 \conga{27} -9 \not\conga{27} 0 \\
	}\\
$

\noindent Después de fumarme eso: $(a^{97} - 36 : 135) \en\set{1,9}$, porque todas las soluciones cumplen que:\\
$3 \divideA a^{97} \entonces 9 \divideA a^{97}\entonces 27 \divideA a^{97}$ y como $9 \divideA 36$ y $27 \noDivide 36$
siempre el mayor divisor de la expresión va a ser 9.
