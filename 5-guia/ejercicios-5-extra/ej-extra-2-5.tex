\ejExtra

Sea $a \en \enteros$ tal que $(a^{197} - 26 : 15) = 1$. Hallar los posibles valores de
$(a^{97} - 36 : 135)$\\

\separadorCorto
\underline{Nota: } No perder foco en que \textit{no} hay que encontrar "para que
$a$ el mcd vale tanto", sino se pone más complicado en el final.\\

$(a^{97} - 36 : \ob{135}{3^3\cdot 5}) = 3^\alpha \cdot 5^\beta$ con
$\llamada1 \llaves{l}{
		0 \leq \alpha \leq 3\\
		0 \leq \beta \leq 1\\
	}$.\\
Luego $(a^{197} - 26 : \ub{15}{3\cdot 5}) = 1$ se debe cumplir que:
$
	\llave{l}{
		5 \noDivide a^{197} - 26\\
		3 \noDivide a^{197} - 26
	}
$\\

\underline{\textit{Análisis de } $(a^{197} - 26 : 15) = 1$:}\\
\textit{Estudio la divisibilidad $5$:}\\

$ 5 \noDivide a^{197} - 26
	\sisolosi
	\noCongruencia{a^{197} - 26}{0}{5}
	\sisolosi
	\noCongruencia{a^{197} - 1}{0}{5}
	\flecha{analizo casos}[$5\divideA a$ o $5\divideA a$]\\
	\noCongruencia{a^{197}}{1}{5}
	\sii
	\llave{l}{
		\text{(\magenta{rama $5\noDivide a$})}
		\flecha{5 es primo}[$5\noDivide a$]
		\noCongruencia{a\cdot(\ob{a^4}{\conga5 1})^{49}}{1}{5}
		\sii
		\noCongruencia{a}{1}{5} \Tilde\\
		%---
		\text{(\magenta{rama $5\divideA a$})}
		\flecha{5 es primo}[$5\divideA a$]
		\noCongruencia{0}{1}{5} \to \congruencia{a}{0}{5}
	}
$\\

\textit{Conclusión divisilidad 5: }\\
Para que
\boxed{5 \noDivide a^{197} - 26
	\sisolosi
	\noCongruencia{a}{1}{5} \llamada2
}\\

\textit{Estudio la divisibilidad $3$:}\\

$ 3 \noDivide a^{197} - 26
	\sisolosi
	\noCongruencia{a^{197} - 2}{0}{3}
	\sisolosi
	\noCongruencia{a^{197} - 2}{0}{3}
	\flecha{analizo casos}[$3\divideA a$ o $3\divideA a$]\\
	\noCongruencia{a^{197}}{2}{3}
	\sii
	\llave{l}{
		\text{(\magenta{rama $3\noDivide a$})}
		\flecha{3 es primo}[$5\noDivide a$]
		\noCongruencia{a\cdot(\ob{a^2}{\conga3 1})^{98}}{2}{3}
		\sii
		\noCongruencia{a}{2}{3} \Tilde\\
		%---
		\text{(\magenta{rama $3\divideA a$})}
		\flecha{3 es primo}[$3\divideA a$]
		\noCongruencia{0}{2}{3} \to \congruencia{a}{0}{3}
	}
$\\

\textit{Conclusión divisilidad 3: }\\
Para que
\boxed{
	3 \noDivide a^{197} - 26
	\sisolosi
	\noCongruencia{a}{2}{3}\llamada3
}\\

\separadorCorto
\underline{\textit{Análisis de } $(a^{97} - 36 : 135)$:}\\
Necesito que
$ \llaves{c}{
		3 \divideA a^{97} - 36\\
		\text{o bien,}\\
		5 \divideA a^{97} - 36
	}$, para obtener valores distintos de 1 para el MCD.
\\

\textit{Estudio la divisibilidad $5$ (sujeto a $\llamada2$ y $\llamada3$):}\\
Si
$ 5 \divideA a^{97} - 36
	\sisolosi
	\congruencia{a^{97} - 1}{0}{5}
	\sisolosi
	\congruencia{a^{97}}{1}{5}
	\flecha{analizo casos}[$5\divideA a$ o $5\divideA a$]\\
	\congruencia{a^{97}}{1}{5}
	\sii
	\llave{l}{
		\text{(\magenta{rama $5\noDivide a$})}
		\flecha{5 es primo}[$5\noDivide a$]
		\congruencia{a\cdot(\ob{a^4}{\conga5 1})^{24}}{1}{5}
		\sii
		\congruencia{a}{1}{5},\text{ absurdo con $\llamada2$} \skull\\
		%---
		\text{(\magenta{rama $5\divideA a$})}
		\flecha{5 es primo}[$5\divideA a$]
		\congruencia{0}{1}{3}
		\to \text{ si } \congruencia{a}{0}{5}
		\entonces
		\noCongruencia{a^{97}}{1}{5}
	}
$\\

\textit{Conclusión divisilidad 5: }\\
\boxed{5 \noDivide a^{97} - 36\quad \paratodo a \en \enteros \to} el MCD no puede tener un 5 en su factorización. \\\


\textit{Estudio la divisibilidad $3$ (sujeto a $\llamada2$ y $\llamada3$):}\\

$ 3 \divideA a^{97} - 36
	\sisolosi
	\congruencia{a^{97}}{0}{3}
	\sisolosi
	\congruencia{a^{97}}{0}{3}
	\flecha{analizo casos}[$3\divideA a$ o $3\divideA a$]\\
	\congruencia{a^{97}}{0}{3}
	\sii
	\llave{l}{
		\text{(\magenta{rama $3\noDivide a$})}
		\flecha{3 es primo}[$3\noDivide a$]
		\congruencia{a\cdot(\ob{a^2}{\conga5 1})^{48}}{0}{3}
		\sii
		\congruencia{a}{0}{3} \Tilde\\
		%---
		\text{(\magenta{rama $3\divideA a$})}
		\flecha{3 es primo}[$3\divideA a$]
		\congruencia{a}{0}{3}
		\sii
		\congruencia{0}{0}{3}
		\to
		\text{ si } \congruencia{a}{0}{3}
		\entonces
		\congruencia{a^{97}}{0}{3}
	}
$\\

\textit{Conclusión divisilidad 3: }\\
\boxed{3 \divideA a^{97} - 36
	\sisolosi
	\congruencia{a}{0}{3}
\llamada4}\\

De $\llamada1$ 3 es un posible MCD, tengo que ver si $3^2$ o $3^3$ también dividen.\\

\textit{Estudio la divisibilidad $9$ en $a = 3k$ por $\llamada4$:}\\
$ 9 \divideA (3k)^{97} - 36
	\sisolosi
	\congruencia{3k^{97}}{0}{9}
	\sisolosi
    \congruencia{3 \cdot (3^2)^{48}\cdot k^{97}}{0}{9}
    \sisolosi
	\congruencia{0}{0}{9}\Tilde \paratodo k \en \enteros 
$\\

\textit{Conclusión divisilidad 9: }\\
\boxed{ 9 \divideA a^{97} - 36 \text{ puede ser que } (a^{97} - 26:135) = 9 }\Tilde


\textit{Estudio la divisibilidad $27$ en $a = 3k$ por $\llamada4$:}\\
$ 27 \divideA (3k)^{97} - 36
	\sisolosi
    \congruencia{(3k)^{97}}{9}{27}
	\sisolosi
    \congruencia{3 \cdot (3^3)^{32} \cdot k^{97}}{9}{27}
    \sisolosi
    \congruencia{0}{9}{27} 
$\\

\textit{Conclusión divisilidad 27: }\\
Si $\congruencia{a}{0}{3} \entonces 27 \noDivide a^{97} - 36$\\

\magenta{Finalmente: el mcd es 9} 
