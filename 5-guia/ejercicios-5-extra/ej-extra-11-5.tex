\begin{enunciado}{\ejExtra}
  Hallar \textbf{todos} los $p \en \naturales$ primos tales que
  $$
    p \divideA 15^{2p-2} + 7^{p+3} + 174
    \sii
    15^{2p-2} + 7^{p+3} + 174.
  $$
\end{enunciado}

Teniendo en cuenta que $174 = 3 \cdot 2 \cdot 29$
$$
  \begin{array}{rcl}
    p \divideA 15^{2p-2} + 7^{p+3} + 3 \cdot 2 \cdot 29
     & \sii &
    \congruencia{15^{2p-2} + 7^{p+3} +3 \cdot 2 \cdot 29}{0}{p}
  \end{array}
$$

Viendo esa expresión tengo ciertos $p$ primos de interés que no me dejan aplicar PTF para toda la expresión. Estos serían:
$$
  p \en \set{2, 3, 5, 7, 29}
$$

Hago PTF con $p$ genérico que no divide a ninguno de ese conjunto para ver que sale:

\textit{Caso $p \notin\set{2, 3, 5, 7, 29}$}:\par
$$
  \congruencia{15^{2p-2} + 7^{p+3} +3 \cdot 2 \cdot 29}{0}{p}
  \Sii{PTF}
  \congruencia{15^2 + 7^4 + 3 \cdot 2 \cdot 29}{0}{p}
  \sii
  \congruencia{2800}{0}{p}
  \sii
  \congruencia{2^4 \cdot 5^3}{0}{p}
$$

De acá saco que no hay $p$ fuera del conjunto que tengo que puedan ser solución.
En el cálculo consideré que $p$ no era ni 2 ni 5. Arranco a analizar con cada uno de los  $p \en \set{2, 3, 5, 7, 29}$ en particular:

\begin{enumerate}[label={\tiny \magenta{\faIcon[regular]{gamepad}}}]
  \item
        \textit{Caso $p = 29$}:\par
        $$
          \congruencia{15^{56} + 7^{32} }{0}{29}
          \Sii{PTF}
          \congruencia{15^{0} + 7^{4} }{0}{29}
          \sii
          \congruencia{7^{4}}{0}{29}
          \sii
          \congruencia{23}{0}{29}
        $$
        $p = 29$ \underline{no cumple}.

  \item
        \textit{Caso $p = 7$}:\par
        $$
          \congruencia{1^{12} + 174}{0}{7}
          \sii
          \congruencia{7}{0}{7}
          \sii
          \congruencia{0}{0}{7} \blue{\Tilde}
        $$
        $p = 7$ \fcolorbox{orange}{white}{cumple}.

  \item

        \textit{Caso $p = 5$}:\par
        $$
          \congruencia{ 2^8 + 4}{0}{5}
          \sii
          \congruencia{5}{0}{5}
          \sii
          \congruencia{0}{0}{5} \blue{\Tilde}
        $$
        $p = 5$ \fcolorbox{orange}{white}{cumple}.

  \item
        \textit{Caso $p = 3$}:\par
        $$
          \congruencia{1^6}{0}{3}
          \sii
          \congruencia{0}{0}{5}
        $$
        $p = 3$ \underline{no cumple}.

  \item
        \textit{Caso $p = 2$}:\par
        $$
          \congruencia{1^0 + 1^5}{0}{2}
          \sii
          \congruencia{0}{0}{2} \blue{\Tilde}
        $$
        $p = 2$ \fcolorbox{orange}{white}{cumple}.
\end{enumerate}

Hay 3 números \textit{primos} que cumplen lo pedido.

\bigskip

El resultado sería:
$$
  p \en \set{2, 5, 7}
$$

\begin{aportes}
  \item \aporte{https://github.com/nad-garraz}{Nad Garraz \github}
\end{aportes}
