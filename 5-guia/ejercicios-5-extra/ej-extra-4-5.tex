\ejExtra
Determinar para cada $n \en \naturales$ el resto de dividir a $8^{3^n-2}$ por 20.

\separadorCorto

Quiero encontrar $r_{20}(8^{3^n-2})$ entonces analizo congruecia:\\
$
	\congruencia{8^{3^n-2}}{X}{20}
	\flecha{quebrar}
	\llave{l}{
		\congruencia{8^{3^n-2}}{3^{3^n-2}}{5} \llamada1 \\
		\congruencia{8^{3^n-2}}{0}{4} \to \paratodo n \en \naturales
	}\\
$

Laburo con $\llamada1$:\\

$
	\congruencia{8^{3^n-2}}{ \ub{3^{3^n-2}}{    \conga5 3^{r_4(3^n-2)}\llamada2    } }{5} \\
	\flecha{$\llamada2$}
	3^{r_4(3^n-2)}
	\flecha{$n$ par}[$n$ impar]
	\llave{rl}{
      \text{si $n$ par}   & 3^{r_4(3^n-2)} \underset{\llamada3}{\conga5} 3^{1-2} \conga5 \congruencia{3^3}{2}{5} \\
		\text{si $n$ impar} & 3^1 \conga5 3 \ (5)
	}
	\quad
	\begin{array}{|c|l|l|l|l|}
		\hline
		r_4(n)   & 0 & 1 & 2 & 3 \\ \hline
		r_4(3^n) & 1 & 3 & 1 & 3 \\ \hline
	\end{array} \llamada3
$\\

$
	\llave{lll}{
		\congruencia{8^{3^n-2}}{0}{4}\llamada4 & \text{ si } & \paratodo n \en naturales\\
		\congruencia{8^{3^n-2}}{2}{5}\llamada5 & \text{ si } & \congruencia{n}{0}{2} \\
		\congruencia{8^{3^n-2}}{3}{5}\llamada6 & \text{ si } & \congruencia{n}{1}{2}
	}\\
$

Si $
	\congruencia{n}{0}{2}
	\flecha{$\llamada4$}[$\llamada5$]
	\llave{l}{
		8^{3^n-2} = 4j
		\to
		\congruencia{4j}{2}{5}
		\sii
		\congruencia{j}{3}{5}\\
		\sii
		j = 5k + 3
		\entonces
		8^{3^n-2} = 4(5k+3)
		\sii
		\boxed{
			\congruencia{8^{3^n-2}}{12}{20}
			\sii
			\congruencia{n}{0}{2}.
		} \Tilde
	}
$

Si $
	\congruencia{n}{1}{2}
	\flecha{$\llamada4$}[$\llamada6$]
	\llave{l}{
		8^{3^n-2} = 4j
		\to
		\congruencia{4j}{3}{5}
		\sii
		\congruencia{j}{2}{5}\\
		\sii
		j = 5k + 2
		\entonces
		8^{3^n-2} = 4(5k+2)
		\sii
		\boxed{
			\congruencia{8^{3^n-2}}{8}{20}
			\sii
			\congruencia{n}{1}{2}.
		} \Tilde
	}
$\\

Se concluye que
\boxed{
	r_{20}(8^{3^n-2}) = 12
	\text{ si } n \text{ par y }
	r_{20}(8^{3^n-2}) = 8
	\text{ si } n \text{ impar  con }
	n \en \naturales
}
