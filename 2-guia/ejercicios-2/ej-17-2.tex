\begin{enunciado}{\ejercicio}
    \begin{enumerate}[label=\roman*)]
        \item Sea $(a_n)_{n \en \naturales }$ la sucesión definida por
        \begin{align*}
            a = 1  \ytext  a_{n+1} = a_n + n \cdot n!,\ \paratodo n \en \naturales 
        \end{align*}
        Probar que $a_n = n!$, y, aplicando el Ej.9i), calcular $\displaystyle \sumatoria{i=1}{n} i \cdot i!$.
        
        \item Sea $(a_n)_{n \en \naturales }$ la sucesión definida por
        \begin{align*}
            a = 1  \ytext  a_{n+1} = a_n + 3n^2 + 3n + 1,\ \paratodo n \en \naturales 
        \end{align*}
        Probar que $a_n = n^3$, y, aplicando el Ej.9i, calcular de otra forma $\displaystyle \sumatoria{i=1}{n} i^2$ 
        (comparar con Ej.6).
    \end{enumerate}
\end{enunciado}

\begin{enumerate}[label=\roman*)]
    \item Sea $(a_n)_{n \en \naturales }$ definida por
    \setcounter{equation}{0}
    \begin{align}
        a_1 &= 1 \\
        a_{n+1} &= a_n + n \cdot n!,\ \paratodo n \en \naturales 
    \end{align} 
    Tenemos que probar que la sucesión dada por recurrencia satisface la sucesión $a_n = n!$ para todo $n \en 
    \naturales $. 
    Hagamoslo por inducción
    \begin{align*}
        P(n): a_n = n!,\ \paratodo n \en \naturales 
    \end{align*}
    \underline{Caso Base}, $n = 1$:
	    \begin{align*}
		    &P(1): a_1 = 1! = 1 \igual{(1)} 1 \entonces P(1):V
	    \end{align*}
	\underline{Paso inductivo.} Sea $n \geq 1$:
	\begin{enumerate}
        \item[HI.] $P(n): V$
        \item[TI.] $P(n+1): a_{n+1} = (n+1)!$
    \end{enumerate}
 	Desarrollemos el lado izquierdo de la igualdad en la TI
    \begin{align*}
  	    a_{n+1} \igual{(2)} a_n + n \cdot n! \igual{HI} n! + n \cdot n! = n!(n + 1) = (n+1)! 
        \entonces P(n+1):V
    \end{align*}
    Hemos probado el caso base y el paso inductivo. Concluimos que $P(n):V,$ $\paratodo n \en \naturales $.\\

    Calculemos $\displaystyle \sumatoria{i=1}{n} i \cdot i! $:
    \begin{align*}
            \sumatoria{i=1}{n}  i \cdot i! \taa{\text{Aux}}{1}= \sumatoria{i=1}{n}  (a_{i+1} - a_i) \taa{\text{Aux}}{2}=
        a_{n+1} - a_1 = (n+1)! - 1
    \end{align*}

    \paragraph{Auxiliar 1}{
        Usemos la Ec.(2)
        \begin{align*}
            a_{n+1} = a_n + n \cdot n! \entonces n \cdot n! = a_{n+1} - a_n
        \end{align*}
    }
    \paragraph{Auxiliar 2}{
        Ecuación dada por el ejercicio 19.i)
        \begin{align*}
            \sumatoria{i=1}{n} (a_{i+1} - a_1) = a_{n+1} - a_1
        \end{align*}
    }

    \item Sea $(a_n)_{n \en \naturales }$ definida por
    \setcounter{equation}{0}
    \begin{align}
        a_1 &= 1 \\
        a_{n+1} &= a_n + 3n^2 + 3n + 1,\ \paratodo n \en \naturales 
    \end{align} 
    Tenemos que probar que la sucesión dada por recurrencia satisface la sucesión $a_n = n^3$ para todo $n \en 
    \naturales $. 
    Hagamoslo por inducción
    \begin{align*}
        P(n): a_n = n^3,\ \paratodo n \en \naturales 
    \end{align*}
    \underline{Caso Base}, $n = 1$:
	    \begin{align*}
		    &P(1): a_1 = 1! = 1 \igual{(1)} 1 \entonces P(1):V
	    \end{align*}
	\underline{Paso inductivo.} Sea $n \geq 1$:
	\begin{enumerate}
        \item[HI.] $P(n): V$
        \item[TI.] $P(n+1): a_{n+1} = (n+1)^3$
    \end{enumerate}
 	Desarrollemos el lado izquierdo de la igualdad en la TI
    \begin{align*}
            a_{n+1} \igual{(2)} a_n + 3n^2 + 3n + 1 \igual{HI} n^3 + 3n^2 + 3n + 1 = (n+1)^3 
        \entonces P(n+1):V
    \end{align*}
    Hemos probado el caso base y el paso inductivo. Concluimos que $P(n):V,$ $\paratodo n \en \naturales $. \\

    Calculemos $\displaystyle \sumatoria{i=1}{n} i^2 $:

    \begin{align*}
        \sumatoria{i=1}{n}  i^2 \overset{\text{(Aux.1)}}&{=} \sumatoria{i=1}{n}  \frac{a_{i+1} - a_i - 3i - 1}{3} =
        \frac{1}{3}\sumatoria{i=1}{n}  (a_{i+1} - a_i - 3i - 1) \\
        &= \frac{1}{3} \left( \sumatoria{i=1}{n}  (a_{i+1} - a_i) - \sumatoria{i=1}{n}  3i - \sumatoria{i=1}{n} 1 \right)
        = \frac{1}{3} \left( \sumatoria{i=1}{n}  (a_{i+1} - a_i) - 3\sumatoria{i=1}{n}  i - \sumatoria{i=1}{n} 1 \right) \\
        &= \frac{1}{3} \left( \sumatoria{i=1}{n}  (a_{i+1} - a_i) - 3\frac{n(n+1)}{2} - n \right) 
        = \frac{1}{3}\sumatoria{i=1}{n}  (a_{i+1} - a_i) + \frac{1}{3} \left(- 3\frac{n(n+1)}{2} - n \right) \\
        \overset{\text{(Aux.2)}}&{=} \frac{1}{3}(a_{n+1} - a_1) + \left(-\frac{n(n+1)}{2} - \frac{n}{3} \right)
        = \frac{1}{3}((n+1)^3 - 1) + \left(-\frac{n(n+1)}{2} - \frac{n}{3} \right) \\
        &=  \frac{(n+1)^3}{3} - \frac{1}{3} -\frac{n(n+1)}{2} - \frac{n}{3} = 
        \frac{(n+1)^3}{3} -\frac{n(n+1)}{2} - \frac{n+1}{3} \\
        &= (n+1) \left( \frac{(n+1)^2}{3} -\frac{n}{2} - \frac{1}{3} \right)  
        = (n+1) \left( \frac{n^2 + 2n + 1}{3} -\frac{n}{2} - \frac{1}{3} \right) \\
        &= (n+1) \left( \frac{2n^2 + 4n + 2 - 3n - 2}{6} \right) = (n+1) \left( \frac{2n^2 + n}{6} \right)
        = \frac{n(n+1)(2n+1)}{6} \\
        \sumatoria{i=1}{n}  i^2 &= \frac{n(n+1)(2n+1)}{6}
    \end{align*}

    \paragraph{Auxiliar 1}{
        Usemos la Ec.(2)
        \begin{align*}
            a_{n+1} &= a_n + 3n^2 + 3n + 1 \\ 
            3n^2 &= a_{n+1} - a_n - 3n - 1 \\
            n^2 &= \frac{a_{n+1} - a_n - 3n - 1}{3}
        \end{align*}
    }
    \paragraph{Auxiliar 2}{
        Ecuación dada por el ejercicio 19.i)
        \begin{align*}
            \sumatoria{i=1}{n} (a_{i+1} - a_1) = a_{n+1} - a_1
        \end{align*}
    }
\end{enumerate}

% Contribuciones
\begin{aportes}
  %% iconos : \github, \enstagram, \tiktok, \linkedin
  %\aporte{url}{nombre icono}
  \item \aporte{https://github.com/koopardo/}{Marcos Zea \github}
\end{aportes}
