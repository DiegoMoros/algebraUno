\begin{enunciado}{\ejercicio}
  Sea $a,\, b \en \reales$.
  Probar que para todo $n \en \naturales,\, a^n - b^n = (a-b)
    \sumatoria{i = 1}{n}a^{i-1}b^{n-i}$.
  Deducir la fórmula de la serie geométrica: para todo
  $a \distinto 1,\, \sumatoria{i = 0}{n}a^i = \frac{a^{n+1} - 1}{a-1}$.
\end{enunciado}

Quiero probar por inducción:
$$
  p(n) : a^n - b^n = (a-b) \sumatoria{i=1}{n} a^{i-1} b^{n-i} \quad \paratodo n \en \naturales
$$

\textit{Caso base:}
$$
  p(\blue{1}) : (a^{\blue{1}} - b^{\blue{1}})
  \sumatoria{i = 1}{\blue{1}} a^{i-1} \cdot b^{\blue{1}-i} =
  a - b = a^1 - b^1
$$
Resulta que el \textit{caso base} es verdadero.

\medskip

\textit{Paso inductivo:}

Asumo como verdadero para algún $k \en \enteros$:
$$
  p(\blue{k}) :  \ub{
    a^{\blue{k}} - b^{\blue{k}} = (a - b) \sumatoria{i = 1}{\blue{k}} a^{i-1} \cdot b^{\blue{k}-i}
  }{\purple{\text{hipótesis inductiva}}}
$$

y quiero probar que:

$$
  p(\blue{k+1}) : a^{\blue{k+1}} - b^{\blue{k+1}} \igual{?} (a - b) \sumatoria{i = 1}{\blue{k+1}} a^{i-1} \cdot b^{\blue{k+1}-i}
$$

Arranco por el paso $\blue{k+1}$ y busco de usar la \purple{hipótesis inductiva} para probar lo que quiero:

$$
  (a - b) \sumatoria{i = 1}{\blue{k+1}} a^{i-1} \cdot b^{\blue{k+1}-i}
  \igual{\red{!!}}
  \blue{b} \cdot (a - b) \sumatoria{i = 1}{\blue{k+1}} a^{i-1} \cdot b^{\blue{k} - i}
  \igual{\red{!}}
  \blue{b} \cdot (a - b) \left(
  \sumatoria{i = 1}{k} a^{i-1} \cdot b^{k-i} +
  a^k \cdot b^{\cyan{-1}}
  \right)
  \igual{$\llamada1$}
$$
En el \red{!!} hice un factor común y en el \red{!} el truquito de separar un término de la sumatoria.
Acomodo la expresión:
$$
  \igual{$\llamada1$}
  \blue{b} \cdot (a - b) \sumatoria{i = 1}{k} a^{i-1} \cdot b^{\blue{k} - i} +
  (a^{k+1} - a^k b)
  \igual{\purple{HI}}
  \blue{b} \cdot (\purple{a^k - b^k}) + (a^{k+1} - a^k b) = a^{k+1} - b^{k+1}
$$

Por lo que $p(k+1)$ es verdadero.

\bigskip

Como $p(1), p(k) \ytext p(k+1)$ resultaron verdaderas, por el principio de inducción $p(n)$ también lo
es $\paratodo n \en \naturales$.

\bigskip

Para deducir la fórmula de la serie geométrica pongo $b = 1$ en la expresión original:
$$
  a^n - 1 =
  (a - \magenta{1}) \sumatoria{i=1}{n} a^{i-1} \cdot \magenta{1}
  \Sii{$a\neq 1$}
  \frac{a^n - 1}{a-1} = \sumatoria{i=1}{n} a^{i-1}
$$

Multiplico por $\magenta{a}$ miembro a miembro:
$$
  \magenta{a} \cdot \frac{a^n - 1}{a-1} = \magenta{a} \cdot \sumatoria{i=1}{n} a^{i-1}
  \Sii{\red{!}}
  \frac{a^{n+1} - a}{a-1} = \sumatoria{i=1}{n} a^{i}
  \igual{\red{!!}}
  \red{-1} + \sumatoria{i = \red{0}}{n}a^i
$$
Por lo tanto:
$$
  \frac{a^{n+1} - a}{a-1}
  =
  \red{-1} + \sumatoria{i = \red{0}}{n}a^i
  \Sii{\red{!}}
  \cajaResultado{
    \displaystyle
    \frac{a^{n+1} - 1}{a-1} =
    \sumatoria{i = 0}{n} a^i
  }
$$

\begin{aportes}
  \item \aporte{\dirRepo}{naD GarRaz \github}
\end{aportes}

