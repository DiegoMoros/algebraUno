\begin{enunciado}{\ejercicio}

  Escribir los dos primeros y los dos últimos términos de las expresiones siguientes
  \begin{multicols}{5}
    \begin{enumerate}[label=\roman*)]
      \item $\sumatoria{i=6}{n} 2(i - 5)$
      \item $\sumatoria{i=n}{2n} \frac{1}{i(i+1)}$
      \item $\sumatoria{i=1}{n} \frac{n + i}{2i}$
      \item $\sumatoria{i=1}{n^2} \frac{n}{i}$
      \item $\productoria{i=1}{n} \frac{n + i}{2i - 3}$
    \end{enumerate}
  \end{multicols}

\end{enunciado}
Llamo $t_1$, $t_2$ a los primeros términos y $t_{m-1}$, $t_m$ a los últimos\\

\begin{enumerate}[label=\roman*)]
  \item $\sumatoria{i=6}{n}2(i - 5)$\par

        $t_1 = 2(6 - 5) = 2 \quad t_2 = 2(7 - 5) = 4$\par
        $t_{m-1} = 2((n - 1) - 5) = 2n - 12 \quad t_m = 2(n - 5) = 2n - 10$

  \item $\sumatoria{i=n}{2n} \frac{1}{i(i+1)}$\par
        $t_1 = \frac{1}{n(n + 1)} = \frac{1}{n^2 + n} \quad t_2 = \frac{1}{(n+1)((n+1) + 1)} = \frac{1}{n^2 + 3n + 2}$\par
        $t_{m-1} = \frac{1}{(2n-1)(2n-1+1)} = \frac{1}{4n^2-2n} \quad t_m = \frac{1}{2n(2n+1)} = \frac{1}{4n^2 + 2n}$

  \item $\sumatoria{i=1}{n} \frac{n + i}{2i}$\par
        $t_1 = \frac{n + 1}{2} \quad t_2 = \frac{n + 2}{4}$\par
        $t_{m-1} = \frac{n + (n-1)}{2(n-1)} = \frac{2n - 1}{2n - 2} \quad t_m = \frac{n + n}{2n} = \frac{2n}{2n} = 1$

  \item $\sumatoria{i=1}{n^2} \frac{n}{i}$\par
        $t_1 = n \quad t_2 = \frac{n}{2}$\par
        $t_{m-1} = \frac{n}{n^2 - 1} \quad t_m = \frac{n}{n^2} = \frac{1}{n}$

  \item $\productoria{i=1}{n} \frac{n + i}{2i - 3}$\par
        $t_1 = \frac{n + 1}{2 - 3} = - n - 1 \quad t_2 = \frac{n + 2}{4 - 3} = n + 2$\par
        $t_{m-1} = \frac{n + (n-1)}{2(n-1) - 3} = \frac{2n - 1}{2n - 5} \quad t_m = \frac{n + n}{2n - 3} = \frac{2n}{2n - 3}$
\end{enumerate}

% Contribuciones
\begin{aportes}
  %% iconos : \github, \instagram, \tiktok, \linkedin
  %\aporte{url}{nombre icono}
  \item \aporte{https://https://www.instagram.com/GabiEGV}{Gabriel Garcia \instagram}
  \item \aporte{https://github.com/nad-garraz}{Nad Garraz \github}
\end{aportes}
