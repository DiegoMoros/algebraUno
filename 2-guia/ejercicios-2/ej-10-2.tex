\begin{enunciado}{\ejercicio}
  Probar que las siguientes desigualdades son verdaderas para todo $n \en \naturales$
  \begin{multicols}{2}
    \begin{enumerate}[label=\roman*)]
      \item $3^n + 5^n \geq 2^{n+2}$
      \item $3^n \leq n^3$
      \item $ \sumatoria{i=1}{n} \frac{n+i}{i+1} \leq 1 + n(n-1)$
      \item $ \sumatoria{i=n}{2n} \frac{i}{2^i} \leq n$
      \item $ \sumatoria{i=1}{2^n} \frac{1}{2i-1} > \frac{n+3}{4}$
      \item $ \sumatoria{i=1}{n} \frac{1}{i!} \leq 2 - \frac{1}{2^{n-1}}$
      \item $ \productoria{i=1}{n} \frac{4i-1}{n+i} \geq 1$
    \end{enumerate}
  \end{multicols}
\end{enunciado}

\begin{enumerate}[label=\roman*)]
  \item $P(n): 3^n + 5^n \geq 2^{n+2},\ n \en \naturales$
        \begin{enumerate}[label=\arabic*)]
          \item Caso base, $n = 1$:
                \begin{align*}
                  P(1) & : 3^1 + 5^1 \geq 2^{1+2}     \\
                  P(1) & : 8 \geq 8 \entonces P(1): V
                \end{align*}
          \item Paso inductivo. Sea $n \en \naturales$:
                \begin{enumerate}
                  \item[HI.] $P(n): V$
                  \item[TI.] $P(n+1): 2^{n+3} \leq 3^{n+1} + 5^{n+1}$
                \end{enumerate}

                Desarrollemos el lado izquierdo de la desigualdad:
                \begin{align*}
                  2^{n+3}   & = 2 \cdot 2^{n+2} \igual{HI} 2 \cdot (3^n + 5^n) = 2 \cdot 3^n + 2 \cdot 5^n \leq
                  3 \cdot 3^n + 5 \cdot 5^n = 3^{n+1} + 5^{n+1}                                                 \\
                  2^{n+3}   & \leq 3^{n+1} + 5^{n+1}                                                            \\
                  \entonces & P(n+1): V
                \end{align*}
        \end{enumerate}

        Hemos probado el caso base y el paso inductivo. Concluimos que $\paratodo n \en \naturales, \ P(n): V$.

  \item $P(n): 3^n \geq n^3, \ n \en \naturales$.
        \begin{enumerate}[label=\arabic*)]
          \item Caso base, $n = 1$:
                \begin{align*}
                  P(1) & : 3^1 \geq 1^3               \\
                  P(1) & : 3 \geq 1 \entonces P(1): V
                \end{align*}
          \item Paso inductivo. Sea $n \en \naturales$:
                \begin{enumerate}
                  \item[HI.] $P(n): V$
                  \item[TI.] $P(n+1): 3^{n+1} \geq (n+1)^3$
                \end{enumerate}

                Desarrollemos el lado izquierdo de la desigualdad:
                \begin{align*}
                            & 3^{n+1} = 3 \cdot 3^n \mayorIgual{HI} 3 \cdot n^3 \mayorIgual{Aux} (n+1)^3, \ n \geq 3 \\
                            & 3^{n+1} \geq (n+1)^3, \ n \geq 3                                                       \\
                  \entonces & P(n+1): V, \ n \geq 3
                \end{align*}
        \end{enumerate}

        Hemos probado el caso base y el paso inductivo, este último para los $n \geq 3$. Como solo probamos el paso
        inductivo para $n \geq 3$, deberiamos ver que $P(2)$ y $P(3)$ son verdaderas.
        \begin{align*}
          P(2) & : 3^2 \geq 2^3 \entonces P(2): 9 \geq 9 \entonces P(2): V \\
          P(3) & : 3^3 \geq 3^3 \entonces P(3): V
        \end{align*}
        Tenemos que

        $P(1): V \land \ P(2): V \land  P(3): V \\
          \text{si } n \geq 3, \ P(n): V \entonces P(n+1): V$

        Concluimos que $\paratodo n \en \naturales, \ P(n): V$.

        \subsubsection*{Auxiliar}
        $3n^3 \geq (n+1)^3 \sisolosi \sqrt[3]{3n^3} \geq \sqrt[3]{(n+1)^3} \sisolosi
          \sqrt[3]{3}n \geq n+1 \sisolosi \sqrt[3]{3}n - n \geq 1 \\
          \sisolosi n(\sqrt[3]{3} - 1) \geq 1 \sisolosi  n \geq \frac{1}{\sqrt[3]{3} - 1}
          \approx 2.6 \sisolosi n \geq 3$ \\
        $\therefore 3n^3 \geq (n+1)^3 \sisolosi n \geq 3$

  \item $P(n):  \sumatoria{i=1}{n} \frac{n+i}{1+i} \leq 1 + n(n-1), \ n \en \naturales$.
        \begin{enumerate}[label=\arabic*)]
          \item Caso base, $n = 1$:
                \begin{align*}
                  P(1) & : \sumatoria{i=1}{1} \frac{1+i}{1+i} \leq 1 + 1(1-1) \\
                  P(1) & : \frac{1+1}{1+1}\leq 1 \entonces P(1): V
                \end{align*}
          \item Paso inductivo. Sea $n \en \naturales$:
                \begin{enumerate}
                  \item[HI.] $P(n): V$
                  \item[TI.] $ \sumatoria{i=1}{n+1} \frac{n+1+i}{1+i} \leq 1 + (n+1)n$
                \end{enumerate}

                Desarrollemos el lado izquierdo de la desigualdad:
                \begin{align*}
                            & \sumatoria{i=1}{n+1} \frac{n+1+i}{1+i} = \frac{2(n+1)}{n+2} + \sumatoria{i=1}{n} \frac{n+1+i}{1+i}
                  = \frac{2(n+1)}{n+2} + \sumatoria{i=1}{n} \left(\frac{1}{1+i} + \frac{n+i}{1+i} \right)                        \\
                            & = \frac{2(n+1)}{n+2} + \sumatoria{i=1}{n} \frac{1}{1+i} + \sumatoria{i=1}{n} \frac{n+i}{1+i}
                  \menorIgual{Aux} \frac{2\cancelto{1}{(n+1)}}{\cancelto{1}{(n+1)}} + \sumatoria{i=1}{n} 1
                  + \sumatoria{i=1}{n} \frac{n+i}{1+i} \igual{*}                                                                 \\
                            & \igual{*} 2 + n + \sumatoria{i=1}{n} \frac{n+i}{1+i} \menorIgual{HI} 2 + n + 1 + n(n-1)
                  = 2 + n + 1 + n^2 - n \igual{**}                                                                               \\
                            & \igual{**} 1 + n^2 + 2 \overset{(2 \leq n)}{\leq} 1 + n^2 + n = 1 + (n+1)n                         \\
                  \entonces & \sumatoria{i=1}{n+1} \frac{n+1+i}{1+i} \leq 1 + (n+1)n, \text{ si } n \geq 2                       \\
                  \entonces & P(n+1): V,\text{ si } n \geq 2
                \end{align*}
        \end{enumerate}

        Hemos probado el caso base y el paso inductivo, este último para los $n \geq 2$. Como solo probamos el paso
        inductivo para $n \geq 2$, deberiamos ver que $P(2)$ es verdadera.
        \begin{align*}
          P(2) & : \sumatoria{i=1}{2} \frac{2+i}{1+i} \leq 1 + 2(2-1) \\
          P(2) & : \frac{2+1}{1+1} + \frac{2+2}{1+2} \leq 3           \\
          P(2) & : \frac{17}{6} \leq 3 \entonces P(2): V
        \end{align*}
        Tenemos que

        $P(1): V \land P(2): V \\
          \text{si } n \geq 2, \ P(n): V \entonces P(n+1): V$

        Concluimos que $\paratodo n \en \naturales, \ P(n): V$.

        \subsubsection*{Auxiliar}
        Acotemos $1/(n+2)$
        \begin{align*}
          n+1 \leq n + 2 , \ \paratodo n \en \naturales \sisolosi \frac{1}{n+2} \leq \frac{1}{n+1},
          \ \paratodo n \en \naturales
        \end{align*}

        Acotemos la sumatoria
        \begin{align*}
          \frac{1}{1+i} \leq 1, \ \paratodo i \en \naturales \entonces \sumatoria{i=1}{n} \frac{1}{1+i} \leq \sumatoria{i=1}{n} 1
        \end{align*}

  \item $P(n):  \sumatoria{i=n}{2n} \frac{i}{2^i} \leq n, \ n \en \naturales$.
        \begin{enumerate}[label=\arabic*)]
          \item Caso base, $n = 1$:
                \begin{align*}
                  P(1) & : \sumatoria{i=1}{2 \cdot 1} \frac{i}{2^i} \leq 1    \\
                  P(1) & : \frac{1}{2} + \frac{2}{4} \leq 1 \entonces P(1): V
                \end{align*}
          \item Paso inductivo. Sea $n \en \naturales$:
                \begin{enumerate}
                  \item[HI.] $P(n): V$
                  \item[TI.] $P(n+1):  \sumatoria{i=n+1}{2(n+1)} \frac{i}{2^i} \leq n+1$
                \end{enumerate}

                Desarrollemos el lado izquierdo de la desigualdad:
                \begin{align*}
                  \sumatoria{i=n+1}{2(n+1)} \frac{i}{2^i}           & = \frac{2n+2}{2^{2n+2}} + \frac{2n+1}{2^{2n+1}}
                  + \sumatoria{i=n+1}{2n} \frac{i}{2^i} = \frac{n}{2^n} - \frac{n}{2^n} + \frac{2n+2}{2^{2n+2}}
                  + \frac{2n+1}{2^{2n+1}} + \sumatoria{i=n+1}{2n} \frac{i}{2^i}                                                                                             \\
                                                                    & = - \frac{n}{2^n} + \frac{2n+2}{2^{2n+2}} + \frac{2n+1}{2^{2n+1}} + \sumatoria{i=n}{2n} \frac{i}{2^i}
                  = - \frac{2^{n+2}}{2^{n+2}}\frac{n}{2^n} + \frac{2n+2}{2^{2n+2}} + \frac{2}{2}\frac{2n+1}{2^{2n+1}}
                  + \sumatoria{i=n}{2n} \frac{i}{2^i}                                                                                                                       \\
                                                                    & = - \frac{4n 2^n}{2^{2n+2}} + \frac{2n+2}{2^{2n+2}} + \frac{4n+2}{2^{2n+2}}
                  + \sumatoria{i=n}{2n} \frac{i}{2^i} = \frac{-4n 2^n + 6n + 4}{2^{2n+2}} + \sumatoria{i=n}{2n} \frac{i}{2^i}
                  \menorIgual{HI}                                                                                                                                           \\
                                                                    & \menorIgual{HI} \frac{-4n 2^n + 6n + 4}{2^{2n+2}} + n
                  \menorIgual{Aux} 1 + n, \ n \geq 2                                                                                                                        \\
                  \entonces \sumatoria{i=n+1}{2(n+1)} \frac{i}{2^i} & \leq n + 1, \ n \geq 2 \entonces P(n+1): V, \ n \geq 2
                \end{align*}

        \end{enumerate}

        Hemos probado el caso base y el paso inductivo, este último para los $n \geq 2$. Como solo probamos el paso
        inductivo para $n \geq 2$, deberiamos ver que $P(2)$ es verdadera.
        \begin{align*}
          P(2) & : \sumatoria{i=2}{2 \cdot 2} \frac{i}{2^i} \leq 2 \\
          P(2) & : \frac{2}{4} + \frac{3}{8} + \frac{4}{16} \leq 2 \\
          P(2) & : \frac{9}{8} \leq 2 \entonces P(2): V
        \end{align*}

        Tenemos que

        $P(1): V \land P(2): V \\
          \text{si } n \geq 2, \ P(n): V \entonces P(n+1): V$

        Concluimos que $\paratodo n \en \naturales, \ P(n): V$.

        \subsubsection*{Auxiliar}
        Acotemos el termino $-4n 2^n$
        \begin{align*}
          -4n 2^n \leq -4n \cdot 2 = -8n \entonces -4n 2^n \leq -8n
        \end{align*}

        Usemos esto para acotar toda la fracción
        \begin{align*}
          \frac{-4n 2^n + 6n + 4}{2^{2n+2}} \leq \frac{-8n + 6n + 4}{2^{2n+2}} \leq \frac{-2n + 4}{2^{2n+2}}
          \overset{(n \geq 2)}{\leq} \frac{1}{2^{2n+2}} \leq 1
        \end{align*}

        Por último, veamos porque $-2n+4 \leq 1$
        \begin{align*}
          -2n + 4 & \leq 0 \sisolosi -2n \leq -4 \sisolosi n \geq \frac{-4}{-2}
          \sisolosi n \geq 2                                                                    \\
          -2n + 4 & \leq 0, \text{ si } n \geq 2 \entonces -2n + 4 \leq 1, \text{ si } n \geq 2
        \end{align*}

  \item $P(n):  \sumatoria{i=1}{2^n} \frac{1}{2i-1} > \frac{n+3}{4}, \ n \en \naturales$.
        \begin{enumerate}[label=\arabic*)]
          \item Caso base, $n = 1$:
                \begin{align*}
                  P(1) & : \sumatoria{i=1}{2^1} \frac{1}{2i-1} > \frac{1+3}{4} \\
                  P(1) & : \frac{1}{1} + \frac{1}{3} > 1  \entonces P(1): V
                \end{align*}

          \item Paso inductivo. Sea $n \en \naturales$:
                \begin{enumerate}
                  \item[HI.] $P(n): V$
                  \item[TI.] $P(n+1):  \sumatoria{i=1}{2^{n+1}} \frac{1}{2i-1} > \frac{n+4}{4}$
                \end{enumerate}

                Desarrollemos el lado derecho de la desigualdad:
                \begin{align*}
                  \frac{n+4}{4}           & = \frac{1}{4} + \frac{n+3}{4} \overset{\text{HI}}{<} \frac{1}{4} + \sumatoria{i=1}{2^n}
                  \frac{1}{2i-1} = \frac{1}{4} + \sumatoria{i=2^n + 1}{2^{n+1}} \frac{1}{2i-1} - \sumatoria{i=2^n + 1}{2^{n+1}}
                  \frac{1}{2i-1} + \sumatoria{i=1}{2^n} \frac{1}{2i-1} \igual{*}                                                             \\
                                          & \igual{*} \frac{1}{4} - \sumatoria{i=2^n + 1}{2^{n+1}} \frac{1}{2i-1} + \sumatoria{i=1}{2^{n+1}}
                  \frac{1}{2i-1} \overset{\text{Aux.1}}{<} 0 + \sumatoria{i=1}{2^{n+1}} \frac{1}{2i-1} = \sumatoria{i=1}{2^{n+1}}
                  \frac{1}{2i-1}                                                                                                             \\
                  \entonces \frac{n+4}{4} & < \sumatoria{i=1}{2^{n+1}} \frac{1}{2i-1} \entonces P(n+1):V
                \end{align*}
        \end{enumerate}

        Hemos probado el caso base y el paso inductivo. Concluimos que $\paratodo n \en \naturales, \ P(n): V$.

        \subsubsection*{Auxiliar 1}
        \begin{align*}
          \sumatoria{i=2^n + 1}{2^{n+1}} \frac{1}{2i-1}           & = \frac{1}{2^{n+1}+1} + \frac{1}{2^{n+1}+3} + \dots
          + \frac{1}{2^{n+1}-1} > \frac{1}{2^{n+2}} + \frac{1}{2^{n+2}} + \dots + \frac{1}{2^{n+2}}
          \igual{*}                                                                                                                                                                 \\
                                                                  & \igual{*} \sumatoria{i=2^n + 1}{2^{n+1}} \frac{1}{2^{n+2}} = \frac{1}{2^{n+2}} \sumatoria{i=2^n + 1}{2^{n+1}} 1
          \igual{\tiny Aux.2}
          \frac{1}{2^{n+2}} 2^n = \frac{\cancelto{1}{2^n}}{2^2\cancelto{1}{2^n}} = \frac{1}{4}                                                                                      \\
          \entonces \sumatoria{i=2^n + 1}{2^{n+1}} \frac{1}{2i-1} & > \frac{1}{4}
          \entonces -\sumatoria{i=2^n + 1}{2^{n+1}} \frac{1}{2i-1} < -\frac{1}{4}
          \entonces \frac{1}{4} -\sumatoria{i=2^n + 1}{2^{n+1}} \frac{1}{2i-1} < 0
        \end{align*}

        \subsubsection*{Auxiliar 2}

        Cantidad de sumandos en una sumatoria
        \begin{align*}
          \sumatoria{i=A}{B}a_i = a_A + \dots + a_B \\
          \# Elementos = \sumatoria{i=A}{B}1 = B + 1 - A
        \end{align*}

        Calculemos la cantidad de sumandos en $ \sumatoria{i=2^n + 1}{2^{n+1}} \frac{1}{2i-1}$
        \begin{align*}
          B = 2^{n+1}, A = 2^n + 1 \entonces & \# Elementos = B + 1 - A = 2^{n+1} + 1  - (2^n + 1)
          = 2 \cdot 2^n + 1 - 2^n -1                                                               \\
          \entonces                          & \# Elementos = 2^n
        \end{align*}

  \item $P(n): \sumatoria{i=1}{n} \frac{1}{i!} \leq 2 - \frac{1}{2^{n-1}}, \ n \en \naturales$.
        \begin{enumerate}[label=\arabic*)]
          \item Caso base, $n = 1$:
                \begin{align*}
                  P(1) & : \sumatoria{i=1}{1} \frac{1}{i!} \leq 2 - \frac{1}{2^{1-1}} \\
                  P(1) & : \frac{1}{1} \leq 1 \entonces P(1):V
                \end{align*}
          \item Paso inductivo. Sea $n \en \naturales$:
                \begin{enumerate}
                  \item[HI.] $P(n): V$
                  \item[TI.] $P(n+1):  \sumatoria{i=1}{n+1} \frac{1}{i!} \leq 2 - \frac{1}{2^{n}}$
                \end{enumerate}

                Desarrollemos el lado izquierdo de la desigualdad:
                \begin{align*}
                  \sumatoria{i=1}{n+1} \frac{1}{i!}           & = \frac{1}{(n+1)!} + \sumatoria{i=1}{n} \frac{1}{i!} \menorIgual{HI}
                  \frac{1}{(n+1)!} + 2 - \frac{1}{2^{n-1}} = \frac{1}{(n+1)!} + 2 - \frac{2}{2^{n}} \igual{*}                                        \\
                                                              & \igual{*}  2 - \frac{1}{2^{n}} - \frac{1}{2^{n}} + \frac{1}{(n+1)!} \menorIgual{Aux}
                  2 - \frac{1}{2^{n}} - \frac{1}{2^{n}} + \frac{1}{2^{n}} = 2 - \frac{1}{2^{n}}                                                      \\
                  \entonces \sumatoria{i=1}{n+1} \frac{1}{i!} & \leq 2 - \frac{1}{2^{n}} \entonces P(n+1):V
                \end{align*}
        \end{enumerate}

        Hemos probado el caso base y el paso inductivo. Concluimos que $\paratodo n \en \naturales, \ P(n): V$.

        \subsubsection*{Auxiliar}
        \begin{align*}
          \frac{1}{(n+1)!} \leq \frac{1}{2^n} \sisolosi 2^n \leq (n+1)!
        \end{align*}
        Probemos esto último usando inducción. Sea $Q(n): 2^n \leq (n+1)!, \ n \en \naturales$.

        \begin{enumerate}[label=\arabic*)]
          \item Caso base, $n = 1$:
                \begin{align*}
                  Q(1) & : 2^1 \leq (1+1)!           \\
                  Q(1) & : 2 \leq 2 \entonces Q(1):V
                \end{align*}
          \item Paso inductivo. Sea $n \en \naturales$:
                \begin{enumerate}
                  \item[HI.] $Q(n): V$
                  \item[TI.] $Q(n+1): 2^{n+1} \leq (n+2)! $
                \end{enumerate}

                Desarrollemos el lado izquierdo de la desigualdad:
                \begin{align*}
                  2^{n+1}           & = 2 \cdot 2^n \menorIgual{HI} 2 (n+1)! \overset{(2\leq n+2)}{\leq} (n+2)(n+1)! = (n+2)! \\
                  \entonces 2^{n+1} & \leq (n+2)! \entonces Q(n+1):V
                \end{align*}
        \end{enumerate}

        Hemos probado el caso base y el paso inductivo. Concluimos que $\paratodo n \en \naturales, \ Q(n): V$.

  \item $P(n):  \productoria{i=1}{n}\frac{4i-1}{n+i} \geq 1, \ n \en \naturales$.
        \begin{enumerate}[label=\arabic*)]
          \item Caso base, $n = 1$:
                \begin{align*}
                  P(1) & : \productoria{i=1}{1} \frac{4i-1}{1+i} \geq 1 \\
                  P(1) & : \frac{3}{2} \geq 1 \entonces P(1): V
                \end{align*}
          \item Paso inductivo. Sea $n \en \naturales$:
                \begin{enumerate}
                  \item[HI.] $P(n): V$
                  \item[TI.] $P(n+1):  \productoria{i=1}{n+1} \frac{4i-1}{n+1+i} \geq 1 $
                \end{enumerate}

                Desarrollemos el lado derecho de la desigualdad:
                \begin{align*}
                  1           & \menorIgual{HI} \productoria{i=1}{n} \frac{4i-1}{n+i} = \frac{4(n+1)-1}{4(n+1)-1} \cdot
                  \frac{n + (n+1)}{n + (n+1)} \cdot \productoria{i=1}{n} \frac{4i-1}{n+i} \igual{*}                                   \\
                              & \igual{*}\frac{n+(n+1)}{4(n+1)-1} \cdot
                  \frac{4(n+1)-1}{n+(n+1)} \cdot \productoria{i=1}{n} \frac{4i-1}{n+i} = \frac{n+(n+1)}{4(n+1)-1} \cdot
                  \productoria{i=1}{n+1} \frac{4i-1}{n+i} \igual{**}                                                                  \\
                              & \igual{**} \frac{2n+1}{4n+3} \cdot \productoria{i=1}{n+1} \frac{4i-1}{n+i} \cdot \frac{n+1+i}{n+1+i}
                  = \frac{2n+1}{4n+3} \cdot \productoria{i=1}{n+1} \frac{4i-1}{n+1+i} \cdot \frac{n+1+i}{n+i} \igual{***}             \\
                              & \igual{***} \frac{2n+1}{4n+3} \cdot \productoria{i=1}{n+1} \frac{4i-1}{n+1+i}
                  \productoria{i=1}{n+1}  \frac{n+1+i}{n+i} \igual{\tiny Aux.1} \frac{2n+1}{4n+3} \cdot
                  \productoria{i=1}{n+1} \frac{4i-1}{n+1+i} \cdot 2 =                                                                 \\
                              & = 2 \cdot \frac{2n+1}{4n+3} \cdot \productoria{i=1}{n+1} \frac{4i-1}{n+1+i} = \frac{4n+2}{4n+3} \cdot
                  \productoria{i=1}{n+1} \frac{4i-1}{n+1+i} \menorIgual{\tiny Aux.2} 1 \cdot
                  \productoria{i=1}{n+1} \frac{4i-1}{n+1+i}                                                                           \\
                  \entonces 1 & \leq \productoria{i=1}{n+1} \frac{4i-1}{n+1+i} \entonces P(n+1):V
                \end{align*}
        \end{enumerate}

        Hemos probado el caso base y el paso inductivo. Concluimos que $\paratodo n \en \naturales, \ Q(n): V$.

        \subsubsection*{Auxiliar 1}
        Recordemos la formula del factorial de $n$
        \begin{align*}
          n! = \productoria{i=1}{n}  i = 1 \cdot 2 \dots n
        \end{align*}
        Veamos que pasa si sumamos una constante $k \en \naturales$ a la parte de la productoria
        \begin{align*}
          \productoria{i=1}{n}  (k + i) & = (k+1) \cdot (k+2) \dots (k+n) = \frac{1 \cdot 2 \dots k}{1 \cdot 2 \dots k}
          (k+1) \cdot (k+2) \dots (k+n)                                                                                                         \\
                                        & = \frac{1 \cdot 2 \dots k \cdot (k+1) \cdot (k+2) \dots (k+n)}{1 \cdot 2 \dots k} = \frac{(k+n)!}{k!} \\
          \productoria{i=1}{n}  (k + i) & = \frac{(k+n)!}{k!}
        \end{align*}

        Calculemos $ \productoria{i=1}{n+1}  \frac{n+1+i}{n+i}$
        \begin{align*}
          \productoria{i=1}{n+1}  \frac{n+1+i}{n+i}           & = \frac{ \productoria{i=1}{n+1}  n+1+i}
          { \productoria{i=1}{n+1}  n+i}
          = \frac{\frac{(n+1 + n+1)!}{(n+1)!}}{ \frac{(n+n+1)!}{n!}}
          = \frac{ \frac{(2n+2)!}{(n+1)!}}{ \frac{(2n+1)!}{n!}}                                                                                                \\
                                                              & = \frac{(2n+2)!}{(n+1)!}: \frac{(2n+1)!}{n!} = \frac{(2n+2)!}{(n+1)!} \cdot \frac{n!}{(2n+1)!} \\
                                                              & = \frac{(2n+2)(2n+1)!}{(n+1)n!} \cdot \frac{n!}{(2n+1)!}
          = \frac{2(n+1)(2n+1)!}{(n+1)n!} \cdot \frac{n!}{(2n+1)!}                                                                                             \\
                                                              & = 2 \frac{(n+1)(2n+1)!n!}{(n+1)(2n+1)!n!} = 2                                                  \\
          \entonces \productoria{i=1}{n+1}  \frac{n+1+i}{n+i} & = 2
        \end{align*}

        \subsubsection*{Auxiliar 2}
        \begin{align*}
          2 \leq 3 \entonces 4n + 2 \leq 4n + 3 \entonces \frac{4n+2}{4n+3} \leq 1
        \end{align*}
\end{enumerate}

% Contribuciones
\begin{aportes}
  %% iconos : \github, \instagram, \tiktok, \linkedin
  %\aporte{url}{nombre icono}
  \item \aporte{https://github.com/koopardo/}{Marcos Zea \github}
\end{aportes}
