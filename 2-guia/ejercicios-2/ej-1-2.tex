\ejercicio
\begin{enumerate}[label=\roman*)]
    \item Reescribir cada una de las siguientes sumas usando el símbolo de sumatoria\\ 
    
    ¿Cómo resolver este ejercicio?\\
    Lo que queremos hacer es compactar la suma para evitar el uso\\
    de puntos suspensivos, la notación ideal para esos casos es el símbolo de sumatoria.\\
    El primer paso es fijarse en el comportamiento de cada término de\\
    nuesta suma. Por ejemplo, en el punto \ref{referenciaAlItem:1-2} notamos que cada término comienza a\\
    duplicarse.


    \begin{enumerate}[label=\alph*)] 
        \item  $1 + 2 + 3 + 4 + \dots + 100$\\
        Respuesta: $ \sumatoria{i=1}{100}i $\\

        \item\label{referenciaAlItem:1-2}  $1 + 2 + 4 + 8 + 16 + \dots + 1024$\\
        Respuesta: $ \sumatoria{i=0}{10}2^i $\\

      \item  $1 + (-4) + 9 + (-16) + 25 + \dots + (-144)$\\
        Respuesta: $ \sumatoria{i=1}{12}i^2(-1)^{n+1} $\\

        \item $1 + 9 + 25 + 49 + \dots + 441$\\

        Respuesta: $ \sumatoria{i=0}{10}(1+2i)^2 $\\

        \item $1 + 3 + 5 + \dots + (2n + 1)$\\
        Respuesta: $ \sumatoria{i=0}{n} 2i+1 $\\

        \item $n + 2n + 3n + \dots + n^2$\\
        Respuesta: $ \sumatoria{i=1}{n} in $\\
    \end{enumerate}

    \item

	  \begin{enumerate}[label=\alph*)]
      \item $5 \cdot 6 \cdots 99 \cdot 100$\\
      Respuesta: $\productoria{i=5}{100}i = \frac{100!}{4!}$\\

      \item $1 \cdot 2 \cdot 4 \cdot 8 \cdot 16 \cdots 1024$\\
      Respuesta: $\productoria{i=0}{10}2^i$\\

      \item $n \cdot 2n \cdot 3n \cdots n^2$\\
      Respuesta: $\productoria{i=1}{n}in = n^n \cdot n!$\\
\end{enumerate}
