\ejercicio
\begin{enumerate}[label=\roman*)]
    \item Reescribir cada una de las siguientes sumas usando el símbolo de sumatoria\\ 
    
    ¿Como resolver este ejercicio?\\
    Lo que queremos hacer es compactar la suma para evitar evitar el uso\\
    de suspensivos, la notacion ideal para esos casos es el simbolo de sumatoria.\\
    El primer paso es fijarse en el comportamiento de cada termino de\\
    nuesta suma. Por ejemplo, en el punto (b) notamos que cada termino comienza a\\
    duplicarse.


    \begin{enumerate}[label=\alph*)] 
        \item  1 + 2 + 3 + 4 + · · · + 100\\
        Respuesta: $ \sumatoria{i=1}{100}i $\\

        \item  1 + 2 + 4 + 8 + 16 + · · · + 1024\\
        Respuesta: $ \sumatoria{i=0}{10}2^i $\\

        \item  1 + (-4) + 9 + (-16) + 25 + · · · + (-144)\\
        Respuesta: $ \sumatoria{i=1}{12}i^2(-1)^{n+1} $\\

        \item 1 + 9 + 25 + 49 + · · · + 441\\
        Respuesta: $ \sumatoria{i=0}{10}(1+2i)^2 $\\

        \item 1 + 3 + 5 + · · · + (2n + 1)\\
        Respuesta: $ \sumatoria{i=0}{n} 2i+1 $\\

        \item n + 2n + 3n + · · · + n^2\\
        Respuesta: $ \sumatoria{i=1}{n} in $\\
    \end{enumerate}
\end{enumerate}

