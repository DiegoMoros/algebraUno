\separador

\textit{Recurrencia}

\begin{enunciado}{\ejercicio}
  \begin{enumerate}[label=\roman*)]
    \item
          Sea $(a_n)_{n\ en \naturales}$ la sucesión de números reales definida recursivamente por
          $$
            a_1 = 2 \quad\ytext\quad a_{n+1} = 2 n a_n+ 2^{n+1}n!,\, \paratodo n \en \naturales
          $$
          Probar que $a_n = 2^n n!$.

    \item
          Sea $(a_n)_{n\ en \naturales}$ la sucesión de números reales definida recursivamente por
          $$
            a_1 = 0 \quad\ytext\quad a_{n+1} = a_n + n(3n + 1),\, \paratodo n \en \naturales
          $$
          Probar que $a_n = n^2 (n-1)$.
  \end{enumerate}
\end{enunciado}

\begin{enumerate}[label=\roman*)]
  \item
        Inducción. \textit{Proposición: }
        $$
          p(n) : a_n = 2^n n! \quad \paratodo n \en \naturales
        $$
        \textit{Caso base: }
        $ p(\magenta1) :
          \llave{l}{
          a_{\magenta1} \igual{def} 2 = 2^{\magenta1} \cdot \magenta1! \Tilde \\
              a_{\magenta1 + 1} = a_2 \igual{def} 2 \cdot \magenta1 \cdot a_{\magenta1} + 2^{\magenta1 + 1}\magenta1! =
              8
              \igual!
              2^2 \cdot 2 \Tilde.
            }
        $\par Resulta que $p(1)$ es verdadera.\medskip

        \textit{Paso inductivo: }\par
        $p(\blue{k}): \ub{ a_{\blue k} = 2^{\blue k} \blue k! }{\textit{\purple{hipótesis inductiva}}} $ asumo verdadera
        para algún $k \en \enteros \entonces$
        quiero que $p(\blue{k} + 1) : a_{\blue k + 1} = 2^{\blue k + 1}  (\blue k + 1)!$ también lo sea.\par

        $ a_{\blue{k} + 1} \igual{def}
          2 \blue{k} \cdot a_{\blue k} + 2^{\blue k + 1} \blue k! \igual{\purple{HI}}
          2k \cdot \purple{2^k k!} +  2^{k + 1} k! = 2^{k+1}k!(k + 1) \igual{\red{!}} 2^{k+1}(k+1)! \Tilde
        $. Resulta que $p(k+1)$ es verdadera.\medskip

        Como $p(1),\, p(k) \ytext p(k+1)$ son verdaderas, por el principio de inducción también lo es
        $p(n) \paratodo n \en \naturales$

  \item
        Inducción. \textit{Proposición: }
        $$
          p(n) : a_n = n^2(n-1) \quad \paratodo n \en \naturales
        $$
        \textit{Caso base: }
        $ p(\magenta1) :
          \llave{l}{
            a_{\magenta1} \igual{def} 0 = 1^2 \cdot (\magenta1 - 1) \Tilde \\
            a_{\magenta1 + 1} = a_2
            \igual{def}
            a_{\magenta1} + \magenta1 (3\cdot \magenta1 + 1) =
            4 = 2^2 \cdot (2 - 1) \Tilde.
          }
        $\par Resulta que $p(1)$ es verdadera.\medskip

        \textit{Paso inductivo: }\par
        $p(\blue{k}): \ub{ a_{\blue k} = \blue k^2 (\blue{k} - 1) }{\textit{\purple{hipótesis inductiva}}} $ asumo verdadera
                para algún
                $k \en \enteros \entonces$ quiero que\par
                $p(\blue{k} + 1) : a_{\blue k + 1} =
                (\blue k + 1)^2 (\blue{k} + 1 - 1) =
                \blue{k} (\blue k + 1)^2$
        también lo sea.\par

        $ a_{\blue{k} + 1} \igual{def}
                a_{\blue k} + \blue k (3\blue k + 1)
                \igual{\purple{HI}}
                \purple{k^2 (k - 1)} + 3 k^2 + k = k^3 + 2k^2 + k
                \igual{!}
                k(k+1)^2 \Tilde
        $. Resulta que $p(k+1)$ es verdadera.\medskip

        Como $p(1),\, p(k) \ytext p(k+1)$ son verdaderas, por el principio de inducción también lo es
        $p(n) \paratodo n \en \naturales$
\end{enumerate}
