\begin{enunciado}{\ejercicio}
    Sea $f:\mathbb{R}-\{0,1\}\rightarrow \mathbb{R}-\{0,1\}$ definida por $f(x) = \frac{1}{1-x}$. Para 
    $n \in \mathbb{N}$ se define:
    \begin{align*}
        f^n = \underbrace {f \circ f \circ \cdots \circ f}_\text{$n$ veces}
    \end{align*}
    Probar que $f^{3k}(x) = x$ para todo $k \in \mathbb{N}$.
\end{enunciado}
Sea $f:\mathbb{R}-\{0,1\}\rightarrow \mathbb{R}-\{0,1\}$ definida por
\setcounter{equation}{0}
\begin{align}
    f(x) = \frac{1}{1-x}
\end{align}
Sea $n \in \mathbb{N}$, definimos $f^n$ como la composición de $f$ consigo misma $n$ veces

\begin{align}
    f^n = \underbrace{f \circ f \circ \cdots \circ f}_\text{$n$ veces}
\end{align}
Probemos lo que pide el ejercicio por inducción
\begin{align*}
    P(k):f^{3k}(x) = x,\ k \in \mathbb{N}
\end{align*}
\underline{Caso Base}, $k = 1$:
\begin{align*}
    P(1): f^{3 \cdot 1}(x) &= f^3(x) \overset{(2)}{=} f \circ f \circ f(x) = f(f(f(x))) 
    \overset{(1)}{=} f \left(f\left(\frac{1}{1-x}\right)\right) 
    \overset{(1)}{=} f \left(\frac{1}{1 - \displaystyle \frac{1}{1-x} }\right) \\
    &= f \left(\frac{1}{\displaystyle \frac{1-x-1}{1-x}}\right)
    = f \left(\frac{1}{\displaystyle \frac{-x}{1-x}}\right) = f \left(\frac{1-x}{-x}\right) 
    = f \left(\frac{x-1}{x}\right) \\
    \overset{(1)}&{=} \frac{1}{1 - \displaystyle \frac{x-1}{x}} = \frac{1}{\displaystyle \frac{x-x+1}{x}}
    = \frac{1}{\displaystyle \frac{1}{x}} = x \\
    P(1): f^3(x) &= x \implies P(1):V
\end{align*}
\underline{Paso inductivo.} Sea $k \geq 1$:
\begin{enumerate}
    \item[HI.] $P(k): V$
    \item[TI.] $P(k+1): f^{3(k+1)}(x) = x$
\end{enumerate}
Desarrollemos el lado izquierdo de la igualdad en la TI
\begin{align*}
    f^{3(k+1)}(x) &= f^{3k+3}(x) \overset{(2)}{=} f^{3k} \circ f^3(x) = f^{3k}\left( f^3(x)\right)
    \overset{P(1)}{=} f^{3k}(x) \overset{\text{HI}}{=} x \\
    f^{3(k+1)}(x) &= x\implies P(k+1):V
\end{align*}
Hemos probado el caso base y el paso inductivo. Concluimos que $P(k):V,$ $\forall k \in \mathbb{N}$. \\

