\begin{enunciado}{\ejercicio}
    Probar que todo número natural $n$ se escribe como suma de distintas potencias de 2, incluyendo $2^0 = 1$.
\end{enunciado}
Probemos que todo número natural $n$ se escribe como suma de distintas potencias de 2 usando inducción
\begin{align*}
    P(n): n = \sumatoria{i=1}{r}  2^{\alpha_i},\, 0 \leq \alpha_1 < \alpha_2 < \cdots < \alpha_r,\, n \en \naturales 
\end{align*}
\underline{Caso Base}, $n = 1$:
\begin{align*}
    P(1): 1 = 2^0 \Entonces{Obs} P(1):V
\end{align*}
\underline{Paso inductivo}. Sean $n,k \en \naturales $, donde $1 \leq k \leq n$:
\begin{enumerate}
    \item[HI.] $P(n):V,\, \paratodo k$
    \item[TI.] $P(n+1): n + 1 = \displaystyle \sumatoria{i=1}{s}  2^{\beta_i},\, 
    0 \leq \beta_1 < \beta_2 < \cdots < \beta_k$
\end{enumerate}
Desarrollemos el lado izquierdo de la igualdad en la TI. Para ello analizemos el caso donde $n$ es par y otro donde es impar.\\
Caso $n$ impar: \\
Si $n$ impar esto implica que $n+1$ es par. Podemos escribir a $n+1$ como $n + 1 = 2h,\, h \en \naturales $
\begin{align*}
    h \overset{\text{Aux.1}}{\leq} n \overset{\text{HI}}&{\entonces}
    h = \sumatoria{i=1}{r} 2^{\alpha_i} \entonces 2h = 2 \sumatoria{i=1}{r} 2^{\alpha_i} 
    \entonces 2h = \sumatoria{i=1}{r} 2 \cdot 2^{\alpha_i} \entonces 2h = \sumatoria{i=1}{r} 2^{\alpha_i + 1} \\
    \overset{2h = n+1}&{\entonces} n + 1 = \sumatoria{i=1}{r} 2^{\alpha_i + 1},\, 
    1 \leq \alpha_1+1 < \alpha_2+1 < \cdots < \alpha_r+1\\
    \overset{\text{Obs}}&{\entonces} P(n+1):V,\, n \text{ impar}
\end{align*}
Caso $n$ par: \\
Si $n$ par esto implica que $n+1$ es impar. Podemos escribir a $n+1$ como $n + 1 = 2h + 1,\, h \en \naturales $
\begin{align*}
    h \overset{\text{Aux.2}}{\leq} n \overset{\text{HI}}&{\entonces}
    h = \sumatoria{i=1}{r} 2^{\alpha_i} \entonces 2h = 2 \sumatoria{i=1}{r} 2^{\alpha_i} 
    \entonces 2h = \sumatoria{i=1}{r} 2 \cdot 2^{\alpha_i} \entonces 2h = \sumatoria{i=1}{r} 2^{\alpha_i + 1} \\
    &\entonces 1 + 2h = 1 + \sumatoria{i=1}{r} 2^{\alpha_i + 1}
    \entonces 2h + 1 = 2^0 + \sumatoria{i=1}{r} 2^{\alpha_i + 1} \\
    \overset{2h+1 = n+1}&{\entonces} n + 1 = 2^0 + \sumatoria{i=1}{r} 2^{\alpha_i + 1},\, 
    1 \leq \alpha_1+1 < \alpha_2+1 < \cdots < \alpha_r+1\\
    \overset{\text{Obs}}&{\entonces} P(n+1):V,\, n \text{ impar}
\end{align*}
Por lo tanto, $P(n+1):V$ para cualquier $n$ natural. Hemos probado el caso base y el paso inductivo. Concluimos que $P(n):V,$ $\paratodo n \en \naturales $.

\paragraph{Auxiliar 1}{
    Acotemos $h$
    \begin{align*}
        2h = n + 1 \entonces h = \frac{n+1}{2} \entonces h = \frac{n+1}{2} \leq \frac{2n}{2} \entonces h \leq n
    \end{align*}
}
\paragraph{Auxiliar 1}{
    Acotemos $h$
    \begin{align*}
        2h + 1 = n + 1 \entonces h = \frac{n}{2} \entonces h = \frac{n}{2} \leq n \entonces h \leq n
    \end{align*}
}

\paragraph{Observación}{
    Aclaraciones sobre el uso de las preposiciones $P(n)$ y $P(n+1)$
    \begin{itemize}
        \item El caso base nos queda verdadero porque encontramos una forma de escribir a 1 como una suma de potencias
        de 2, en este caso la sumatoria solo posee un término que es $2^0$. En nuestra proposición $P(n)$ aparecen $r$ 
        y los $\alpha_i$ (exponentes de 2). Para nuestro caso base, tenemos que $r = 1$ y $\alpha_1 = 0$.
        \item En el paso inductivo, caso $n$ impar, llegamos a escribir $n+1$ como suma de potencias de 2, lo cual
        hace que $P(n+1)$ sea verdadera. Otra forma de verlo, es que encontramos los $\beta_i$ mencionados en la TI. 
        Basta tomar $\beta_i = \alpha_i + 1$, con $i = 1, \cdots , r$, con lo cual $s = r$. Estos $\beta_i$ son válidos
        pues se deducen de la HI. 
        \item En el paso inductivo, caso $n$ par, llegamos a escribir $n+1$ como suma de potencias de 2, lo cual
        hace que $P(n+1)$ sea verdadera. Otra forma de verlo, es que encontramos los $\beta_i$ mencionados en la TI. 
        Basta tomar $\beta_1 = 0$ y $\beta_{i+1} = \alpha_i + 1$, con $i = 1, \cdots , r$, con lo cual $s = r + 1$.
        Estos $\beta_{i+1}$ son válidos pues se deducen de la HI. 
    \end{itemize}
}

