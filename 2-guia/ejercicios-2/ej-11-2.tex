\begin{enunciado}{\ejercicio}
  \begin{enumerate}[label=\roman*)]
    \item Sea $(a_n)_{n \en \naturales }$ una sucesión de números reales todos del mismo signo y tales que 
    $a_n > -1$ para todo $n \in \mathbb{N}$. Probar que 
    \begin{align*}
      \prod_{i=1}^{n} (1 + a_i) \geq 1 + \sum_{i=1}^{n}a_i
    \end{align*}
          
    ¿En qué paso de la demostración se usa que $a_n > -1$ para todo $n \in \mathbb{N}$? ¿Y que todos los términos
    de la sucesión $(a_n)_{n \in \mathbb{N}}$ tienen el mismo signo?

    \item Deducir que si $a \in \mathbb{R}$ tal que $a > -1$, entonces $(1+a)^n \geq 1 + na$.
  \end{enumerate}
\end{enunciado}
\setcounter{equation}{0}
\begin{enumerate}[label=\roman*)]
  \item Sea $(a_n)_{n \in \mathbb{N}}$ sucesión de números reales tales que
  \begin{align}
    &sg(a_n) = sg(a_1), \ \forall n \in \mathbb{N} \\
    &a_n > -1, \ \forall n \in \mathbb{N} \iff 1 + a_n > 0, \ \forall n \in \mathbb{N}
  \end{align}
  donde $sg(x)$ es la función signo de $x$
  \begin{align*}
    sg(x) = \begin{cases}
              -1, \text{ si } x < 0 \\
              \phantom{-}1, \text{ si } x > 0
            \end{cases}
  \end{align*}

  Definamos nuestra proposición $P(n)$ 
  \begin{align*}
    P(n): \prod_{i=1}^{n} (1 + a_i) \geq 1 + \sum_{i=1}^{n}a_i, \ n \in \mathbb{N}
  \end{align*}

  \underline{Caso Base}, $n = 1$:
  \begin{align*}
  P(1)&: \prod_{i=1}^{n} (1 + a_i) \geq 1 + \sum_{i=1}^{1}a_i \\
  P(1)&: 1 + a_1 \geq 1 + a_1 \implies P(1):V
  \end{align*}

  \underline{Paso inductivo}. Sea $n \in \mathbb{N}$:
  \begin{enumerate}
    \item[HI.] $P(n): V$
    \item[TI.] $P(n+1): \displaystyle \prod_{i=1}^{n+1} (1 + a_i) \geq 1 + \sum_{i=1}^{n+1}a_i$
  \end{enumerate}
  Desarrollemos el lado izquierdo de la desigualdad
  \begin{align*}
    \prod_{i=1}^{n+1} (1 + a_i) &= (1 + a_{n+1}) \prod_{i=1}^{n} (1 + a_i) \overset{\text{HI}}{\underset{(2)}{\geq}} 
    (1 + a_{n+1}) \left(1 + \sum_{i=1}^{n}a_i\right) = 1 + \sum_{i=1}^{n}a_i + a_{n+1} + a_{n+1}\sum_{i=1}^{n}a_i 
    \overset{*}{=} \\
    &\overset{*}{=} 1 + \sum_{i=1}^{n+1}a_i + a_{n+1}\sum_{i=1}^{n}a_i \overset{\text{Aux}}{\geq}
    1 + \sum_{i=1}^{n+1}a_i \implies P(n+1):V 
  \end{align*}
  Hemos probado el caso base y el paso inductivo, entonces $P(n): V, \ \forall n \in \mathbb{N}$.
  \paragraph{Auxiliar}{Queremos ver que $a_{n+1}\sum_{i=1}^{n}a_i > 0$}. La Ec.(1) nos dice que $a_n$ nunca cambia de 
  signo mientras que la Ec.(2) nos dice que $a_n$ puede ser positiva o negativa. Entonces
  \begin{align*}
    sg(a_n) < 0 \ \lor sg(a_n) > 0
  \end{align*}
  Caso $sg(a_n) < 0 $
  \begin{align}
    sg(a_n) < 0 &\overset{\text{(1)}}{\implies} sg(a_{n+1}) < 0 \\
    sg(a_n) < 0 &\overset{\text{(1)}}{\implies} \sum_{i=1}^{n}a_i < 0
  \end{align}
  Por la Ec.(3) y la Ec.(4), tenemos que
  \begin{align*}
    a_{n+1}\sum_{i=1}^{n}a_i > 0
  \end{align*}

  Caso $sg(a_n) > 0 $
  \begin{align}
    sg(a_n) > 0 &\overset{\text{(1)}}{\implies} sg(a_{n+1}) > 0 \\
    sg(a_n) > 0 &\overset{\text{(1)}}{\implies} \sum_{i=1}^{n}a_i > 0
  \end{align}
  Por la Ec.(5) y la Ec.(6), tenemos que
  \begin{align*}
    a_{n+1}\sum_{i=1}^{n}a_i > 0
  \end{align*}
  \paragraph{¿En qué paso de la demostración se usa que $a_n > -1$ para todo $n \in \mathbb{N}$? }{En la misma parte 
  donde utilizamos la HI, pues la desigualdad de la HI sigue siendo valida solo si multiplicamos a ambos lados por un 
  número positivo. Ese número positivo esta en la Ec.(2).}
  \paragraph{¿Y que todos los términos de la sucesión $(a_n)_{n \in \mathbb{N}}$ tienen el mismo signo?}{En la parte
  donde utilizamos el calculo auxiliar.}

  \item Como $a > -1$ y $a \in \mathbb{R}$, cumple con las condiciones impuestas sobre $(a_n)_{n \in \mathbb{N}}$ en el
  punto (i). Tomamos $a_n = a$, $\forall n \in \mathbb{N}$. Como $P(n)$ es verdadera
  \begin{align*}
    \prod_{i=1}^{n} (1 + a) &\geq 1 + \sum_{i=1}^{n}a \\
    (1 + a)^n &\geq 1 + a \sum_{i=1}^{n}1 \\
    (1 + a)^n &\geq 1 + n a
  \end{align*}
\end{enumerate}