\begin{enunciado}{\ejercicio}
  Hallar una fórmula para el término general de las sucesiones definidas recursivamente a continuación
  y probar su validez.
  \begin{enumerate}[label=\roman*)]
    \item $a_1 = 1,\ a_2 = 2  \ytext a_{n+2} = n a_{n+1} + 2(n+1)a_n,\ \paratodo n \en \naturales $
    \item $a_1 = 1,\ a_2 = 4  \ytext a_{n+2} = 4 \sqrt{a_{n+1}} + a_n,\ \paratodo n \en \naturales $
    \item $a_1 = 1,\ a_2 = 3  \ytext 2 a_{n+2} = a_{n+1} + a_n + 3n + 5,\ \paratodo n \en \naturales $
    \item $a_1 = -3,\ a_2 = 6 \text{ y }
            a_{n+2} =
            \begin{cases}
              -a_{n+1} - 3,        & \text{si $n$ es impar} \\
              a_{n+1} + 2 a_n + 9, & \text{si $n$ es par}
            \end{cases}
            \  (n \en \naturales )$
  \end{enumerate}
\end{enunciado}

\begin{enumerate}[label=\roman*)]
  \item Sea $(a_n)_{n \en \naturales }$ definida por
        \setcounter{equation}{0}

        \begin{align}
          a_1     & = 1                                                  \\
          a_2     & = 2                                                  \\
          a_{n+2} & = n a_{n+1} + 2(n+1)a_n,\ \paratodo n \en \naturales
        \end{align}
        Veamos si hay algún patrón entre los términos de la sucesión para poder conjeturar una fórmula para su término
        n-ésimo
        \begin{align}
          a_3 & = 1 \cdot a_2 + 2(1 + 1)a_1 = 1 \cdot 2 + 2(1 + 1)1 = 6 = 3! \nonumber \\
          a_4 & = 2 a_3 + 2(2 + 1)a_2 = 2 \cdot 6 + 2(2 + 1)2 = 24 = 4! \nonumber      \\
          a_5 & = 3 a_4 + 2(3 + 1)a_3 = 3 \cdot 24 + 2(3 + 1)6 = 120 = 5! \nonumber    \\
              & \vdotswithin{=} \nonumber                                              \\
          a_n & = n!
        \end{align}

        Tenemos que probar que la sucesión dada por recurrencia satisface la sucesión que conjeturamos en la Ec.(4).
        Hagámoslo por inducción

        \begin{align*}
          P(n): a_n = n!, \ n \en \naturales
        \end{align*}

        \underline{Caso Base}, $n = 1$ y $n = 2$:
        \begin{align*}
           & P(1): a_1 = 1! = 1 \igual{(1)} 1 \entonces P(1):V \\
           & P(2): a_2 = 2! = 2 \igual{(2)} 2 \entonces P(2):V \\
        \end{align*}
        \underline{Paso inductivo}. Sea $n \en \naturales $:
        \begin{enumerate}
          \item[HI.] $P(n):V  \ytext P(n+1):V$, donde $P(n+1): a_{n+1} = (n+1)!$
          \item[TI.] $P(n+2): a_{n+2} = (n+2)!$
        \end{enumerate}
        Desarrollemos el lado izquierdo de la igualdad en la TI
        \begin{align*}
          a_{n+2} \overset{(3)} & {=} n a_{n+1} + 2(n+1)a_n \igual{HI} n(n+1)! + 2(n+1)n! = n(n+1)! + 2(n+1)! \\
          a_{n+2}               & = (n+1)!(n+2) = (n+2)!
          \entonces P(n+2):V
        \end{align*}
        Hemos probado el caso base y el paso inductivo. Concluimos que $P(n):V,\paratodo n \en \naturales $.

  \item Sea $(a_n)_{n \en \naturales }$ definida por
        \setcounter{equation}{0}
        \begin{align}
          a_1     & = 1                                                   \\
          a_2     & = 4                                                   \\
          a_{n+2} & = 4 \sqrt{a_{n+1}} + a_n,\ \paratodo n \en \naturales
        \end{align}

        Veamos si hay algún patrón entre los términos de la sucesión para poder conjeturar una fórmula para su término
        n-ésimo
        \begin{align}
          a_3 & = 4 \sqrt{a_2} + a_1 = 4 \sqrt{4} + 1 = 9 = 3^2 \nonumber   \\
          a_4 & = 4 \sqrt{a_3} + a_2 = 4 \sqrt{9} + 4 = 16 = 4^2 \nonumber  \\
          a_5 & = 4 \sqrt{a_4} + a_3 = 4 \sqrt{16} + 9 = 25 = 5^2 \nonumber \\
              & \vdotswithin{=} \nonumber                                   \\
          a_n & = n^2
        \end{align}
        Tenemos que probar que la sucesión dada por recurrencia satisface la sucesión que conjeturamos en la Ec.(4).
        Hagámoslo por inducción
        \begin{align*}
          P(n): a_n = n^2, \ n \en \naturales
        \end{align*}
        \underline{Caso Base}, $n = 1$ y $n = 2$:
        \begin{align*}
           & P(1): a_1 = 1^2 = 1 \igual{(1)} 1 \entonces P(1):V \\
           & P(2): a_2 = 2^2 = 4 \igual{(2)} 4 \entonces P(2):V \\
        \end{align*}

        \underline{Paso inductivo}. Sea $n \en \naturales $:

        \begin{enumerate}
          \item[HI.] $P(n):V  \ytext P(n+1):V$, donde $P(n+1): a_{n+1} = (n+1)^2$
          \item[TI.] $P(n+2): a_{n+2} = (n+2)^2$
        \end{enumerate}

        Desarrollemos el lado izquierdo de la igualdad en la TI

        \begin{align*}
          a_{n+2} \overset{(3)} & {=} 4 \sqrt{a_{n+1}} + a_n \igual{HI} 4 \sqrt{(n+1)^2} + n^2
          = 4(n+1) + n^2 = n^2 + 4n + 4 = (n+2)^2                                              \\
          a_{n+2}               & = (n+2)^2 \entonces P(n+2):V
        \end{align*}
        Hemos probado el caso base y el paso inductivo. Concluimos que $P(n):V,$ $\paratodo n \en \naturales $.

  \item Sea $(a_n)_{n \en \naturales }$ definida por
        \setcounter{equation}{0}
        \begin{align}
          a_1      & = 1                                                              \\
          a_2      & = 3                                                              \\
          2a_{n+2} & = a_{n+1} + a_n + 3n + 5,\ \paratodo n \en \naturales  \entonces
          a_{n+2} = \frac{a_{n+1} + a_n + 3n + 5}{2},\ \paratodo n \en \naturales
        \end{align}

        Veamos si hay algún patrón entre los términos de la sucesión para poder conjeturar una fórmula para su término
        n-ésimo

        \begin{align}
          a_3 & = \frac{a_{2} + a_1 + 3 \cdot 1 + 5}{2} = \frac{3 + 1 + 3 + 5}{2}  = 6 = 3 + 2 + 1 \nonumber           \\
          a_4 & = \frac{a_{3} + a_2 + 3 \cdot 2 + 5}{2} = \frac{6 + 3 + 6 + 5}{2}  = 10 = 4 + 3 + 2 + 1 \nonumber      \\
          a_5 & = \frac{a_{4} + a_3 + 3 \cdot 3 + 5}{2} = \frac{10 + 6 + 9 + 5}{2}  = 15 = 5 + 4 + 3 + 2 + 1 \nonumber \\
              & \vdotswithin{=} \nonumber                                                                              \\
          a_n & = \sumatoria{i=1}{n}  i \entonces a_n = \frac{n(n+1)}{2}
        \end{align}

        Tenemos que probar que la sucesión dada por recurrencia satisface la sucesión que conjeturamos en la Ec.(4).
        Hagámoslo por inducción:

        \begin{align*}
          P(n): a_n = \frac{n(n+1)}{2}, \ n \en \naturales
        \end{align*}

        \underline{Caso Base}, $n = 1$ y $n = 2$:

        \begin{align*}
           & P(1): a_1 = \frac{1(1+1)}{2} = 1 \igual{(1)} 1 \entonces P(1):V \\
           & P(2): a_2 = \frac{2(2+1)}{2} = 3 \igual{(2)} 3 \entonces P(2):V \\
        \end{align*}
        \underline{Paso inductivo}. Sea $n \en \naturales $:
        \begin{enumerate}
          \item[HI.] $P(n):V  \ytext P(n+1):V$, donde $P(n+1): a_{n+1} = \displaystyle \frac{(n+1)(n+2)}{2}$
          \item[TI.] $P(n+2): a_{n+2} = \displaystyle \frac{(n+2)(n+3)}{2}$
        \end{enumerate}
        Desarrollemos el lado izquierdo de la igualdad en la TI

        \begin{align*}
          a_{n+2} \overset{(3)} & {=} \frac{a_{n+1} + a_n + 3n + 5}{2} = \frac{1}{2}(a_{n+1} + a_n + 3n + 5)
          \igual{HI} \frac{1}{2} \left( \frac{(n+1)(n+2)}{2} + \frac{n(n+1)}{2} + 3n + 5 \right)             \\
          a_{n+2}               & = \frac{1}{2} \left( \frac{(n+1)(n+2) + n(n+1) + 6n + 10}{2}\right) =
          \frac{1}{2} \left( \frac{n^2 + 3n + 2 + n^2 + n + 6n + 10}{2}\right)                               \\
          a_{n+2}               & = \frac{1}{2} \left( \frac{2n^2 + 10n + 12}{2}\right) =
          \frac{1}{2} \left( \frac{2(n+2)(n+3)}{2}\right) = \frac{(n+2)(n+3)}{2}                             \\
          a_{n+2}               & = \frac{(n+2)(n+3)}{2} \entonces P(n+2):V
        \end{align*}

        Hemos probado el caso base y el paso inductivo. Concluimos que $P(n):V,$ $\paratodo n \en \naturales $.

  \item Sea $(a_n)_{n \en \naturales }$ definida por
        \setcounter{equation}{0}

        %===========================
        %===========================
        % Entiendo que queda bien en el pdf, pero
        % tratemos de evitar este código. 
        % Es muy difícil de leer y mantener, no aporta.

        %\newsavebox{\mycases}
        %\begin{align}
        %  a_1 = -3 \hphantom{\hspace{54.6mm}}                                                 \\
        %  a_2 = 6 \hphantom{\hspace{57.8mm}}                                                  \\
        %  \sbox{\mycases}{$\displaystyle a_{n+2} =
        %      \left\{\begin{array}{c}
        %               \hphantom{} \\
        %               \hphantom{}
        %             \end{array}
        %      \right.\kern-\nulldelimiterspace$}
        %  \raisebox{-.5\ht\mycases}[0pt][0pt]{\usebox{\mycases}}
        %  -a_{n+1} - 3, \hphantom{\hspace{7mm}} \quad \text{si $n$ es impar} \label{positive} \\
        %  a_{n+1} + 2 a_n + 9, \quad \text{si $n$ es par \hphantom{\hspace{3.1mm}}} \label{negative}
        %\end{align}
        %===========================
        %===========================

        \begin{align}
          a_1 & = -3 \\
          a_2 & = 6
        \end{align}
        Y como $a_{n+2}$ está partida:
        \begin{align}
          a_{n+2} & = -a_{n+1} - 3,       & \text{si $n$ es impar} \\
          a_{n+2} & = a_{n+1} + 2a_n + 9, & \text{si $n$ es par}
        \end{align}

        Veamos si hay algún patrón entre los términos de la sucesión para poder conjeturar una fórmula para su término
        n-ésimo

        \begin{align}
          a_3 & = -a_2 - 3 = -6 - 3  = -9 = (-1) \cdot 3 \cdot 3 \nonumber                       \\
          a_4 & = a_3 + 2a_2 + 9 = -9 + 2 \cdot 6 + 9  = 12 = (-1)^2 \cdot 3 \cdot 4 \nonumber   \\
          a_5 & = -a_4 - 3 = -12 - 3  = -15 = (-1) \cdot 3 \cdot 5  \nonumber                    \\
          a_6 & = a_5 + 2a_4 + 9 = -15 + 2 \cdot 12 + 9  = 18 = (-1)^2 \cdot 3 \cdot 6 \nonumber \\
              & \vdotswithin{=} \nonumber                                                        \\
          a_n & = (-1)^n 3n
        \end{align}
        Tenemos que probar que la sucesión
        dada por recurrencia satisface la sucesión que
        conjeturamos en la Ec.(5).
        Hagámoslo por inducción
        \begin{align*}
          P(n): a_n = (-1)^n 3n, \ n \en \naturales
        \end{align*}
        \underline{Caso Base}, $n = 1$ y $n = 2$:
        \begin{align*}
           & P(1): a_1 = (-1)^1 \cdot 3\cdot 1 = -3 \igual{(1)} 1 \entonces P(1):V \\
           & P(2): a_2 = (-1)^2 \cdot 3\cdot 2 = 6 \igual{(2)} 6 \entonces P(2):V  \\
        \end{align*}
        \underline{Paso inductivo}. Sea $n \en \naturales $:
        \begin{enumerate}
          \item[HI.] $P(n):V \ytext P(n+1):V$, donde $P(n+1): a_{n+1} = (-1)^{n+1}3(n+1)$
          \item[TI.] $P(n+2): a_{n+2} = (-1)^{n+2}3(n+2)$
        \end{enumerate}
        Desarrollemos el lado izquierdo de la igualdad en la TI. Como este depende de la paridad de $n$ tendremos que
        analizar dos casos, uno con $n$ impar y otro con $n$ par. \\
        Caso $n$ impar:
        \begin{align*}
          a_{n+2} \overset{(3)} & {=} -a_{n+1} - 3 \igual{HI} -(-1)^{n+1}3(n+1) - 3
          \taa{Aux}{1}{=} (-1)^{n+2}3(n+1) - 3
          \taa{Aux}{2}{=} -3(n+1) - 3                                                                          \\
                a_{n+2}               & = -3n - 3 - 3 = -3n -6 = -3(n+2) = (-1)3(n+2) \taa{Aux}{2}{=} (-1)^{n+2}3(n+2) \\
          a_{n+2}               & = (-1)^{n+2}3(n+2) \entonces P(n+2):V,\ n \text{ impar}
        \end{align*}
        Caso $n$ par:
        \begin{align*}
          a_{n+2} \overset{(4)} & {=} a_{n+1} + 2 a_n + 9 \igual{HI} (-1)^{n+1}3(n+1) + 2(-1)^n 3n + 9
          \taa{Aux}{4}{=} -3(n+1) + 2 \cdot 3n + 9                                                                    \\
          a_{n+2}               & = -3n - 3 + 6n + 9 = 3n + 6 = 3(n+2) = 1 \cdot 3(n+2)\igual{Aux.5} (-1)^{n+2}3(n+2) \\
          a_{n+2}               & = (-1)^{n+2}3(n+2) \entonces P(n+2):V,\ n \text{ par}
        \end{align*}
        Por lo tanto, $P(n+2):V$ para cualquier $n$ natural.
        Hemos probado el caso base y el paso inductivo. Concluimos que $P(n):V,$ $\paratodo n \en \naturales $.

        \paragraph{Auxiliar}{
          Sabemos que
          \begin{align*}
            (-1)^n =
            \begin{cases}
              \hphantom{-}1, \text{ si $n$ es par} \\
              -1, \text{ si $n$ es impar}
            \end{cases}
          \end{align*}
        }

        \paragraph{Auxiliar 2}{
          Tenemos $n$ impar
          \begin{align*}
            -(-1)^{n+1} = (-1)^1 \cdot (-1)^{n+1} = (-1)^{n+2}
          \end{align*}
        }

        \paragraph{Auxiliar 2}{
          Tenemos $n$ impar
          \begin{align*}
            (-1)^{n+2} = (-1)^n \cdot (-1)^2 = (-1)^n \cdot 1 \igual{Aux} -1
          \end{align*}
        }

        \paragraph{Auxiliar 3}{
          Tenemos $n$ impar
          \begin{align*}
            -1 = (-1) \cdot (-1)^2 \igual{Aux} (-1)^n \cdot (-1)^2 = (-1)^{n+2}
          \end{align*}
        }

        \paragraph{Auxiliar 4}{
          Tenemos $n$ par
          \begin{itemize}
            \item $(-1)^{n+1} = (-1)^n \cdot (-1)^1 \igual{Aux} 1 \cdot (-1) = -1$
            \item $(-1)^n \igual{Aux} 1$
          \end{itemize}
        }

        \paragraph{Auxiliar 3}{
          Tenemos $n$ par
          \begin{align*}
            1 = 1 \cdot (-1)^2 \igual{Aux} (-1)^n \cdot (-1)^2 = (-1)^{n+2}
          \end{align*}
        }
        
  \item Sea $(a_n)_{n \in \mathbb{N}_0}$ definida por
    \setcounter{equation}{0}
    \begin{align}
        a_0 &= 1 \\
        a_1 &= 4 \\
        a_{n+2} &= 4 a_{n+1} - 4 a_n,\ \forall n \in \mathbb{N}_0 
    \end{align}
    Veamos si hay algún patrón entre los términos de la sucesión para poder conjeturar una formula para su termino 
    n-ésimo
    \begin{align}
        a_2 &= 4a_1 - 4a_0 = 4 \cdot 4 - 4 \cdot 1 = 2^2 \cdot 2^2 - 2^2 = 2^2(2^2 - 1) = 2^2 \cdot 3 \nonumber \\
        a_3 &= 4a_2 - 4a_1 \overset{\text{Aux}}{=} 4\cdot 2^2 \cdot 3 - 4 \cdot 2 \cdot 2 =
        2^2\cdot 2^2 \cdot 3 - 2^2 \cdot 2 \cdot 2 = 2^3 (2 \cdot 3 - 2) = 2^3 \cdot 4 \nonumber \\
        a_4 &= 4a_3 - 4a_2 = 4 \cdot 2^3 \cdot 4 - 4 \cdot 2^2 \cdot 3 = 2^2 \cdot 2^3 \cdot 4 - 2^2 \cdot 2^2 \cdot 3 
        = 2^4 (2 \cdot 4 - 3) = 2^4 \cdot 5  \nonumber \\
        a_5 &= 4a_4 - 4a_3 = 4 \cdot 2^4 \cdot 5 - 4 \cdot 2^3 \cdot 4 = 2^2 \cdot 2^4 \cdot 5 - 2^2 \cdot 2^3 \cdot 4 
        = 2^5(2 \cdot 5 - 4) = 2^5 \cdot 6 \nonumber \\
        &\vdotswithin{=} \nonumber \\
        a_n &= 2^n (n+1)
    \end{align}
    Tenemos que probar que la sucesión dada por recurrencia satisface la sucesión que conjeturamos en la Ec.(4). 
    Hagamoslo por inducción
    \begin{align*}
        P(n): a_n = 2^n (n+1), \ n \in \mathbb{N}_0   
    \end{align*}
    \underline{Caso Base}, $n = 0$ y $n = 1$:
  \begin{align*}
    &P(0): a_0 = 2^0 (0+1) = 1 \overset{(1)}{=} 1 \implies P(0):V \\
        &P(1): a_1 = 2^1 (1+1) = 4 \overset{(2)}{=} 4 \implies P(1):V \\
  \end{align*}
  \underline{Paso inductivo}. Sea $n \in \mathbb{N}$:
  \begin{enumerate}
        \item[HI.] $P(n):V \text{ y } P(n+1):V$, donde $P(n+1): a_{n+1} = 2^{n+1}(n+2)$
        \item[TI.] $P(n+2): a_{n+2} = 2^{n+2}(n+3)$
    \end{enumerate}
  Desarrollemos el lado izquierdo de la igualdad en la TI
    \begin{align*}
        a_{n+2} \overset{(3)}&{=} 4 a_{n+1} - 4 a_n \overset{\text{HI}}{=} 
        4 \cdot 2^{n+1}(n+2) - 4 \cdot 2^n(n+1) = 2^2 \cdot 2^{n+1}(n+2) - 2^2 \cdot 2^n(n+1) \\
        a_{n+2} &= 2^{n+2}(2(n+2) - (n+1)) = 2^{n+2}(2n + 4 - n - 1) = 2^{n+2}(n+3) \\
        a_{n+2} &= 2^{n+2}(n+3) \implies P(n+2):V
    \end{align*}
    Hemos probado el caso base y el paso inductivo. Concluimos que $P(n):V,$ $\forall n \in \mathbb{N}$.

    \paragraph{Auxiliar}{
        Escribimos a $a_1 = 4$ como $a_1 = 2 \cdot 2$ para simplificar pasos en la obtención del termino general de 
        $a_n$. Si utilizamos $a_1 = 4$, llegamos a que $a_3 = 2^5$, despues a que $a_5 = 2^6 \cdot 3$ y no es tan fácil 
        ver el patrón de nuestro termino generale.
    }
\end{enumerate}

% Contribuciones
\begin{aportes}
  %% iconos : \github, \enstagram, \tiktok, \linkedin
  %\aporte{url}{nombre icono}
  \item \aporte{https://github.com/koopardo/}{Marcos Zea \github}
\end{aportes}


