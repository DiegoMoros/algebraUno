\ejercicio
\begin{enumerate}[label=\roman*)]
	\item $\llave{ l }{
			      \text{Primer caso } n = 1 \to \sumatoria{1}{1} (-1)^{i+1} i^2 = (-1)^2 \cdot 1 = 1 \Tilde\\
			      \text{Paso inductivo }
			      \llave{ l }{
				      n = k \to \sumatoria{1}{k} (-1)^{i+1} i^2 = (-1)^{k+1} \frac{k(k+1)}{2} \\
				      \entonces\\
				      n = k+1 \to \sumatoria{1}{k+1}  (-1)^{i+1} i^2 \eq?= (-1)^{(k+1)+1} \frac{(k+1)(k+2)}{2}\\
				      \to \sumatoria{1}{k+1}  (-1)^{i+1} i^2 =
				      \HI{
					      \sumatoria{1}{k} (-1)^{i+1} i^2
				      } +
				      \kmasuno{
					      (-1)^{k+2} (k+1)^2
				      } =\\
				      (-1)^{k+1} \frac{k(k+1)}{2} + (-1)^k (-1)^2 (k+1)^2 \flecha{acomodar}[fáctor cómun] (-1)^k (k+1)\left[ -\frac{k}{2} + (k+1) \right] =\\
				      (-1)^k (k+1)\frac{(k+2)}{2} \Tilde
			      }
		      }\\ \to
		      \boxed{\sumatoria{i=1}{n} (-1)^{i+1} i^2 = (-1)^{n+1} \frac{n(n+1)}{2}}
	      $

	\item
	      \hacer

	\item
	      \hacer

	\item
	      $\productoria{i=1}{n} \parentesis{ 1 + a^{2^{i-1}} } = \frac{1-a^{2^n}}{1-a}$\\
	      $\llave{l}{
		      \text{Primer caso } n = 1 \to
		      \productoria{i=1}{1} ( 1 + a^{2^{i-1}} ) =
		      1 + a^{2^0} = \magenta{1 + a} =
		      \frac{1-a^{2^1}}{1 - a} = \frac{(1-a)(1+a)}{1-a} =
		      \magenta{1 + a} \Tilde \\
		      \text{Paso inductivo } n = k \to
		      \productoria{i=1}{k} ( 1 + a^{2^{i-1}} ) =
		      \frac{1-a^{2^k}}{1-a} \entonces
		      n = k+1 \to  \productoria{i=1}{k+1} ( 1 + a^{2^{i-1}} ) \eq?=
		      \frac{1 - a^{2^{k+1}}}{1-a}\\
		      \llave{l}{
		      \productoria{i=1}{k+1} ( 1 + a^{2^{i-1}} ) =
		      \HI{
			      \productoria{i=1}{k} ( 1 + a^{2^k} )
		      } \cdot
		      \kmasuno{
			      1 + a^{2^{i-1}}
		      }  =
		      \frac{1-a^{2^k}}{1-a} \cdot 1 + a^{2^k}
		      \flecha{diferencia}[de cuadrados]
		      \frac{1 - ( a^{2^k})^2}{1-a} =\\
		      \frac{1 - a^{2 \cdot 2^k}}{1-a} = \frac{1 - a^{2^{k+1}}}{1-a}\Tilde
		      }
		      }
	      $



	\item
	      $\productoria{i=1}{n} \frac{n+i}{2i-3} = 2^n (1 -2n)$\\
	      En este ejercicio conviene abrir la productoria y acomodar los factores. Por inducción:\\
	      $\llave{l}{
			      p(n):\ \productoria{i=1}{n} \frac{n+i}{2i-3} = 2^n (1 -2n)  \\
			      \textit{Caso Base: } p(1) \text{ V?} \to \productoria{i=1}{1} \frac{1+i}{2i-3} = \frac{1+1}{2 \cdot 1 - 3} = 2^1 (1 - 2\cdot 1) = -2\\
			      \textit{Paso inductivo: Supongo } p(k) \text{ Verdadero } \flecha{quiero ver}[que] p(k+1) \text{ Verdadero para algún } k \en \naturales.\\
			      \textit{Hipótesis inductiva: Supongo } \productoria{i=1}{k} \frac{k+i}{2i-3} = 2^k (1 -2k), \qvq   \productoria{i=1}{k+1} \frac{k+1+i}{2i-3} = 2^{k+1} (1 -2(k+1))\\

			      \llave{l}{
				      \productoria{i=1}{k} \frac{k+i}{2i-3} = \frac{k+1}{2 \cdot 1 - 3} \cdot \frac{k+2}{2 \cdot 2 - 3} \cdot \frac{k+3}{2 \cdot 3 - 3} \cdots \frac{2k}{2 \cdot k - 3} = 2^k(1 - 2k)\\
				      \productoria{i=1}{k+1} \frac{k+1+i}{2i-3} = \frac{k+2}{2 \cdot 1 - 3} \cdot \frac{k+3}{2 \cdot 2 - 3} \cdots \frac{k+1 + (k-1)}{2(k-1) - 3} \cdot \frac{k+1 + k}{2k - 3}\cdot \frac{k+1 + (k+1)}{2(k + 1) - 3} \\
				      \flecha{Masajear: multiplico por $1 = \frac{\cyan{k+1}}{\cyan{k+1}}$}[corro los denominadores una fracción hacia $\leftarrow$]
				      \frac{\cyan{k+1}}{\cyan{k+1}} \cdot \frac{1}{2 \cdot 1 - 3} \cdot \frac{k+2}{2 \cdot 2 - 3} \cdot \frac{k + 3}{2 \cdot 3 - 3} \cdots \frac{2 k}{2k - 3} \cdot \frac{2k + 1}{2(k+1) - 3}  \cdot \frac{2k+2}{1} = \\
				      \flecha{acomodo para que}[aparezca la HI]
				      \HI{
					      \textstyle \frac{\cyan{k+1}}{2 \cdot 1 - 3} \cdot \frac{k+2}{2 \cdot 2 - 3} \cdot \frac{k + 3}{2 \cdot 3 - 3} \cdots \frac{2 k}{2k - 3} \cdot \frac{2k + 1}{2(k+1) - 3}
				      }  \cdot \frac{2k+2}{\cyan{k+1}} =\\
				      = 2^k (1 -2k) \cdot \frac{2k + 1}{2(k+1) - 3}  \cdot \frac{2k+2}{\cyan{k+1}} = 2^k \cancel{(1 - 2k)} \cdot \frac{2k + 1}{(-1)\cancel{(1 - 2k)}}  \cdot \frac{2\cancel{(k + 1)}}{\cancel{\cyan{k + 1}}} = 2^{k+1}(-1)(2k + 1)=\\
				      = 2^{k+1} \parentesis{1 - 2(k+1)} \Tilde

			      }
		      }$
	      Como $p(1)$ es verdadero y $p(k)$ es verdadero y $p(k+1)$ también lo es, por el principio de inducción $p(n)$ es verdadera $\paratodo n\en \naturales $
\end{enumerate}

