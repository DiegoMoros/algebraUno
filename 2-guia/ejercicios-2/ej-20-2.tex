\begin{enunciado}{\ejercicio}
  Hallar una fórmula para el término general de las sucesiones definidas recursivamente a continuación
  y probar su validez.
  \begin{enumerate}[label=\roman*)]
    \item $a_1 =   \frac{1}{2} \ytext a_{n+1} = \frac{1}{2}\left(1 - \sumatoria{i=1}{n} a_i\right),\
            \paratodo n \en \naturales $
    \item $a_0 = 5 \ytext
            a_n =
            \begin{cases}
              2a_{n-1},                     & \text{si $n$ es impar} \\
              \frac{1}{5}a^2_{\frac{n}{2}}, & \text{si $n$ es par}
            \end{cases}
            ,\, \paratodo n \en \naturales $
  \end{enumerate}
\end{enunciado}

\begin{enumerate}[label=\roman*)]
  \item Sea $(a_n)_{n \en \naturales _0}$ definida por
        \setcounter{equation}{0}
        \begin{align}
          a_1     & = \frac{1}{2}                                                                      \\
          a_{n+1} & = \frac{1}{2}\left(1 - \sumatoria{i=1}{n} a_i\right),\, \paratodo n \en \naturales
        \end{align}
        Veamos si hay algún patrón entre los términos de la sucesión para conjeturar una fórmula de su término
        n-ésimo
        \begin{align}
          a_2 & = \frac{1}{2}\left(1 - \sumatoria{i=1}{1} a_i\right) = \frac{1}{2}\left(1 - \frac{1}{2}\right) = \frac{1}{2}\left(1 - \frac{1}{2}\right)
          = \frac{1}{4} = \frac{1}{2^2} \nonumber                                                                                                        \\
          a_3 & = \frac{1}{2}\left(1 - \sumatoria{i=1}{2} a_i\right) = \frac{1}{2}\left(1 - \frac{1}{2} - \frac{1}{4}\right)
          = \frac{1}{8} = \frac{1}{2^3} \nonumber                                                                                                        \\
          a_4 & = \frac{1}{2}\left(1 - \sumatoria{i=1}{3} a_i\right)
          = \frac{1}{2}\left(1 - \frac{1}{2} - \frac{1}{4} - \frac{1}{8}\right)
          = \frac{1}{16} = \frac{1}{2^4} \nonumber                                                                                                       \\
              & \vdotswithin{=} \nonumber                                                                                                                \\
          a_n & = \frac{1}{2^n}
        \end{align}
        Tenemos que probar que la sucesión dada por recurrencia satisface la sucesión que conjeturamos en la Ec.(3).
        Hagamoslo por inducción
        \begin{align*}
          P(n): a_n = \frac{1}{2^n}, \, n \en \naturales
        \end{align*}
        \underline{Caso Base}, $n = 1$:
        \begin{align*}
          P(1): a_1 = \frac{1}{2^1} \igual{(1)} \frac{1}{2} \entonces P(1):V \\
        \end{align*}

        \underline{Paso inductivo}. Sean $n,k \en \naturales $, donde $1 \leq k \leq n$:
        \begin{enumerate}
          \item[HI.] $P(k):V,\, \paratodo k$
          \item[TI.] $P(n+1): a_{n+1} =   \frac{1}{2^{n+1}}$
        \end{enumerate}
        Desarrollemos el lado izquierdo de la igualdad en la TI
        \begin{align*}
          a_{n+1} \overset{(2)} & {=} \frac{1}{2}\left(1 - \sumatoria{i=1}{n} a_i\right) \overset{\text{HI}}{=}
          \frac{1}{2}\left(1 - \sumatoria{i=1}{n} \frac{1}{2^i}\right)  \overset{\text{Aux}}{=}
          \frac{1}{2}\left(\frac{1}{2^n}\right) = \frac{1}{2^{n+1}}                                             \\
          a_{n+1}               & = \frac{1}{2^{n+1}} \entonces P(n+1):V
        \end{align*}
        Hemos probado el caso base y el paso inductivo. Concluimos que $P(n):V,$ $\paratodo n \en \naturales $.

        \paragraph{Auxiliar}{Recordemos la suma geométrica para $q \en \reales - \set{0,1}$
          \begin{align}
            \sumatoria{i=0}{n} q^i = \frac{q^{n+1} - 1}{q-1}
          \end{align}
          Calculemos $1 - \sumatoria{i=1}{n} 1/2^i$
          \begin{align*}
             & 1 - \sumatoria{i=1}{n} \frac{1}{2^i} =
            1 - 1 + 1 - \sumatoria{i=1}{n} \frac{1}{2^i} =
            1 + 1 - \sumatoria{i=0}{n} \frac{1}{2^i}  =
            2 - \sumatoria{i=0}{n} \frac{1}{2^i}
            \overset{(1 = 1^i)}{=} 2 - \sumatoria{i=0}{n} \frac{1^i}{2^i}                   \\
             & = 2 - \sumatoria{i=0}{n} \left(\frac{1}{2}\right)^i
            \igual{(4)} 2 - \left(\frac{\left( \frac{1}{2}\right)^{n+1} - 1}{  \frac{1}{2} - 1}\right)
            = 2 - \left(\frac{\left( \frac{1}{2}\right)^{n+1} - 1}{  \frac{-1}{2}}\right)   \\
             & = 2 + \left(\frac{\left( \frac{1}{2}\right)^{n+1} - 1}{  \frac{1}{2}}\right)
            \igual{\red{!!}}
            \frac{1}{2^n}                                                                   \\
             & \entonces 1 - \sumatoria{i=1}{n} \frac{1}{2^i} = \frac{1}{2^n}
          \end{align*}
        }

  \item Sea $(a_n)_{n \en \naturales_0}$ definida por
        \setcounter{equation}{0}
        \begin{align}
          a_0 & = 5                                                      \\
          a_n & = 2a_{n-1},\, \text{si $n$ es impar}                     \\
          a_n & =   \frac{1}{5}a^2_{\frac{n}{2}},\, \text{si $n$ es par}
        \end{align}
        Veamos si hay algún patrón entre los términos de la sucesión para conjeturar una fórmula de su término
        n-ésimo
        \begin{align}
          a_1 & = 2 a_0 = 2 \cdot 5 = 10 = 2 \cdot 5 \nonumber                            \\
          a_2 & = \frac{1}{5} a_1^2 = \frac{1}{5} \cdot 10^2 = 20 = 2^2 \cdot 5 \nonumber \\
          a_3 & = 2 a_2 = 2 \cdot 20 = 40 = 2^3 \cdot 5 \nonumber                         \\
          a_4 & = \frac{1}{5} a_2^2 = \frac{1}{5} 20^2 = 80 = 2^4 \cdot 5 \nonumber       \\
              & \vdotswithin{=} \nonumber                                                 \\
          a_n & = 2^n \cdot 5
        \end{align}
        Tenemos que probar que la sucesión dada por recurrencia satisface la sucesión que conjeturamos en la Ec.(4).
        Hagamoslo por inducción
        \begin{align*}
          P(n): a_n = 2^n \cdot 5,\, n \en \naturales _0
        \end{align*}
        \underline{Caso Base}, $n = 0$:
        \begin{align*}
          P(0): a_0 = 2^0 \cdot 5 = 5 \igual{(1)} 5 \entonces P(0):V \\
        \end{align*}
        \underline{Paso inductivo}. Sean $n,k \en \naturales _0$, donde $0 \leq k \leq n$:
        \begin{enumerate}
          \item[HI.] $P(k):V,\, \paratodo k$
          \item[TI.] $P(n+1): a_{n+1} = 2^{n+1} \cdot 5$
        \end{enumerate}
        Desarrollemos el lado izquierdo de la igualdad en la TI. Como $a_{n}$ es sucesión partida, analizemos los casos
        donde $n$ es par y otro donde es impar. \\
        Caso $n$ par ($n \text{ par} \entonces n+1 \text{ impar}$):
        \begin{align*}
          a_{n+1} \overset{(2)} & {=} 2 a_{n + 1 -1} = 2 a_n \overset{\text{HI}}{=} 2 \cdot 2^n \cdot 5 = 2^{n+1} \cdot 5 \\
          a_{n+1}               & = 2^{n+1} \cdot 5 \entonces P(n+1):V,\, n \text{ par}
        \end{align*}
        Caso $n$ impar ($n \text{ impar} \entonces n+1 \text{ par}$):
        \begin{align*}
          a_{n+1} \overset{(3)} & {=} \frac{1}{5} a_{\frac{n+1}{2}}^2 \overset{\text{HI}}{=}
          \frac{1}{5} \cdot (2^{\frac{n+1}{2}}\cdot 5)^2 = \frac{1}{5} \cdot 2^{2\frac{n+1}{2}}\cdot 5^2
          =2^{n+1} \cdot 5                                                                   \\
          a_{n+1}               & = 2^{n+1} \cdot 5 \entonces P(n+1):V,\, n \text{ impar}
        \end{align*}
        Por lo tanto, $P(n+1):V$ para cualquier $n$ natural.
        Hemos probado el caso base y el paso inductivo. Concluimos que $P(n):V,$ $\paratodo n \en \naturales $.
\end{enumerate}
