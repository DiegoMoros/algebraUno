\begin{enunciado}{\ejExtra}
Probar para todo $n \en \naturales$ se cumple la siguiente desigualdad:
$$
	\frac{(2n)!}{(n!)^2} \leq (n+1)!
$$
\end{enunciado}

Se prueba usando el principio de inducción $\en \naturales$.\par

\textit{Proposición: }\par
$$
	p(n): \frac{(2n)!}{(n!)^2} \leq (n+1)!
$$

\textit{Caso base: } Evalúo en $n=1$.

$$
	p(\blue{1}):
	\frac{(2 \cdot \blue{1})!}{\blue{1}!^2} = 2 \leq (1+1)! \Tilde
$$
Se concluye que $p(1)$ es verdadera.

\textit{Paso inductivo: }
$$
	p(k): \HI{\frac{(2k)!}{(k!)^2} \leq (k+1)!}
$$ la supongo verdadera.\par
Quiero probar que:
$$
	p(k+1): \frac{(2(k + \magenta{1}))!}{(k + \magenta{1})!^2} \leq (k + \magenta{1} + 1)!
$$ también lo es.\par

$$
	\llave{c}{
		\frac{(2k+2)!}{(k+1)!^2} \leq  (k+2)!
		\Sii{abrir}[factorial]\\
		\frac{(2k + 2) \cdot (2k + 1) \cdot \blue{(2k)!}}{(k + 1)^2 \cdot \blue{(k!)^2}}
		\stacktext{HI}\leq
		\frac{
			\ob{
				(2k + 2) \cdot (2k + 1)
			}{ 4\cdot \cancel{(k+1)} (k + \frac{1}{2}) }
		} {(k + 1)^{\cancel{2}} } \blue{(k + 1)!} =
		\frac{
			\ob{ 4\cdot (k + \frac{1}{2}) }{\frac{4(k+\frac{1}{2})}{k+1}
				\stackrel{\llamada1}\leq \yellow{k+2}}
		}{(k + 1)} (k+1)!
		\leq \yellow{(k+2)} (k+1)! = (k+2)!\\
		\boxed{\frac{(2(k+1))!}{(k+1)!^2} \leq  (k+2)!}
	}
$$
$\llamada1$ se prueba fácil en 2 cuentas, queda como ejercicio para vos
\href{https://github.com/nad-garraz/algebraUno}{\Large\faIcon{hands-wash}}.
Es así que $p(1), p(k), \ytext p(k+1)$ resultaron verdaderas y por el principio de inducción
$p(n)$ también lo será $\paratodo n \en \naturales$.
