\begin{enunciado}{\ejExtra}
  Calcular el resto de dividir
  $$
    \sumatoria{k = 4}{134} (k! + k^3)
  $$
  por 7.
\end{enunciado}

Nos piden calcular el resto 7 de esa porquería:
$$
  \sumatoria{k = 4}{134} (k! + k^3)
  =
  \sumatoria{k = 4}{134} k! + \sumatoria{k = 4}{134} k^3
$$

Arranco por estudiar $\sumatoria{k = 4}{134} k^3$. Tabla de restos 7 de $k^3$:
$$
  \begin{array}{|c||c|c|c|c|c|c|c|}
    \hline
    r_7(k)   & 0 & 1 & 2 & 3 & 4 & 5 & 6 \\ \hline
    r_7(k^3) & 0 & 1 & 1 & 6 & 1 & 6 & 6 \\ \hline
  \end{array}
$$
Pensar que $\congruencia{6}{-1}{7}$ y eso nos ayuda a anular muchas cosas:
$$
  \sumatoria{k = 4}{134} k^3 = \ub{4^3 + 5^3 + 6^3 + 7^3 + 8^3 + \cdots + 130^3 + 131^3 + 132^3 + 134^3}{131 \text{ términos}}
$$
Todos esos términos tienen $r_7$ igual a $0, 1$ o $-1$. \underline{Sumando 7 términos consecutivos se obtiene como resultado 0}. Organizo los términos teniendo en cuenta que
$134 = 19 \cdot 7 + 1 $, es decir que tengo 19 sumas de 7 términos que dan 0 y me sobra el último término:
$$
  \scriptstyle
  \congruencia{r_7\Big(\sumatoria{k = 4}{134} k^3 \Big)}{
    \ub{\scriptstyle 4^3 + 5^3 + 6^3 + 7^3 + 8^3 + 9^3 + 10^3}{1^3 + (-1)^3 + (-1)^3 + 0^3 + 1^3 + 1^3 + (-1)^3=0} +
    \cdots +
    \ub{\scriptstyle126^3+ 127^3 + 128^3 + 129^3 + 130^3 + 131^3 + 132^3}{1^3 + (-1)^3 + (-1)^3 + 0^3 + 1^3 + 1^3 + (-1)^3=0} + \ub{\scriptstyle134^3}{=1}}{7}
  \congruente
  19 \cdot 0 + 1\ (7)
  \congruente 1 \ (7) \llamada1
$$

\bigskip

Ahora quiero ver $\sumatoria{k = 4}{134}k!$. Noto primero que cuando $k \geq 7$ el número $k!$ es un múltiplo de 7, es decir:
$$
  \congruencia{k!}{0}{7} \quad \text{ con } k \en \naturales_{\geq 7}
$$
Por lo tanto me quedaría con los primero 3 términos:
$$
  \sumatoria{k = 4}{134} k! = 4! + 5! + 6! + \ub{ 0 + \cdots + 0}{131 \text{ términos igual a 0}}
  \congruente
  3 + 1 + 6 \ (7)
  \congruente
  3 + 1 + 6 \ (7)
  \congruente
  3 \ (7) \llamada2
$$

Por último juntando los resultados de $\llamada1 \ytext \llamada2$:
$$
  r_7 \Big( \sumatoria{k = 4}{134} (k! + k^3) \Big)  = 4
$$

% Contribuciones
\begin{aportes}
  %% iconos : \github, \instagram, \tiktok, \linkedin
  %\aporte{url}{nombre icono}
  \item \aporte{https://github.com/nad-garraz}{Nad Garraz \github}
\end{aportes}
