\begin{enunciado}{\ejExtra}
  Pruebe que la siguiente igualdad es cierta para todo $n \en \naturales$.
  $$
    \frac{1}{9} \sumatoria{k = 1}{n}k \parentesis{\frac{2}{3}}^{k-1} =
    \parentesis{-\frac{n}{3} -1}\parentesis{\frac{2}{3}}^n + 1.
  $$
\end{enunciado}

Este es simple, muy directo. \textit{induccionemos}!

\bigskip

Quiero probar que:
$$
  p(n) \ : \
  \frac{1}{9} \sumatoria{k = 1}{n}k \parentesis{\frac{2}{3}}^{k-1}
  =
  \parentesis{-\frac{n}{3} -1}\parentesis{\frac{2}{3}}^n + 1
  \quad \paratodo n \en \naturales
$$

\textit{Caso base: ¿Es $p(1)$ verdadera? }
$$
  p(\blue{1}) \ : \
  \frac{1}{9} \sumatoria{k = 1}{\blue{1}}k \parentesis{\frac{2}{3}}^{k-1} =
  \frac{1}{9}
  \igual{\red{!}}
  \parentesis{-\frac{\blue{1}}{3} - 1}\parentesis{\frac{2}{3}}^{\blue{1}} + 1 =
  \frac{1}{9}
$$
Por lo que $p(1)$ resultó ser verdadera.

\textit{Paso inductivo, asumo que}:
$$
  p(\blue{h})\ : \
  \ub{\displaystyle
    \frac{1}{9} \sumatoria{k = 1}{\blue{h}}k \parentesis{\frac{2}{3}}^{k-1}
    =
    \parentesis{-\frac{\blue{h}}{3} - 1}\parentesis{\frac{2}{3}}^{\blue{h}} + 1
  }{
    \text{\purple{hipótesis inductiva}}
  }
  \quad \text{para algún } \blue{h} \en \naturales
$$
es verdadero, entonces quiero probar que:
$$
  p(\blue{h+1})\ : \
  \frac{1}{9} \sumatoria{k = 1}{\blue{h+1}}k \parentesis{\frac{2}{3}}^{k-1}
  =
  \parentesis{-\frac{\blue{h+1}}{3} - 1}\parentesis{\frac{2}{3}}^{\blue{h+1}} + 1
$$
también lo sea.

Parto de $p(h+1)$ trato de enchufar la \purple{hipótesis inductiva} en algún lado. Acomodo la sumatoria:
$$
  \begin{array}{c}
    \frac{1}{9} \sumatoria{k = 1}{\blue{h+1}}k \parentesis{\frac{2}{3}}^{k-1}
    =
    \frac{1}{9} \sumatoria{k = 1}{\blue{h}}k \parentesis{\frac{2}{3}}^{k-1} + \frac{1}{9}(\blue{h+1}) \parentesis{\frac{2}{3}}^{\blue{h}}
    \igual{\purple{HI}}
    \parentesis{-\frac{\blue{h}}{3} - 1}\parentesis{\frac{2}{3}}^{\blue{h}} + 1 + \frac{1}{9}(\blue{h+1}) \parentesis{\frac{2}{3}}^{\blue{h}}
    \igual{\red{!}} \\
    \igual{\red{!}}
    \parentesis{\frac{2}{3}}^{\blue{h}}\parentesis{ -\frac{2}{9}\blue{h} - \frac{8}{9}} + 1 =
    \parentesis{\frac{2}{3}}^{\blue{h}+1}\parentesis{ -\frac{1}{3}\blue{h} - \frac{4}{3}} + 1
    \igual{\red{!}}
    \parentesis{\frac{2}{3}}^{\blue{h}+1}\parentesis{ -\frac{\blue{h} + 1}{3} - 1} + 1
  \end{array}
$$
Probando así que $p(h+1)$ también es verdadera.

\bigskip

Dado que $p(1),\, p(h) \ytext p(h+1)$ son verdaderas por criterio de inducción también lo es $p(n) \paratodo n \en \naturales$.

\begin{aportes}
  \item \aporte{\dirRepo}{naD GarRaz \github}
\end{aportes}
