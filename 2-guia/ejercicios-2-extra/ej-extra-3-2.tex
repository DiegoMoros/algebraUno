%Macro local
\def\h{\magenta h}

\begin{enunciado}{\ejExtra}
  Probar que
  $$
    \sumatoria{k = 1}{n} (2k-1)^2 \geq \frac{(2n-1)^3}{6} \quad
    \paratodo n \en \naturales.
  $$
\end{enunciado}

Ejercicio de inducción. Voy a probar que la preposición
$
  p(n) :
  \sumatoria{k = 1}{n} (2k-1)^2 \geq \frac{(2n-1)^3}{6}
$ sea verdadera para todos los naturales.\par

\textit{Caso base: }
$p(\blue1): \sumatoria{k = 1}{\blue1} (2k-1)^2  =
  (2\cdot \blue1 - 1)^2 = 1
  \geq
  \frac{(2 \cdot \blue1 - 1)^3}{6} =
  \frac{1}{6}$.
por lo tanto $p(1)$ es verdadera \Tilde\par

\textit{Paso inductivo: }\par
Asumo $p(\h)$ verdadera, entonces quiero probar que $p(\magenta{h+1})$ también lo sea. En este caso:
$$
  p(\h) :
  \sumatoria{k = 1}{\h} (2k-1)^2 \geq \frac{(2\h-1)^3}{6}
$$ para algún $h \en \enteros$. Ésta será nuestra \textit{hipótesis inductiva}: HI.\par

Quiero probar que:

$$
  \sumatoria{k = 1}{\magenta{h + 1}} (2k-1)^2 \geq \frac{(2(\magenta{h+1})-1)^3}{6} = \frac{(2h+1)^3}{6},
$$

sea verdadera para algún $h \en \enteros$.\par
$
  \sumatoria{k = 1}{\magenta{h+1}} (2k-1)^2 =
  \sumatoria{k = 1}{\magenta h} (2k-1)^2 + (2(\magenta{h+1}) - 1) =
  \sumatoria{k = 1}{\magenta h} (2k-1)^2 + (2\h +1)^2
$\par

\textit{Nota innecesaria pero que quizás aporta: }\par
Lo que acabamos de hacer recién nos deja la HI regalada.
Pero atento que esto \underline{solo} suele funcionar cuando \underline{no}  tenemos a la '$n$'
en el término principal de la sumatoria. Después de hacerte éste, mirá el \refEjExtra{ejExtra:2}\par
\textit{Fin nota innecesaria pero que quizás aporta}.\par

$
  \sumatoria{k = 1}{\magenta{h+1}} (2k-1)^2 =
  \sumatoria{k = 1}{\h} (2k-1)^2 + (2\h +1)^2 \stackrel{HI} \geq
  \ub{
    \frac{(2\h-1)^3}{6} + (2\h + 1)^2
    \geq
    \frac{(2h+1)^3}{6}
  }
  {
    \mathclap{\text{Si ocurre esto, $p(h+1)$ será verdadera}}
  }
  \Sii{$\times 6$}
  (2h-1)^3 + 6(2h + 1)^2
  \geq
  (2h+1)^3\\
  \Sii{distribuyo}[a morir]
  \cancel{8 h^3} + \cancel{12 h^2} + 30 h + 5
  \geq
  \cancel{8 h^3} + \cancel{12 h^2} + 6 h + 1
  \sii
  24h + 4
  \stackrel{\Tilde}\geq
  0\quad \paratodo h \en \naturales\\
  \entonces
  \sumatoria{k = 1}{h+1}(2k-1)^2 \geq \frac{(2h+1)-1)^3}{6},
$
concluyéndose que $p(h+1)$ también es verdadera.\medskip

Como tanto $p(1), p(h) \ytext p(h+1)$ resultaron verdaderas, por el principio de inducción se tiene que
$p(n)$ es verdadera para todo $n \en \naturales$.



