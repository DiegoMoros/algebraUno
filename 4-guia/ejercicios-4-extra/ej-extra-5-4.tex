\ejExtra

Estudiar los valores parar \textbf{todos} los $a \en \enteros$ de $(a^3 + 1 : a^2 - a + 1)$.\\

\separadorCorto

Primero hay que notar que el lado $a^2-a+1$ es siempre impar ya que:\\
$\llaves{l}{
		(2k-1)^2 - (2k -1) + 1 \conga2 (-1)^2 -1 + 1 \conga2 1\\
		(2k)^2 - (2k) +1 \conga2 (0)^2 - 0 + 1\conga2 1.
	}$ Por lo tanto 2 no puede ser un divisor de ambas expresiones  y si $2 \noDivide A \entonces 2 \cdot k \noDivide A$ tampoco.\\
Se ve fácil contrarecíproco: $\ub{2k}{par} \divideA A \entonces 2 \divideA A$. Porque existe un $k$ tal que
$2 \cdot c \cdot k = A \entonces 2 \cdot (c\cdot k) = A.$\\
Ahora cuentas para simplificar la expresión y encontrar número del lado derecho.\\
$
	\llave{l}{
		d \divideA  a^3 + 1\\
		d \divideA  a^2 -a +1
	}
	\to d \divideA 30
	\to\divsetP{d}{1,2,3,5,6,10,15,30}
	\flecha{por lo de antes}[no hay divisores pares] \divsetP{d}{1,3,5,15}\\
	\flecha{hacer tabla de restos}[empezar por los números chicos]
	\llaves{ccc}{
		r_3(a^3 + 1) = 0 & \text{ si }& \congruencia{a}{2}{3}\\
		\y& &\\
		r_3(a^2 -a + 1) = 0 & \text{ si } & \congruencia{a}{2}{3}
	}
	\to
	\llaves{cc}{
		r_5(a^3 + 1) \distinto 0 & \paratodo a \en \enteros\\
	}
$.\\
Luego si  $5 \noDivide (a^3 + 1 : a^2 - a + 1) \entonces \ub{15}{5\cdot 3} \noDivide (a^3 + 1 : a^2 - a + 1)
	\flecha{se achica el}[conjunto de divisores] \divsetP{d}{1,3}\\
	d = \llave{ccc}{
		3 & \text{ si } & \congruencia{a}{2}{3}\\
		1 & \text{ si } & \congruencia{a}{1 \o 2}{3}\\
	}
$

