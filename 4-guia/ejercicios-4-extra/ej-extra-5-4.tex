\begin{enunciado}{\ejExtra}
  Estudiar los valores parar \textbf{todos} los $a \en \enteros$ de $(a^3 + 31 : a^2 - a + 1)$.\\
\end{enunciado}

Simplifico la expresión $(a^3 + 31 : a^2 - a + 1)$ con el querido algoritmo de Euclides:
$$
  \divPol{X^3 + 31}{X^2 - X+ 1}
$$

Por lo tanto el mcd $d = (a^3 + 31 : a^2 - a + 1) = (a^2-a+1: 30)$, es decir que:
$$
  d \divideA 30 \entonces d \en \set{1, 2, 3, 5, 6, 10, 15, 30}
$$

Muchos divisores. Se pueden elimiar unos cuantos notando que $a^2 - a + 1$ es una expresión siempre impar. Una forma de mostrar esto:

$$
  a^2 - a + 1 \text{ es impar }
  \sii
  \congruencia{a^2 - a + 1}{1}{2}
  \Sii{\red{!}}
  \congruencia{a \cdot (a - 1)}{0}{2}
$$

La última expresión $a \cdot (a-1)$ es siempre \magenta{par}, dado que es un número multiplicado por su consecutivo.
{\tiny Otra forma de mostrar la paridad sería reemplazando
por $2k$ y luego por $2k+1$ y ver que los resultados son siempre impares. Hacé lo que más te guste \faIcon[regular]{smile}!}\par\medskip

Dado que esa expresión es impar podemos reducir el conjunto de divisores a:

$$
  d \divideA 30 \ytext \congruencia{d}{1}{2} \entonces d \en \set{1,3,5,15}.
$$

\textit{Tabla de restos}: Siempre empezando por el menor valor

$$
  \begin{array}{|r|ccc|}
    \hline
    r_3(a)           & 0 & 1 & 2           \\ \hline
    r_3(a^2 - a + 1) & 1 & 1 & \magenta{0} \\ \hline
  \end{array}
$$

Obtenemos que 3 \textit{es un potencial mcd} cuando $r_3(a) = 2$ o dicho de otro modo $\congruencia{a}{2}{3}$.

$$
  \begin{array}{|r|ccccc|}
    \hline
    r_5(a)           & 0 & 1 & 2 & 3 & 4 \\ \hline
    r_5(a^2 - a + 1) & 1 & 1 & 3 & 2 & 3 \\ \hline
  \end{array}
$$
Obtenemos que 5 \textit{no es un potencial mcd}, por lo que 15 tampoco será un divisor de la expresión  $a^2 - a + 1$.

Con la información obtenida se puede concluir que:
$$
d = 
\llave{l}{
        3 \text{ si } \congruencia{a}{2}{3}\\ 
        1 \text{ si } \noCongruencia{a}{2}{3}
}
$$

% Contribuciones
\begin{aportes}
  %% iconos : \github, \instagram, \tiktok, \linkedin
  %\aporte{url}{nombre icono}
  \item \aporte{https://github.com/nad-garraz}{Nad Garraz \github}
  \item \aporte{https://github.com/maxiitietze}{Maxi T. \github}

\end{aportes}
