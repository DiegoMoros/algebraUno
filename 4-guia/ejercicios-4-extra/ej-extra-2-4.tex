\ejExtra

Hallar el menor $n \en \naturales$ tal que:

\begin{enumerate}[label=\roman*)]
	\item $(n:2528) = 316 $
	\item $n$ tiene exáctamente 48 divisores positivos
	\item $27 \noDivide n$
\end{enumerate}

\separadorCorto

$
	\llave{l}{
		\flecha{factorizo}[2528] 2528 = 2^5 \cdot 79 \Tilde\\
		\flecha{factorizo}[316] 316 = 2^2 \cdot 79 \Tilde\\
		\flecha{reescribo}[condición] (n:2^5 \cdot 79) = 2^2 \cdot 79
	}\\
	\flecha{quiero}[encontrar] n = 2^{\alpha_2} \cdot 3^{\alpha_3} \cdot 5^{\alpha_5} \cdot 7^{\alpha_7} \cdots 79^{\alpha_79} \cdots .\\
	\flecha{como} (n:2^5 \cdot 79) = 2^2 \cdot 79
	\flecha{tengo}[que]
	\llave{ll}{
		\alpha_2 = 2, & \text{dado que $2^2 \cdot 79 \divideA n$. \blue{Recordar que busco el menor $n$!}.}\\
		\alpha_{79} \geq 1, & \text{Al igual que antes.} \\
		\flecha{notar}[que] \alpha_3 < 3 &  \text{ si no } 3^3 = 27 \divideA n
	}\\
	\flecha{la estrategia sigue con}[el primo más chico que haya]
	\llave{l}{
		48 = \ub{(\alpha_2 + 1)}{2 + 1} \cdot (\alpha_3 + 1) \cdots\\
		48 = 3 \cdot (\alpha_3 + 1) \cdots\\
		16 = (\alpha_3 + 1) \cdot (\alpha_5 + 1) \cdot (\alpha_7 + 1) \cdots \ub{(\alpha_{79} + 1)}{=2\text{ quiero el menor}} \cdots \\
		8 = (\alpha_3 + 1) \cdot (\alpha_5 + 1) \cdot (\alpha_7 + 1) \cdots \\
		8 = \ub{(\alpha_3 + 1)}{=2} \cdot \ub{(\alpha_5 + 1)}{=2} \cdot \ub{(\alpha_7 + 1)}{=2} \cdot 1 \cdots 1 \\
	}
$
El $n$ que cumple lo pedido sería $n = 2^2 \cdot 3^1 \cdot 5^1 \cdot 7^1 \cdot 79^1$
