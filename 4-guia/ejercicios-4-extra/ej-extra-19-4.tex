\begin{enunciado}{\ejExtra}
	Probar que $6^{2n} - 35n -1$ es divisible por 245 para todo $n \en \naturales$.
\end{enunciado}

Noto que:
$$
	245 = 5^2 \cdot 7
$$

Los que nos piden se puede escribir como:
$$
	\congruencia{6^{2n} - 35n -1}{0}{245}
$$

\textit{Inducción:}

Quiero probar que:
$$
	p(n) : \congruencia{6^{2n} - 35n -1}{0}{245} \quad \paratodo n \en \naturales
$$

\medskip

\textit{Caso base:}
$$
	p(\blue{1}) :
	\congruencia{
		6^{2 \cdot \blue{1}} - 35 \cdot {\blue{1}} -1 = 0 }{0}{245}
$$
Por lo que $p(1)$ resultó ser verdadera.

\medskip

\textit{Paso inductivo:}
Asumo que
$$
	p(\blue{k}) : \ub{\congruencia{ 6^{2 \cdot \blue{k}} - 35 \cdot {\blue{k}} -1}{0}{245}}{\purple{\text{hipótesis inductiva}}}
$$
es verdadera para algún $k \en \naturales$. Entonces quiero probar que:
$$
	p(\blue{k+1}) : \congruencia{ 6^{2 \cdot (\blue{k+1})} - 35 \cdot ({\blue{k+1}}) -1}{0}{245}
$$
Partiendo de esto último:
$$
	6^{2 \cdot (\blue{k+1})} - 35 \cdot ({\blue{k+1}}) -1
	=
	36 \cdot 6^{2k} - 35k -35 -1
	=
	36 \cdot (6^{2k} - 1) - 35k \congruente 0 \ (245) \llamada1
$$
Acomodo la \purple{hipótesis inductiva}:
$$
	\congruencia{ 6^{2 \blue{k}} - 35 \cdot {\blue{k}} -1}{0}{245}
	\sisolosi
	\congruencia{ 6^{2 k} -1 }{35k}{245} \llamada2
$$
Uso $\llamada2$ eso en $\llamada1$
$$
	36 \cdot (\purple{35k}) - 35k = 35^2k = 5^2 \cdot 7^2= 245 \cdot 7 \congruente 0 \ (245) \llamada1
$$
Es así que $p(k+1)$ también es verdadera.

\bigskip

Dado que $p(1),\, p(k),\, p(k+1)$ resultaron verdaderas por principio de inducción $p(n)$ también lo es $\paratodo n \en \naturales$

\begin{aportes}
	\item \aporte{\dirRepo}{naD GarRaz \github}
\end{aportes}
