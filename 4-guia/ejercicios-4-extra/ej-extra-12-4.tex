\begin{enunciado}{\ejExtra}
  Determinar para cada par $(a,b) \en \enteros^2$ tal que $(a:b) = 7 $ el valor de
  $$
    (a^2b^4 : 7^5(-a + b)).
  $$
\end{enunciado}

\textit{Coprimizar: }
$$
  d = (a^2b^4 : 7^5(-a + b)) \Sii{$a = 7A$}[$b = 7B$] 7^6 \cdot (A^2B^4 : B - A ) \sii d = 7^6\cdot D
$$

$$
  \llave{l}{
    D \divideA A^2 B^4\\
    D \divideA B - A \Sii{def} \congruencia{B}{A}{D}\llamada1
  }
$$

$$
  \llave{l}{
    D \divideA A^2 B^4 \Sii{$\llamada1$} \boxed{\congruencia{B^6}{0}{D}}\\
    \text{ y también }\\
    D \divideA A^2 B^4 \Sii{$\llamada1$} \boxed{\congruencia{A^6}{0}{D}}
  }
$$

El resultado dice que $D \divideA A^6$ y que $D \divideA B^6$ lo cual está \underline{complicado} porque $A$ y $B$
son coprimos, por lo tanto $A^6$ y $B^6$ también y $(A^6:B^6) = 1 = D$ .\par\medskip

Creo que hay que justificar con algo más, pero no sé, con algo de primos? Bueh, más o menos algo así:\par
Si $D \divideA A^6$ entonces la \textit{descomposición en primos} de $D = p_1^{\purple{i_d}} \cdots p_n^{\purple{j_d}}$
tiene que tener solo factores de la \textit{descomposición en primos} de
$A = p_1^{\blue{i}} \cdots p_n^{\blue{j}} \cdot p_{n+1}^{\blue{k}} \cdots p_m^{\blue{l}}$ con los exponentes de los factores de $D\, (\purple{i_d}, \purple{j_d},\dots),$  menores o iguales
a los exponentes de $A\, (\blue{i}, \blue{j}, \dots)$ de manera que al dividir:
$$
  \frac{A}{D}
  =
  \frac{
  p_1^{\blue i} \cdots p_n^{\blue j} \cdot p_{n+1}^{\blue k}  \cdots  p_m^{\blue l}
  }
  {
  p_1^{\purple{i_d}} \cdots p_n^{\purple{j_d}} \cdot p_{n+1}^{\purple{k_d}}  \cdots  p_m^{\purple{l_d}}
  }
  =
  \frac{
  p_1^{     \ob{ \blue i - \purple{i_d}} {0 \menorIgual{\red{!} } }} \cdots
  p_n^{     \ob{ \blue j - \purple{j_d}} {0 \menorIgual{\red{!} } }} \cdots
  p_{n+1}^{ \ob{ \blue k - \purple{k_d}} {0 \menorIgual{\red{!} } }} \cdots
  p_{m}^{   \ob{ \blue l - \purple{l_d}} {0 \menorIgual{\red{!} } }}
  }
  {
  \red 1
  },
$$

es decir que se cancele todo de manera que quede un \red{1} en el denominador.
Eso es que $D \divideA A$ ni más ni menos.\par

Y sí, \textit{muy rico todo}, pero esa cantinela es la misma para $D \divideA B$, \red{pero} la
\textit{descomposición en primos} de $B$ tiene los $p_i$ \textbf{distintos} a los de $A$, porque
\red{¡}$(A:B) = 1$\red{!} y ahí llegamos al \underline{absurdo}. $D$ no puede dividir a ambos
a la vez a menos que $D = 1$ \Tilde.

$$
  D = 1 \entonces \boxed{d = 7^6}, \text{para cada } (a,b) \en \enteros^2 \Big / (a:b) = 7
$$
