\begin{enunciado}{\ejExtra}
  Calcular, para cada $n \en \naturales$, el resto de dividir por 18 a
  $$
    6 \cdot 35^n + 73^{3021} + \sumatoria{k=1}{n}3^k \cdot k!
  $$
\end{enunciado}

Simplifiquemos esa expresión espantosa calculando el $r_{18}$ y aplicando las propiedades:
$$
  \begin{array}{rcl}
    r_{18} (6 \cdot 35^n + 73^{3021} + \sumatoria{k=1}{n}3^k \cdot k!   )
                                                         & \igual{\red{!}}         &
    r_{18} (6 \cdot (-1)^n + 1^{3021} + r_{18}(\sumatoria{k=1}{n}3^k \cdot k!))      \\
                                                         & \igual{$\llamada1$}     &
    \llave{ll}{
    r_{18} (7 + r_{18}(\sumatoria{k=1}{n}3^k \cdot k!))  & \text{ si $n$ es par}     \\
    r_{18} (-5 + r_{18}(\sumatoria{k=1}{n}3^k \cdot k!)) & \text{ si $n$ es impar}
    }
  \end{array}
$$

La para está ahora en calcular: $r_{18}(\sumatoria{k=1}{n} 3^k \cdot k!)$

Dado que tiene un 3 ahí dando vueltas y que la $k!$ en algún momento tendrá el factor $6 = 3! = 2 \cdot 3$,
es esperable que el término general de la sumatoria sea un múltiplo de 18.\par

Acomodo la expresión:

$$
  r_{18}( \sumatoria{k=1}{n} 3^k \cdot k!) =
  r_{18} (3 + \sumatoria{k=\magenta{2}}{n}3^k \cdot k!)
  \igual{$\llamada2$}
  3 + r_{18}(\sumatoria{k=2}{n} 3^k \cdot k!)
$$

A ojo se puede ver que $r_{18}(\sumatoria{k=2}{n} 3^k \cdot k!) = 0 \paratodo n \en \naturales_{\geq 2}$
Pero como no sabemos si el que nos corrige está de mal humor probemos eso por inducción:\par\medskip
Quiero probar que:
$$
  p(n) : r_{18}(\sumatoria{k=2}{n} 3^k \cdot k!) = 0 \paratodo n \en \naturales_{\geq 2}
$$

\textit{Caso base:}
$$
  p(2) : r_{18}(\sumatoria{k=2}{2} 3^k \cdot k!) = r_{18} (3^2 \cdot 2) = 0
$$
Por lo que el caso $p(2)$ es verdadero.

\textit{Paso inductivo:}
Asumo que para algún $k \geq 2$
$$
  p(\blue{h}) : \ub{ r_{18}(\sumatoria{k=2}{\blue{h}} 3^k \cdot k!) = 0 }{\text{\purple{hipótesis inductiva}}}
$$

es verdadero. Y quiero probar que:

$$
  p(\blue{h+1}) : r_{18}(\sumatoria{k=2}{\blue{h+1}} 3^k \cdot k!) = 0
$$
también lo sea.\par

Partiendo de $p(h+1)$
$$
  \begin{array}{rcl}
    r_{18}(\sumatoria{k=2}{h+1} 3^k \cdot k!) & =                   &
    r_{18}\bigl(\sumatoria{k=2}{\magenta{h}} 3^k \cdot k! + 3^{h+1} \cdot (h+1)!\bigr) \\
                                              & \igual{\purple{HI}} &
    r_{18}\bigl(3^{h+1} \cdot (h+1)!\bigr)                                             \\
                                              & \igual{\red{!}}     &
    r_{18}\bigl(3 \cdot \magenta{6} \cdot 3^h \cdot \frac{(h+1)!}{\magenta{3!}}\bigr)  \\
                                              & =                   & 0
  \end{array}
$$
Ahí en el \red{!} me las arreglé para que aparezca el 18 que hace que el resto de 0. Debe haber otras formas
de hacerlo, tenés licencia para dibujar.\par

Como $p(2), p(h) \ytext p(h+1)$ resultaron verdaderas, por  criterio de inducción $p(n)$ también lo es para todo
$n \en \naturales_{\geq 2}$

Volviendo a $\llamada2$:
$$
  r_{18}( \sumatoria{k=1}{n} 3^k \cdot k!) = 3
$$

por lo tanto en $\llamada1$:

$$
  \begin{array}{rcl}
    r_{18} (6 \cdot 35^n + 73^{3021} + \sumatoria{k=1}{n}3^k \cdot k!)
                                        & =                       & \llave{ll}{
    r_{18} (6 + 1 + 3)  = 10                 & \text{ si $n$ es par}                 \\
    r_{18} (-6 + 1 +  3) \igual{\red{!}} 16 & \text{ si $n$ es impar}
    }
  \end{array}
$$

% Contribuciones
\begin{aportes}
  \item \aporte{https://github.com/nad-garraz}{Nad Garraz \github}
  \item \aporte{https://github.com/daniTadd}{Dani Tadd \github}
\end{aportes}
