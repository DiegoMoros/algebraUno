\begin{enunciado}{\ejExtra}
  Sea $(a_n)_{n \en \naturales_0}$ con
  $\llave{l}{
      a_0 = 1 \\
      a_1 = 3 \\
      a_n = a_{n-1} - a_{n-2}\ \paratodo n \geq 2
    }$\par

  \begin{multicols}{2}
    \begin{enumerate}[label=\alph*)]
      \item Probar que $a_{n+6} = a_n$

      \item Calcular $\sumatoria{k=0}{255} a_k$
    \end{enumerate}
  \end{multicols}
\end{enunciado}

\begin{enumerate}[label=(\alph*)]
  \item
        Por inducción:
        $$
          p(n):   a_{n+6} = a_n \paratodo n \geq \naturales_0
        $$
        Primero notar que:                                                                                                          \par
        $$
          \llaves{l}{
            a_0 = 1\\
            a_1 = 3\\
            a_2 \igual{def}  2\llamada1\\
            a_3 \igual{def}  -1\\
            a_4 \igual{def}  -3\\
            a_5 \igual{def}  -2
          } \to
          \llaves{l}{
            a_6 \igual{def}  1\\
            a_7 \igual{def}  3\\
            a_8 \igual{def} 2\llamada1\\
            a_9 \igual{def}  -1\\
            a_{10} \igual{def}  -3\\
            a_{11} \igual{def}  -2
          }
        $$

        \text{Se ve que tiene un período de 6 elementos.}\par

        \textit{Caso Base:}
        $
          p(2): a_8 \igual{?}[$\llamada1$]   a_2 \Tilde
        $\par

        \textit{Paso inductivo: }
        Asumo que
        $$
          p(\blue{k}):   \ub{a_{\blue{k}+6} =
            a_{\blue{k}} \text{ para algún } \blue{k} \geq
            \naturales_{\geq2}}{\text{\purple{hipótesis inductiva}}}
        $$
        entonces quiero probar que,
        $$
          p(\blue{k+1}): a_{\blue{k+1} + 6} = a_{\blue{k + 1}}
        $$
        también sea verdadera.\par
        Parto desde $p(\blue{k+1})$
        $$
          \green{a_{k+7}} \igual{def}
          a_{k+6} - a_{k+5} \igual{\purple{HI}}
          \purple{a_k} - a_{k+5}
          \igual{def}[\red{!}]
          a_k - ( a_k + a_{k+4}) = - a_{k+4}
          \entonces
          a_{k+7} = - a_{k+4} \Tilde
        $$
        Ahora uso la definición de manera sucesiva:
        $$
          \green{a_{k+7}} = -a_{k+4} \igual{def}
          -(a_{k+3} - a_{k+2}) \igual{def}
          - ( a_{k+2} - \green{a_{k+1}} - a_{k+2} ) =
          \green{a_{k+1}}
          \entonces
          a_{k+7} = a_{k+1}\Tilde
        $$\par
        Como $p(2),\, p(3),\, p(4),\, p(5),\, p(k) \ytext p(k+1)$ son verdaderas
        por el principio de inducción $p(n)$ también es verdadera $\paratodo n \en \naturales_{\geq 2}$

  \item
        $\sumatoria{k=0}{255} a_k =
          \underbrace{\textstyle a_0 + a_1 + a_2 + a_3 + a_4 + a_5}_{= 0} +
          \underbrace{\textstyle a_6 + a_7 + a_8 + a_9 + a_{10} + a_{11} }_{=0} +
          \cdots +
          a_{252} + a_{253} + a_{254} + a_{255}
        $\par
        En la sumatoria hay \red{256 términos}. $256 = 42 \cdot 6 + 4$ por lo tanto van a haber 42 bloques que dan 0 y sobreviven los últimos 4 términos.
        $\sumatoria{k=0}{255} a_k = \underbrace{\textstyle 0 + 0 + \dots + 0}_{42 \text{ ceros}} + a_{252} + a_{253} + a_{254} + a_{255} =
          \cancel{a_{252}} + a_{253} + a_{254} + \cancel{a_{255}} = a_{253} + a_{254} = 5\\
        $ Donde usé que: $a_n =
          \llave{rcl}{
            1  & \text{si} & n\mod6 = 0 \\
            3  & \text{si} & n\mod6 = 1 \\
            2  & \text{si} & n\mod6 = 2 \\
            -1 & \text{si} & n\mod6 = 3 \\
            -3 & \text{si} & n\mod6 = 4 \\
            -2 & \text{si} & n\mod6 = 5
          }\longrightarrow
        $
        \boxed{\sumatoria{k=0}{255} a_k = 5} \Tilde
\end{enumerate}

\begin{aportes}
  \item \aporte{\dirRepo}{naD GarRaz \github}
\end{aportes}
