\begin{enunciado}{\ejExtra}
  Sea $n \en \naturales$ con $n \congruente 2 \ (\text{mód } 6)$. Hallar los posibles valores de $(2n^4 + 6n^2 + 10: n^3 + 2n)$ y dar
  un ejemplo para cada uno.
\end{enunciado}

Bautizo al MCD:
$$
  d = (2n^4 + 6n^2 + 10: n^3 + 2n)
$$
Según Euclides:
$$
  \polyset{vars=n}
  \divPol{2n^4+6n^2+10}{n^3+2n}
$$
Por lo que:
$$
  d = (2n^4 + 6n^2 + 10: n^3 + 2n) = (n^3 + 2n : 2n^2 + 10)
$$

$$
  \polyset{vars=n}
  \divPol{n^3 + 2n}{2n^2 + 10}
$$
Por lo que:
$$
  d = (n^3 + 2n : 2n^2 + 10) = (2n^2 + 10 : -3n)
$$

$$
  \polyset{vars=n}
  \divPol{2n^2 + 10}{-3n}
$$
Por lo que:
$$
  d = (2n^2 + 10 : -3n) = (3n : 10)
$$

Si $d = (3n : 10)$ entonces los \textit{potenciales} posibles valores para el MCD serían:
$$
  d \en \set{1, 2, 5, 10}
$$

\textit{Quiero ver si encuentro algún $n$ tal que $d = 1$:}

Teniendo en cuenta que los valores de $n$ que interesan son:
$$
  \congruencia{n}{2}{6}
  \Sii{def}[$\llamada1$]
  n \en \set{k \en \enteros_{\geq 0} : 6k + 2}
$$
Los $n$ son siempre par, por lo que $3n$ siempre será par, descartando así el $d = 1$. También puedo sacar el 5,
porque si $3n$ es par y encima múltiplo de 5, entonces es múltiplo de 10.

Hasta el momento:
$$
  d \neq 1
  \ytext
  d \neq 5
  \entonces
  d \en \set{2, 10}
$$

\textit{Quiero ver si encuentro algún $n$ tal que $d = 2$:}

Con $n = 2$:
$$
  \cajaResultado{
    n = 2
    \entonces
    d = (6 : 10) = 2
  }
$$

\textit{Quiero ver si encuentro algún $n$ tal que $d = 10$:}
$$
  \Entonces{$\llamada1$}[\red{!}]
  n \igual{$\llamada2$} 6k + 2
$$
$$
  \congruencia{6k + 2}{0}{5}
  \sii
  \congruencia{k}{3}{5}
  \Entonces{$k = 3$} n \igual{$\llamada{2}$} 20
$$
Es con $n = 20$ que:
$$
  \cajaResultado{
    n = 20
    \entonces
    d = (60 : 10) = 10
  }
$$

\begin{aportes}
  \item \aporte{\dirRepo}{naD GarRaz \github}
\end{aportes}
