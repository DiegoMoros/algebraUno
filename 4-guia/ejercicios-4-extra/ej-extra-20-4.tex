\begin{enunciado}{\ejExtra}
  Sea $n \en \naturales$ con $n \congruente 2 \ (\text{mód } 6)$. Hallar los posibles valores de $(2n^4 + 6n^2 + 10: n^3 + 2n)$ y dar
  un ejemplo para cada uno.
\end{enunciado}

Bautizo al MCD:
$$
  d = (2n^4 + 6n^2 + 10: n^3 + 2n)
$$
Según Euclides:
$$
  \polyset{vars=n}
  \divPol{2n^4+6n^2+10}{n^3+2n}
$$
Por lo que:
$$
  d = (2n^4 + 6n^2 + 10: n^3 + 2n) = (n^3 + 2n : 2n^2 + 10)
$$

Limpiando un poco eso:
$$
  \llave{l}{
    d \divideA  n^3 + 2n \\
    d \divideA  2n^2 + 10
  }
  \Entonces{$2F_1 - n F_2 \to F_1$}
  \llave{l}{
    d \divideA  6n \\
    d \divideA  2n^2 + 10
  }
  \Entonces{$3 F_2 - nF_1 \to F_2$}
  \llave{l}{
    d \divideA  6n \\
    d \divideA  30
  }
$$

Si $d = (6n : 30)$ entonces los \textit{potenciales} posibles valores para el MCD serían:
$$
  d \en \set{1, 2, 3, 5, 6, 10, 15, 30}
$$

\medskip

Teniendo en cuenta que los valores de $n$ que interesan son:
$$
  \congruencia{n}{2}{6}
  \Sii{def}[$\llamada1$]
  n \en \set{k \en \enteros_{\geq 0} : 6k + 2}
$$

\underline{Los $n$ son siempre par}, descartando así los valores $\set{1, 3, 5, 15}$. Esto porque siempre va a estar el 2 como factor en $d$,
ya que $d \divideA 6n$.

\medskip

Similarmente:
$$
  6 \divideA 6n,
$$
por lo tanto el 2 no puede ser MCD, porque el 6 siempre va a ser un divisor común y obviamente $6 > 2$ {\tiny\faIcon[regulat]{meh-rolling-eyes}}.

Hasta el momento los posibles valores que puede tomar el MCD, $d$:
$$
  d \en \set{6, 10, 30}
$$

\textit{Me parece que $d = 10$ no va a ser un d}, ya que si encuentro algún $n$ es divisible por 5, ocurre lo siguiente:
$$
  \purple{n} = \purple{5k}
  \entonces
  d \divideA 6\cdot \purple{5k}=30k,
$$
por lo que \underline{tengo que eliminar al 10 de los posibles $d$} \red{!} Siempre que sea divisible por 10 va a ser divisible por 30.

\bigskip

Ahora busco ese $\purple{n}$ multiplo de 5, para probar que $30$ es un podible $d$:
$$
  \Entonces{$\llamada1$}[\red{!}]
  \purple{n} = \ub{6k + 2}{\llamada2}
  \Entonces{analizar si $\purple{n}$}[es multiplo de 5]
  \congruencia{6k + 2}{0}{5}
  \sii
  \congruencia{k}{3}{5}
  \Entonces{$k = 3$}
  \purple{n} \igual{$\llamada{2}$} 20
$$
Es con $\purple{n} = 20$ que:
$$
  \cajaResultado{
    \purple{n} = 20
    \entonces
    d
    \igual{\checkmark}
    30
  }
$$

Es así que los posibles valores para $d = (2n^4 + 6n^2 + 10: n^3 + 2n)$ son:
$$
  \cajaResultado{
    d \en \set{6, 30},
    \ \text{ para }
    \congruencia{n}{2}{6}
    \quad \text{y además} \quad
    \llave{llcl}{
      \text{si} & n = 2  & \entonces & d = 6  \\
      \text{si} & n = 20 & \entonces & d = 30
    }
  }
$$

\begin{aportes}
  \item \aporte{\dirRepo}{naD GarRaz \github}
\end{aportes}
