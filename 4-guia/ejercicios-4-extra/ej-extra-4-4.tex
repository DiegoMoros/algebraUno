\begin{enunciado}{\ejExtra}
  Sea $n \en \naturales$. Probar que
  $
    81 \divideA (16 n^2 + 8^{2n} - 15n - 7)^{2024}
  $ si y solo si
  $3 \divideA n$.
\end{enunciado}
\begin{itemize}
  \item[$\red{\entonces}$)]
        $$
          \begin{array}{c}
            81 \divideA (16 n^2 + 8^{2n} - 15n - 7)^{2024}
            \Entonces{\red{!!!}}
            3 \divideA (16 n^2 + 8^{2n} - 15n - 7)^{506}
            \Sii{def}             \\
            \Sii{def}
            \congruencia{(16 n^2 + 8^{2n} - 15n - 7)^{2024}}{0}{3}
            \Sii{\red{!}}
            \congruencia{(n^2)^{2024}}{0}{3}
            \sii
            \congruencia{n^{4048}}{0}{3}
            \Entonces{\red{!!}}
            \congruencia{n}{0}{3} \\
            \boxed{
              81 \divideA (16 n^2 + 8^{2n} - 15n - 7)^{2024}
              \entonces
              3 \divideA n
            }
          \end{array}
        $$

        En el \red{!!!} uso \hyperlink{teoria4:exponentes}{esto $p^n \divideA a^n \sii p \divideA a$}.
        En \red{!} son cuentas de congruencia.
        Y en \red{!!} uso \hyperlink{teoria4:priProductos}{esto, $p \divideA a^n \entonces p \divideA a$}.

  \item[$\red{\Leftarrow}$)]
        $$
          \begin{array}{c}
            3 \divideA n
            \Sii{def}
            \congruencia{n}{0}{3}
            \Sii{\red{!}}
            \congruencia{n^2}{0}{3}
            \Sii{\red{!}}
            \congruencia{16n^2 + 8^{2n} -15n - 7}{0}{3}
            \Sii{\red{!}}
            \\
            \Sii{\red{!}}
            \congruencia{(16n^2 + 8^{2n} -15n - 7)^4}{0}{3^4}
            \Entonces{\red{!}}
            \congruencia{(16n^2 + 8^{2n} -15n - 7)^{2024}}{0}{3^4} \\
            \boxed{
              3 \divideA n
              \entonces
              81 \divideA (16n^2 + 8^{2n} -15n - 7)^{2024}
            }
          \end{array}
        $$
        En el primero y último \red{!} uso que
        $
          \congruencia{n}{0}{d}
          \entonces
          \congruencia{n^m}{0}{d}
        $ y en los otros la mismas cosas que antes... \textit{ponele}
\end{itemize}

\begin{aportes}
  \item \aporte{\dirRepo}{naD GarRaz \github}
\end{aportes}
