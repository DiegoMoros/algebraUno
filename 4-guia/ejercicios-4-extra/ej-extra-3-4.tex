\begin{enunciado}{\ejExtra}
  Sabiendo que $(a:b) = 5$. Probar que $(3ab: a^2 + b^2) = 25$
\end{enunciado}

Arranco \textit{comprimizando: }\par
$$
  \llave{l}{ a = 5c \\
    b = 5d
  }
  \entonces
  (3ab : a^2 + b^2) = 25
  \Sii{coprimizar}[\red{!}]
  (3 cd : c^2 + d^2) = 1
$$
Esto último nos dice que las expresiones $3cd$ y $c^2 + d^2$ son coprimas entre
sí, en otras palabras, que \textit{no hay ningún p primo} que divida ambas expresiones
a la vez.\par
Pruebo por absurdo que no existe $p$ primo que divida a ambas expresiones, es decir
que no existe un $p$, tal que $(3cd : c^2 + d^2) = p$.
Supongo que $\existe p$ primo tal que:
$$
  p \divideA 3 \cdot c \cdot d
  \sii
  \llave{cc}{
    p \divideA 3 & \llamada1 \\
    \otext       &           \\
    p\divideA c  & \llamada2 \\
    \otext       &           \\
    p\divideA d  & \llamada3
  }
$$

Si ocurre que $p \divideA 3 \sii p = 3$. Quiero entonces ver si $3 \divideA c^2 + d^2 \sii c^2 + d^2 \conga3 0$. Hago
una tabla para estudiar esa última ecuación:
$$
  \begin{array}{|c||c|c|c|}
    \hline
    r_3(c)         & 0 & 1 & 2 \\ \hline
    r_3(d)         & 0 & 1 & 2 \\ \hline
    r_3(c^2 + d^2) & 0 & 2 & 2 \\ \hline
  \end{array}
$$

De la tabla concluímos que para que $c^2 + d^2 \conga3 0$ debe ocurrir que:
$c \conga3 0$ y también que $d \conga3 0$, es decir que tanto $c$ como $d$ sean
múltiplos de 3. Esto es una contradicción, ya que \red{no puede} ocurrir porque $(c:d) = 1$. Por
lo tanto no puede ser que $\llamada1 p \divideA 3$ \par \medskip

Si ocurre ahora que $\llamada2 p \divideA c$, estudio a ver si también $p \divideA c^2 + d^2$:
$$
  \llave{l}{
    p \divideA c \\
    p \divideA c^2 + d^2
  }
  \Entonces{}[$F_2 - c \cdot F_1 \to F_2$]
  \llave{l}{
    p \divideA c \\
    p \divideA d^2 \Sii{$p$}[primo] p \divideA d
  }
$$

Entonces si $p \divideA c$ y también $p \divideA c^2 + d^2$ debe ocurrir que
$p \divideA d$. Nuevamente contraticción ya que \red{no puede ocurrir} debido a que $(c:d) = 1$.\par \medskip

El caso $\llamada3$ es lo mismo que el caso $\llamada2$.\par

Se concluye entonces que $(3cd : c^2 + d^2) = 1$ con $(c:d) = 1$.
Así probando que $(3ab : a^2 + b^2) = 25$ con
$\llave{l}{
    a = 5c \\
    b = 5d
  }$

\begin{aportes}
  \item \aporte{\dirRepo}{naD GarRaz \github}
\end{aportes}
