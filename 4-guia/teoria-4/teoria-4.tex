\textit{Divisibilidad:}
\begin{itemize}
  \item Definición divisibilidad:
        $$
          \begin{array}{c}
            \text{$d$ divide a $a$} \Sii{es lo mismo}[que decir] \text{$a$ es un múltiplo entero de $d$} \\
            d \divideA a \sisolosi \existe k \en \enteros \text{ tal que } a = k \cdot d
          \end{array}
        $$

  \item Conjunto de divisores de a:
        $$
          \divset{-a}{-|a|,\dots,-1,1,\dots,|a|}.
        $$

  \item $d \divideA 0 $, dado que $0 = 0\cdot d$. Se desprende que $\divset{0}{\enteros - \set{0}}$

  \item A la hora de divisibilidad \textit{los signos no importan:}\par
        $$
          \llave{rcl}{
            d \divideA a & \sisolosi & -d \divideA a \text{ (pues }a = k \cdot d \sisolosi a = (-k) \cdot (-d))  \\
            d \divideA a & \sisolosi & d \divideA -a \text{ (pues } a = k \cdot d \sisolosi (-a) = (-k) \cdot d)
          }
          \entonces \boxed{d \divideA a \sisolosi |d| \,\divideA\, |a|}
        $$

  \item Propiedades súper útiles para justificar los cálculos en los ejercicios:\par
        $$\llave{l}{
            d \divideA a \ytext d \divideA b \entonces d \divideA a \pm b                         \\
            d \divideA a \entonces d \divideA c \cdot a, \paratodo c \en \enteros                 \\
            d \divideA a \Sii{\red{!!}} d^n \divideA a^n \paratodo n \en \naturales \\
          }$$
        $$
          \text{\red{Error recurrente: }} d \divideA a \cdot b \not\entonces
          \llave{c}{
            d \divideA a \\
            \otext       \\
            d \divideA b
          }. \text{ Por ejemplo } 6 \divideA 3 \cdot 4
          \text{ pero }
          \llave{c}{
            6 \noDivide 3 \\
            \text{ni}     \\
            6 \noDivide 4
          }
        $$
\end{itemize}\bigskip

\textit{Definición congruencia: }

\begin{itemize}[label=$\scriptscriptstyle\blacksquare$]
  \item \textit{Definición congruencia: }\par
        $$\llave{l}{
            \textit{'$a$' es congruente a '$b$' módulo '$d$'} \text{ si }   d \divideA a-b. \quad \text{ Notación } \boxed{\congruencia{a}{b}{d}} \\
            \congruencia{a}{b}{d} \sisolosi d \divideA a-b
          }
        $$

  \item Sumar ecuaciones de congruencia \textit{de mismo módulo}, conserva la congruencia: \par
        $$
          \llave{c}{
            \congruencia{a_1}{b_1}{d} \\
            \vdots                    \\
            \congruencia{a_n}{b_n}{d}
          }
          \entonces \congruencia{a_1 + \cdots + a_n}{a_b + \cdots + b_n}{d}
        $$.
  \item Multiplicar ecuacioines de congruencia \textit{de mismo módulo}, conserva la congruencia: \par
        $$
          \llave{c}{
            \congruencia{a_1}{b_1}{d} \\
            \vdots                    \\
            \congruencia{a_n}{b_n}{d}
          }
          \entonces \congruencia{a_1 \cdots a_n}{a_b \cdots b_n}{d}
        $$
        Un caso particular con un simpático resultado:
        $$
          n \  \text{ecuaciones}
          \llave{c}{
            \congruencia{a}{b}{d} \\
            \vdots                \\
            \congruencia{a}{b}{d}
          }
          \entonces \boxed{\congruencia{a^n}{b^n}{d}}
        $$
\end{itemize}\bigskip

\textit{\underline{Algoritmo de división:}}\par
% macro local
\newcommand{\condicionResto}[1]{\ub{0 \leq #1 < |d|}{\text{\tiny cumple condición de resto}}}
\begin{itemize}
  \item
        Dados $a,\, d \en \enteros$ con $d \distinto 0$, \textit{\underline{existen únicos}} $q$ (cociente),
        $r \text{(resto)} \en \enteros$ tales que:
        $$
          \llave{l}{
            a =  q \cdot d + r, \\
            \text{con } 0 \leq r < |d|.
          }
        $$

  \item \textit{Notación: } \boxed{r_d(a)} es el resto de dividir a $a$ entre $d$

  \item $\condicionResto{r} \entonces r = r_d(r)$. Un número que cumple condición de resto, \underline{es su resto}.

  \item Así es como me gusta pensar a la congruencia. La derecha es el resto de dividir a $a$ entre $d$:
        $$
          \congruencia{a}{r_d(a)}{d}.
        $$

  \item Si $d$ divide al número $a$, entonces el resto de la división es 0:
        $$
          r_d(a) = 0 \sisolosi d \divideA a \sisolosi \congruencia{a}{0}{d}
        $$

  \item El resto es único:
        $$
          \congruencia{a}{r}{d} \text{ con }  \condicionResto{r} \entonces r = r_d(a)
        $$

        $$
          \congruencia{r_1}{r_2}{d} \text{ con } \condicionResto{r_1,r_2} \entonces r_1 = r_2
        $$

  \item Dos números que son congruentes módulo $d$ entre sí, tienen igual resto al dividirse por $d$:
        $$
          \congruencia{a}{b}{d} \sisolosi r_d(a) = r_d(b).
        $$

  \item Propiedades útiles para los ejercicios de calcular restos:
        $$
          r_d(a+b) = r_d(r_d(a) + r_d(b)) \ytext  r_d(a \cdot b) = r_d(r_d(a) \cdot r_d(b))
        $$
        ya que si,
        $$ \llaves{c}{
            \congruencia{a}{r_d(a)}{d}\\
            \congruencia{b}{r_d(b)}{d}
          }
          \Entonces{sumo}[ecuaciones]
          \congruencia{a + b}{r_d(a) + r_d(b)}{d}
        $$
        y,
        $$ \llaves{c}{
            \congruencia{a}{r_d(a)}{d}\\
            \congruencia{b}{r_d(b)}{d}
          }
          \Entonces{multiplico}[ecuaciones]
          \congruencia{a \cdot b}{r_d(a) \cdot r_d(b)}{d}
        $$
\end{itemize}\bigskip

%%%%%% Macro local
\def\mcd{(a:b)}
\def\D{\mathcal D}
\def\cz{s\cdot a + t \cdot b}
%%%%%% fin Macro local

\textit{Máximo común divisor: }

\begin{itemize}[label=\tiny\faIcon{grimace}]
  \item Sean $a,b \en \enteros$, \underline{no ambos nulos}. El MCD entre $a$ y $b$ es el mayor de los divisores
        común entre $a$ y $b$ y se nota:
        $$
          \boxed{
            \text{máximo común divisor: MCD}= \mcd
          }
        $$

  \item $\mcd \en \naturales$ (pues $\mcd \geq 1$) \textit{siempre existe} y es \textit{único}.

  \item Propiedades del $(a:b)$, con $a$ y $b \en \enteros$, no ambos nulos.\par
        \begin{itemize}[label={\tiny\faIcon{atom}}]
          \item Los signos no importan: $\mcd = (\pm a : \pm b)$
          \item Es simétrico: $\mcd = (b:a)$
          \item Entre 1 y $a \en \enteros$ siempre $(a:1) = 1$
          \item Entre 0 y $a$ siempre $(a:0) = |a|,\ \paratodo a \en \enteros -\set{0}$
          \item si $b \divideA a \entonces \mcd = |b|$ con $b \en \enteros - \set{0}$
          \item \red{Útil para ejercicios}: $\mcd = (a: b+na)$ con $n \en \enteros$
          \item \red{Útil para ejercicios}: $\mcd = (a: r_a(b))$ con $n \en \enteros$
          \item \red{Útil para ejercicios}: Sean $a,\ b \en \enteros$ no ambos nulos, y sea $k \en \naturales$
                $$
                  (ka:kb) = k(a:b)
                $$
        \end{itemize}

  \item \textit{Algoritmo de Euclides}: Para encontrar el $(a:b)$ con números feos. Hay que saber hacer esto. Fin.
        \red{¡Se usa de acá hasta el final de la materia!}.\par

  \item \textit{Combinacion Entera}: Otra herramienta gloriosa que sale de hacer \textit{Euclides}.
        \red{¡Se usa de acá hasta el final de la materia!}.\par
        Sean $a,b \en \enteros$ no ambos nulos, entonces $\existe s,\ t \en \enteros$ tal que $\mcd = \cz$.
        \begin{itemize}[label=\tiny\faIcon{atom}]
          \item Todos los divisores comunes entre $a$ y $b$ dividen al $\mcd$. Sean $a,b \en \enteros$ no ambos nulos, $d \en \enteros - \set{0}$. Entonces:
                $$
                  d \divideA a \ytext d \divideA b \sisolosi d \divideA \ub{\mcd}{\cz}.
                $$

          \item Sea $c \en \enteros$ entonces $\existe s', t' \en \enteros$ con $c = s'a + t'b \sisolosi \mcd \divideA c$.

          \item Todos los números múltiplos del MCD se escriben como combinación entera de $a$ y $b$.

          \item Si un número es una combinación entera de $a$ y $b$ entonces es un múltiplo del MCD.
        \end{itemize}
\end{itemize}
\bigskip

\textit{Coprimos: }

\begin{itemize}
  \item Definición coprimos:\par
        Dados $a,b \en \enteros$, no ambos nulos, se dice que son \textit{coprimos} si $\mcd = 1$
        $$
          \begin{array}{rcl}
            a \cop b       & \sisolosi & \mcd = 1                                            \\
            \qquad a\cop b & \sisolosi & \existe s,\ t \en \enteros \text{ tal que } 1 = \cz
          \end{array}
        $$

  \item Sean $a,b \en \enteros$ no ambos nulos.\textit{coprimizar} los números es dividirlos por su máximos común divisor, para
        obtener un nuevo par que sea coprimo:
        $$
          (a:b)\distinto 1 \flecha{coprimizar} a' = \frac{a}{\mcd},\, b' = \frac{b}{\mcd}, \entonces \boxed{(a':b') = 1} \Tilde
        $$

  \item \red{¡Causa de muchos errores!} Sean $a, c, d \en \enteros$ con $c,d$ no nulos. Entonces:
        $$
          c \divideA a \ytext d \divideA a \ytext c \cop d \Sii{\red{!!}} c\cdot d \divideA a
        $$
        Al ser $c$ y $d$ coprimos, pienso a $a$ como un número cuya factorización tiene a $c$, $d$ y la coprimicidad hace que en la factorización
        aparezca $c \cdot d$. {\tiny(no sé, así lo piensa mi {\color{pink}\faIcon{brain}})}.

  \item Sean $a, b, d \en \enteros$ con $d \distinto 0$. Entonces:
        \[
          d \divideA a \cdot b \ytext d \cop a   \entonces d \divideA b
        \]
\end{itemize}

\begin{itemize}
  \item \textit{Primos y Factorización: }
        \begin{itemize}[label=\tiny\faIcon{meh}]
          \item Sea $p$ primo y sean $a,b \en \enteros$. Entonces:
                $$
                  p \divideA a\cdot b \entonces p \divideA a \quad \otext\quad p \divideA b
                $$
          \item\hypertarget{teoria4:priProductos}{\textit{Si $p$ divide a algún producto de números, tiene que dividir a alguno de los factores $\to$}}\par
                Sean $a_1,\dots, a_n \en \enteros$:\par
                \begin{center}
                  $
                    \llave{l}{
                      p \divideA a_1 \cdot a_2 \cdots a_n \entonces p \divideA a_i \text{ para algún } i \text{ con } 1 \leq i \leq n. \\
                      p \divideA a^n \entonces p \divideA a.
                    }$
                \end{center}

          \item Si $a \en \enteros$, $p$ primo:\par
                \begin{center}
                  $\llave{l}{
                      (a:p) = 1 \sisolosi p \noDivide a \\
                      (a:p) = p \sisolosi p \divideA a
                    }$
                \end{center}

          \item Sea $n \en \enteros - \set{0},\,
                  n = \ub{s }{\set{-1,1}} \cdot \productoria{i=1}{k} p_i^{\alpha_i} =
                  p_1^{\alpha_1} \cdots p_k^{\alpha_k}$
                su factorización en primos. Entonces todo divisor $m$ positivo de $n$ se escribe como:\par
                $$
                  \llave{c}{
                    \text{Si } m \divideA n \to  m = p_1^{\beta_1} \cdots p_k^{\beta_k}
                  \text{ con } 0 \leq \beta_i \leq \alpha_i,\, \paratodo i\, 1\leq i \leq k\\
                  \text{ y hay } \\
                  (\alpha_1 + 1) \cdot (\alpha_2 + 1)\cdots (\alpha_k + 1) = \productoria{i=1}{k} \alpha_i +1 \\
                  \text{divisores positivos de } n.
                  }
                $$
          \item Sean $a$ y $b \en \enteros$ no nulos, con
                $$
                  \llave{l}{
                    a = \pm p_1^{m_1}\cdots p_r^{m_r} \text{ con } m_1,\cdots, m_r \en \enteros_0\\
                  b = \pm p_1^{n_1}\cdots p_r^{n_r} \text{ con } n_1,\cdots, n_r \en \enteros_0\\
                  \llave{l}{
                    \entonces \mcd = p_1^{min\set{m_1,n_1}}\cdots p_r^{min\set{m_r,n_r}}\\
                  \entonces [a:b] = p_1^{max\set{m_1,n_1}}\cdots p_r^{max\set{m_r,n_r}}
                  }
                  }
                $$

          \item\hypertarget{teoria4:exponentes} Sean $a, d \en\enteros$ con $d \distinto 0$ y sea $n \en \naturales$. Entonces
                $$
                  d \divideA a \sisolosi d^n \divideA a^n.
                $$

          \item Sean $a,b,c \en \enteros$ no nulos:
                \begin{itemize}
                  \item $a \cop b \sisolosi \text{no tienen primos en común}.$
                  \item $\mcd = 1 \ytext (a : c) = 1 \sisolosi (a : bc) = 1$
                  \item $\mcd = 1 \sisolosi (a^m: b^n) = 1 ,\, \paratodo m, n \en \naturales$
                  \item $(a^n:b^n) = (a:b)^n \paratodo n \en \naturales$
                \end{itemize}

          \item Si $a \divideA m \ \y \  b \divideA m$, entonces $[a:b] \divideA m$

          \item $\mcd \cdot [a:b] = |a \cdot b|$
        \end{itemize}

\end{itemize}
