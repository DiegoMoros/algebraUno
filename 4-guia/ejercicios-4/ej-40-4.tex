\begin{enunciado}{\ejercicio}
    Hallar todos los $a,b \en \naturales$ tales que 
    \begin{multicols}{2}
      \begin{enumerate}[label=(\alph*)]
        \item $(a:b)=10$ y $[a:b]=1500$.
        \item $3 \divideA a$, $(a:b)=20$ y $[a:b]=9000$.
      \end{enumerate}
    \end{multicols}
\end{enunciado}

\begin{enumerate}[label=(\alph*)]

    \item 

    Veamos la primera condición

    $$
    (a:b)=10
    \sisolosi
    (a:b)= 2 \cdot 5
    $$

    De aca tenemos que tanto $a$ y $b$ poseen en su factorización, como mínimo, un 2 y un 5. \bigskip

    Veamos la segunda condición

    $$
    [a:b]=1500
    \sisolosi
    [a:b]=2^2 \cdot 3 \cdot 5^3
    $$

    De aca tenemos que alguno entre $a$ y $b$ tiene un $2^2$, pero nos los dos a la vez, pues en el MCD aparece un 2. Por la misma razón,
    alguno tiene un 3, pero no los dos y alguno tiene un $5^3$, pero no los dos. \bigskip

    Resumiendo, tenemos que $a$ y $b$ tienen un 2 y un 5 siempre y debemos repartir un 2, un 3 y un $5^2$ 
    para formar todas las combinaciones posibles. Así, todos los $a$ y $b$ son 

    \begin{align*}
        (a,b)= \boxed{(2^2 \cdot 3 \cdot 5^3 , 2 \cdot 5)} \\
        (a,b)= \boxed{(2^2 \cdot 3 \cdot 5 , 2 \cdot 5^3)} \\
        (a,b)= \boxed{(2^2 \cdot 5 , 2 \cdot 3 \cdot 5^3)} \\
        (a,b)= \boxed{(2^2 \cdot 5^3 , 2 \cdot 3 \cdot 5)} \\
        (a,b)= \boxed{(2 \cdot 5 , 2^2 \cdot 3 \cdot 5^3)} \\
        (a,b)= \boxed{(2 \cdot 5^3 , 2^2 \cdot 3 \cdot 5)} \\
        (a,b)= \boxed{(2 \cdot 3 \cdot 5^3 , 2^2 \cdot 5)} \\
        (a,b)= \boxed{(2 \cdot 3 \cdot 5 , 2^2 \cdot 5^3)} \\
    \end{align*}


    \item

    Veamos la segunda condición

    $$
    (a:b)=20
    \sisolosi
    (a:b)= 2^2 \cdot 5
    $$

    De aca tenemos que tanto $a$ y $b$ poseen en su factorización, como mínimo, un $2^2$ y un 5. \bigskip

    Veamos la tercera condición

    $$
    [a:b]=9000
    \sisolosi
    [a:b]=2^3 \cdot 3^2 \cdot 5^3
    $$

    De aca tenemos que alguno entre $a$ y $b$ tiene un $2^3$, pero nos los dos a la vez, pues en el MCD aparece un $2^2$. Por la misma razón,
    alguno tiene un $5^3$, pero no los dos. \par
    En el caso del $3^2$, como tenemos la primera condición que nos dice que $3 \divideA a$, el $3^2$ debe estar en la factorización de $a$ 
    si o si. \bigskip

    Resumiendo, tenemos que $a$ tiene un $2^2$, un $3^2$ y un 5, mientras $b$ posee un $2^2$ y un 5. Ahora solo queda repartir un 2 y un $5^2$
    para formar todas las combinaciones posibles. Así, todos los $a$ y $b$ son

    \begin{align*}
      (a,b)= \boxed{(2^3 \cdot 3^2 \cdot 5^3 , 2^2 \cdot 5)} \\
      (a,b)= \boxed{(2^2 \cdot 3^2 \cdot 5 , 2^3 \cdot 5^3)} \\
      (a,b)= \boxed{(2^3 \cdot 3^2 \cdot 5 , 2^2 \cdot 5^3)} \\
      (a,b)= \boxed{(2^2 \cdot 3^2 \cdot 5^3 , 2^3 \cdot 5)} \\
  \end{align*}


\begin{aportes}
    \item \aporte{https://github.com/Nunezca}{Nunezca \github}
\end{aportes}
