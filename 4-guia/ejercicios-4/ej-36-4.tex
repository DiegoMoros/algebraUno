\begin{enunciado}{\ejercicio}
    Sean $a, b \en \enteros$. Probar que si $(a:b)=1$ entonces $(a^{2} \cdot b^{3}: a+b)=1$.
\end{enunciado}

La estrategia es suponer que $(a^{2} \cdot b^{3}: a+b) \neq 1$ sabiendo que $(a:b)=1$ y llegar a una contradicción. \par
Sea $d = (a^{2} \cdot b^{3}: a+b)$ con $d \neq 1$, entonces $\existe p$ primo positivo tal que $p \divideA d$. \par
Luego

$$
 \llave{l}{
    d \divideA a^2 \cdot b^3 \\
    d \divideA a+b
}
\Entonces{Transitividad}
 \llave{l}{
    p \divideA a^2 \cdot b^3 \Entonces{p primo} p \divideA a \otext p \divideA b \\
    p \divideA a+b
}
$$

Esto nos deja dos opciones:

\begin{itemize}

    \item Caso $p \divideA a$

    $$
    \llave{l}{
        p \divideA a \\
        p \divideA a+b
    }
    \Entonces{$F_2 - F_1$}
    p \divideA b
    $$

    Lo cual es absurdo, pues $p \divideA a$ y $p \divideA b$, pero dijimos que $(a:b)=1$.

    \item Caso $p \divideA b$ 

    $$
    \llave{l}{
        p \divideA b \\
        p \divideA a+b
    }
    \Entonces{$F_2 - F_1$}
    p \divideA a
    $$

    Lo cual es absurdo, pues $p \divideA a$ y $p \divideA b$, pero dijimos que $(a:b)=1$.

\end{itemize}

Sea como fuera, en ambos casos llegamos a un absurdo suponiendo que $d \neq 1$. Luego, $d=1 \Tilde$


\begin{aportes}
    \item \aporte{https://github.com/Nunezca}{Nunezca \github}
\end{aportes}