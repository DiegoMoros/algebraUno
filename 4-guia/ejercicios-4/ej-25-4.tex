\begin{enunciado}{\ejercicio}
    Sea $p$ primo positivo.
  \begin{enumerate}[label=\alph*)]
	\item Probar que si $0 < k < p \divideA \binom{p}{k}$.
	\item Probar que si $a, b \en \enteros$, entonces $\congruencia{(a+b)^p}{a^p + b^p}{p}$.
  \end{enumerate}
\end{enunciado}

\begin{enumerate}[label=\alph*)]
	\item 

	Como $ 0 < k < p$, tenemos que $p \noDivide k!$ y que $p \noDivide (p-k)!$, pues $p$ es primo y no divide a ningun factor de ambos números.
	Por la misma razón, se tiene que $p \noDivide k!(p-k)!$.
	Entonces

	$$
	\frac{p!}{k!(p-k)!}=\binom{p}{k}
	\sisolosi
	p!=\binom{p}{k} \cdot k!(p-k)!
	\sisolosi
	p(p-1)!=\binom{p}{k} \cdot k!(p-k)!
	\Entonces{$(p-1)! \en \enteros$}
	p \divideA \binom{p}{k} \cdot k!(p-k)! 
	$$ 

	$$
	p \divideA \binom{p}{k} \cdot k!(p-k)!
	\Entonces{$p$ primo}[$p \noDivide k!(p-k)!$]
	p \divideA \binom{p}{k} \Tilde
	$$


	\item
	Usando el binomio de Newton, tenemos que 

	$$
	(a+b)^{p}=\sumatoria{k=0}{p} \binom{p}{k} \cdot a^{k} \cdot b^{p-k}=a^{p}+b^{p} + \sumatoria{k=1}{p-1} \binom{p}{k} \cdot a^{k} \cdot b^{p-k}
	$$

	Como en la nueva sumatoria tenemos que $0 < k < p$, podemos aplicar lo probado en el inciso a, obteniendo que

	$$
	\congruencia{\sumatoria{k=1}{p-1} \binom{p}{k} \cdot a^{k} \cdot b^{p-k}}{0}{p}
	$$

	Ahora solo queda juntar todo

	$$
	(a+b)^{p} \equiv a^{p}+b^{p} + \sumatoria{k=1}{p-1} \binom{p}{k} \cdot a^{k} \cdot b^{p-k} \equiv a^{p}+b^{p} \pmod{p} \Tilde
	$$

	\begin{aportes}
		\item \aporte{https://github.com/Nunezca}{Nunezca \github}
	\end{aportes}