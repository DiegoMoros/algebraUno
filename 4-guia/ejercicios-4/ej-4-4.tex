\begin{enunciado}{\ejercicio}
  Sea $a \in \enteros$ impar.
  Probar que $2^{n+2} \divideA a^{2^n} - 1$ para todo $n \en \naturales$
\end{enunciado}

Pruebo por inducción:

$$
  p(n): 2^{n+2} \divideA a^{2^n} - 1,\, \text{ con } a \en \enteros \text{ e impar.} \paratodo n \en \naturales.
$$

\textit{Caso base: } $$
  \begin{array}{c}
    p(1)\ :\ 2^3 = 8 \divideA a^2 - 1 = (a - 1) \cdot (a + 1)                       \\
    \flecha{$a$ es impar, si $m \en \enteros$}[$a \igual{}[$\llamada1$] 2m -1$]     \\
    (a - 1) \cdot (a + 1) \igual{$\llamada1$}
    (2m - 2)\cdot(2m) \igual{!}
    4 \cdot \ub{m \cdot (m-1)}{par:\; \green{2h},\, h \en \enteros} =
    4 \cdot \green{2 h} = 8 * h                                                     \\
    \flecha{por lo}[tanto]                                                          \\
    8 \divideA 8h = (a - 1) \cdot (a + 1) \text{ para algún } h \en \enteros \Tilde \\
  \end{array}
$$

Por lo tanto $p(1)$ es verdadera.\par

\textit{Paso inductivo: }\par
Asumo que:
$ p(\blue{k}): \ob{2^{\blue{k}+2} \divideA a^{2^{\blue{k}}} - 1}{\purple{\text{hipótesis inductiva}}}, \text{ es verdadera }
  \entonces \text{ Quiero ver que }
  p(\blue{k}+1) : 2^{\blue{k} + 3} \divideA a^{2^{\blue{k} + 1}} - 1,
  \text{ también lo sea}.
$
$$
  \begin{array}{c}
    2^{k+3} \divideA a^{2^{k+1}} - 1
    \Sii{\red{!}}
    2^{k+2} \cdot 2 \divideA (a^{2^k} - 1)
          \cdot
          \ob{(a^{2^k} + 1)}{\text{\green{par !}}}\\
          \Sii{Si $a \divideA b \ytext c \divideA d \entonces ac \divideA bd$}[\purple{hipótesis inductiva}]\\
          \purple{2^{k+2}} \cdot 2 \divideA \purple{(a^{2^k} - 1)} \cdot \ub{(a^{2^k} + 1)}{\text{\green{par}}}.
  \end{array}
$$

El \red{!} es todo tuyo, \textit{hints:} diferencia de cuadrados, propiedades de exponentes... \faIcon{hands-wash}

En el último paso se comprueba que $p(k+1)$ es vedadera.\par
Como $p(1), p(k) \ytext p(k+1)$ resultaron verdaderas,
por el principio de inducción también lo será ${p(n) \paratodo n \en \naturales}$.


% Contribuciones
\begin{aportes}
  %% iconos : \github, \instagram, \tiktok, \linkedin
  %\aporte{url}{nombre icono}
  \item \aporte{https://github.com/nad-garraz}{naD GarRaz \github}
\end{aportes}
