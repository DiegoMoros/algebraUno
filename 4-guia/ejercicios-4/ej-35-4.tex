\begin{enunciado}{\ejercicio}
 \begin{enumerate}[label=(\alph*)]

    \item Sea $k \en \naturales$. Probar que $(2^k+7^k:2^k-7^k)=1$.

    \item Sea $k \en \naturales$. Probar que $(2^k+5^{k+1}:2^{k+1}+5^k)=3 \otext 9$, y dar un ejemplo para cada caso.

    \item Caracterizar para cada $k \en \naturales$ el valor que toma $(12^k-1:12^k+1286)$.

 \end{enumerate}
\end{enunciado}

\begin{enumerate}[label=(\alph*)]

    \item 
    
    Sea $d=(2^k+7^k:2^k-7^k)=1$. Entonces 

    $$
     \llave{l}{
        d \divideA 2^k+7^k \\
        d \divideA 2^k-7^k
     }
     \llave{l}{
        \Entonces{$F_1+F_2$} d \divideA 2 \cdot 2^k \\
        \Entonces{$F_1-F_2$} d \divideA 2 \cdot 7^k
     }
     \entonces 
     d \divideA (2 \cdot 2^k:2 \cdot 7^k) = 2(2^k:7^k) \igual{\red{!!}}2 \cdot 1=2
     \entonces
     d \en \set{1,2}
     $$

     En $\red{!!}$ uso que $(2:7)= 1 \sisolosi (2^k:7^k)=1$ \par
     Ahora nos gustaria descartar que $d$ pueda ser 2, con lo que basta ver que 2 no divide a alguna de las expresiones.
     Para esto, miremos la congruencia módulo 2 de $2^k+7^k$:

     $$
     2^k+7^k \equiv 0^k+1^k \equiv 1 \pmod{2}
     \entonces
     r_{2}(2^k+7^k) = 1 , \paratodo ~ k \en \naturales
     \entonces
     2 \noDivide 2^k+7^k, \paratodo ~ k \en \naturales
     $$

     De aca, tenemos que $2 \noDivide d$. Entonces, queda que $\boxed{d=1}$, tal como queriamos probar.


     \item 
     
     Sea $d=(2^k+5^{k+1}:2^{k+1}+5^k)=1$. Entonces

     $$
     \llave{l}{
        d \divideA 2^k+5^{k+1} \\
        d \divideA 2^{k+1}+5^k
     }
     \llave{l}{
        \Entonces{$2 \cdot F_1$} 
        \llave{l}{
        d \divideA 2^{k+1}+2 \cdot 5^{k+1} \\
        d \divideA 2^{k+1}+5^k
        }
        \Entonces{$F_1-F_2$}
        d \divideA 9 \cdot 5^k \\
        \Entonces{$5 \cdot F_2$} 
        \llave{l}{
        d \divideA 2^{k}+ 5^{k+1} \\
        d \divideA 5 \cdot 2^{k+1}+5^{k+1}
        }
        \Entonces{$F_2-F_1$}
        d \divideA 9 \cdot 2^k 
     }
     \entonces
     d \divideA (9 \cdot 5^k:9 \cdot 2^k) = 9(5^k:2^k) \igual{\red{!!}}9 \cdot 1=9
     $$

     $$
     \entonces
     d \en \set{1,3,9}
     $$

     En $\red{!!}$ uso que $(5:2)= 1 \sisolosi (5^k:2^k)=1$ \par

     Veamos ahora que $d$ puede ser igual a 3 o a 9:

     $$
     \llave{l}{
        k=1 \rightarrow d= (2^1+5^{1+1}:2^{1+1}+5^1)=(27:9)=(9:0)=9 \Tilde \\
        k=2 \rightarrow d= (2^2+5^{2+1}:2^{2+1}+5^2)=(129:33)=(33:30)=(30:3)=(3:0)=3 \Tilde
     }
     $$

     En estos pasos usé el algoritmo de Euclides. \bigskip

     Ahors tenemos que ver que $d$ no puede ser 1, con lo que debemos verificar que ambas expresiones son siempre divisibles por 3.
     Para esto, miramos la congruencia módulo 3:

     $$
     \llave{l}{
        2^k + 5^{k+1} \equiv 2^k+2^{k+1} \equiv 3 \cdot 2^k \equiv 0 \pmod{3} \entonces r_3(2^k+5^{k+1})=0, \paratodo ~ k \en \naturales \\
        2^{k+1} + 5^k \equiv 2^{k+1}+2^k \equiv 3 \cdot 2^k \equiv 0 \pmod{3} \entonces r_3(2^{k+1}+5^k)=0, \paratodo ~ k \en \naturales
     }
     $$

     Entonces, tenemos que 

     $$
     3 \divideA 2^k+5^{k+1} \ytext 3 \divideA 2^{k+1}+5^k, \paratodo ~ k \en \naturales
     $$

     Con lo que d no puede ser 1. Entonces $\boxed{d=3 \otext 9}$, tal como queriamos ver.


     \item 

     Sea $d = (12^k-1:12^k+1286)$. \par

     Notemos que $(12^k + 1286)-(12^k-1)=1287$. De modo que, haciendo Euclides, tenemos que
     
     $$
     d = (12^k-1:12^k+1286) = (12^k-1:1287)=(12^k-1:3^2 \cdot 11 \cdot 13)
     $$

     Miremos ahora la congruencia módulo 3, 11 y 13 de $12^k-1$:
      
      \begin{itemize}

        \item mod 3

        $$
        12^k-1 \equiv 0^k+2 \equiv 2 \pmod{3}
        \entonces
        r_3(12^k-1)=2
        \entonces
        3 \noDivide 12^k-1, \paratodo ~ k \en \naturales
        $$

        Luego, $3 \noDivide d$, de modo que $d \en \set{11,13,11 \cdot 13}$

        \item mod 11

        $$
        12^k-1 \equiv 1^k-1 \equiv 1-1 \equiv 0 \pmod{11}
        \entonces
        r_{11}(12^k-1)=0
        \entonces
        11 \divideA 12^k-1, \paratodo ~ k \en \naturales
        $$

        Luego, $11 \divideA d$, de modo que $d \en \set{11,11 \cdot 13}$

        \item mod 13

        $$
        12^k-1 \equiv (-1)^k-1 \pmod{13}
        $$

        Aca se abren dos opciones, dependiendo si $k$ es par o impar.\bigskip

        Si $k$ es par, tenemos que 
        
        $$
        (-1)^k=1
        \entonces
        \congruencia{12^k-1}{0}{13}
        \entonces 
        r_{13}(12^k-1)=0
        \entonces
        13 \divideA 12^k-1
        $$
        
        Luego, tenemos que $13 \divideA d$, de modo que $d= 11 \cdot 13=143$. \bigskip

        Si $k$ es impar, tenemos que 
        
        $$
        (-1)^k=-1
        \entonces
        \congruencia{12^k-1}{11}{13}
        \entonces 
        r_{13}(12^k-1)=11
        \entonces
        13 \noDivide 12^k-1
        $$
        
        Luego, tenemos que $13 \noDivide d$, de modo que $d= 11$. 

      \end{itemize}
    
    Resumiendo

    $$
    \llave{ll}{
      \text{$\boxed{d=11}$ si $k$ es impar} \\
      \text{$\boxed{d=143}$ si $k$ es par}
    }
    $$

\end{enumerate} 


\begin{aportes}
     \item \aporte{https://github.com/Nunezca}{Nunezca \github}
\end{aportes}
     