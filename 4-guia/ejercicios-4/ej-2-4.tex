\def\enumeracion{\alph*)}
\begin{enunciado}{\ejercicio}
  Hallar todos los $n \en \naturales$ tales que:
  \begin{multicols}{2}
    \begin{enumerate}[label=\enumeracion]
      \item $3n - 1 \divideA n+7$
      \item $3n -2 \divideA 5n - 8$
      \item $2n+1 \divideA n^2 + 5$
      \item $n-2 \divideA n^3 - 8$
    \end{enumerate}
  \end{multicols}
\end{enunciado}

\begin{enumerate}[label=\enumeracion]
  \item
        Busco eliminar la $n$ del \textit{miembro} derecho.
        $$
          \llave{l}{
            3n - 1 \divideA n + 7 \\
            3n - 1 \divideA 3n - 1
          }
          \Entonces{$3F_1 - F_2 \to F_1$}
          \llave{l}{
            3n - 1 \divideA 22 \\
            3n - 1 \divideA 3n - 1
          }
        $$
        Con ese resultado sé que $3n -1$ es un \textit{\underline{posible divisor}} de 22:
        $$
          \blue{d} = 3n - 1 \en \set{\pm1,\pm2, \pm11, \pm22}
        $$

        Pero ahora hay que probar para cuáles valores de $n$, obtengo alguno de esos valores $\blue{d}$.

        Probando a manopla:
        $$
          \begin{array}{lcl}
            n = 1 & \entonces & d = 2  \\
            n = 6 & \entonces & d = 11
          \end{array}
        $$

  \item Trato de sacar la $n$ del lado derecho:
        $$
          \llave{l}{
            3n -2 \divideA 5n - 8 \\
            3n -2 \divideA 3n - 2
          }
          \Entonces{$3F_1 - 4F_2 \to F_1$}
          \llave{l}{
            3n -2 \divideA -16 \\
            3n -2 \divideA 3n - 2
          }
        $$
        Con ese resultado sé que $3n -2$ es un \textit{\underline{posible divisor}} de  16:
        $$
          \blue{d} = 3n - 2 \en \set{\pm1, \pm2, \pm4, \pm8 \pm 16}
        $$
        Pero ahora hay que probar para cuáles valores de $n$, obtengo alguno de esos valores $\blue{d}$,
        pero me da pajilla \rosa{\faIcon[regular]{grimace}}.

  \item
        Trato de sacar la $n$ del lado derecho:
        $$
          \llave{l}{
            2n - 1 \divideA n^2 + 5 \\
            2n - 1 \divideA 2n - 1
          }
          \Entonces{$nF_2 - 2F_1 \to F_1$}
          \llave{l}{
            2n - 1 \divideA -n - 10 \\
            2n - 1 \divideA 2n - 1
          }
          \Entonces{$2F_1 + F_2 \to F_1$}
          \llave{l}{
            2n - 1 \divideA -21 \\
            2n - 1 \divideA 2n - 1
          }
        $$
        Con ese resultado sé que $2n - 1$ es un \textit{\underline{posible divisor}} de  21:
        $$
          \blue{d} = 3n - 2 \en \set{\pm1, \pm3, \pm7, \pm21}
        $$
        Pero ahora hay que probar para cuáles valores de $n$, obtengo alguno de esos valores $\blue{d}$,
        pero me da pajilla \rosa{\faIcon[regular]{grimace}}.

  \item A veces se ve y a veces no, pero:
        $$
          \polyset{vars=n}
          \divPol{n^3-8}{n-2}
        $$
        Por lo que:
        $$
          n^3 - 8
          \igual{\red{!}}
          (n-2) \cdot (n^2 + 2n +4)
          \entonces
          n-2 \divideA n^3 - 8 \quad \paratodo n \en \naturales_{\neq 2}
        $$
\end{enumerate}

\begin{aportes}
  \item \aporte{\dirRepo}{naD GarRaz \github}
\end{aportes}
