\begin{enunciado}{\ejercicio}
  Sabiendo que el resto de la división de un entero
  $a$ por 18 es 5, calcular el resto de:
  \begin{enumerate}[label=\alph*)]
    \item la división de $a^2 -3a +11$ por 18.
    \item la división de $a$ por 3.
    \item la división de $4a+1$ por 9.
    \item la división de $7a^2 + 12$ por 28.
  \end{enumerate}
\end{enunciado}

\begin{enumerate}[label=\alph*)]
  \item $r_{18}(a) =
          r_{18}( \ub{r_{18}(a)^2}{5^2} - \ub{r_{18}(3)}{3} \cdot \ub{r_{18}(a)}{5} + \ub{r_{18}(11)}{11} ) =
          r_{18}(21) = 3 $

        \separadorCorto

  \item $
          \llaves{l}{
            a = 3 \cdot q + r_3(a)\\
            6 \cdot a = 18 \cdot q + \ub{\green{6 \cdot r_3(a)}}{r_{18}(6a)}\\
          } \to
          r_{18}(6a) = r_{18}( r_{18}(6) \cdot r_{18}(a) ) = r_{18}(30) = 12\\
          \entonces \green{6 \cdot r_3(a)} = r_{18}(6a) \to  r_3(a) = 2
        $
        \separadorCorto

  \item $r_9(4a+1) = \ub{r_9(4 \cdot r_9(a) + 1)}{\blue{*1}} \to\\
          a = 18 \cdot q + 5 = 9 \cdot \ub{( 9 \cdot q)}{q'} + \ub{5}{r_9(a)}
          \flecha{\blue{*1}}
          r_9(a) = r_9(21) = 3
        $

  \item
        $r_{28}(7a^2 + 12) = r_{28}(7 \cdot r_{28}(a)^2 + 12) \flecha{¿qué es} r_{28}(a)$\\
        $\llave{l}{
            a = 18 \cdot q + 5 \flecha{busco algo}[para el 28]\\
            14 \cdot a = \ub{252 \cdot q}{28 \cdot 9\cdot q } + 70
            \flecha{corrijo según}[condición resto]
            28 \cdot 9\cdot q + \ub{2\cdot28 +14}{70} = 28\cdot (9\cdot q + 2) + 14  \Tilde\\
            \flecha{por lo}[tanto] 14a = 28\cdot q' + 14 \entonces \congruencia{14\cdot a}{14}{28} \sisolosi  \congruencia{a}{1}{28}
          }$\\
        Ahora que sé que $r_{28}(a) = 1$ sale que $r_{28}(7a^2 + 12) = r_{28}(7 \cdot \ub{r_{28}(a)^2}{1} + 12) = r_{28}(19)=19 \Tilde$
\end{enumerate}
