\begin{enunciado}{\ejercicio}

    \begin{enumerate}[label=(\alph*)]

        \item Sean $a,b \en \enteros$ tales que $(a:b)=3$. Calcular los posibles valores de $(a^2+15b+57:4050)$ y dar un ejemplo en cada caso.

        \item Sean $a,b \en \enteros$. Sabiendo que $\congruencia{b}{6}{24}$ y que $(a:b)=13$, calcular $(5a^2+11b+117:624)$.
    
    \end{enumerate}

\end{enunciado}

    \begin{enumerate}[label=(\alph*)]

        \item 

        Coprimizo: defino $3x=a$ y $3y=b$, con lo que $(a:b)=(3x:3y)=3(x:y)=3$, de modo que $(x:y)=1$. \bigskip

        Reemplazo en $d=(a^2+15b+57:4050):$ 

        $$
        d=(a^2+15b+57:4050) =
        (9x^2 + 45y +57:2 \cdot 3^4 \cdot 5^2)=
        3(3x^2 + 15y +19:2 \cdot 3^3 \cdot 5^2)
        $$

        Sea ahora $d'=(3x^2 + 15y +19:2 \cdot 3^3 \cdot 5^2)$. Entonces, $d=3d'$.\par
        Sabiendo todo esto, tenemos que

        $$
        d' \divideA 2 \cdot 3^3 \cdot 5^2
        \entonces
        d' \en \set{1,2,3,5,6,9,10,15,18,25,27,30,45,50,54,75,90,135,150,225,270,450,675,1350}
        $$

        Miremos ahora la congruencia de $3x^2 + 15y +19$ módulo 3 y 5:

          \begin{itemize}

            \item mod 5

            Notemos que $\congruencia{3x^2 + 15y +19}{3x^2+4}{5}$. Entonces, veo la tabla de restos de $3x^2+4$.

            $$
            \begin{array}{|r|c|c|c|c|c|}
              \hline
              r_5(x)      & 0 & 1 & 2 & 3 & 4 \\ \hline
              r_5(3x^2+4) & 4 & 2 & 1 & 1 & 2 \\ \hline
            \end{array}
           $$

           De aca tenemos que $5 \noDivide 3x^2 + 15y +19 \paratodo ~ x \en \enteros$. Luego, $5 \noDivide d'$. Con lo que 

           $$
           d' \en \set{1,2,3,6,9,18,27,54}
           $$

           \item mod 3

           Acá no hace falta ver la tabla de restos, pues notemos que $\congruencia{3x^2 + 15y +19}{1}{3}$. Entonces 
           $5 \noDivide 3x^2 + 15y +19 \paratodo ~ x \en \enteros$. Luego, $3 \noDivide d'$. Con lo que 

           $$
           d' \en \set{1,2}
           $$

          \end{itemize}

           Como $d'= 1 \otext 2$, entonces, $d= 3 \otext 6$. Veamos ahora ejemplos de que cada uno es posible:

           $$
           \llave{l}{
             (a,b)=(3,3)
             \flecha{$(3,3)=3$}
             d=(111:4050)=(111:54)=(54:3)=(3:0)=3 \Tilde \\
             (a,b)=(6,3)
             \flecha{$(6,3)=3$}
             d=(138:4050)=(138:48)=(48:42)=(42:6)=(6:0)=6 \Tilde
           }
           $$

           Luego, $\boxed{d= 3 \otext 6}$.


        \item

        Coprimizo: defino $13x=a$ y $13y=b$, con lo que $(a:b)=(13x:13y)=13(x:y)=13$, de modo que $(x:y)=1$. \bigskip

        Reemplazo en $d=(5a^2+11b+117:624):$ 

        $$
        d=(5a^2+11b+117:624) =
        (5 \cdot 13^2 \cdot x^2 + 143y +117:2^4 \cdot 3 \cdot 13)=
        13(65x^2+ 11y +9:2^4 \cdot 3)
        $$

        Sea ahora $d'=(65x^2+ 11y +9:2^4 \cdot 3)$. Entonces, $d=13d'$.\par
        Sabiendo todo esto, tenemos que

        $$
        d' \divideA 2^4 \cdot 3
        \entonces
        d' \en \set{1,2,3,4,6,8,12,16,24,48}
        $$


        Antes de mirar las congruencias, veamos la condición que dice que $\congruencia{b}{6}{24}$. De esta obtenemos lo siguiente

        $$
        \congruencia{b}{6}{24}
        \entonces 
         \llave{l}{
          \congruencia{b}{0}{2} \\
          \congruencia{b}{0}{3} \\
          \congruencia{b}{2}{4} \\
          \congruencia{b}{6}{8} 
         }
        $$

        Para obtener condiciones sobre $y$, usamos $b=13y$. Entonces

        $$
        \llave{l}{
          \congruencia{b}{0}{2} \entonces \congruencia{13y}{0}{2} \Entonces{$\congruencia{13}{1}{2}$} \congruencia{y}{0}{2} \\
          \congruencia{b}{0}{3} \entonces \congruencia{13y}{0}{3} \Entonces{$\congruencia{13}{1}{3}$} \congruencia{y}{0}{3} \\
          \congruencia{b}{2}{4} \entonces \congruencia{13y}{2}{4} \Entonces{$\congruencia{13}{1}{4}$} \congruencia{y}{2}{4} \\
          \congruencia{b}{6}{8} \entonces \congruencia{13y}{6}{8} \Entonces{$\congruencia{5}{5}{8}$} \congruencia{65y}{30}{8} \Entonces{$\congruencia{65}{1}{8}$} \congruencia{y}{6}{8}
        }
        $$

        Miremos ahora las congruencias con la expresión $65x^2+ 11y +9$.

        \begin{itemize}

          \item mod 3

          Usando que $\congruencia{y}{0}{3}$, tenemos que $\congruencia{65x^2+ 11y +9}{2x^2}{3}$. \bigskip

          Miremos la tabla de restos con $2x^2$

          $$
            \begin{array}{|r|c|c|c|}
              \hline
              r_3(x)    & 0 & 1 & 2  \\ \hline
              r_3(2x^2) & 0 & 2 & 2  \\ \hline
            \end{array}
           $$

           Notemos que el resto es 0 si y solo $\congruencia{x}{0}{3}$, pero esto no puede ser, pues tendriamos que $3 \divideA x$ y que
           $3 \divideA y$ y no se cumpliria que $(x:y)=1$. Luego, $3 \noDivide 65x^2+ 11y +9$, con lo que $ 3 \noDivide d'$. Con lo que

           $$
           d' \en \set{1,2,4,8,16}
           $$

           \item mod 8

          Usando que $\congruencia{y}{6}{8}$, tenemos que $\congruencia{65x^2+ 11y +9}{x^2+3}{8}$. \bigskip

          Miremos la tabla de restos con $x^2+3$

          $$
            \begin{array}{|r|c|c|c|c|c|c|c|c|}
              \hline
              r_8(x)     & 0 & 1 & 2 & 3 & 4 & 5 & 6 & 7 \\ \hline
              r_8(x^2+3) & 3 & 4 & 7 & 4 & 3 & 4 & 7 & 4 \\ \hline
            \end{array}
           $$

           De aca tenemos que $8 \noDivide 65x^2+ 11y +9$. Luego, $8 \noDivide d'$. Con lo que 

           $$
           d' \en \set{1,2,4}
           $$

           \item mod 2

          Usando que $\congruencia{y}{0}{2}$, tenemos que $\congruencia{65x^2+ 11y +9}{x^2+1}{2}$. \bigskip

          Miremos la tabla de restos con $x^2+1$

          $$
            \begin{array}{|r|c|c|}
              \hline
              r_2(x)     & 0 & 1 \\ \hline
              r_2(x^2+1) & 1 & 0 \\ \hline
            \end{array}
           $$

           De aca tenemos que el resto es 0 si y solo si $\congruencia{x}{1}{2}$. Notemos que en realidad esta es la unica opción, 
           pues no puede ser que $\congruencia{x}{0}{2}$, pues tendriamos que $(x:y) \neq 1$. Luego $2 \divideA 65x^2+ 11y +9$, 
           con lo que $2 \divideA d'$. Así tenemos

           $$
           d' \en \set{2,4}
           $$

           \item mod 4

           Usando que $\congruencia{y}{2}{4}$, tenemos que $\congruencia{65x^2+ 11y +9}{x^2+3}{4}$. \bigskip

           Como del caso anterior obtuvimos que $x$ debe ser impar, basta ver la congruencia módulo 1 y 3:

           $$
            \begin{array}{|r|c|c|}
              \hline
              r_4(x)     & 1 & 3 \\ \hline
              r_4(x^2+3) & 0 & 0 \\ \hline
            \end{array}
           $$
           
           Como en ambos casos el resto es 0, tenemos que $4 \divideA 65x^2+ 11y +9$, de modo que $4 \divideA d'$. Así, llegamos a que 
           el único valor que puede tomar $d'$ es 4. \bigskip

          \end{itemize}
        
        Finalmente, tenemos que $d=13 \cdot 4= \boxed{52}$
    
        \end{enumerate}

        \begin{aportes}
          \item \aporte{https://github.com/Nunezca}{Nunezca \github}
      \end{aportes}

           







           



           







        



        







