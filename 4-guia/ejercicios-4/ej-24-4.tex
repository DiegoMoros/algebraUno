\begin{enunciado}{\ejercicio}
  Probar que existen infinitos primos positivos congruentes a 3 módulo 4.

  \textit{Sugerencia:} probar primero que si $a \en \naturales$ satisface $\congruencia{a}{3}{4}$, entonces existe $p$ primo con
  $\congruencia{p}{3}{4}$ tal que $p \divideA a$. Luego probar que si existieran sólo finitos primos congruentes a 3 módulo 4,
  digamos $p_1, p_2, \dots, p_n$, entonces $a= -1 + 4\productoria{i=1}{n} p_i$ sería mayor que 1 y no es divisible por ningún
  primo congruente a 3 módulo 4.
\end{enunciado}

Comencemos probando la primera parte de la sugerencia:

$$
  \text{Dado $a \en \naturales, \congruencia{a}{3}{4}$}
  \entonces
  \text{$\existe p$ primo con $\congruencia{p}{3}{4}$ tal que $p \divideA a$}
$$

Como $\congruencia{a}{3}{4} \sisolosi a= 4k+3$ y dado que $a \en \naturales$, es evidente que $a > 1$.
Luego sabemos que $\existe p$ primo tal que $p \divideA a$. Ahora debemos ver que $\congruencia{p}{3}{4}$.
Para esto, apliquemos el TFA:

$$
  a = (P_1)^{n_1} \cdot (P_2)^{n_2} \cdots (P_r)^{n_r}, ~ n_1,n_2 \dots, n_r \en \naturales
$$

Notemos ahora que ninguno de los primos en la factorización de $a$ puede ser 2, pues $a=4k+3=2(2k+1) + 1$ es impar.
Esto nos descarta que $\congruencia{p}{0}{4}$ o que $\congruencia{p}{2}{4}$, pues el único primo que cumple alguna es el 2.
De modo que nos quedan dos opciones:

$$
  \congruencia{p}{1}{4}
  \otext
  \congruencia{p}{3}{4}
$$

Prestemos atención a lo siguiente. Si todos los primos en la factorización de $a$ fueran congruentes a 1 módulo 4, esto es

$$
  \congruencia{P_1}{1}{4},
  \congruencia{P_2}{1}{4},
  \dots,
  \congruencia{P_r}{1}{4}
  \entonces
  \congruencia{(P_1)^{n_1}}{1}{4},
  \congruencia{(P_2)^{n_2}}{1}{4},
  \dots,
  \congruencia{(P_r)^{n_r}}{1}{4}
$$

tendriamos que

$$
  a=\congruencia{(P_1)^{n_1} \cdot (P_2)^{n_2} \cdots (P_r)^{n_r}}{1}{4}
$$

lo cual contradice nuestra hipótesis de que $\congruencia{a}{3}{4}$. \par
Así, probamos que al menos debe existir un $p$ en la factorización de $a$ (esto asegura que $p \divideA a$), que cumpla que $\congruencia{p}{3}{4}$ si es que tenemos que $\congruencia{a}{3}{4}$,
que era lo que queriamos probar. \bigskip

Veamos ahora la segunda parte de la sugerencia (no voy a probar eso exactamente, pero es parecido). \par
Supongamos que existen finitos primos congruentes a 3 módulo 4, digamos $p_1, p_2, \dots, p_n$.
Esto nos permite definir $a$ como $a = -1 + 4\productoria{i=1}{n} p_i$.
Notemos que como $a \en \naturales$, $a > 1$ y $\congruencia{a}{3}{4}$, podemos aplicar lo que probamos en la primera parte.
Esto es: existe $p$ primo con $\congruencia{p}{3}{4}$ tal que $p \divideA a$. Notemos que este $p$ debe ser alguno de los $p_i$. \par
Luego

$$
  \llave{l}{
    p_i \divideA -1 + 4\productoria{i=1}{n} p_i \\
    p_i \divideA 4\productoria{i=1}{n} p_i
  }
  \Entonces{$F_2-F_1$}
  p_i \divideA 1
$$

Lo cual es absurdo. Esta contradicción proviene de la única suposición que hicimos, que existen finitos primos congruentes a 3 módulo 4. \par
Luego, existen infinitos primos congruentes a 3 módulo 4, que era lo que queriamos probar.

\begin{aportes}
  \item \aporte{https://github.com/Nunezca}{Nunezca \github}
\end{aportes}
