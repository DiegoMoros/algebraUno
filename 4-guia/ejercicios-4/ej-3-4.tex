\begin{enunciado}{\ejercicio}
  Sean $a.\,b \en \enteros$.
  \begin{enumerate}[label=\alph*)]
    \item Probar que $a-b \divideA a^n - b^n$ para todo $n \en \naturales \ytext a \distinto b \en \enteros$
    \item Probar que si $n$ es un número natural par y $a \distinto -b$, entonces $a+b \divideA a^n - b^n$.
    \item Probar que si $n$ es un número natural impar y $a \distinto -b$, entonces  $a+b \divideA a^n + b^n$.
  \end{enumerate}
\end{enunciado}

\begin{enumerate}[label=\alph*)]
  \item
        \textit{Inducción: }\par
        \textit{Proposición: }
        $$p(n) : a-b \divideA a^n - b^n \paratodo n \en \naturales
          \ytext a \distinto b \en \enteros$$

        \textit{Caso Base: }
        $$
          p(1): a - b \divideA a^{\blue1} - b^{\blue1},$$
        $p(1)$ es verdadera. \Tilde\par

        \textit{Paso inductivo:}\par
        Asumo que $p(\blue{k}): a-b \divideA a^{\blue k} - b^{\blue k}$ es verdadera
        $\entonces$ quiero probar que
        $p(\blue{k + 1}) : a-b \divideA a^{\blue{k + 1}} - b^{\blue{k + 1}}$
        también lo sea.

        $$
          \llave{l}{
            a - b \divideA a^k -b^k\\
            a - b \divideA a^k -b^k
          }
          \Entonces{$\times \magenta{a}$}[$\times \magenta{b}$]
          \llave{l}{
          a - b \divideA a^{k+1} - ab^k\\
          a - b \divideA ba^k -b^{k+1}
          }
          \Entonces{$+$}
          \llave{l}{
            a - b \divideA a^{k+1} - b^{k+1}.\\
          }\Tilde
        $$

        Como $p(1),\, p(k) \ytext p(k+1)$ resultaron verdaderas por el principio de
        inducción $p(n)$ también lo es.

  \item  Sé que
        $$
          a+b \divideA a+b
          \Sii{def}
          \congruencia{a}{-b}{a+b}
        $$
        Multiplicando la ecuación de congruencia por $\magenta{a}$ sucesivas
        veces me formo:\par
        $
          \llave{rcl}{
            \magenta{a} \cdot a = a^2
            &\conga{a+b}&
            \magenta{a} \cdot (-b)
            \conga{a+b}
            (-1)^2 b \\
            \quad &\vdots& \quad \ot\llamada{1}\\
            a^n &\conga{a+b}& (-1)^n \cdot b^n
            \to
            \llave{ll}{
              \congruencia{a^n}{ b^n}{a+b} & \text{con n par}  \\
              \congruencia{a^n}{(-1)^n \cdot b^n}{a+b}   & \text{con n impar} \\
            }
          }\\
          \boxed{
            \llave{rccl}{
              \text{Con $n$ par: } & \congruencia{a^n}{b^n}{a+b} &\entonces& a+b \divideA a^n - b^n  \\
              \text{Con $n$ impar: }  &\congruencia{a^n}{- b^n}{a+b}& \entonces& a+b \divideA a^n + b^n
            }
          }
        $\par

        \textit{$\llamada1$Inducción:}
        $$
          p(n) : \magenta{\congruencia{a}{-b}{a+b}}
          \entonces
          \congruencia{a^n}{(-1)^n \cdot b^n}{a+b} \paratodo n \en \naturales.
        $$

        \textit{Caso base: }
        $$
          p(1) :
          \magenta{\congruencia{a}{-b}{a+b}}
          \entonces
          \congruencia{a^1}{(-1)^1\cdot b^1}{a+b}
        $$
        $p(1)$ es verdadera.\par

        \textit{Paso inductivo: }\par
        $
          \begin{array}{c}
            p(\blue{k}) : \congruencia{a}{-b}{a+b}
            \entonces
            \congruencia{a^{\blue{k}}}{(-1)^{\blue{k}} \cdot b^{\blue k}}{a+b}
            \text{ asumo verdadera para algún } k \en \enteros                           \\
            \entonces                                          \text{quiero probar que } \\
            p(\blue{k+1}) : \congruencia{a}{-b}{a+b}
            \entonces
            \congruencia{a^{\blue{k+1}}}{(-1)^{\blue{k}} \cdot b^{\blue k}}{a+b}
          \end{array}\\
          \text{Partiendo de } p(\blue{k}):
          \llave{c}{
            \congruencia{a}{-b}{a+b}
            \entonces
            \congruencia{a^k}{(-1)^k\cdot b^k}{a+b}\\
            \Entonces{multiplico}[por $\yellow{a}$]\\
            \yellow a \cdot a^k =
            \congruencia{a^{k+1}}{(-1)^k\cdot \ub{\yellow a}{ \conga{a+b} -b} \cdot b^k}{a+b}\\
            \entonces\\
            \congruencia{a^{k+1}}{(-1)^{k+1} \cdot b^{k+1}}{a+b}
            \sisolosi
            a + b \divideA a^{k+1} - (-1)^{k+1} b^{k+1} \Tilde
          }
        $\par

        Como $p(1),\ p(k) \ytext p(k+1)$
        son verdaderas por principio de inducción lo es también $p(n) \paratodo n \en \naturales$


  \item Hecho en el anterior {\Large\faIcon{hands-wash}}.
\end{enumerate}
