\begin{enunciado}{\ejercicio}
    Sea $n \en \naturales, n \geq 2$. Probar que si $p$ es un número primo positivo entonces $\sqrt[n]{p} \notin \racionales$.
\end{enunciado}

Supongamos que $\sqrt[n]{p} \en \racionales$ y lleguemos a una contradicción:

$$
\sqrt[n]{p} \en \racionales 
\entonces
\sqrt[n]{p}=\frac{a}{b}, ~ a,b ~  \en \enteros \ytext b \neq 0
\entonces
b\cdot\sqrt[n]{p}=a
\entonces
b^{n}\cdot p=a^{n}
$$

Por otro lado, por TFA, tenemos que

$$
\llave{l}{
 a = (P_1)^{m_1}...(P_r)^{m_r}, ~ m_1,...,m_r ~ \en \naturales_0 \\
 b = (P_1)^{l_1}...(P_r)^{l_r}, ~ l_1,...,l_r ~ \en \naturales_0
}
\entonces
\llave{l}{
 a^{n} = (P_1)^{n\cdot m_1}...(P_r)^{n\cdot m_r} \\
 b^{n} = (P_1)^{n\cdot l_1}...(P_r)^{n\cdot l_r}
}
$$

Luego

$$
b^{n}\cdot p = a^{n}
\sisolosi
(P_1)^{n\cdot l_1}...(P_r)^{n\cdot l_r}\cdot p^1=(P_1)^{n\cdot m_1}...(P_r)^{n\cdot m_r}
$$

En el lado izquierdo de la igualdad, el primo $p$ aparece con el exponente $n\cdot l_p +1$, mientras que en el lado izquierdo aparece con el exponente $n\cdot m_p$ 

Entonces, por la unicidad de la factorización, tenemos que

\begin{align*}
    n\cdot l_p +1 &= n\cdot m_p \\
    1 &= n\cdot m_p - n\cdot l_p \\
    \frac{1}{n} &=m_p-l_p
\end{align*}

Lo cual es absurdo, pues $m_p-l_p \en \enteros$ pero claramente $\frac{1}{n} \notin \enteros ~ $para$ ~ n \geq 2$.
La contradicción proviene de la única suposición que hicimos, que $\sqrt[n]{p} \en \racionales$. Luego, $\sqrt[n]{p} \notin \racionales$, tal como queriamos probar.

\begin{aportes}
    \item \aporte{https://github.com/Nunezca}{Nunezca \github}
\end{aportes}