\begin{enunciado}{\ejercicio}
    Sea $n \en \naturales, n \geq 2$. Probar que si $p$ es un número primo positivo entonces $\sqrt[n]{p} \notin \racionales$.
\end{enunciado}

Supongamos que $\sqrt[n]{p} \en \racionales$ y lleguemos a una contradicción.

$$
\sqrt[n]{p} \en \racionales 
\entonces
\sqrt[n]{p}=\frac{a}{b}, ~ a,b ~  \en \enteros \ytext b \neq 0
$$

Tomemos $\frac{a}{b}$ como una fracción irreducible, es decir, con $a$ y $b$ coprimos.

Luego, 

$$
\sqrt[n]{p}=\frac{a}{b}
\entonces
b\cdot\sqrt[n]{p}=a
\entonces
b^{n} \cdot p=a^{n}
\entonces
p \divideA a^{n}
\Entonces{$p$ primo}
p \divideA a
$$

Como $p \divideA a$, entonces $a= p \cdot k, \ k \en \enteros$. Reemplazando, tenemos que

$$
b^{n} \cdot p=a^{n}
\entonces
b^{n} \cdot p=(p \cdot k)^{n}
\entonces 
b^{n} \cdot p=p^n \cdot k^{n}
\Entonces{\red{!!}}
b^{n}=p^{n-1} \cdot k^{n}
\Entonces{\red{!!!}}
b^{n}=p \cdot p^{n-2} \cdot k^{n}
\entonces
p \divideA b^{n}
\Entonces{$p$ primo}
p \divideA b
$$

El paso en $\red{!!}$ tiene sentido porque $n \en \naturales$ y en $\red{!!!}$ porque $n \geq 2$. Esto asegura que las
expresiones $p^{n-1}$ y $p^{n-2}$ pertenezcan $\naturales_0$. \bigskip

Así, obtuvimos que $p \divideA a$ y $p \divideA b$, lo cual contradice el hecho que $a$ y $b$ son coprimos.
La contradicción proviene de la única suposición que hicimos, que $\sqrt[n]{p} \en \racionales$. 
Luego, $\sqrt[n]{p} \notin \racionales$, tal como queriamos probar.


\begin{aportes}
    \item \aporte{https://github.com/Nunezca}{Nunezca \github}
\end{aportes}