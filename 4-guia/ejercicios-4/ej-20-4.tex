\ejercicio
Sea $a \en \enteros$.
\begin{enumerate}[label=\alph*)]
	\item Probar que $(5a + 8 : 7a + 3) = 1 \otext 41.$
    Exhibir un valor de $a$ para el cual da 1, y verificar
	      que efectivamente para $a = 23$ da $41$.

	\item Probar que $(2 a^2 + 3a : 5a +6) = 1 \otext 43.$ Exhibir un valor de $a$ para el cual da 1, y verificar
	      que efectivamente para $a = 16$ da $43$

	\item Probar que $(a^2-3a+2 : 3a^3 -5a^2) = 2 \otext 4$, y exhibir un valor de $a$ para cada caso.\\
	      (Para este item es \textbf{indispensable} mostrar que el máximo común divisor nunca puede ser 1).
\end{enumerate}

\separadorCorto

\begin{enumerate}[label=\roman*)]
	\item \hacer 
	\item \hacer
	\item
	      $(a^2-3a+2 : 3a^3 -5a^2)
		      \flecha{Euclides}
		      (\ub{a^2 - 3a + 2}{\llamada{1} par} : \ub{6a -8}{\llamada{1}{} par})\\
		      \flecha{busco}[divisor]
		      \llaves{l}{
			      d \divideA a^2 - 3a + 2\\
			      d \divideA 6a - 8
		      }
		      \flecha{$\times 6$}[$\times a$]
		      \llaves{l}{
			      d \divideA 10a -12\\
			      d \divideA 6a - 8
		      }
		      \flecha{$\times 6$}[$\times 10$]
		      \llaves{l}{
			      d \divideA 8
		      }\to
		      \divsetP{8}{1,2,4,8} \llamada{1}= \set{2,4,8}\\
		      \llave{ll}{
			      a = 1 & (0:-2) = 2\\
			      a = 2 & (0:4) = 4
		      }$\\
	      \red{Parecido al hecho en clase.}\\
	      \red{¿Qué onda el 8? Hice mal cuentas? Si no, cómo lo descarto?}
\end{enumerate}
