\documentclass[12pt,a4paper, spanish]{article}
% Sacar draft para que aparezcan las imagenes.
% Opciones: 12pt, 10pt, 11pt, landscape, twocolumn, fleqn, leqno...
% Opciones de clase: article, report, letter, beamer...

% Paquetes:
% =========
\usepackage[headheight=110pt, top = 2cm, bottom = 2cm, left=1cm, right=1cm]{geometry} %modifico márgenes
\usepackage[T1]{fontenc} % tildes
\usepackage[utf8]{inputenc} % Para poder escribir con tildes en el editor.
\usepackage[english]{babel} % Para cortar las palabras en silabas, creo.
\usepackage[ddmmyyyy]{datetime}
\usepackage{amsmath} % Soporte de mathmatics
\usepackage{amssymb} % fuentes de mathmatics
\usepackage{array} % Para tablas y eso
\usepackage{caption} % Configuracion de figuras y tablas
\usepackage[dvipsnames]{xcolor} % Para colorear el texto: black, blue, brown, cyan, darkgray, gray, green, lightgray, lime, magenta, olive, orange, pink, purple, red, teal, violet, white, yellow.
\usepackage{graphicx} % Necesario para poner imagenes
\usepackage{enumitem} % Cambiar labels y más flexibilidad para el enumerate
\usepackage{tikz} % para graficar
\usepackage{cancel}
\usepackage{titlesec} % para editar titulos y hacer secciones con formato a medida
\usepackage{ulem}
\usepackage{centernot} % tacha cosas
\usepackage{bbding} % símbolos de donde uso FiveStar
\usepackage{skull} % símbolos de donde uso Skull
% \usepackage{lipsum}
\usepackage{soul} % para tachar en mathmode -> \hbox{\sout{$x+1$}}

% para hacer los graficos tipo grafos
\usetikzlibrary{shapes,arrows.meta, chains, matrix, calc, trees, positioning, fit}
\usetikzlibrary{external,angles,quotes}

\begin{document}
% Definiciones y nuevos comandos:
% =============
\def\partes{\mathcal P}
\def\relacion{\,\mathcal{R}\,}
\def\norelacion{\,\cancel{\relacion}\,}
\def\universo{\mathcal U}
\def\reales{\mathbb R}
\def\naturales{\mathbb N}
\def\enteros{\mathbb Z}
\def\complejos{\mathbb C}
\def\i{\text{i}}
\def\vacio{\varnothing}
\def\union{\cup}
\def\inter{\cap}
\def\y{\land}
\def\o{\lor}
\def\neg{\sim}
\def\entonces{\Rightarrow}
\def\sisolosi{\iff}
\def\clase{\overline}


\def\existe{\,\exists\,}
\def\noexiste{\,\nexists\,}
\def\paratodo{\forall}
\def\distinto{\neq}
\def\en{\in}
\def\talque{\;|\;}

% =====
\def\qvq{\text{ quiero ver que }}

%funciones
\def\imagen{\text{Im}}
\def\dominio{$\text{Dom}$}
\def\comp{\circ}
\def\inv{^{-1}}
\def\infinito{\infty}

% Llaves, paréntesis, contenedores
\newcommand{\llave}[2]{ \left\{ \begin{array}{#1} #2 \end{array}\right. }
\newcommand{\llaves}[2]{ \left\{ \begin{array}{#1} #2 \end{array} \right\} }
\newcommand{\matriz}[2]{\left( \begin{array}{#1} #2 \end{array} \right)}
\newcommand{\deter}[2]{\left| \begin{array}{#1} #2 \end{array} \right|}
\newcommand{\lista}[2][(1)]{\begin{enumerate}[\bf #1]\setlength\itemsep{-0.6ex} #2 \end{enumerate}}
\newcommand{\listal}[2][-0.6ex]{\begin{enumerate}[\bf(a)]\setlength\itemsep{#1} #2 \end{enumerate}}

% naturales
\newcommand{\sumatoria}[2]{\sum\limits_{#1}^{#2}}
\newcommand{\productoria}[2]{\prod\limits_{#1}^{#2}}
\newcommand{\kmasuno}[1]{\underbrace{#1}_{k+1\text{-ésimo}}}
\newcommand{\HI}[1]{\underbrace{#1}_{\text{HI}}}

% enteros
\def\divide{\,|\,}
\def\congruente{\, \equiv \,}
\newcommand{\congruencia}[3]{#1 \equiv #2 \;(\text{mod}\;#3)}
\newcommand{\divset}[2]{\mathcal{D}(#1) = \set{#2}}



% =====
% Miscelanea
% =====
\newcommand{\estabien}{{\color{blue} Consultado, está bien. \checkmark}}
\newcommand{\hacer}{{\color{black!30!red}Hacer!}}
\newcommand{\Hacer}{{\color{black!30!red}\Large Hacer!}}

\def\llamadaI{\stackrel{\cyan{$*^1$}}}
\def\llamadaII{\stackrel{\cyan{$*^2$}}}
\def\llamadaIII{\stackrel{\cyan{$*^3$}}}

% separador
\def\separador{\noindent\rule{\linewidth}{0.4pt}\\}
\def\separadorCorto{\noindent\rule{0.5\linewidth}{0.4pt}\\}

% sección ejercicio con su respectivo formato y contador
\newcounter{ejercicio}[subsubsection] % contador que se resetea en cada sección
\renewcommand{\theejercicio}{\arabic{ejercicio}} % el contador es un número arabic
\newcommand{\ejercicio}{%
	\stepcounter{ejercicio}% incremento en uno
	\titleformat{\section}[runin]{\normalfont\bfseries}{\theejercicio}{1em}{}%
	\section*{\noindent\theejercicio. \noindent}%
}

% Colores
\newcommand{\red}[1]{ {\color{red} \text{#1}}}
\newcommand{\green}[1]{ {\color{olive} \text{#1}}}
\newcommand{\blue}[1]{ {\color{blue} \text{#1}}}
\newcommand{\cyan}[1]{ {\color{cyan} \text{#1}}}
\newcommand{\magenta}[1]{ {\color{magenta} \text{#1}}}

% Conjuntos entre llaves
\newcommand{\set}[1] { \left\{ #1 \right\} }
\newcommand{\parentesis}[1] { \left( #1 \right) }

% Stackrel text
\newcommand{\stacktext}[2]{ \stackrel{\text{#1}}{#2} }
\def\eq?{\stackrel{\text{?}}}

% Flecha con texto
\NewDocumentCommand{\flecha}{m o}{%
	\IfNoValueTF{#2}{%
		\xrightarrow[]{\text{#1}}
	}{
		\xrightarrow[\text{#2}]{\text{#1}}
	}
}
 % idem con las definiciones

\pagestyle{empty} % Para que no muestre el número en pie de página

% Info para armar título.
\title{Práctica 6 de álgebra 1} % título
\author{D. Garraz} % autor
\date{last update: \today} % Cambiar de ser necesario

\maketitle  % para que aprezca el título en el documento

\section{Definiciones y fórmulas útiles}

\begin{itemize}
	\item $G_n = \set{w \en \complejos / w^n = 1} = \set{e^{\frac{2\pi i k}{n}}\ :\ 0\leq k \leq n}$

	\item $(G_n, \cdot)$ es un grupo abeliano, o conmutativo.
	      \begin{itemize}
		      \item $\paratodo w, z \en G_n, w\cdot z = z \cdot w \en G_n$.
		      \item $1 \en G_n,\ w \cdot 1 = 1 \cdot w = w\qquad \paratodo w \en G_n$.
		      \item $w \en G_n \entonces \existe w^{-1} \en G_n,\ w \cdot w^{-1} = w^{-1}\cdot w = 1$
		            \begin{itemize}
			            \item $\conj w \en G_n,\ w \cdot \conj w = |w|^2 = 1 \entonces \conj w = w^{-1}$
		            \end{itemize}
	      \end{itemize}
	\item \textit{Propiedades: $w \en G_n$}
	      \begin{itemize}
		      \item $m \en \enteros$ y $n \dividea m \entonces w^m = 1$
		      \item $\congruencia{m}{m'}{n} \entonces w^m = w^{m'}\quad (w^m = w^{r_n(m)})$
		      \item $n \dividea m \sisolosi G_n \subseteq G_m$
		      \item $G_n \inter G_m = G_{(n:m)}$
		      \item Si $(G, \cdot)$ es un grupo y $\#G = n$ decimos que $G$ siempre es cíclico si
		            $\existe w\en G / G = \set{1,w, w^2,\dots, w^{n-1}}$\\
		            \begin{itemize}
			            \item \textit{Observación: } $G_n$ es cíclico, ej, $w := \ub{e^\frac{2\pi i}{n}}{w_1} = \cos(\frac{2\pi}{n}) + i \sin(\frac{2\pi}{n})$
			            \item $w_1$ genera $G_n = \set{1, w_1, w_1^2,\dots,w_1^{n-1}}$
		            \end{itemize}
		      \item $w$ es raíz primitiva $n-$ésima de la unidad si: $G_n = \set{1,w,w^2,\dots,w^{n-1}} = \set{i^k\ :\ 0\leq k \leq n-1}$\\
		            Ejemplo: $i, -i$ son primitivas de $G_4 = \set{1,i,-1,-i} = \set{i^k\ :\ 0 \leq k \leq 3}$, pero 1 y -1 no son raíces primitivas de $G_4$.
	      \end{itemize}
	\item \textit{Definición: } El orden de $w \en G_n$ se define como $\ord(w) = \text{min}\set{k \en \naturales / w^k = 1}$
	      \begin{itemize}
		      \item \textit{Observación: } Si $w \en G_n \entonces \ord(w) \dividea n$
	      \end{itemize}
\end{itemize}

\subsubsection*{Ejercicios dados en clase:}
\ejercicio

\ejercicio

\newpage

%=========================
% Ejercicios guia
%=========================

\section*{Ejercicios de la guía:}
\setcounter{ejercicio}{0} % Reset the custom counter

%1
\ejercicio

\setcounter{ejercicio}{6}

%7
\ejercicio
Hallar todos los $n \en \naturales$ tales que
\begin{enumerate}[label=\roman*)]
	\item $(\sqrt3 -i)^n = 2^{n-1}(-1 + \sqrt3 i)$ \\
	      \separadorCorto
	      $(\sqrt3 -i)^n = 2^n e^{i\frac{11}{12}\pi n} = 2^{n+1}\cdot 2e^{i \frac{2}{3} \pi}
		      \llave{l}{
			      2^n = 2^n\\
			      \frac{11}{12}\pi n = \frac{2}{3}\pi + 2k \pi \to 11n = 8+8k \flecha{8(k+1)} \boxed{\congruencia{n}{0}{8}}
		      }$

	\item $(-\sqrt3 + i)^n \cdot \parentesis{\frac{1}{2} + \frac{\sqrt3}{2}i}$ es un número real negativo.\\
	      \separadorCorto
	      Un número real negativo tendrá un arg$(z) = \pi$\\
	      $\ub{(-\sqrt3 + i)^n}{2^ne^{i\frac{5}{6}\pi n}} \cdot \ub{\parentesis{\frac{1}{2} + \frac{\sqrt3}{2}i}}{e^{\frac{\pi}{3}i}} =
		      2^ne^{i(\frac{5}{6}n + \frac{1}{3}) \pi} \to \theta = (\frac{5}{6}n + \frac{1}{3}) \pi $\\
	      $\flecha{$\theta = \pi + 2k\pi$}
		      \cancel\pi \frac{5}{6}n + \frac{\cancel\pi}{3} = \cancel\pi + 2k\cancel\pi
		      \flecha{acomodo}[congruencia]
		      \congruencia{5n}{4}{12}
		      \flecha{multiplico}[por 5]
		      \boxed{\congruencia{n}{8}{12}} $

	\item $\text{arg}((-1+i)^{2n}) = \frac{\pi}{2}$ y $\text{arg}((1-\sqrt3 i)^{n-1}) = \frac{2}{3}\pi$
	      \separadorCorto


\end{enumerate}
\end{document}
