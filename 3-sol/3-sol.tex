\documentclass[12pt,a4paper, spanish]{article}
% Sacar draft para que aparezcan las imagenes.
% Opciones: 10pt, 11pt, landscape, twocolumn, fleqn, leqno...
% Opciones de clase: article, report, letter, beamer...

% Paquetes:
% =========
\usepackage[headheight=110pt, top = 2cm, bottom = 2cm, left=1cm, right=1cm]{geometry} %modifico márgenes
\usepackage[T1]{fontenc} % tildes
\usepackage[utf8]{inputenc} % Para poder escribir con tildes en el editor.
\usepackage[english]{babel} % Para cortar las palabras en silabas, creo.
\usepackage[ddmmyyyy]{datetime}
\usepackage{amsmath} % Soporte de mathmatics
\usepackage{amssymb} % fuentes de mathmatics
\usepackage{array} % Para tablas y eso
\usepackage{caption} % Configuracion de figuras y tablas
\usepackage[dvipsnames]{xcolor} % Para colorear el texto: black, blue, brown, cyan, darkgray, gray, green, lightgray, lime, magenta, olive, orange, pink, purple, red, teal, violet, white, yellow.
\usepackage{graphicx} % Necesario para poner imagenes
\usepackage{enumitem} % Cambiar labels y más flexibilidad para el enumerate
\usepackage{multicol} 
\usepackage{tikz} % para graficar
\usepackage{cancel}
\usepackage{titlesec} % para editar titulos y hacer secciones con formato a medida

% para hacer los graficos tipo grafos
\usetikzlibrary{shapes,arrows.meta, chains, matrix, calc, trees, positioning, fit}
\usetikzlibrary{external}

% Definiciones y nuevos comandos:
% =============
\def\partes{\mathcal P}
\def\relacion{\,\mathcal{R}\,}
\def\norelacion{\,\cancel{\relacion}\,}
\def\universo{\mathcal U}
\def\reales{\mathbb R}
\def\naturales{\mathbb N}
\def\enteros{\mathbb Z}
\def\complejos{\mathbb C}
\def\i{\text{i}}
\def\vacio{\varnothing}
\def\union{\cup}
\def\inter{\cap}
\def\y{\land}
\def\o{\lor}
\def\neg{\sim}
\def\entonces{\Rightarrow}
\def\sisolosi{\iff}
\def\clase{\overline}


\def\existe{\,\exists\,}
\def\noexiste{\,\nexists\,}
\def\paratodo{\forall}
\def\distinto{\neq}
\def\en{\in}
\def\talque{\;|\;}

% =====
\def\qvq{\text{ quiero ver que }}

%funciones
\def\imagen{\text{Im}}
\def\dominio{$\text{Dom}$}
\def\comp{\circ}
\def\inv{^{-1}}
\def\infinito{\infty}

% Llaves, paréntesis, contenedores
\newcommand{\llave}[2]{ \left\{ \begin{array}{#1} #2 \end{array}\right. }
\newcommand{\llaves}[2]{ \left\{ \begin{array}{#1} #2 \end{array} \right\} }
\newcommand{\matriz}[2]{\left( \begin{array}{#1} #2 \end{array} \right)}
\newcommand{\deter}[2]{\left| \begin{array}{#1} #2 \end{array} \right|}
\newcommand{\lista}[2][(1)]{\begin{enumerate}[\bf #1]\setlength\itemsep{-0.6ex} #2 \end{enumerate}}
\newcommand{\listal}[2][-0.6ex]{\begin{enumerate}[\bf(a)]\setlength\itemsep{#1} #2 \end{enumerate}}

% naturales
\newcommand{\sumatoria}[2]{\sum\limits_{#1}^{#2}}
\newcommand{\productoria}[2]{\prod\limits_{#1}^{#2}}
\newcommand{\kmasuno}[1]{\underbrace{#1}_{k+1\text{-ésimo}}}
\newcommand{\HI}[1]{\underbrace{#1}_{\text{HI}}}

% enteros
\def\divide{\,|\,}
\def\congruente{\, \equiv \,}
\newcommand{\congruencia}[3]{#1 \equiv #2 \;(\text{mod}\;#3)}
\newcommand{\divset}[2]{\mathcal{D}(#1) = \set{#2}}



% =====
% Miscelanea
% =====
\newcommand{\estabien}{{\color{blue} Consultado, está bien. \checkmark}}
\newcommand{\hacer}{{\color{black!30!red}Hacer!}}
\newcommand{\Hacer}{{\color{black!30!red}\Large Hacer!}}

\def\llamadaI{\stackrel{\cyan{$*^1$}}}
\def\llamadaII{\stackrel{\cyan{$*^2$}}}
\def\llamadaIII{\stackrel{\cyan{$*^3$}}}

% separador
\def\separador{\noindent\rule{\linewidth}{0.4pt}\\}
\def\separadorCorto{\noindent\rule{0.5\linewidth}{0.4pt}\\}

% sección ejercicio con su respectivo formato y contador
\newcounter{ejercicio}[subsubsection] % contador que se resetea en cada sección
\renewcommand{\theejercicio}{\arabic{ejercicio}} % el contador es un número arabic
\newcommand{\ejercicio}{%
	\stepcounter{ejercicio}% incremento en uno
	\titleformat{\section}[runin]{\normalfont\bfseries}{\theejercicio}{1em}{}%
	\section*{\noindent\theejercicio. \noindent}%
}

% Colores
\newcommand{\red}[1]{ {\color{red} \text{#1}}}
\newcommand{\green}[1]{ {\color{olive} \text{#1}}}
\newcommand{\blue}[1]{ {\color{blue} \text{#1}}}
\newcommand{\cyan}[1]{ {\color{cyan} \text{#1}}}
\newcommand{\magenta}[1]{ {\color{magenta} \text{#1}}}

% Conjuntos entre llaves
\newcommand{\set}[1] { \left\{ #1 \right\} }
\newcommand{\parentesis}[1] { \left( #1 \right) }

% Stackrel text
\newcommand{\stacktext}[2]{ \stackrel{\text{#1}}{#2} }
\def\eq?{\stackrel{\text{?}}}

% Flecha con texto
\NewDocumentCommand{\flecha}{m o}{%
	\IfNoValueTF{#2}{%
		\xrightarrow[]{\text{#1}}
	}{
		\xrightarrow[\text{#2}]{\text{#1}}
	}
}
 % idem con las definiciones

\begin{document}

\pagestyle{empty} % Para que no muestre el número en pie de página

% Info para armar título.
\title{Práctica 3 de álgebra 1} % título
\author{D. Garraz} % autor
\date{last update: \today} % Cambiar de ser necesario
% \maketitle  % Para que aprezca el título en el documento
\section*{Ejercicios fuera de la guía}
\textbf{26/4}\\
Sea $\relacion \subseteq \partes(\naturales) \times \partes(\naturales)$ la relación de equivalencia
$\to X \relacion Y \sisolosi X \triangle Y \subseteq \set{4,5,6,7,8}$.\\
¿Cuántos conjuntos hay en la clase de equivalencia de $X = \set{x \en \naturales : x \geq 6}$?

\separadorCorto

\begin{enumerate}
	\item La relación toma valores de $\partes(\naturales)$

	\item Los elementos del conjunto $\relacion \subseteq \partes(\naturales) \times \partes(\naturales)$

	\item El conjunto $X = \set{\foreach \i in {6,...,10}{\i, }\dots}$ es simplemente un elemento de $\partes(\naturales)$.
	      Los conjuntos $Y \in \partes(\naturales)$ tales que $X \relacion Y$ van a ser los conjuntos que junto a $X$ formarán la
	      clase de equivalencia.\\
	      $ \overline{X} = \set{Y \in \partes{\naturales} : X \relacion Y}$
\end{enumerate}

Para tener una relación de equivalencia deben cumplirse:
\begin{itemize}
	\item Reflexividad. $X \triangle X = \vacio \stackrel{\checkmark}\subseteq \set{4,5,6,7,8}$
	\item Simetría. $X \triangle Y \stackrel{\checkmark}= Y \triangle X,\, \paratodo X,Y \in \partes(\naturales)$
	\item Transitividad.
\end{itemize}

Condiciones que debería cumplir un elemeto $Y$ para pertenecer a la la clase de equivalencia, en otras palabras estar relacionado con $X$:\\
Los elementos $\to$\\
$\llaves{l}{
		1, 2, 3 \text{ no deben pertenecer a } Y \flecha{por}[ejemplo]
		\llave{l}{
			X \triangle \underbrace{\set{3,8,9,\dots}}_Y = \set{3,6,7} \cancel{\subseteq} \set{4,5,6,7,8}\\
			X \triangle \underbrace{\set{1,2,3}}_Y = \set{1,2,3,6,7,\dots} \cancel{\subseteq} \set{4,5,6,7,8}\\
		}\\ \hline

		4,5,6,7,8 \text{ pueden o no pertenecer a } Y \flecha{por}[ejemplo]
		\llave{l}{
			X \triangle \underbrace{\set{4,6,8,9,\dots}}_Y = \set{4,7} \stackrel{\checkmark}\subseteq \set{4,5,6,7,8}\\
			X \triangle \underbrace{\set{9,\dots}}_Y = \set{6,7,8} \stackrel{\checkmark}\subseteq \set{4,5,6,7,8}
		}\\ \hline

		9,10, \dots \text{ deben pertenecer a } Y \flecha{por}[ejemplo]
		\llave{l}{
			X \triangle \underbrace{\set{6,7,8}}_Y = \set{9,10,\dots} \cancel\subseteq \set{4,5,6,7,8}\\
			X \triangle \underbrace{\set{10,\dots}}_Y = \set{9} \cancel\subseteq \set{4,5,6,7,8}\\
			X \triangle \underbrace{\set{9,\dots}}_Y = \set{6,7,8} \stackrel{\checkmark}\subseteq \set{4,5,6,7,8}
		}
	}$

Se concluye que la clase de equivalencia será el conjunto $\overline{X}$ (\red{notación inventada}):\\
$\overline{X} = \set{ Y_1  \union \set{9,10,\dots}, Y_2  \union \set{9,10,\dots}, \dots, Y_{32}  \union \set{9,10,\dots}}$
con
$Y_i \en \partes(\set{4,5,6,7,8})\; i \in [1,2^5] $ donde $\#\overline{X} = 2^5$\\

\noindent\red{Entiendo que el enunciado no pedía encontrar la clase de equivalencia, no obstante ¿estaría bien así?}\\


\section*{Ejercicios de la guía}

%1
\ejercicio

%2
\ejercicio
%3
\ejercicio
%4
\ejercicio
%5
\ejercicio
%6
\ejercicio
%7
\ejercicio
%8
\ejercicio


%9
\ejercicio
Si $A$ es un conjunto con $n$ elementos ¿Cuántas relaciones en $A$ hay?
¿Cuántas de ellas son reflexivas?
¿Cuántas de  ellas son simétricas? ¿Cuántas de ellas son reflexivas y simétricas? \\

\separadorCorto

Dado que para dos conjuntos $A = \set{a,b,c}$ y $B = \set{1,2}$ la cantidad de relaciones
que hay entre ellos es igual a la cantidad de subconjuntos de $\partes(A \times B)$, entonces si
$A = \set{1,\cdots, n}$ el cardinal $\# \partes(A \relacion A ) = 2^{n^2}$\\

Las relaciones reflexivas son de la forma $a_i \relacion a_i$, por lo que solo será una relación por cada
elemento del conjunto $\# (A \relacion A)_{ref} = n$. Voy a calcular la cantidad de elementos que tiene
el conjunto $\partes \parentesis{ (A \relacion A)_{ref} }$, porque estoy buscando todos los subconjuntos que puedo
formar con los elementos de $(A \relacion A)_{ref}$, entonces $ \#\partes((A \relacion A)_{ref}) = 2^n$\\
\red{Corroborar}

Las relaciones simétricas serán aquellas que $a_i \relacion a_j \entonces a_j \relacion a_i$. Pensando esto como los elementos de la diagonal
para abajo de una matriz de $n\times n$ tengo $\sumatoria{i=1}{n} i = \frac{n \cdot (n+1)}{2}$ elementos matriciales.\\
$\sumatoria{k = 0}{n} \binom{\frac{n \cdot (n+1)}{2}}{k} = 2^{\frac{n \cdot (n+1)}{2}}$
\red{Corroborar}

$
	\begin{array}{c|c|c|c|c|c|c|c|}
		\multicolumn{1}{c}{} & \multicolumn{1}{c}{a_1} & \multicolumn{1}{c}{a_2} & \multicolumn{1}{c}{a_3} & \multicolumn{1}{c}{\cdots} & \multicolumn{1}{c}{a_{n-2}} & \multicolumn{1}{c}{a_{n-1}} & \multicolumn{1}{c}{a_n} \\ \cline{2-8}
		a_1                  & R,\, S                  & \cdot                   & \cdot                   & \cdots                     & \cdot                       & \cdot                       & \cdot                   \\ \cline{2-8}
		a_2                  & S                       & R,\, S                  & \cdot                   & \cdots                     & \cdot                       & \cdot                       & \cdot                   \\ \cline{2-8}
		a_3                  & S                       & S                       & R,\, S                  & \cdots                     & \cdot                       & \cdot                       & \cdot                   \\ \cline{2-8}
		\vdots               & \vdots                  & \vdots                  & \vdots                  & \ddots                     & \cdot                       & \cdot                       & \cdot                   \\ \cline{2-8}
		a_{n-2}              & S                       & S                       & S                       & \ddots                     & R,\, S                      & \cdot                       & \cdot                   \\ \cline{2-8}
		a_{n-1}              & S                       & S                       & S                       & \ddots                     & S                           & R,\, S                      & \cdot                   \\ \cline{2-8}
		a_n                  & S                       & S                       & S                       & \cdots                     & S                           & S                           & R,\, S                  \\ \cline{2-8}
	\end{array}
$\\

%10
\ejercicio
Sean $A = \set{1,2,3,4,5}$ y $B = \set{1,2,3,4,5,6,7,8,9,10,11,12}$. Sea $\mathcal F$ el conjunto de todas las funciones
$f: A \to B$.
\begin{enumerate}[label=\roman*)]
	\item  ¿Cuántos elementos tiene le conjunto $\mathcal F$
	\item  ¿Cuántos elementos tiene le conjunto $\set{f \en \mathcal F : 10 \en \imagen(f) }$
	\item  ¿Cuántos elementos tiene le conjunto $\set{f \en \mathcal F : 10 \en \imagen(f)}$
	\item  ¿Cuántos elementos tiene le conjunto $\set{f \en \mathcal F : f(1) \en \set{2,4,6} }$
\end{enumerate}

\separadorCorto

Cuando se calcula la cantidad de funciones, haciendo el árbol se puede ver que va a haber $\#\imagen(f)$ de funciones que provienen
de un elemento del dominio. Por lo tanto si tengo un conjunto $A_n$ y uno $B_m$, la cantidad de funciones $f : A \to B$ será
de $m^n$\\

\begin{enumerate}[label=\roman*)]
	\item $\# \mathcal F = 12^5$

	\item $\# \mathcal F = 11^5$

	\item Tengo una que va a parar al $10$ y cuento que queda. Por ejemplo si $f(2) = 10$: $A = \set{1,\cancel{2},3,4,5}$ y $B = \set{1,2,3,4,5,6,7,8,9,10,11,12}$.
	      Por lo tanto tengo $\# \mathcal F = 12^4 \cdot \underbrace{\normalsize 1}_{f (2) = 10}$\\
	      \red{Corroborar}

	\item Me dicen que $f(\set1) = \set{2,4,6}$,
	      Si lo pienso como el anterior ahora tengo 3 veces más combinaciones, entonces
	      $\# \mathcal F = 12^4 \cdot \underbrace{\tiny 3}_{ f (\set{1})= \set{2,4,6}}$
\end{enumerate}

%11
\ejercicio
Sean $A = \set{1,2,3,4,5,6,7}$ y $\set{8,9,10,11,12,13,14}$.
\begin{enumerate}[label=\roman*)]
	\item ¿Cuántas funciones biyectivas $f: A \to B$ hay?
	\item ¿Cuántas funciones biyectivas $f: A \to B$ hay tales que $f(\set{1,2,3}) = \set{12,13,14}$?
\end{enumerate}

\separadorCorto

Cuando cuento funciones biyectivas, el ejercicio es como reordenar los elementos del conjunto de llegada de todas las
formas posibles. Dado un conjunto Im($f$), la cantidad de funciones biyectivas será $\#\imagen(f)$

\begin{enumerate}[label=\roman*)]
	\item Hay $7!$ funciones biyectivas.
	\item Dado que hay 3 valores fijos, juego con los 4 valores restantes, por lo tanto habrá $4!$ funciones biyectivas
\end{enumerate}

%12
\ejercicio
¿Cuántos números de 5 cifras distintas se pueden armar usando los dígitos del 1 al 5?
¿ Y usando los dígitos del 1 al 7? ¿ Y usando los dígitos del 1 al 7 de manera que el dígito de las centenas no sea el 2?
\begin{enumerate}[label=\arabic*)]
	\item Hay que usar $\set{1,2,3,4,5}$ y reordenarlos de todas las formas posibles. $5!$

	\item Hay que usar $\set{1,2,3,4,5,6,7}$ y ver de cuantas formas posibles pueden ponerse en 5 lugares:
	      $\llave{c c c c c}{
			      \_ & \_ & \_& \_& \_ \\
			      1 & 2 &3 &4 &5
		      }$, dado que no puedo repetir, a medida que voy llenando los valores, me voy quedando cada vez con menos valores
	      para elegir del conjunto de datos, por lo tanto queda algo así:
	      $\llave{c c c c c}{
			      \#7 & \#6 & \#5 & \#4 &\#3 \\
			      \downarrow &\downarrow &\downarrow &\downarrow &\downarrow\\
			      \_ & \_ & \_& \_& \_ \\
			      1 & 2 &3 &4 &5
		      }\to$ Tengo $7 \cdot 6 \cdot 5 \cdot 4 \cdot 3 = \frac{7!}{2!} $ \red{interpretar?}

	\item Parecido al anterior pero fijo el 2 en el dígito de las centenas:\\
	      $\llave{c c c c c}{
			      \#6 & \#5 & \#4 & \#1 &\#3 \\
			      \downarrow &\downarrow &\downarrow &\downarrow &\downarrow\\
			      \_ & \_ & \_& 2 & \_ \\
			      1 & 2 &3 &4 &5
		      }
		      \to $ Tengo $6 \cdot 5 \cdot 4 \cdot 1 \cdot 3 = \frac{6!}{2!} $ \red{interpretar?}
\end{enumerate}

%13
\ejercicio
Sean $A = \set{1,2,3,4,5,6,7}$ y $B = \set{1,2,3,4,5,6,7,8,9,10}$.
\begin{enumerate}[label=\roman*)]
	\item ¿Cuántas funciones inyectivas $f; A \to B$ hay?
	\item ¿Cuántas de ellas son tales que $f(1)$ es par?
	\item ¿Y cuántas tales que $f(1)$ y $f(2)$ son pares?
\end{enumerate}

\separadorCorto

\begin{enumerate}[label=\roman*)]
	\item Una pregunta equivalente a si tengo 10 pelotitas distintas y 7 cajitas cómo puedo ordenarlas.
	      $\llave{c c c c c c c}{
			      \#10 & \#9 & \#8 & \#7 & \#6 & \#5 & \#4 \\
			      \downarrow &\downarrow & \downarrow &\downarrow &\downarrow &\downarrow &\downarrow\\
			      f(1) & f(2) &f(3) &f(4) &f(5) &f(6) &f(7)
		      }
		      \to\frac{10!}{3!} = \magenta{$\frac{\#B}{\#B - \#A}$
		      }$

	\item Hay 5 números pares para elegir como imagen de $f(1)$\\
	      $\llave{c c c c c c c}{
			      \#5 & \#9 & \#8 & \#7 & \#6 & \#5 & \#4 \\
			      \downarrow &\downarrow & \downarrow &\downarrow &\downarrow &\downarrow &\downarrow\\
			      f(1) & f(2) &f(3) &f(4) &f(5) &f(6) &f(7)
		      }
		      \to 5 \cdot \frac{9!}{3!}$

	\item Hay 5 números pares para elegir como imagen de $f(1)$, luego habrá 4 números pares para $f(2)$\\
	      $\llave{c c c c c c c}{
			      \#5 & \#4 & \#8 & \#7 & \#6 & \#5 & \#4 \\
			      \downarrow &\downarrow & \downarrow &\downarrow &\downarrow &\downarrow &\downarrow\\
			      f(1) & f(2) &f(3) &f(4) &f(5) &f(6) &f(7)
		      }
		      \to 5 \cdot 4 \cdot \frac{8!}{3!}$
\end{enumerate}

%14
\ejercicio
¿Cuántas funciones biyectivas $f: \set{1,2,3,4,5,6,7} \to \set{1,2,3,4,5,6,7}$ tales que $f(\set{1,2,3}) \subseteq \set{3,4,5,6,7}$ hay?

\separadorCorto

Primero veo la condición  $f(\set{1,2,3}) \subseteq \set{3,4,5,6,7}$, donde podría formar $\frac{5!}{(5-3)!} = 60$ combinaciones biyectivas.
Para obtener la cantidad de funciones pedidas, tengo que usar todos los valores del $\set{1,2,3,4,5,6,7}$. Primero fijo la cantidad de
valores que pueden tomar $f(\set{1,2,3}) \subseteq \set{3,4,5,6,7}$ luego lo que reste.
\\
$\llave{c c c c c c c}{
		\#5 & \#4 & \#3 & \#4 & \#3 & \#2 & \#1 \\
		\downarrow & \downarrow & \downarrow &\downarrow &\downarrow &\downarrow &\downarrow\\
		f(1) & f(2) &f(3) &f(4) &f(5) &f(6) &f(7)\\
		\multicolumn{3}{c}{\blue{Condiciones pedidas}} & \multicolumn{4}{c}{\green{Lo que resta para completar}}
	}
	\to \blue{$5 \cdot 4 \cdot 3$} \cdot \green{$4 \cdot 3 \cdot 2 \cdot 1$} = \frac{5!}{(5-3)!} \cdot 4!$

%15
\ejercicio
Sea $A = \set{f : \set{1,2,3,4} \to \set{1,2,3,4,5,6,7,8} \text{ tal que $f$ es una función inyectiva}}.$\\
Sea $\relacion$ la relación de equivalencia en $A$ definida por: $f \relacion g \sisolosi f(1) + f(2) = g(1) + g(2)$.\\
Sea $f \en A$ la función definida por $f(n) = n+2$ ¿Cuántos elementos tiene su clase de equivalencia?

\separadorCorto

Uf... enunciado feo.\\

\red{¿Qué onda esa $g$?}

%16
\ejercicio

Determinar cuántas funciones $f : \set{1,2,3,4,5,6,7,8} \to \set{\foreach \i in {1,...,11}{ \i, }12}$ satisfacen
simultáneamente las condiciones:\\
\begin{multicols}{3}
	\begin{itemize}
		\item $f$ es inyectiva,
		\item $f(5) + f(6) = 6$,
		\item $f(1) \leq 6$.
	\end{itemize}
\end{multicols}

\separadorCorto

\begin{itemize}
	\item $f$ inyectiva hace que mi conjunto de llegada se reduzca en 1 con cada elección.

	\item Si $f(5) + f(6) = 6$ entonces $f: \set{5,6} \to \set{1,2,4,5}$. Una vez que $f(5)$ tome
	      un valor de los 4 posibles e.g. $f(5) = 1 \flecha{condiciona}[única opción] f(6) = 5 $

	\item $f(1) \leq 6 \to f : \set{1} \to \set{\cancel{1},2,3,\cancel{4},5,6}$ donde cancele el 1
	      y el 4, para sacar 2 números que sí o sí deben irse en la condición de $f(5) + f(6) = 6$. Por lo
	      tanto $f(1)$ puede tomar 4 valores.
\end{itemize}

$\llave{c c c c c c c c}
	{
		\#4 & \#9 & \#8 & \#7 & \#4 & \#1 & \#6 & \#5 \\
		\downarrow & \downarrow & \downarrow &\downarrow & \downarrow & \downarrow &\downarrow &\downarrow\\
		f(1) & f(2) &f(3) &f(4) &f(5) &f(6) & f(7) & f(8)
	}
	\to 4 \cdot 9 \cdot 8 \cdot 7 \cdot 4 \cdot 1 \cdot 6 \cdot 5 = 4 \cdot 4 \cdot \frac{9!}{4!} = 241.920$\\

\red{Siento todo esto muy artesanal y poco justificable suficientemente \it{mathy-snobby}}

\separador

{\it \underline{Número combinatorio}}

%17
\ejercicio

\begin{enumerate}[label=\roman*)]
	\item ¿Cuántos subconjuntos de 4 elementos tiene el conjunto $\set{\foreach \i in {1,...,6}{ \i, }7} $
	\item ¿ Y si se pide que 1 pertenezca al subconjunto?
	\item ¿ Y si se pide que 1 no pertenezca al subconjunto?
	\item ¿ Y si se pide que 1 o 2 pertenezca al subconjunto, pero no simultáneamente los dos?
\end{enumerate}

\separadorCorto

El problema de tomar $k$ elementos de un conjunto de $n$ elementos se calcula con $\binom{n}{k} = \frac{n!}{k!(n-k)!}$

\begin{enumerate}[label=\roman*)]
	\item  $\binom{7}{4} = \frac{7!}{4!(7-4)!} = \frac{7\cdot \cancel{6} \cdot 5 \cdot \cancel{4!}}{\cancel{4!}(\cancel{3!})} = 35$

	\item $\binom{6}{3} = \frac{6!}{3!\cdot 3!} = 20$.

	\item $\binom{6}{4} = \frac{6!}{4!\cdot 2!} = 15$.

	\item $\binom{5}{3} \cdot 2 = \frac{5!}{3!\cdot 2!} \cdot 2 = 20$
\end{enumerate}

%18
\ejercicio
Sea $A = \set{n \in \naturales : n \leq 20}$. Calcular la cantidad de subconjuntos $B \subseteq A$ que cumplen las siguientes condiciones:
\begin{enumerate}[label=\roman*)]
	\item $B$ tiene 10 elementos y contiene exactamente 4 múltiplos de 3.
	\item $B$ tiene 5 elementos y no hay dos elementos de $B$ cuya suma sea impar.
\end{enumerate}

\separadorCorto

El conjunto $A = \set{\foreach \i in {1,...,19}{\i, }20}$\\
\begin{enumerate}[label=\roman*)]
	\item
	      $ \flecha{multiplos}[de 3] C = \set{3, 6, 9, 12, 15, 18}$, agarro 4 elementos del conjunto $C$ y luego 6 de los restantes del conjunto $A$ sin contar
	      el múltiplo de 3 que ya usé.\\
	      $\llave{l}{
			      \binom{6}{4} \cdot \binom{9}{6} = \frac{\cancel{6!}}{4! 2!} \cdot \frac{9!}{\cancel{6!} 3!} \flecha{simplificando} 9 \cdot 4 \cdot 7 \cdot 5 = 1260  \\
			      \red{Verificar y preguntar por la justificación.}
		      }$

	\item
	      La condición de que la suma \textit{no sea impar} implica que todos los elementos deben ser par o todos impar.\\
	      $\llave{l}{
			      \flecha{todos}[pares]   \set{2,4,6,8,10,12,14,16,18, 20} \flecha{10 elementos}[quiero 5] \binom{10}{5} = \frac{10!}{5!\cdot 5!} = 252\\
			      \flecha{todos}[impares] \set{1,3,5,9,11,13,15,17,19} \flecha{9 elementos}[quiero 5] \binom{9}{5} = \frac{9!}{5!\cdot 4!} = 126
		      }$
\end{enumerate}

%19
\ejercicio
Dadas dos rectas paralelas en el plano, se marcan $n$ puntos distintos sobre una y $m$ puntos distintos sobre la otra.
¿Cuántos triángulos se pueden formar con vértices en esos puntos?\\
\hacer

%20
\ejercicio
Determinar cuántas funciones $f: \set{1,2,3,\dots,11} \to \set{1,2,3,\dots,16}$ satisfacen simultáneamente las condiciones:
\begin{multicols}{3}
	\begin{itemize}
		\item $f$ es inyectiva,
		\item Si $n$ es par, $f(n)$ es par,
		\item $f(1) \leq f(3) \leq f(5) \leq f(7)$.
	\end{itemize}
\end{multicols}

\separadorCorto

\begin{itemize}
	\item
	      La función es inyectiva y cuando \textit{inyecto un conjunto de m elementos en uno de n elementos } $\to \frac{m!}{(m-n)!}$.

	\item
	      Para cumplir la segunda condición el Dom($f$) tengo 5 números par $\set{2,4,6,8,10}$ y en el codominio tengo 8 números par
	      $\set{\foreach \i in {2,4,...,14}{\i,}16}$ al \textit{inyectar} obtengo $\frac{8!}{(8-5)!}$ permutaciones.

	\item
	      La condición de las desigualdades se piensa con los elementos de la Im($f$) restantes después de la \text{inyección}, que son $16 - 5 = 11$.
	      De esos 11 elementos quiero tomar 4. El cuántas formas distintas de tomar 4 elementos de un conjunto de 11 elementos se calcula con $\binom{11}{4}$,
	      número de combinación que cumple las desigualdades, porque todos los números son \underline{distintos}. Para la combinación
	      \textbf{no hay órden}, elegir $\set{16,1,15,13}$ es lo mismo
	      \footnote{Que sea lo mismo quiere decir que no lo cuenta nuevamente, el contador aumenta solo si cambian los
		      elementos y \underline{no} el lugar de los elementos}
	      que $\set{1,16,13,15}$. Es por eso que \textit{con $4$ elementos seleccionados}
	      solo hay \underline{una} \textit{permutación} que cumple las desigualdades; en este ejemplo sería $\set{1,13,15,16}$

	\item
	      Por último inyecto los número del dominio restantes $\set{9,11}$ en los 7 elementos de Im($f$) que quedaron luego de la combinación de las
	      desigualdades $\to \frac{7!}{(7-2)!}$\\

	      Concluyendo: Habrían $\frac{8!}{(8-5)!} \cdot \binom{11}{4} \cdot \frac{7!}{(7-2)!} = 93.139.200$\\
	      \red{Corroborar}
\end{itemize}


%21
\ejercicio
¿Cuántos anagramas tienen las palabras \textit{estudio, elementos} y \textit{combinatorio}\\
\separadorCorto

El anagrama equivale a permutar los elementos. Si no hay letras repetidas es una biyección $\#(letras)!$\\
La palabra \textit{estudio} tiene $7!$ anagramas.\\

\textit{Elementos} tiene 3 letras \underline{e}, por lo tanto los elementos no repetidos son 6 $\set{l,m,n,t,o,s}$; esto
es una \textit{inyección}
\footnote{Primero ubico lo que no está repetido. Luego agrego, en una dada posición, a eso 3 o más elementos repetidos. Esta última acción
	no altera la cantidad de permutaciones. Pensar en esto: lmntosEEE cuenta como lmntos\_\_\_.}
$ \to \frac{9!}{(9-6)!} = \frac{9!}{3!}$.\\
También puedo pensar esto con combinatoria: Primero ubico a las 3 letras \textit{e} en los lugares de las letras, por ejemplo
$\llave{c c c c c c c c c} % hay 9 en total
	{
		e & \_ & e & \_ & e & \_ & \_ & \_ & \_\\
		1 & 2 & 3 & 4 & 5 & 6 & 7 & 8 & 9
	} \to$
donde esta es una de un total de $\binom{9}{3}$ formas de hacer eso, y los elementos que quedan en el conjunto de letras se \textit{inyectan}
en los lugares vacíos que quedan, en este caso tengo 6 elementos para ubicar en 6 lugares, lo que sería una biyección $\#(letras)!$.\\
$\to \binom{9}{3} \cdot 6!  = \frac{9!}{3!}$

\textit{Combinatorio} tiene repetidas las letras \textit{i} (x2) y la \textit{o} (x3). Tengo un conjunto de 7 elementos $\set{c,m.b,n,a,t,r}$
sin repetición. Puedo ubicar las letras con combinación en los 12 lugares \textit{o} y luego las \textit{i} en los 9 lugares restantes. Una vez hecho
eso puedo \textit{inyectar (biyectar?)} las letras no repetidas restantes:\\
$\to \binom{12}{3} \cdot \binom{9}{2} \cdot 7! =
	\underbrace{\normalsize \frac{12!}{3! 2!} }_{
		\text{ notar \footnotemark}
	}
	= \frac{\foreach \i in {12,11,...,5}{\i \cdot}4}{2} = 39.916.800$

\footnotetext{Esto es el total de biyecciones dividido entre las cantidades de repeticiones de los elementos en cuestión.}

%22
\ejercicio

¿Cuántas palabras se pueden formar permutando las letras de $cuadros$
\begin{enumerate}[label=\roman*)]
	\item con la condición de que todas las vocales estén juntas?
	\item con la condición de que las consonantes mantengan el orden relativo original?
	\item con la condición de que nunca haya dos (o más) consonantes juntas?
\end{enumerate}

\separadorCorto

El conjunto de consonantes es $C = \set{c,d,r,s}$ y de vocales $V = \set{u,a,o}$
\begin{enumerate}[label=\roman*)]
	\item  Para que las vocales estén juntas tengo que considerar las permutaciones dentro del conjunto $V$, que son iguales a 3! y ahora \textit{inyecto} el conjunto de
	      5 elementos $CV = \set{c,d,r,s,{uao}}$ que da 5!.\\
	      Finalmente se pueden formar $3! \cdot 5!$ palabras con la condición pedida.

	\item \red{¿Qué sería el orden relativo?}

	\item \hacer

\end{enumerate}

%23
\ejercicio

Con la palabra $polinomios$,

\begin{enumerate}[label=\roman*)]
	\item ¿Cuántos anagramas pueden formarse en las que las 2 letras $i$ no estén juntas?
	\item ¿Cuántos anagramas puede formarse en los que la letra $n$ aparezca a la izquierda de la letra $s$ y
	      la letra $s$ aparezca a la izquierda de la letra $p$ (no necesariamente una al lado de la otra)?
\end{enumerate}

\separadorCorto

\begin{enumerate}[label=\roman*)]
	\item
	      Tengo 10 letras, $\set{p,l,n,m,s,o,o,o,i,i}$. Para que no hayan $"ii"$ calculo $\binom{10}{3} = 120$, pensando que en un conjunto de 3, siempre
	      puedo poner las letras $"\underline{i}\, \_ \, \underline{i} "$. Para cada uno de estas 120 configuraciones de la pinta: \\
	      $\llave{c c c c c c c c c c} % hay 9 en total
		      {
			      \magenta{i} & \magenta{\_} & \magenta{i} & \_ & \_ & \_ & \_ &\_ & \_ & \_\\
			      \_ & \_ & \magenta{i} & \magenta{\_} & \_ & \_ & \_ &\magenta{i} & \_ & \_\\
			      \_ & \_ & \_ & \magenta{i} & \_ & \magenta{\_} & \_ &\_ & \_ & \magenta{i}\\ \hline
			      1 & 2 & 3 & 4 & 5 & 6 & 7 & 8 & 9 & 10
		      } \to
	      $ tengo que rellenar con 8 letras los lugares que sobran, teniendo en cuenta las repeticiones de las $"o"$: $\binom{10}{3} \cdot \frac{8!}{3!}$

	\item
	      Tengo 10 letras, $\set{p,l,n,m,s,o,o,o,i,i}$. Para que se forme  $"n\dots s \dots p"$ calculo $\binom{10}{3} = 120$, pensando que en un conjunto de 3, siempre
	      puedo poner las letras $"\underline{n}\dots \underline s \dots \underline{p} "$. Para cada uno de estas 120 configuraciones de la pinta: \\
	      $\llave{c c c c c c c c c c} % hay 9 en total
		      {
			      \magenta{n} & \magenta{s} & \magenta{p} & \_ & \_ & \_ & \_ &\_ & \_ & \_\\
			      \_ & \_ & \magenta{n} & \magenta{s} & \_ & \_ & \_ &\magenta{p} & \_ & \_\\
			      \_ & \_ & \_ & \magenta{n} & \_ & \magenta{s} & \_ &\_ & \_ & \magenta{p}\\ \hline
			      1 & 2 & 3 & 4 & 5 & 6 & 7 & 8 & 9 & 10
		      } \to
	      $ tengo que rellenar con 7 letras los lugares que sobran, teniendo en cuenta las repeticiones de las $"o"$ y de las $"i"$:
	      $\binom{10}{3} \cdot \frac{7!}{3!2!}$
\end{enumerate}


%24
\ejercicio

%25
\ejercicio

%26
\ejercicio

%27
\ejercicio

%28
\ejercicio
En este ejercicio no hace falta usar inducción.
\begin{enumerate}[label=\roman*)]
	\item Probar que $\sumatoria{k = 0}{n} \binom{n}{k}^2 = \binom{2n}{n}$. \qquad sug: $\binom{n}{k} = \binom{n}{n-k}$.
	\item Probar que $\sumatoria{k = 0}{n} (-1)^k \binom{n}{k} = 0$.
	\item Probar que $\sumatoria{k = 0}{2n} \binom{2n}{k} = 4^n$ y deducir que $\binom{2n}{n} < 4^n$.
	\item Calcular $\sumatoria{k = 0}{2n+1} \binom{2n+1}{k}$ y deducir que $\sumatoria{k=0}{n} \binom{2n+1}{k}$.
\end{enumerate}

\separadorCorto

Binomio de Newton: $(x + y)^n = \sumatoria{k=0}{n} \binom{n}{k} x^n y^{n-k}$
\begin{enumerate}[label=\roman*)]
	\item

	\item  Binomio $\to
		      \llaves{lcr}{
			      x &=& 1\\
			      y &=& -1
		      } \to 0^n = \sumatoria{k=0}{n} \binom{n}{k} 1^n (-1)^{n-k}  = \sumatoria{k=0}{n} \binom{n}{k} (-1)^{n-k} = 0 \to\\
		      \llave{l}{
		      \flecha{si $n$ es par}[primer término positivo] \sumatoria{k=0}{n} (-1)^k  \binom{n}{k} =
		      \binom{n}{0} - \binom{n}{1} + \dots + (-1)^\frac{n}{2} \binom{n}{\frac{n}{2}} +\dots  - \binom{n}{k-1} + \binom{n}{n} \to\\
		      \flecha{uso sugerencia}[$\binom{n}{k} = \binom{n}{n-k}$ ]
		      2\cdot\binom{n}{0} - 2 \cdot \binom{n}{1} + \dots + 2 \cdot (-1)^{\frac{n}{2} + 1} \binom{n}{\frac{n}{2} + 1} + (-1)^\frac{n}{2} \binom{n}{\frac{n}{2}} = 0 \red{¿Qué onda?}\\

		      \flecha{si $n$ es impar}[primer término negativo] \sumatoria{k=0}{n} (-1)^{k+1} \binom{n}{k} =
		      - \binom{n}{0} + \binom{n}{1} - \dots  - \binom{n}{k-1} + \binom{n}{n} \flecha{uso sugeerencia}[$\binom{n}{k} = \binom{n}{n-k}$ \checkmark] 0
		      }$

	\item \hacer
	\item \hacer
\end{enumerate}

%29
\ejercicio

Sea $X = \set{\foreach \i in {1,...,19}{\i,}20}$, y sea $R$ la relación de orden en $\partes(X)$ definida por:
$A \relacion B \sisolosi A - B = \vacio.$\\
¿Cuántos conjuntos $A \en \partes(X)$ cumplen simultáneamente $\#A \geq 2$ y $A \relacion \set{1,2,3,4,5,6,7,8,9}$?

%30
\ejercicio
Sea $X = \set{1,2,3,4,5,\red{$\cancelto{6}{5}$},7,8,9,10}$, y sea $R$ la relación de equivalencia en $\partes(X)$ definida por:
$A \relacion B \sisolosi A \inter \set{1,2,3} = B \inter \set{1,2,3}.$\\
¿Cuántos conjuntos $B \en \partes(X)$ de exactamente 5 elementos tiene la clase de equivalencia $\overline A $ de $A = \set{1,3,5}$?

\separadorCorto

Como $A$ tiene al 1 y al 3, los elementos $B$, \textit{conjuntos en este caso}, pertenecientes a la clase $\overline A$
deberían cumplir que si $B \subseteq \overline A \entonces
	\llaves{cc}{
		1 \en B&\\
		3 \en B&\\
		2 \notin B &\to \text{ si } 2 \en B \entonces A \cancel\relacion B
	} $.\\
Los conjuntos de 5 elementos serán de la forma:\\
$\llave{c c c c c} % hay 5 en total
	{
		1 & 3 & \_ & \_ & \_   \\
	} \flecha{5 elementos}[$ \inter \set{1,2,3} \stackrel{\checkmark}= \set{1,3}$] \binom{7}{3} = 35$. Los 7 números usados son $\set{4,5,\red{6},7,8,9,10}$ \\
\red{¿Es solo eso o interpreto mal la $\relacion$ u otra cosa?}

%31
\ejercicio
Sean $X = \set{n \en \naturales : n\leq 100}$ y $A = \set{1}$ ¿Cuántos subconjuntos $B\subseteq X$ satisfacen que el conjunto $A \triangle B$
tiene a lo sumo 2 elementos?

\separadorCorto
...\\
\textit{a lo sumo = como mucho = como máximo}\\
\textit{al menos =  por poco = como mínimo}\\
...\\

La diferencia simétrica es la unión de los elementos no comunes a los conjuntos $A$ y $B$. Si me piden que:\\
$\#( A \triangle B) \leq 2 \entonces B =
	\llave{ccl}{
		1 \en B & \to & \#B \leq 3
		\flecha{Busco conjuntos}[de la forma]
		\llave{lll}{
			\underline{1}\; \_ \;\_ &\flecha{el 1 está usado}[quedan  99. Elijo 2.] &   \binom{99}{2}\\
			\underline{1} \;\_      &\flecha{el 1 está usado}[quedan  99. Elijo 1.] &   \binom{99}{1}\\
			\underline{1}           &\flecha{el 1 está usado}[quedan  99. Elijo 0.] &  \binom{99}{0}
		}\\
		1 \notin B & \to & \#B \leq 1
		\flecha{Busco conjuntos}[de la forma]
		\llave{lll}{
			\_ & \flecha{tengo 99 números para}[elegir \red{$1 \notin B$}. Elijo 1]     & \binom{99}{1}\\
			\vacio & \flecha{tengo 99 números para}[elegir \red{$1 \notin B$}. Elijo 0] & \binom{99}{0}
		}\\
	}$

Por último habría un total de $ \binom{99}{2} + \binom{99}{1} + \binom{99}{0} +\binom{99}{1} + \binom{99}{0}$ subconjuntos $B \subseteq X$
para cumplir lo pedido.


%32
\ejercicio

\begin{enumerate}[label=\roman*)]
	\item Sea $A$ un conjunto con $2n$ elementos. ¿Cuántas relaciones de equivalencia pueden definirse en $A$ que cumplan la condición
	      de que para todo $a \en A$ la clase de equivalencia de $a$ tenga $n$ elementos?

	\item Sea $A$ un conjunto con $3n$ elementos. ¿Cuántas relaciones de equivalencia pueden definirse en $A$ que cumplan la condición
	      de que para todo $a \en A$ la clase de equivalencia de $a$ tenga $n$ elementos?

\end{enumerate}

\end{document}
