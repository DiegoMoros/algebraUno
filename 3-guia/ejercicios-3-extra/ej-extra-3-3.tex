\begin{enunciado}{\ejExtra}
  Calcular la cantidad de anagramas de HIPOPOTAMO que preserven el orden
  relativo orifinal de las letras I y A, es decir, los que tengan la I
  a la izquierda de la A.
\end{enunciado}

No sé si ésta es la mejor forma de hacer esto, pero es la forma que
se me ocurrió.\par
En total hay \underline{10 letras}, \red{con repeticiones}.
Primero voy a atacar el tema de la posición relativa de la I y la A.
Calculo todas las posibles posiciones respetando que la I esté a las izquierda de la A.\par
La I fija y cuento posibles lugares para la A:
$$
  \begin{array}{cccccccccc | l}
    1  & 2  & 3  & 4  & 5  & 6  & 7  & 8  & 9  & 10 &                                   \\ \hline
    I  & A  & \_ & \_ & \_ & \_ & \_ & \_ & \_ & \_ & \to \text{ 9 posibles posiciones} \\
    \_ & I  & A  & \_ & \_ & \_ & \_ & \_ & \_ & \_ & \to \text{ 8 posibles posiciones} \\
    \_ & \_ & I  & A  & \_ & \_ & \_ & \_ & \_ & \_ & \to \text{ 7 posibles posiciones} \\
    \_ & \_ & \_ & I  & A  & \_ & \_ & \_ & \_ & \_ & \to \text{ 6 posibles posiciones} \\
    \_ & \_ & \_ & \_ & I  & A  & \_ & \_ & \_ & \_ & \to \text{ 5 posibles posiciones} \\
    \_ & \_ & \_ & \_ & \_ & I  & A  & \_ & \_ & \_ & \to \text{ 4 posibles posiciones} \\
    \_ & \_ & \_ & \_ & \_ & \_ & I  & A  & \_ & \_ & \to \text{ 3 posibles posiciones} \\
    \_ & \_ & \_ & \_ & \_ & \_ & \_ & I  & A  & \_ & \to \text{ 2 posibles posiciones} \\
    \_ & \_ & \_ & \_ & \_ & \_ & \_ & \_ & I  & A  & \to \text{ 1 posible posición}
  \end{array}
$$
De ahí salen en total $1 + 2+3+4+5+6+7+8+9 = \sumatoria{i=1}{9} i = \frac{10 \cdot 9}{2} = 45$ lugares los cuales hay que
rellenar con las letras faltantes.\par
Para cada una de las 45 posiciones de la I y la A correctamente ubicadas tengo que ubicar 8 letras, de donde saldrían
$8!$ posiciones, peeeeero, al tener repeticiones y para no contar cosas de más, divido por la cantidad de letras repetidas tanto para la O como para la P:\par
$$
  \text{Total de anagramas: } 45 \cdot (\frac{8!}{\ub{3!}{\text{O}} \cdot \ub{2!}{\text{P}} }).
$$

\begin{aportes}
\item \aporte{https://github.com/nad-garraz/algebraUno}{Nad Garraz \github}
\end{aportes}
