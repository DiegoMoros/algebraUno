\ejercicio
¿Cuántos anagramas tienen las palabras \textit{estudio, elementos} y \textit{combinatorio}\\
\separadorCorto

El anagrama equivale a permutar los elementos. Si no hay letras repetidas es una biyección $\#(letras)!$\\
La palabra \textit{estudio} tiene $7!$ anagramas.\\

\textit{Elementos} tiene 3 letras \underline{e}, por lo tanto los elementos no repetidos son 6 $\set{l,m,n,t,o,s}$; esto
es una \textit{inyección}
\footnote{Primero ubico lo que no está repetido. Luego agrego, en una dada posición, a eso 3 o más elementos repetidos. Esta última acción
	no altera la cantidad de permutaciones. Pensar en esto: lmntosEEE cuenta como lmntos\_\_\_.}
$ \to \frac{9!}{(9-6)!} = \frac{9!}{3!}$.\\
También puedo pensar esto con combinatoria: Primero ubico a las 3 letras \textit{e} en los lugares de las letras, por ejemplo
$\llave{c c c c c c c c c} % hay 9 en total
	{
		e & \_ & e & \_ & e & \_ & \_ & \_ & \_\\
		1 & 2 & 3 & 4 & 5 & 6 & 7 & 8 & 9
	} \to$
donde esta es una de un total de $\binom{9}{3}$ formas de hacer eso, y los elementos que quedan en el conjunto de letras se \textit{inyectan}
en los lugares vacíos que quedan, en este caso tengo 6 elementos para ubicar en 6 lugares, lo que sería una biyección $\#(letras)!$.\\
$\to \binom{9}{3} \cdot 6!  = \frac{9!}{3!}$

\textit{Combinatorio} tiene repetidas las letras \textit{i} (x2) y la \textit{o} (x3). Tengo un conjunto de 7 elementos $\set{c,m.b,n,a,t,r}$
sin repetición. Puedo ubicar las letras con combinación en los 12 lugares \textit{o} y luego las \textit{i} en los 9 lugares restantes. Una vez hecho
eso puedo \textit{inyectar (biyectar?)} las letras no repetidas restantes:\\
$\to \binom{12}{3} \cdot \binom{9}{2} \cdot 7! =
	\underbrace{\normalsize \frac{12!}{3! 2!} }_{
		\text{ notar \footnotemark}
	}
	= \frac{\foreach \i in {12,11,...,5}{\i \cdot}4}{2} = 39.916.800$

\footnotetext{Esto es el total de biyecciones dividido entre las cantidades de repeticiones de los elementos en cuestión.}
\newpage
