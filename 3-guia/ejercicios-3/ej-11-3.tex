\ejercicio
Sean $A = \set{1,2,3,4,5,6,7}$ y $\set{8,9,10,11,12,13,14}$.
\begin{enumerate}[label=\roman*)]
	\item ¿Cuántas funciones biyectivas $f: A \to B$ hay?
	\item ¿Cuántas funciones biyectivas $f: A \to B$ hay tales que $f(\set{1,2,3}) = \set{12,13,14}$?
\end{enumerate}

\separadorCorto

Cuando cuento funciones biyectivas, el ejercicio es como reordenar los elementos del conjunto de llegada de todas las
formas posibles. Dado un conjunto $\im(f)$, la cantidad de funciones biyectivas será $\#\im(f)$

\begin{enumerate}[label=\roman*)]
	\item Hay $7!$ funciones biyectivas.
	\item Dado que hay 3 valores fijos, juego con los 4 valores restantes, por lo tanto habrá $4!$ funciones biyectivas
\end{enumerate}
