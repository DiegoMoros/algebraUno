\begin{enunciado}{\ejercicio}
  Sean $A = \set{1,2,3,4,5,6,7}$ y $\set{8,9,10,11,12,13,14}$.
  \begin{enumerate}[label=\roman*)]
    \item ¿Cuántas funciones biyectivas $f: A \to B$ hay?
    \item ¿Cuántas funciones biyectivas $f: A \to B$ hay tales que $f(\set{1,2,3}) = \set{12,13,14}$?
  \end{enumerate}
\end{enunciado}

Cuando cuento funciones biyectivas, el ejercicio es como reordenar los elementos del conjunto de llegada de todas las
formas posibles. Dado un conjunto $\im(f)$, la cantidad de funciones biyectivas será:
$$
  \#\im(f)!
$$

\begin{enumerate}[label=\roman*)]
  \item
        Hay un total de
        $$
          \cajaResultado{
            7! = 5040
          }
        $$
        funciones biyectivas.

  \item  Del enunciado puedo formarme con $f(\set{1,2,3}) = \set{12,13,14}$ un total de:
        $$
          3!
        $$
        combinaciones para respetar esa condición. Luego para definir el resto de la funciones tengo
        que contar:
        $$
          f(\set{4,5,6,7}) = \set{8,9,10,11}.
        $$
        Esto me da un total de $4!$ combinaciones por cada una de las $3!$ combinaciones encontradas antes.
        Por lo tanto el total de funciones será:
        $$
          \cajaResultado{
            3! \cdot 4! = 144
          }
        $$
\end{enumerate}

\begin{aportes}
  \item \aporte{\dirRepo}{naD GarRaz \github}
\end{aportes}
