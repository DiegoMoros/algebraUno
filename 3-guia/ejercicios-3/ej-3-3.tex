\begin{enunciado}{\ejercicio}
  Dados subconjuntos finitos $A, B, C$ de un conjunto referencial $V$, calcular $\#(A \union B \union C)$\\
  en términos de los cadinales de $A, B, C$ y sus intersecciones.
\end{enunciado}

\begin{align*}
  \#(A \union B \union C) & = \#(A \union (B \union C)                                                                       \\
                          & = \#A + \#(B \union C) - \#(A \inter (B \union C))                                               \\
                          & = \#A + \#B + \#C - \#(B \inter C) - \#[(A \inter B) \union (A \inter C)]                        \\
                          & = \#A + \#B + \#C - \#(B \inter C) - [\#(A \inter B) + \#(A \inter C) - \#(A \inter B \inter C)] \\
                          & = \#A + \#B + \#C - \#(B \inter C) - \#(A \inter B) - \#(A \inter C) + \#(A \inter B \inter C)
\end{align*}

