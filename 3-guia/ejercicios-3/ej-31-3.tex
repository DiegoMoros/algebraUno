\begin{enunciado}{\ejercicio}
  Sean $X = \set{n \en \naturales : n\leq 100}$ y $A = \set{1}$
  ¿Cuántos subconjuntos $B\subseteq X$ satisfacen que el conjunto $A \triangle B$
  tiene a lo sumo 2 elementos?
\end{enunciado}
$$
  ...\\
  \textit{a lo sumo = como mucho = como máximo}\\
$$
$$
  \textit{al menos =  por poco = como mínimo}\\
  ...\\
$$

La diferencia simétrica es la unión de los elementos no comunes
a los conjuntos $A$ y $B$. Si me piden que:

$$
  \#( A \triangle B) \leq 2 \entonces
  \llave{ccl}{
    \blue{1} \en B    & \entonces & 1\leq \#B \leq 3
    \flecha{Conjuntos}[de la forma]
    \llave{lll}{
      \#B = 3 \to \set{\blue{1}; \text{\faIcon[regular]{grimace}}; \text{\faIcon[regular]{smile-wink}} } & \flecha{el 1 está usado, de}[99 números elijo 2] & \binom{99}{2} \\
      \#B = 2 \to \set{\blue{1} ; \text{\faIcon[regular]{grimace}} }                       & \flecha{el 1 está usado, de}[99 números elijo 1] & \binom{99}{1} \\
      \#B = 1 \to \set{\blue{1}}                                                 & \flecha{el 1 está usado, de}[99 números elijo 0] & \binom{99}{0}
    }
    \\
    \\
    \blue{1} \notin B & \entonces & 0\leq\#B \leq 1
    \flecha{Conjuntos}[de la forma]
    \llave{lll}{
      \#B = 1 \to  \set{ \text{\faIcon[regular]{grimace}}} & \flecha{De 99 números para}[elegir \red{$1 \notin B$}. Elijo 1] & \binom{99}{1} \\
      \#B = 0 \to \vacio                         & \flecha{De 99 números para}[elegir \red{$1 \notin B$}. Elijo 0] & \binom{99}{0}
    }
  }
$$

Por último el total de subconjuntos $B \subseteq X$ que cumplen lo pedido sería:
$$
  \cajaResultado{
    \binom{99}{2} + \binom{99}{1} + \binom{99}{0} +\binom{99}{1} + \binom{99}{0} = \binom{99}{2} + 200
  }
$$

\begin{aportes}
  \item \aporte{\dirRepo}{naD GarRaz \github}
\end{aportes}
