\begin{enunciado}{\ejercicio}
  Sean $A = \set{1,2,3,4,5}$ y $B = \set{1,2,3,4,5,6,7,8,9,10,11,12}$. Sea $\mathcal F$ el conjunto de todas las funciones
  $f: A \to B$.
  \begin{enumerate}[label=\roman*)]
    \item  ¿Cuántos elementos tiene le conjunto $\mathcal F$
    \item  ¿Cuántos elementos tiene le conjunto $\set{f \en \mathcal F : 10 \en \im(f) }$
    \item  ¿Cuántos elementos tiene le conjunto $\set{f \en \mathcal F : 10 \en \im(fa)}$
    \item  ¿Cuántos elementos tiene le conjunto $\set{f \en \mathcal F : f(1) \en \set{2,4,6} }$
  \end{enumerate}
\end{enunciado}

Cuando se calcula la cantidad de funciones, haciendo el árbol se puede ver que va a haber
$\#\im(f)$ de funciones que provienen de un elemento del dominio.
Por lo tanto si tengo un conjunto $A_n$ y uno $B_m$, la cantidad de funciones $f : A \to B$ será
de $m^n$

\begin{enumerate}[label=\roman*)]
  \item $\# \mathcal F = 12^5$

  \item $\# \mathcal F = 11^5$

  \item Lo planteo por el complemento, entonces pienso: (la cantidad todal de funciones) - (la cantidad de funciones para las cuales el 10 no pertenece a la imagen).
        En el ejercicio 10.i vimos ¿Cuántos elementos tiene le conjunto $\mathcal F$ sin restricciones y nos dio como resultado: $\# \mathcal F = 12^5$
        Luego en el ejercicio 10.ii vimos ¿Cuántos elementos tiene le conjunto $\set{f \en \mathcal F : 10 \en \im(f) }$ y nos dio como resultado: $\# \mathcal F = 11^5$

        De este modo ya sabemos la cantidad total de funciones y la cantidad total de funciones para los cuales el 10 no pertenece a la imagen. Dando como resueltado:
        $\# \mathcal F = 12^5 - 11^5$

  \item Me dicen que $f(\set1) = \set{2,4,6}$,
        De este modo quiero ver los elementos que cumplen que $f(\set1) = \set{2,4,6}$ de aqui saco:
        $$
          \binom{3}{1} = 3
        $$
        Posteriomente dado que tome un elemento del conjunto A entonces $A = \set{2,3,4,5}$ y su cardinal es 4, luego para cada una de las f que me quedan tengo 12 opciones.
        Entonces la cantidad de funciones generales sin restriciones son:  $\# \mathcal F = 12^4$ como en el ejercicio 10.i
        Finalmente el resultado seria:
        $\# \mathcal F = 12^4 \cdot \underbrace{\tiny 3}_{ f (\set{1})= \set{2,4,6}}$
\end{enumerate}
