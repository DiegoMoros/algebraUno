\begin{enunciado}{\ejercicio}
  Con la palabra $polinomios$,
  \begin{enumerate}[label=\roman*)]
    \item ¿Cuántos anagramas pueden formarse en las que las 2 letras $i$ no estén juntas?
    \item ¿Cuántos anagramas puede formarse en los que la letra $n$ aparezca a la izquierda de la letra $s$ y
          la letra $s$ aparezca a la izquierda de la letra $p$ (no necesariamente una al lado de la otra)?
  \end{enumerate}
\end{enunciado}

\begin{enumerate}[label=\roman*)]
  \item
        Tengo 10 letras, $\set{p,l,n,m,s,o,o,o,i,i}$. \sout{Para que no hayan $"ii"$ calculo $\binom{10}{3} = 120$, pensando que en un conjunto de 3, siempre
          puedo poner las letras $"\underline{i}\, \_ \, \underline{i} "$. Para cada uno de estas 120 configuraciones de la pinta:} \red{Esta forma de hacerlo está mal!}
        $$
          \llave{c c c c c c c c c c c} % hay 9 en total
          {
            \magenta{i} & \magenta{\_} & \magenta{i} & \_           & \_ & \_           & \_ & \_          & \_ & \_                                                                                                                             \\
            \_          & \_           & \magenta{i} & \magenta{\_} & \_ & \_           & \_ & \magenta{i} & \_ & \_          & \to \text{\footnotesize Con las \magenta{$i$} donde están el \magenta{"\_"} tiene 4 posiciones} \\
            \_          & \_           & \_          & \magenta{i}  & \_ & \magenta{\_} & \_ & \_          & \_ & \magenta{i} & \to \text{\footnotesize Con las \magenta{$i$} donde están el \magenta{"\_"} tiene 5 posiciones} \\ \cline{1-10}
            1           & 2            & 3           & 4            & 5  & 6            & 7  & 8           & 9  & 10
          } \\
        $$
        \red{
          Estoy contando de más. La cantidad para que las $i$ no estén juntas es 36... salieron contando a mano
          ($\sumatoria{1}{8} k = 36$)
        }.

        Luego inyectando con las repeticiones de la $"o"$: $36 \cdot \frac{8!}{3!} = 241.920 $

        \bigskip

        \blue{Esta forma es la correcta:}

        Pensando en el complemento:

        Las posiciones que pueden tomar las $ii$ juntas, se calculan a mano enseguida. Habrían en total:
        $$
          \to \ub{\frac{10!}{3! \cdot 2!}}{\universo} - \ub{9 \cdot \frac{8!}{3!}}{\text{complemento}}
          = 241.920
        $$

  \item
        Tengo 10 letras, $\set{p,l,n,m,s,o,o,o,i,i}$. Para que se forme  $"n\dots s \dots p"$ calculo $\binom{10}{3} = 120$, pensando que en un conjunto de 3, siempre
        puedo poner las letras $"\underline{n}\dots \underline s \dots \underline{p} "$. Para cada uno de estas 120 configuraciones de la pinta: \\
        $\llave{c c c c c c c c c c} % hay 9 en total
          {
            \magenta{n} & \magenta{s} & \magenta{p} & \_          & \_ & \_          & \_ & \_          & \_ & \_          \\
            \_          & \_          & \magenta{n} & \magenta{s} & \_ & \_          & \_ & \magenta{p} & \_ & \_          \\
            \_          & \_          & \_          & \magenta{n} & \_ & \magenta{s} & \_ & \_          & \_ & \magenta{p} \\ \hline
            1           & 2           & 3           & 4           & 5  & 6           & 7  & 8           & 9  & 10
          } \to
        $ tengo que rellenar con 7 letras los lugares que sobran, teniendo en cuenta las repeticiones de las $"o"$ y de las $"i"$:
        $\binom{10}{3} \cdot \frac{7!}{3!2!}$
\end{enumerate}

\begin{aportes}
  \item \aporte{\dirRepo}{naD GarRaz \github}
\end{aportes}
