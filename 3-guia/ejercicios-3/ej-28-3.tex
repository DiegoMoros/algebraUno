\begin{enunciado}{\ejercicio}
  En este ejercicio no hace falta usar inducción.
  \begin{enumerate}[label=\roman*)]
    \item Probar que $\sumatoria{k = 0}{n} \binom{n}{k}^2 = \binom{2n}{n}$. \qquad sug: $\binom{n}{k} = \binom{n}{n-k}$.
    \item Probar que $\sumatoria{k = 0}{n} (-1)^k \binom{n}{k} = 0$.
    \item Probar que $\sumatoria{k = 0}{2n} \binom{2n}{k} = 4^n$ y deducir que $\binom{2n}{n} < 4^n$.
    \item Calcular $\sumatoria{k = 0}{2n+1} \binom{2n+1}{k}$ y deducir que $\sumatoria{k=0}{n} \binom{2n+1}{k}$.
  \end{enumerate}
\end{enunciado}

Binomio de Newton: $(x + y)^n = \sumatoria{k=0}{n} \binom{n}{k} x^n y^{n-k}$

\begin{enumerate}[label=\roman*)]
  \item
        \hacer
  \item
        \hacer
  \item
        \hacer
  \item
        \hacer
\end{enumerate}

