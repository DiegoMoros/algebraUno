\ejercicio
¿Cuántos números de 5 cifras distintas se pueden armar usando los dígitos del 1 al 5?
¿ Y usando los dígitos del 1 al 7? ¿ Y usando los dígitos del 1 al 7 de manera que el dígito de las centenas no sea el 2?
\begin{enumerate}[label=\arabic*)]
	\item Hay que usar $\set{1,2,3,4,5}$ y reordenarlos de todas las formas posibles. $5!$

	\item Hay que usar $\set{1,2,3,4,5,6,7}$ y ver de cuantas formas posibles pueden ponerse en 5 lugares:
	      $\llave{c c c c c}{
			      \_ & \_ & \_& \_& \_ \\
			      1 & 2 &3 &4 &5
		      }$, dado que no puedo repetir, a medida que voy llenando los valores, me voy quedando cada vez con menos valores
	      para elegir del conjunto de datos, por lo tanto queda algo así:
	      $\llave{c c c c c}{
			      \#7 & \#6 & \#5 & \#4 &\#3 \\
			      \downarrow &\downarrow &\downarrow &\downarrow &\downarrow\\
			      \_ & \_ & \_& \_& \_ \\
			      1 & 2 &3 &4 &5
		      }\to$ Tengo $7 \cdot 6 \cdot 5 \cdot 4 \cdot 3 = \frac{7!}{2!} $ \red{interpretar?}

	\item Parecido al anterior pero fijo el 2 en el dígito de las centenas:\\
	      $\llave{c c c c c}{
			      \#6 & \#5 & \#4 & \#1 &\#3 \\
			      \downarrow &\downarrow &\downarrow &\downarrow &\downarrow\\
			      \_ & \_ & \_& 2 & \_ \\
			      1 & 2 &3 &4 &5
		      }
		      \to $ Tengo $6 \cdot 5 \cdot 4 \cdot 1 \cdot 3 = \frac{6!}{2!} $ \red{interpretar?}
\end{enumerate}
