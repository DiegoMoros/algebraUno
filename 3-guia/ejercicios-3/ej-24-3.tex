\def\Pedro{\href{https://www.youtube.com/watch?v=RXKabdUBiWM}{Pedro}\xspace}

\begin{enunciado}{\ejercicio}
  Pedro compró 14 unidades de fruta: 6 duraznos, 2 naranjas, 1 banana, 1 pera, 1 higo, 1 kiwi, 1 ciruela, 1 mandarina.
  Su propósito es comer una fruta en cada desayuno y merienda. Determinar de cuántas formas puede organizar sus refrigerios de
  esa semana si no quiere consumir más de una naranja por día.
\end{enunciado}

\Pedro tiene que elegir 2 frutas por día, una para desayunar y otra para merendar.
Pero tiene alto mambo con las \orange{naranjas}. Vamos a ver como calcular que el \textit{infeliz} no tenga una sobredosis de
\textit{ácido ascórbico} de dos formas muy parecidas:

\begin{enumerate}[label=\orange{\faIcon{lemon}$_\arabic*$)}]

  \item

        Hay \orange{14 opciones} donde podría comer una naranja:
        $$
          \begin{array}{|c|c|c|c|c|c|c|c|}
            \hline \rowcolor{green!8}
                            & \text{lunes}              & \text{martes} & \text{miércoles} & \text{jueves} & \text{viernes} & \text{sábado} & \text{domingo} \\ \hline
            \text{desayuno} & $\orange{\faIcon{lemon}}$ & 1             & 3                & 5             & 7              & 9             & 11             \\ \hline
            \text{merienda} & \times\red{!}             & 2             & 4                & 6             & 8              & 10            & 12             \\\hline
          \end{array}
        $$

        Dado que solo quiere consumir una \orange{naranja} por día luego le quedarían \orange{12 opciones} \underline{válidas}, por ejemplo:

        $$
          \begin{array}{|c|c|c|c|c|c|c|c|}
            \hline \rowcolor{green!8}
                            & \text{lunes}              & \text{martes} & \text{miércoles} & \text{jueves}             & \text{viernes} & \text{sábado} & \text{domingo} \\ \hline
            \text{desayuno} & $\orange{\faIcon{lemon}}$ &               &                  &                           &                &               &                \\ \hline
            \text{merienda} & \times\red{!}             &               &                  & $\orange{\faIcon{lemon}}$ &                &               &                \\\hline
          \end{array}
        $$
        Peeeeeero, ojo que ¡Estamos contando el doble! A \Pedro no le importa cual de las 2 \orange{naranjas} come cada día.
        El razonamiento usado sirve para dos objetos \underline{distintos}, las \orange{naranjas} en este ejercicio son \textit{indistinguibles},
        por eso hay que dividir entre 2:
        $$
          \frac{14 \cdot 12}{2} = 84 \llamada1
        $$

  \item
        Otra forma de encarar: Tengo que poner 2 objetos \textit{repetidos, iguales, indistinguibles entre sí} en 14 lugares. Recordar, porque me olvido seguido que el número
        combinatorio no cuenta permutaciones:
        $$
          \binom{14}{2} = \frac{14!}{2! \cdot 12!} = 91
        $$
        Peeeeeero, ojo que ahora estamos contando también cuando \Pedro se clava dos jugosas \orange{naranjas} el mismo día. Dado que hay 7 opciones para ubicar
        las \orange{naranjas} juntas, es decir, poner 2 \orange{naranjas} el mismo día, hay que \textit{restar} esas \red{7} opciones:
        $$
          \binom{14}{2} - \red{7} = 84 \llamada1
        $$
\end{enumerate}

Lo que queda por hacer tiene menos rosca, es cuestión de ir ocupando los lugares que quedan con cada fruta:

Quedan \magenta{12 lugares válidos}. Formas de ubicar 6 duraznos:
$$
  \binom{12}{6} = \frac{12!}{6! \cdot 6!} = \orange{924}
$$
Quedan \magenta{6 lugares válidos}. Formas de ubicar 1 banana:
$$
  \binom{6}{1} = \frac{6!}{1! \cdot 5!} = \orange{6}
$$
Quedan \magenta{5 lugares válidos}. Formas de ubicar 1 pera:
$$
  \binom{5}{1} = \frac{5!}{1! \cdot 4!} = \orange{5}
$$
Quedan \magenta{4 lugares válidos}. Formas de ubicar 1 higo:
$$
  \binom{4}{1} = \frac{4!}{1! \cdot 3!} = \orange{4}
$$
Quedan \magenta{3 lugares válidos}. Formas de ubicar 1 kiwi:
$$
  \binom{3}{1} = \frac{3!}{1! \cdot 2!} = \orange{3}
$$
Quedan \magenta{2 lugares válidos}. Formas de ubicar 1 ciruela:
$$
  \binom{2}{1} = \frac{2!}{1! \cdot 1!} = \orange{2}
$$
Quedan \magenta{1 lugar válido}. Formas de ubicar 1 mandarina:
$$
  \binom{1}{1} = \frac{1!}{1! \cdot 0!} = \orange{1}
$$
Juntando todo, las formas que tendrá de balancear su ingesta frutífera:
$$
  84 \cdot 924 \cdot 6 \cdot 5 \cdot 4 \cdot 3 \cdot 2 \cdot 1 = 55.883.520
$$

\begin{aportes}
  \item \aporte{\dirRepo}{naD GarRaz \github}
\end{aportes}
