\begin{enunciado}{\ejercicio}
  ¿Cuántos números naturales hay menores o iguales que 1000 que no son ni múltiplos
  de 3 ni múltiplos de 5?
\end{enunciado}

Empiezo por buscar los múltiplos de 3 y 5:

\textit{Múltiplos de 3:} Si $x$ es un múltiplo de 3, entonces $x = 3\blue{k}$ con $k \en \enteros$. Y si quiero que esos $x$ estén entre 1 y 1000:
$$
  1\leq x \leq 1000
  \sii
  1\leq 3\blue{k} \leq 1000
  \sii
  \frac{1}{3}\leq \blue{k} \leq \frac{1000}{3}
  \Sii{$k \en \enteros$}
  1 \leq \blue{k} \leq 333
$$
Eso me dice que tengo $333\llamada1$ números que son múltiplos de 3 entre 1 y 1000.

Haciendo lo mismo para los mútiplos de 5 encuentro que hay un total de $200\llamada2$ números múltiplos de 5 entre 1 y 1000.

Es tentador llegado este momento usar los resultados de $\llamada1$ y $\llamada2$ para decir que hay un total de
$$
  1000 - 333 - 200 = 467
$$
números que \underline{no} son múltiplo de 3 ni de 5, peeeeero eso estaría mal, porque:
$$
  \begin{array}{l}
    \text{3$k$ } \to 1,2,\cyan{3},4,5,\cyan{6},7,8,\cyan{9},10,11,\cyan{12},13,14,\red{15},16,17,\cyan{18},19,20,\cyan{21},22,23,\cyan{24},25,26,\cyan{27},\cdots \\
    \text{5$k$ } \to 1,2,3,4,\magenta{5},6,7,8,9,\magenta{10},11,12,13,14,\red{15},16,17,18,19,\magenta{20},21,22,23,24,\magenta{25},26,27,\cdots                                \\
  \end{array}
$$
vamos a estar \textit{restando más números de los que debemos}, por ejemplo el \red{15}, que lo estamos restando \red{¡}2 veces\red{!} una
como múltiplo de 3 y otra como múltiplo 5.

Para corregir esa cuenta calculamos los múltiplos de 15, $x = 15\blue{k}$ y los sumamos al número de antes:
$$
  1\leq x \leq 1000
  \sii
  1\leq 15\blue{k} \leq 1000
  \sii
  \frac{1}{15}\leq \blue{k} \leq \frac{1000}{15}
  \Sii{$k \en \enteros$}
  1 \leq \blue{k} \leq 66
$$
Hay un total de $66 \llamada3$ números múltiplos de 15 entre 1 y 1000

Ahora sí la respuesta al enunciado usando $\llamada1, \llamada2 \ytext \llamada3$:
$$
  \cajaResultado{
    1000 - 333 - 200 + 66 = 533
  }
$$

\bigskip

\parrafoDestacado[\atencion]{
Ahora viene la versión más formal. A mí no me gusta, pero funciona para darle forma a esto.
}

\bigskip

Defino un conjunto referencial:
$$
  V = \set{n \en \naturales : n \leq 1000},$$
y dos conjuntos
$$
  A = \set{n \en V : n \text{ no es múltiplo de } 3}
  \ytext
  B = \set{n \en V : n \text{ no es múltiplo de } 5}.
$$

Búsco calcular $\#(A \inter B)$:

$$
  \#(A \inter B)
  \igual{def}
  \#[V - (A \inter B)^c]
  \igual{\red{!}}
  \#(V - A^c \union B^c) = \#V - \#(A^c \union B^c)
  \igual{\red{!}}
  \#V - [\#A^c + \#B^c - \#(A^c \inter B^c)]\llamada4
$$
Donde
$$
  \begin{array}{l}
    A^c = \set{n \en V : n \text{ es múltiplo de } 3}, \\
    B^c = \set{n \en V : n \text{ es múltiplo de } 5}, \\
    (A^c \inter B^c) = \set{n \en V : n \text{ es múltiplo de } 15}
  \end{array}
$$
Calculo sus cardinales y esas barritas son la \textit{función piso}, que lo que hacen es redondear para abajo, así me queda un número entero:
\begin{multicols}{3}
  \begin{itemize}
    \item $\#A^c = \lfloor \frac{1000}{3} \rfloor = 333$
    \item $\#B^c = \lfloor \frac{1000}{5} \rfloor = 200$
    \item $\#(A^c \inter B^c) = \lfloor \frac{1000}{15} \rfloor = 66$
  \end{itemize}
\end{multicols}

Finalmente:
$$
  \cajaResultado{
  \#(A \inter B)
  \igual{$\llamada4$}
  \#V - [\#A^c + \#B^c - \#(A^c \inter B^c)]
  =
  1000 - (333 + 200 - 66) = 533
  }
$$

\begin{aportes}
  \item \aporte{\dirRepo}{naD GarRaz \github}
\end{aportes}
