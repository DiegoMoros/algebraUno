\begin{enunciado}{\ejercicio}
	Sea $A = \set{f : \set{1,2,3,4} \to \set{1,2,3,4,5,6,7,8}} \text{ tal que $f$ es una función inyectiva}.$

	Sea $\relacion$ la relación de equivalencia en $A$ definida por: $f \relacion g \sisolosi f(1) + f(2) = g(1) + g(2)$.

	Sea $f \en A$ la función definida por $f(n) = n+2$ ¿Cuántos elementos tiene su clase de equivalencia?
\end{enunciado}

Las funciónes $f \en A$ del enunciado tiene solo cuatro valores en su $\dom(f)$ y tiene 8 elementos en el $\cod(f)$.
Me dan una $f(\blue{n})$ es específica:
$$
	f(\blue{n}) = \blue{n} + 2
	\entonces
	\llave{rcl}{
		f(\blue{1}) & = & \blue{1} + 2 = 3 \\
		f(\blue{2}) & = & \blue{2} + 2  = 4 \\
		f(\blue{3}) & = & \blue{3} + 2 = 5  \\
		f(\blue{4}) & = & \blue{4}+ 2 = 6
	}
$$
La cual se relacionará según $\relacion$ con otra funciones $g(n)$ del conjunto $A$ siempre que $g(1) + g(2) = 7$.
Formas de que $g(1) + g(2) = 7$ puede contarse a mano:
$$
	\llave{l}{
		\green{g(1)} + \purple{g(2)} = \green{1} + \purple{6} = 7\\
		\green{g(1)} + \purple{g(2)} = \green{2} + \purple{5} = 7\\
		\green{g(1)} + \purple{g(2)} = \green{3} + \purple{4} = 7\\
		\green{g(1)} + \purple{g(2)} = \green{4} + \purple{3} = 7\\
		\green{g(1)} + \purple{g(2)} = \green{5} + \purple{2} = 7\\
		\green{g(1)} + \purple{g(2)} = \green{6} + \purple{1} = 7
	}
$$
\underline{Un total de 6$\llamada1$ formas}. Un ejemplo de una $g$ tal que $f \relacion g$ sería:
$$
	\llave{rcl}{
		g(1) & = & 3  \to \text{ para cumplir } \relacion\\
		g(2) & = & 4 \to \text{ para cumplir } \relacion\\
		g(3) & = & \set{1, 2, 5, 6, 7, 8} \ot \text{algunos de esos que sea \underline{distinto} a } \magenta{g(4)} \\
		g(4) & = & \set{1, 2, 5, 6, 7, 8} \ot \text{algunos de esos que sea \underline{distinto} a } \magenta{g(3)}
	}
$$
Eso último es para que la función sea inyectiva.
Formas de elegir esos últimos 2 valores para $g(3) \ytext g(4)$:
$$
	6 \cdot 5 = 30 \llamada2
$$

Así concluyendo que la cantidad de funciones $g \en A$ tales que $f \relacion g$ donde $f(n) = n + 2$ será:
$$
	\cajaResultado{
		\llamada1\llamada2 \to	30 \cdot 6 = 180
	}.
$$

\begin{aportes}
	\item \aporte{\dirRepo}{naD GarRaz \github}
\end{aportes}
