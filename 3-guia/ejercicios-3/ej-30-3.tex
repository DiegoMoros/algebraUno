\begin{enunciado}{\ejercicio}
  Sea $X = \set{1,2,3,4,5,6,7,8,9,10}$, y sea $R$ la relación de equivalencia en $\partes(X)$ definida por:
  $$
    A \relacion B \sisolosi A \inter \set{1,2,3} = B \inter \set{1,2,3}.
  $$
  ¿Cuántos conjuntos $B \en \partes(X)$ de exactamente 5 elementos tiene la clase de equivalencia $\overline A $ de $A = \set{1,3,5}$?
\end{enunciado}

Como $A = \set{1,3,5}$:
$$
  A \inter \set{1,2,3} = \set{1,3}.
$$

Los conjuntos $B \en \partes(X)$ que tienen $\#B = 5$ pertenecientes a la clase $\clase{A}$
deberían cumplir:
$$
  B \subseteq \clase{A}
  \entonces
  1 \en B
  \ytext
  3 \en B
  \ytext
  2 \notin B,
$$
donde la última condición es necesaria, dado que si:
$$
  2 \en B
  \entonces
  2 \en (B \inter \set{1,2,3})
  \entonces
  A \norelacion B
$$

Con esta info, los conjuntos $B$ con $\#B = 5$ serán de la forma:
$$
  B = \set{ 1, 3, \text{\faIcon[regular]{smile}}, \text{\faIcon[regular]{grimace}}, \text{\faIcon[regular]{smile-wink}} }
$$
Es decir tengo que agarrar 3 elementos de lo que queda del conjunto $X$.
Los 7 números que quedan para elegir son $\set{4,5,6,7,8,9,10}$ y tengo:
$$
  \binom{7}{3} = 35
$$
formas de agarrarlos.

La cantidad de $B$ que cumplen lo pedido son \cajaResultado{35}

\begin{aportes}
  \item \aporte{\dirRepo}{naD GarRaz \github}
\end{aportes}
