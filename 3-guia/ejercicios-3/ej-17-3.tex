\begin{enunciado}{\ejercicio}

  \begin{enumerate}[label=\roman*)]
    \item ¿Cuántos subconjuntos de 4 elementos tiene el conjunto
          $\set{\foreach \j in {1,...,6}{ \j, }7} $
    \item ¿ Y si se pide que 1 pertenezca al subconjunto?
    \item ¿ Y si se pide que 1 no pertenezca al subconjunto?
    \item ¿ Y si se pide que 1 o 2 pertenezca al subconjunto, pero no simultáneamente los dos?
  \end{enumerate}

\end{enunciado}

El problema de tomar $k$ elementos de un conjunto de $n$ elementos se calcula con
$\binom{n}{k} = \frac{n!}{k!(n-k)!}$

\begin{enumerate}[label=\roman*)]
  \item  $\binom{7}{4} =
          \frac{7!}{4!(7-4)!} =
          \frac{7\cdot \cancel{6} \cdot 5 \cdot \cancel{4!}}{\cancel{4!}(\cancel{3!})} = 35$

  \item $\binom{6}{3} = \frac{6!}{3!\cdot 3!} = 20$.

  \item $\binom{6}{4} = \frac{6!}{4!\cdot 2!} = 15$.

  \item $\binom{5}{3} \cdot 2 = \frac{5!}{3!\cdot 2!} \cdot 2 = 20$
\end{enumerate}
