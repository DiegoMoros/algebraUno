\ejercicio

\begin{enumerate}[label=\roman*)]
    \item ¿Cuántos números de exactamente 4 cifras (no pueden empezar con 0) hay que no contienen al dígito 5?
    \item ¿Cuántos números de exactamente 4 cifras hay que contienen al dígito 7?
\end{enumerate}

\separadorCorto

\begin{enumerate}[label=\roman*)]
    \item

    Como las cifras no pueden ser 5 y la primer cifra no puede empezar con 0, se tiene lo siguiente:

    $
        \begin{array}{r|cccc}
            \text{cifras} & \textunderscore & \textunderscore & \textunderscore & \textunderscore \\
            \text{posibilidades} & 8 & 9 & 9 & 9
        \end{array}
    $
    $\entonces$ hay $8 \cdot 9^3 = 5832$ posibiles números

    \item Para hallar la cantidad de números de 4 cifras que contienen al 7 lo calculo con el complemento, o sea\\

    $\# \text{números de 4 cifras con el 7} = \# \text{números de 4 cifras} - \# \text{números de 4 cifras sin el 7}$\\

    \begin{itemize}
        \item \# números de 4 cifras:\\

        $
            \begin{array}{r|cccc}
                \text{cifras} & \textunderscore & \textunderscore & \textunderscore & \textunderscore \\
                \text{posibilidades} & 9 & 10 & 10 & 10
            \end{array}
        $
        $\entonces$ hay $9 \cdot 10^3 = 9000$ números de 4 cifras

        \item \# números de 4 cifras sin el 7:\\
        En el ítem anterior calculamos la cantidad de números de 4 cifras que no contienen al 5, que es la misma cantidad que números de 4 cifras que no contienen al 7, por lo tanto hay 5832 números posibles.
    \end{itemize}

    Así, $\# \text{números de 4 cifras con el 7} = 9000 - 5832 = 3168$
\end{enumerate}