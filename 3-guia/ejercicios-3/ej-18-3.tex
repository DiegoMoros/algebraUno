\ejercicio
Sea $A = \set{n \in \naturales : n \leq 20}$. Calcular la cantidad de subconjuntos $B \subseteq A$ que cumplen las siguientes condiciones:
\begin{enumerate}[label=\roman*)]
	\item $B$ tiene 10 elementos y contiene exactamente 4 múltiplos de 3.
	\item $B$ tiene 5 elementos y no hay dos elementos de $B$ cuya suma sea impar.
\end{enumerate}

\separadorCorto

El conjunto $A = \set{\foreach \i in {1,...,19}{\i, }20}$\\
\begin{enumerate}[label=\roman*)]
	\item
	      $ \flecha{multiplos}[de 3] C = \set{3, 6, 9, 12, 15, 18}$, agarro 4 elementos del conjunto $C$ y luego 6 de los restantes del conjunto $A$ sin contar
	      el múltiplo de 3 que ya usé.\\
	      $\llave{l}{
			      \binom{6}{4} \cdot \binom{9}{6} = \frac{\cancel{6!}}{4! 2!} \cdot \frac{9!}{\cancel{6!} 3!} \flecha{simplificando} 9 \cdot 4 \cdot 7 \cdot 5 = 1260  \\
			      \red{Verificar y preguntar por la justificación.}
		      }$

	\item
	      La condición de que la suma \textit{no sea impar} implica que todos los elementos deben ser par o todos impar.\\
	      $\llave{l}{
			      \flecha{todos}[pares]   \set{2,4,6,8,10,12,14,16,18, 20} \flecha{10 elementos}[quiero 5] \binom{10}{5} = \frac{10!}{5!\cdot 5!} = 252\\
			      \flecha{todos}[impares] \set{1,3,5,9,11,13,15,17,19} \flecha{9 elementos}[quiero 5] \binom{9}{5} = \frac{9!}{5!\cdot 4!} = 126
		      }$
\end{enumerate}
