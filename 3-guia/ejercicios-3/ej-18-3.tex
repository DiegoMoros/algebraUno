\def\enumeracion{\roman*)}

\begin{enunciado}{\ejercicio}
  Sea $A = \set{n \in \naturales : n \leq 20}$.
  Calcular la cantidad de subconjuntos $B \subseteq A$ que cumplen
  las siguientes condiciones:
  \begin{enumerate}[label=\enumeracion]
    \item $B$ tiene 10 elementos y contiene exactamente 4 múltiplos de 3.

    \item $B$ tiene 5 elementos y no hay dos elementos de $B$ cuya suma sea impar.
  \end{enumerate}

\end{enunciado}

\begin{enumerate}[label=\enumeracion]
  \item
        El conjunto $A$:
        $$
          A = \set{\foreach \k in {1,...,19}{\k, }20}
          \quad \text{con} \quad
          \#A = 20
        $$

        El subconjunto de los múltiplos de 3, lo bautizo como $C$:
        $$
          C = \set{3, 6, 9, 12, 15, 18}
          \quad \text{con} \quad
          \#C = 6
        $$

        Quiero armar un conjunto con 10 elemntos. 4 de esos elementos deben ser de $C$
        ¿Cuántas formas de agarrar 4 elementos de $C$ hay?:
        $$
          \binom{6}{4} \llamada1
        $$

        Como quiero después agarrar \textit{6 elementos más de $A$ para llegar a 10}, y que ninguno esté en $C$,
        Tengo para agarrar:
        $$
          \#(A-C) =  14
        $$
        Formas de agarrar 6 elementos de esos 14 que quedaron:
        $$
          \binom{14}{6} \llamada2
        $$

        Finalmente quedan juntando $\llamada1 \ytext \llamada2$, la cantidad de conjuntos $B$ que satisfacen lo pedido:
        $$
          \cajaResultado{
            \binom{6}{4} \cdot \binom{14}{6} = 45.045
          }
        $$

  \item
        La condición de que la suma \textit{no sea impar} implica que \underline{todos} los elementos
        \underline{deben ser} par o todos impar.

        Conjuntos de números pares e impares, los bautizo, $P$ e $I$ respectivamente:
        $$
          P = \set{2,4,6,8,10,12,14,16,18, 20}
          \quad \text{con} \quad
          \#P = 10
        $$
        $$
          I = \set{1,3,5,7,9,11,13,15,17,19}
          \quad \text{con} \quad
          \#I = 10
        $$

        Ahora para armar los conjuntos $B$ con $\#B = 5$ tengo que tomar solo 5
        elementos ya sea de $P$ o 5 elementos de $I$. Eso es tomar 5 elementos de un conjunto de 10:
        $$
          \binom{10}{5} = 252
        $$
        Entonces puedo armarme 252 conjuntos $B$ con los elementos del conjunto $P$
        y también 252 con los de $I$. Por lo tanto la cantidad total de conjuntos que satisfacen lo pedido:
        $$
          \cajaResultado{252 + 252 = 504}
        $$
\end{enumerate}

% Contribuciones
\begin{aportes}
  %% iconos : \github, \instagram, \tiktok, \linkedin
  %\aporte{url}{nombre icono}
  \item \aporte{\dirRepo}{naD GarRaz \github}
  \item \aporte{https://www.instagram.com/jjean.gibert}{Jean \instagram}
  \item \aporte{https://github.com/nicosaadjian}{Nico S \github}
\end{aportes}
