\begin{enunciado}{\ejercicio}
    \begin{enumerate}[label=\alph*)]
        \item  En el listado de inscripciones de un grupo de 150 estudiantes,
        figuran 83 inscripciones en Analisis y 67 en Algebra. Ademas se
        sabe que 45 de los estudiantes se anotaron en ambas materias.
        ¿Cuantos de los estudiantes no estan inscriptos en ningun curso?
        \item  En un instituto de idiomas donde hay 110 alumnos, las clases
        de ingles tienen 63 inscriptos, las de aleman 30 y las de frances
        50. Se sabe que 7 alumnos estudian los tres idiomas, 30 solo
        estudian ingles, 13 solo estudian aleman y 25 solo estudian frances.
        ¿Cuantos alumnos estudian exactamente dos idiomas? Cuantos
        ingles y aleman pero no frances? Cuantos no estudian ninguno
        de esos idiomas?
      \end{enumerate}
\end{enunciado}

\begin{enumerate}[label=\alph*)]
    \item \hacer
    
    \item Bueno para encarar este ejercicio primero quiero definir
    el interior de un conjunto como SOLO los elementos que pertenecen a ese conjunto,
    creo que no hay una notacion estandar para esto, asi que voy a definir una como
    {\large
    \[
    X_i^\circ := X_i \backslash \bigcup_{i \ne j} X_j
    \]
    }
    Basicamente lo que esto quiere decir es que cada vez que leamos
    por ejemplo $A^\circ$, eso son los elementos que estan SOLO en $A$ y no
    en otro conjunto. \\
    Otra aclaración, aca los conjuntos representan personas, pero lo unico
    que nos interesa del conjunto es su cardinalidad, asi que cada vez
    que me refiera a un conjunto por su nombre, en realidad me estoy refiriendo a su cardinalidad
    Tenemos entonces:
    \[I = 63, A = 30, F = 50\]
    \[I^\circ = 30, A^\circ = 13, F^\circ = 25\]
    \[I\cap A\cap F = 7\]
    \[U = I^\circ + A^\circ + F^\circ + (I \cap A)+ (I \cap F) + (A \cap F) + (I \cap A \cap F) + S = 110\]

    Aclaración: $S$ es el conjunto de personas que no estudian ninguno de los idiomas de interes. \\
    Ahora voy a definir cada conjunto $A$,$I$,$F$ en funcion de sus interiores e intersecciones
    \[I = I^\circ \cup (I \cap A) \cup (I \cap F) \cup (I \cap A \cap F) = 63\]
    \[A = A^\circ \cup (A \cap I) \cup (A \cap F) \cup (I \cap A \cap F) = 30\]
    \[F = F^\circ \cup (F \cap I) \cup (F \cap A) \cup (I \cap A \cap F) = 50\]
    La clave es que como ahora todos estos conjuntos son disjuntos puedo tratar la union como una
    suma comun y llamar a cada grupo de intersecciones como una variable y armar un sistema de ecuaciones:
    \[x = I \cap A\]
    \[y = I \cap F\]
    \[z = A \cap F\]
    Ahora, de las definiciones de mas arriba tengo
    \begin{align*}
        30 + x + y + 7 &= 63 \\
        13 + x + z + 7 &= 30 \\
        25 + y + z + 7 &= 50
    \end{align*}
    De este sistema de ecuaciones obtenemos que $x = 9$, $y = 17$, $z = 1$ \\
    Ya tenemos el ejercicio completado practicamente, veamos lo que nos pedia el enunciado... \\
    Alumnos que estudian solo dos idiomas son las intersecciones entre dos conjuntos. Que justamente
    son nuestras $x$, $y$, $z$. A si que tenemos $\cajaResultado{x + y + z = 27}$. \\
    Los alumnos que estudian ingles y aleman pero no frances son la interseccion entre ingles y aleman, osea $\cajaResultado{x = 9}$
    Para determinar los alumnos que no estudian ningun idioma, tenemos que volver a una de las primeras ecuaciones que teniamos. 
    \[U = I^\circ + A^\circ + F^\circ + (I \cap A)+ (I \cap F) + (A \cap F) + (I \cap A \cap F) + S = 110\]
    Reemplazando obtenemos
    \[U = 30 + 13 + 25 + 9 + 17 + 1 + 7 + S = 110\]
    \[30 + 13 + 25 + 9 + 17 + 1 + 7 + S = 110\]
    \[102 + S = 110\]
    \[\cajaResultado{S = 8}\]


    

\end{enumerate}
