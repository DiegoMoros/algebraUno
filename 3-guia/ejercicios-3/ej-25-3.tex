\begin{enunciado}{\ejercicio}
    Un grupo de 15 amigos organiza un asado en un club al que llegarían en 3 autos distintos (4 por auto) 
    y 3 irían caminando. Sabiendo que solo importa en qué auto están o si van caminando, determinar de 
    cuántas formas pueden viajar si se debe cumplir que al menos uno entre Lucía, María y Diego debe 
    ir en auto, y que Juan y Nicolás tienen que viajar en el mismo auto.
\end{enunciado}

Este ejercicio sale contando por el complemento: primero contamos las formas totales de viajar (con el hecho de que Juan y Nicolás tienen
que viajar en el mismo auto) y luego les restamos las formas en las que Lucía, María y Diego van caminando al mismo tiempo.

\begin{itemize}

    \item \underline{Formas totales}

    Teniendo en cuenta que Juan y Nicolás tienen que viajar en el mismo auto, primero elegimos en que auto van. Para esto, 
    tenemos $\binom{3}{1}$ opciones. \bigskip

    Ahora completemos los dos lugares que faltan en el auto en el que van Juan y Nicolás. Como nos quedan 13 personas por asginar, 
    tenemos $\binom{13}{2}$ opciones. \bigskip

    Completemos ahora otro auto. Como nos quedan 11 personas por asignar, tenemos $\binom{11}{4}$ opciones. Pero como cada auto es distinto, 
    debemos contemplar que vayan en el otro auto que queda, de modo que hay que multiplicar ese número por un 2, 
    pues son dos los autos que quedan. \bigskip

    Por último, debemos llenar el último auto. Como quedan 7 personas sin asignar, tenemos $\binom{7}{4}$ opciones. Respecto a los que van caminando,
    no hay que asignar nada, pues ya quedan asignados al haber llenados todos los autos. \bigskip

    Entonces

    $$
    \text{Formas totales} = 
    \underbrace{\binom{3}{1}}_{\text{Auto para J y N}} \cdot
    \underbrace{\binom{13}{2}}_{\text{Completo auto de J y N}} \cdot
    \underbrace{\binom{11}{4}}_{\text{Completo otro auto}} \cdot
    \underbrace{2}_{\text{Dos autos}} \cdot
    \underbrace{\binom{7}{4}}_{\text{Ultimo auto}} 
    =5405400
    $$


    \item \underline{Formas en las que Lucía, María y Diego van caminando}

    Como los tres que van caminando ya están asignados, solo tenemos que asignar en los autos, que es similar a lo que ya hicimos. \bigskip

    Teniendo en cuenta que Juan y Nicolás tienen que viajar en el mismo auto, primero elegimos en que auto van. Para esto, 
    tenemos $\binom{3}{1}$ opciones. \bigskip

    Ahora completemos los dos lugares que faltan en el auto en el que van Juan y Nicolás. Como nos quedan 10 personas por asginar, 
    tenemos $\binom{10}{2}$ opciones. \bigskip

    Completemos ahora otro auto. Como nos quedan 8 personas por asignar, tenemos $\binom{8}{4}$ opciones. Además, por lo mismo que antes,
    hay que multiplicar por 2. Con el último auto no queda nada por hacer, pues se asignan a las cuatro personas que quedan. \bigskip

    Entonces

    $$
    \text{Formas totales} = 
    \underbrace{\binom{3}{1}}_{\text{Auto para J y N}} \cdot
    \underbrace{\binom{10}{2}}_{\text{Completo auto de J y N}} \cdot
    \underbrace{\binom{8}{4}}_{\text{Completo otro auto}} \cdot
    \underbrace{2}_{\text{Dos autos}} \cdot
    =18900
    $$

\end{itemize}


Entonces, la formas totales con Juan y Nicolás en el mismo auto y con al menos uno entre Lucía, María y Diego en auto son

$$
\binom{3}{1} \cdot \binom{13}{2} \cdot \binom{11}{4} \cdot 2 \cdot \binom{7}{4}
-
\binom{3}{1} \cdot \binom{10}{2} \cdot \binom{8}{4} \cdot 2 
=5405400-18900= \boxed{5386500}
$$

\begin{aportes}
    \item \aporte{https://github.com/Nunezca}{Nunezca \github}
\end{aportes}

   


    