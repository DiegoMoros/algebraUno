\documentclass[12pt,a4paper, spanish]{article}
% Sacar draft para que aparezcan las imagenes.
% Opciones: 12pt, 10pt, 11pt, landscape, twocolumn, fleqn, leqno...
% Opciones de clase: article, report, letter, beamer...

% Paquetes:
% =========
\usepackage[headheight=110pt, top = 2cm, bottom = 2cm, left=1cm, right=1cm]{geometry} %modifico márgenes
\usepackage[T1]{fontenc} % tildes
\usepackage[utf8]{inputenc} % Para poder escribir con tildes en el editor.
\usepackage[english]{babel} % Para cortar las palabras en silabas, creo.
\usepackage[ddmmyyyy]{datetime}
\usepackage{amsmath} % Soporte de mathmatics
\usepackage{amssymb} % fuentes de mathmatics
\usepackage{array} % Para tablas y eso
\usepackage{caption} % Configuracion de figuras y tablas
\usepackage[dvipsnames]{xcolor} % Para colorear el texto: black, blue, brown, cyan, darkgray, gray, green, lightgray, lime, magenta, olive, orange, pink, purple, red, teal, violet, white, yellow.
\usepackage{graphicx} % Necesario para poner imagenes
\usepackage{enumitem} % Cambiar labels y más flexibilidad para el enumerate
\usepackage{multicol} 
\usepackage{tikz} % para graficar
\usepackage{cancel}
\usepackage{titlesec} % para editar titulos y hacer secciones con formato a medida
\usepackage{ulem}
\usepackage{centernot} % tacha cosas
\usepackage{bbding} % símbolos de donde uso FiveStar
\usepackage{skull} % símbolos de donde uso Skull
% \usepackage{lipsum}
\usepackage{soul}

% para hacer los graficos tipo grafos
\usetikzlibrary{shapes,arrows.meta, chains, matrix, calc, trees, positioning, fit}
\usetikzlibrary{external}

% Definiciones y nuevos comandos:
% =============
\def\partes{\mathcal P}
\def\relacion{\,\mathcal{R}\,}
\def\norelacion{\,\cancel{\relacion}\,}
\def\universo{\mathcal U}
\def\reales{\mathbb R}
\def\naturales{\mathbb N}
\def\enteros{\mathbb Z}
\def\complejos{\mathbb C}
\def\i{\text{i}}
\def\vacio{\varnothing}
\def\union{\cup}
\def\inter{\cap}
\def\y{\land}
\def\o{\lor}
\def\neg{\sim}
\def\entonces{\Rightarrow}
\def\sisolosi{\iff}
\def\clase{\overline}


\def\existe{\,\exists\,}
\def\noexiste{\,\nexists\,}
\def\paratodo{\forall}
\def\distinto{\neq}
\def\en{\in}
\def\talque{\;|\;}

% =====
\def\qvq{\text{ quiero ver que }}

%funciones
\def\imagen{\text{Im}}
\def\dominio{$\text{Dom}$}
\def\comp{\circ}
\def\inv{^{-1}}
\def\infinito{\infty}

% Llaves, paréntesis, contenedores
\newcommand{\llave}[2]{ \left\{ \begin{array}{#1} #2 \end{array}\right. }
\newcommand{\llaves}[2]{ \left\{ \begin{array}{#1} #2 \end{array} \right\} }
\newcommand{\matriz}[2]{\left( \begin{array}{#1} #2 \end{array} \right)}
\newcommand{\deter}[2]{\left| \begin{array}{#1} #2 \end{array} \right|}
\newcommand{\lista}[2][(1)]{\begin{enumerate}[\bf #1]\setlength\itemsep{-0.6ex} #2 \end{enumerate}}
\newcommand{\listal}[2][-0.6ex]{\begin{enumerate}[\bf(a)]\setlength\itemsep{#1} #2 \end{enumerate}}

% naturales
\newcommand{\sumatoria}[2]{\sum\limits_{#1}^{#2}}
\newcommand{\productoria}[2]{\prod\limits_{#1}^{#2}}
\newcommand{\kmasuno}[1]{\underbrace{#1}_{k+1\text{-ésimo}}}
\newcommand{\HI}[1]{\underbrace{#1}_{\text{HI}}}

% enteros
\def\divide{\,|\,}
\def\congruente{\, \equiv \,}
\newcommand{\congruencia}[3]{#1 \equiv #2 \;(\text{mod}\;#3)}
\newcommand{\divset}[2]{\mathcal{D}(#1) = \set{#2}}



% =====
% Miscelanea
% =====
\newcommand{\estabien}{{\color{blue} Consultado, está bien. \checkmark}}
\newcommand{\hacer}{{\color{black!30!red}Hacer!}}
\newcommand{\Hacer}{{\color{black!30!red}\Large Hacer!}}

\def\llamadaI{\stackrel{\cyan{$*^1$}}}
\def\llamadaII{\stackrel{\cyan{$*^2$}}}
\def\llamadaIII{\stackrel{\cyan{$*^3$}}}

% separador
\def\separador{\noindent\rule{\linewidth}{0.4pt}\\}
\def\separadorCorto{\noindent\rule{0.5\linewidth}{0.4pt}\\}

% sección ejercicio con su respectivo formato y contador
\newcounter{ejercicio}[subsubsection] % contador que se resetea en cada sección
\renewcommand{\theejercicio}{\arabic{ejercicio}} % el contador es un número arabic
\newcommand{\ejercicio}{%
	\stepcounter{ejercicio}% incremento en uno
	\titleformat{\section}[runin]{\normalfont\bfseries}{\theejercicio}{1em}{}%
	\section*{\noindent\theejercicio. \noindent}%
}

% Colores
\newcommand{\red}[1]{ {\color{red} \text{#1}}}
\newcommand{\green}[1]{ {\color{olive} \text{#1}}}
\newcommand{\blue}[1]{ {\color{blue} \text{#1}}}
\newcommand{\cyan}[1]{ {\color{cyan} \text{#1}}}
\newcommand{\magenta}[1]{ {\color{magenta} \text{#1}}}

% Conjuntos entre llaves
\newcommand{\set}[1] { \left\{ #1 \right\} }
\newcommand{\parentesis}[1] { \left( #1 \right) }

% Stackrel text
\newcommand{\stacktext}[2]{ \stackrel{\text{#1}}{#2} }
\def\eq?{\stackrel{\text{?}}}

% Flecha con texto
\NewDocumentCommand{\flecha}{m o}{%
	\IfNoValueTF{#2}{%
		\xrightarrow[]{\text{#1}}
	}{
		\xrightarrow[\text{#2}]{\text{#1}}
	}
}
 % idem con las definiciones

\begin{document}

\pagestyle{empty} % Para que no muestre el número en pie de página

% Info para armar título.
\title{Práctica 5 de álgebra 1} % título
\author{D. Garraz} % autor
\date{last update: \today} % Cambiar de ser necesario

\maketitle  % para que aprezca el título en el documento

\section{Definiciones y fórmulas útiles}

\def\mcd{(a:b)}

\begin{itemize}
	\item Sea $aX + bY = c \text{ con } a,\, b,\, c \en \enteros,\, a \distinto 0 \y b \distinto 0$ y sea
	      $S = \set{ (x,y) \en \enteros^2 : aX + bY = C }$.\\
	      Entonces $S \distinto \vacio \sisolosi (a:b) \divide c$
	      % \sisolosi
	      %  \ub{A}{\frac{a}{\mcd}} x + \ub{B}{\frac{b}{\mcd}} y = \ub{C}{\frac{c}{(a:b)}} \en \enteros$,
	      % \text{también tiene solución para $x$ y $y$.}

	\item Las soluciones al sistema: $S = \set{(x,y) \en \enteros^2 \text{ con }
			      \llaves{l}{
				      x = x_0 + kb'\\
				      y = y_0 + kb'
			      }
			      , k \en \enteros}
	      $

	\item $\congruencia{aX}{c}{b}$ con $ a,\, b \distinto 0$ tiene solución $\sisolosi \mcd \divide c$ tiene solución $\sisolosi \mcd \divide c$. En ese caso, coprimizando:
\end{itemize}

\textit\underline{Ecuaciones de congruencia}

\begin{itemize}
	\item Algoritmo de solución:
	      \begin{enumerate}[label=\arabic*)]
		      \item reducir $a,\, c$ módulo $m$. Podemos suponer $0 \leq a, c < m$
		      \item tiene solución $\sisolosi (a:m) \divide c$. Y en ese caso coprimizo:
		            \[
			            \congruencia{aX}{c}{m} \sisolosi \congruencia{a'X}{c'}{m},\ \text{ con } a' = \frac{a}{(a:m)},\, m' = \frac{m}{(a:m)} \text{ y } c' = \frac{c}{(a:m)}
		            \]
		      \item   Ahora que $a' \cop m'$, puedo limpiar los factores comunes entre $a'$ y $c'$ (los puedo simplificar)
		            \[
			            \congruencia{a'X}{c'}(m') \sisolosi \congruencia{a''X}{c''}{m'} \text{ con } a'' = \frac{a'}(a':c') \text{ y } c'' = \frac{c'}{(a':c')}
		            \]
		      \item Encuentro una solución particular $X_0$ con $ 0\leq X_0 < m'$ y tenemos
		            \[
			            \congruencia{aX}{c}{m} \sisolosi \congruencia{X}{X_0}{m'}
		            \]
	      \end{enumerate}

\end{itemize}

\textit\underline{Ecuaciones de congruencia}
Sean $m_1,\dots m_n \en \enteros$ \underline{coprimos dos a dos} ($\paratodo i \distinto j$, se tiene $m_i \cop m_j$). \\
Entonces, dados $c_1,\dots, c_n \en \enteros$ cualesquiera, el sistema de ecuaciones de congruencia.
\[
	\llave{c}{
		\congruencia{X}{c_1}{m_1}\\
		\congruencia{X}{c_2}{m_2}\\
		\vdots\\
		\congruencia{X}{c_n}{m_n}
	}
\]

es equivalente al sistema (tienen misma soluciones)

\[
	\congruencia{X}{x_0}{m_1\cdot m_2 \cdots m_n}
\]
para algún $x_0$ con $0\leq x_0 < m_1\cdot m_2 \cdots m_n$

\textit{\underline{Pequeño teorema de Fermat}}
\begin{itemize}
	\item Sea $p$ primo, y sea $a \en \enteros$. Entonces:
	      \begin{enumerate}[label=\arabic*.)]
		      \item $ \congruencia{a^p}{a}{p} $
		      \item $ p \noDivide a \entonces \congruencia{a^{p-1}}{1}{p} $
	      \end{enumerate}
	\item Sea $p$ primo, entonces $ \paratodo a \en \enteros$ tal que $ p \noDivide a$ se tiene:
	      \[
		      \congruencia{a^n}{a^{r_{p-1}(n)}}{p} ,\, \paratodo n \en \naturales
	      \]
	\item Sea $a \en \enteros$ y $p > 0$ primo tal que $\ub{(a:p) = 1}{a \cop p}$, y sea  $d \en \naturales$ con $d \leq p-1$
	      el mínimo tal que:
	      \[
		      \congruencia{a^d}{1}{p} \entonces d \divide (p-1)
	      \]
\end{itemize}
\textit{\underline{Aritmética modular:}}
\begin{itemize}
	\item Sea $n \en \naturales, n\geq2$\\
	      $
		      \enteros/_{n\enteros} = \set{\clase{0},\clase{1},\cdots, \clase{n-1} }\\
		      \clase{a},\clase{b} \en \enteros\_{n\enteros}:
		      \llave{l}{
		      \clase{a} \clase{+} \clase{b} := \overline{r_n^{a+b}} \\
		      \clase{a} \clase{\cdot} \clase{b} := \overline{r_n^{a\cdot b}}
		      }
	      $
\end{itemize}



\subsubsection*{Ejercicios dados en clase:}
\ejercicio Hallar los posibles restos de dividir a $a$ por 70,
sabiendo que $(a^{1081}+ 3a + 17 : 105) = 35$\\

\separadorCorto
$ (\ub{a^{1081}+ 3a + 17}{m} : \ub{105}{3\cdot 5 \cdot 7})  = \ub{35}{5 \cdot 7}
	\flecha{notar}[que]
	\llave{l}{
		5 \divide m\\
		7 \divide m\\
		3 \noDivide m \red{ \tiny $\to$ ¡He aquí la más importante info!}
	}\\
	\llave{l}{
		5 \divide m
		\to \congruencia{a^{1081}+ 3a + \ub{17}{\conga{5} 2}}{0}{5}
		\to
		\llave{ll}{
			\text{si} & 5 \divide a \to \congruencia{2}{0}{5} \magenta{ ningún $a$ } \skull  \\
			\text{si} & 5 \noDivide a
			\flecha{$a^{1081} = a \cdot (a^4)^{270}$}[5 primo y $5 \noDivide a$, fermateo]
			\congruencia{a + 3a + 2}{0}{5} \to \congruencia{a}{2}{5}\\
			& \text{si } 5 \divide m \entonces \boxed{\congruencia{a}{2}{5}}\Tilde
		}
		\\
		7 \divide m
		\to \congruencia{a^{1081}+ 3a + \ub{17}{\conga{7} 3}}{0}{7}
		\to
		\llave{ll}{
			\text{si} & 7 \divide a \to \congruencia{3}{0}{7} \magenta{ ningún $a$ } \skull  \\
			\text{si} & 7 \noDivide a
			\flecha{$a^{1081} = a \cdot (a^6)^{180}$}[7 primo y $7 \noDivide a$, fermateo]
			\congruencia{a + 3a + 3}{0}{7} \to \congruencia{4a}{-3}{7}\\
			& \text{si } 7 \divide m \entonces \boxed{\congruencia{a}{1}{7}}\Tilde
		}
		\\
		3 \noDivide m
		\to \congruencia{a^{1081}+ 3a + \ub{17}{\conga{3} 2}}{0}{3}
		\to
		\llave{ll}{
			\text{si} & 3 \divide a \to \congruencia{2}{0}{3}
			% \flecha{$3\noDivide m$ y $3 \divide a$}
			\llave{l}{
				\text{si } 3 \divide m \to \congruencia{2}{0}{3} \to \text{\magenta{ningún $a$} $\skull$ , pero } \\
				\text{quiero } 3 \noDivide m \entonces \magenta{todo $a$} \text{, pero en esta rama}\\
				3 \divide a \entonces \text{si } 3\noDivide m \to \boxed{\congruencia{a}{0}{3}} \Tilde

			}\\
			\text{si} & 3 \noDivide a
			\flecha{$a^{1081} = a \cdot (a^2)^{540}$}[3 primo y $3 \noDivide a$, fermateo]
			\congruencia{a + 0 + 2}{0}{7} \to \congruencia{a}{-2}{3}\\
			& \text{si } 3 \noDivide m \entonces \llamada{1} \boxed{a \not\equiv 1\ (3)}\Tilde
		}
	}$
Debido a la última condición $\llamada{1}$, el problema se ramifica en 2 sistemas:\\
\begin{minipage}{0.5\textwidth}
	\centering
	$
		\llave{l}{
			\congruencia{a}{2}{5} \\
			\congruencia{a}{1}{7} \\
			\congruencia{a}{0}{3}
		}
		\to \boxed{\congruencia{a}{22}{105} }
	$
\end{minipage}
\begin{minipage}{0.5\textwidth}
	\centering
	$\llave{l}{
			\congruencia{a}{2}{5} \\
			\congruencia{a}{1}{7} \\
			\congruencia{a}{2}{3}
		}
		\to \boxed{\congruencia{a}{92}{105} }
	$
\end{minipage}
Veo que para el conjunto de posibles $a$
$\llaves{c}{
		a = 105k_1 + 22 \\
		o \\
		a = 105k_2 + 92
	}\flecha{calculo}[$\conga{70}$]$
$\congruencia{a}{22}{35} \flecha{quiero los restos}[pedidos del enunciado] r_{70}(a) = \set{22, 57}$,
valores de $a$ que cumplan condición de $r_{70}(a)$




\ejercicio

Sea $a \en \enteros$ tal que $(a^{197} - 26 : 15) = 1$. Hallar los posibles valores de
$(a^{97} - 36 : 135)$\\
\separadorCorto
Tengo que dar posible valores para $(a^{97} - 36 : 135)$. Como $135 = 3^3\cdot 5$,
los posibles valores serán de la forma $3^\alpha \cdot 5^\beta$ con
$ \llaves{l}{
		0 \leq \alpha \leq 3\\
		0 \leq \beta \leq 1\\
	}$ potencialmente $\ub{8}{(3+1)\cdot (1+1)} $posibles valores distintos $\set{1, 3, 9, 27, 5, 15, 45, 135}$\\
Como condición mínima para que no sea siempre $(a^{97} - 36 : 135) = 1$ es que
$\llaves{c}{
		3 \divide a^{97} - 36\\
		\text{o bien,}\\
		5 \divide a^{97} - 36
	}$ si no ocurre ninguna de éstas el MCD será 1.\\

$\llave{l}{
		5 \divide a^{97} - 36 \to a^{97} - 36 \conga{5} a^{97} -1 \conga5 0
		\to
		\llave{ll}{
			\flecha{si}[$5 \divide a$] & \congruencia{-1}{0}{5}
			\flecha{ningún $a$ tal que}[$5 \divide a$ logra que] 5 \divide a^{97} - 36 \\

			\flecha{si}[$5 \noDivide a$] & \congruencia{(a^4)^{24} \cdot a -1 }{0}{5}
			\flecha{5 primo, $5 \noDivide a$}[$\congruencia{a^4}{1}{5}$]
			\boxed{\congruencia{a}{1}{5}}
		}\\
		\text{Se concluye de esta rama que si } 5 \divide a^{97} - 36
		\entonces
		\boxed{\congruencia{a}{1}{5}}\llamada3 \Tilde \\

		\separadorCorto

		3 \divide a^{97} - 36 \to a^{97} - 36 \conga{3} a^{97} \conga3 0
		\to
		\llave{ll}{
			\flecha{si}[$3 \divide a$] & \congruencia{0}{0}{3}
			\flecha{dado que en}[esta rama $a \divide 3$] \boxed{ \congruencia{a}{0}{3} } \\
			\flecha{si}[$3 \noDivide a$] & \congruencia{(a^2)^{48} \cdot a}{0}{3}
			\flecha{3 primo, $3 \noDivide a$}[$\congruencia{a^2}{1}{3} $]
			\boxed{\congruencia{a}{0}{3}}
		}\\
		\text{Se concluye de esta rama que si } 3 \divide a^{97} - 36
		\entonces
		\boxed{\congruencia{a}{0}{3}}\llamada4 \Tilde
	}$\\

Todo muy lindo pero los valores de $a$ están condicionados por  $(a^{197} - 26 : 15) = 1$
una condición que "nada" tiene que ver con el MCD, pero que condiciona los valores que puede
tomar $a$. Como $a^{197} - 26 $ y $15 = 3 \cdot 5$ son coprimos sus factorizaciones en primos
no pueden tener ningún número factor en común, dicho de otra forma:
$\llaves{c}{
		5 \noDivide  a^{197} - 26 \\
		\text{ y } \\
		3 \noDivide  a^{197} - 26
	}$ estudiar estas condiciones me va a restringir los valores de $a$ que puedo usar para construir
los posibles MCDs.\\

$\llave{l}{
		\flecha{\magenta{supongo} $5\divide a^{197} - 26$ y me}[\magenta{quedo con el complemento}]
		a^{197} - 1 \conga{5} 0
		\to
		\llave{l}{
			\flecha{si}[$5 \divide a$]
			\congruencia{-1}{0}{5}
			\flecha{\green{ningún} $a$ tal que}[$5 \divide a$ logra que] 5 \divide a^{197} - 26 \\
			\flecha{el \magenta{complemento de "\green{ningún}" es }}[\magenta{todo $a$} pero como $5\divide a$]
			\congruencia{a}{0}{5}\\
			\separadorCorto
			\flecha{si}[$5 \noDivide a$]
			\congruencia{(a^4)^{49} \cdot a - 1}{0}{5}
			\flecha{5 primo, $5\noDivide a$}[$\congruencia{a^4}{1}{5}$]
			\congruencia{a}{1}{5}\llamada1\\
			\flecha{agarro el}[complemento de $\llamada1$]
			\llave{l}{
				\hbox{\sout{$\congruencia{a}{0}{5}$}} \to \text{rama $5\noDivide a$} \\
				\congruencia{a}{2}{5} \\
				\congruencia{a}{3}{5} \\
				\congruencia{a}{4}{5} \\
			}
		}\\

		\separadorCorto

		\flecha{\magenta{supongo} $3\divide a^{197} - 26$ y me}[\magenta{quedo con el complemento}]
		a^{197} - 2 \conga{3} 0
		\to
		\llave{l}{
			\flecha{si}[$3 \divide a$]
			\congruencia{-2}{0}{3}
			\flecha{\green{ningún} $a$ tal que}[$3 \divide a$ logra que] 3 \divide a^{197} - 26 \\
			\flecha{el \magenta{complemento de "\green{ningún}" es }}[\magenta{todo $a$} pero como $3\divide a$]
			\congruencia{a}{0}{3}\\
			\separadorCorto
			\flecha{si}[$3 \noDivide a$]
			\congruencia{(a^2)^{93} \cdot a - 2}{0}{3}
			\flecha{3 primo, $3\noDivide a$}[$\congruencia{a^2}{2}{3}$]
			\congruencia{a}{2}{3}\llamada2\\
			\flecha{agarro el}[complemento de $\llamada2$]
			\llave{l}{
				\hbox{\sout{$\congruencia{a}{0}{3}$}} \to \text{rama $3\noDivide a$} \\
				\congruencia{a}{1}{3}
			}
		}\\

	}$\\
\text{Se concluye del estudio que si } $5 \noDivide a^{197} - 26$ y
$3 \noDivide a^{197} - 26 \to $
\boxed{
	\llave{cr}{
		\congruencia{a}{0}{5} & \text{ o }\\
		\congruencia{a}{2}{5} & \text{ o }\\
		\congruencia{a}{3}{5} & \text{ o }\\
		\congruencia{a}{4}{5}\\
		\text{ y }&\\
		\congruencia{a}{0}{3} & \text{ o }\\
		\congruencia{a}{1}{3}
	}
}, 8 sistemas $\skull$ \\

Para que el MCD \textit{no sea 1}, se deben satisfacer $\llamada3$ o $\llamada4$, lo cual no ocurre
nunca con $\llamada3$. Eso acota los valores de $(a^{97} - 36 : 135)$ a $\set{1,3,9,27}$\\
De los 4 sistemas útiles:\\
$
	\llave{l}{
		\congruencia{a}{0}{5}\\
		\congruencia{a}{0}{3}
	}
	\flecha{solución} \congruencia{a}{0}{15}\flecha{con $a = 0$}
	\llave{l}{
		0^{97} - 36 \conga{3} 0\\
		0^{97} - 36 \conga{9} 0 \Tilde\\
		0^{97} - 36 \conga{27} - 9 \not\conga{27} 0 \\
	}
	\\
	\llave{l}{
		\congruencia{a}{2}{5}\\
		\congruencia{a}{0}{3}
	}
	\flecha{solución} \congruencia{a}{12}{15}
	\flecha{con $a = 12$}
	\llave{l}{
		12^{97} - 36 \conga{3} 0\\
		12^{97} - 36 \conga{9} 4^{97} \cdot (3^2)^{48} \cdot 3 \conga9 0 \Tilde\\
		12^{97} - 36 \conga{27} 4^{97} \cdot (3^3)^{32} \cdot 3^1 - 9 \conga{27} -9 \not\conga{27} 0 \\
	}
	\\
	\llave{l}{
		\congruencia{a}{3}{5}\\
		\congruencia{a}{0}{3}
	}
	\flecha{solución} \congruencia{a}{3}{15}
	\\
	\llave{l}{
		\congruencia{a}{4}{5}\\
		\congruencia{a}{0}{3}
	}
	\flecha{solución} \congruencia{a}{9}{15}
	\flecha{con $a = 9$}
	\llave{l}{
		9^{97} - 36 \conga{3} 0\\
		9^{97} - 36 \conga{9} 0 \to (9^{97} - 36:135) = 9 \Tilde\\
		9^{97} -36 \conga{27} (3^3)^{64} \cdot 3^2 - 9 \conga{27} -9 \not\conga{27} 0 \\
	}\\
$

\noindent Después de fumarme eso: $(a^{97} - 36 : 135) \en\set{1,9}$, porque todas las soluciones cumplen que:\\
$3 \divide a^{97} \entonces 9 \divide a^{97}\entonces 27 \divide a^{97}$ y como $9 \divide 36$ y $27 \noDivide 36$
siempre el mayor divisor de la expresión va a ser 9.

\red{No estoy súper convencido. ¿Podría ocurrir que en algún caso de 27?}\\

\newpage

%=========================
% Ejercicio guia
%=========================

\section*{Ejercicios de la guía:}
\setcounter{ejercicio}{0} % Reset the custom counter
%1
\ejercicio

%2
\ejercicio
Determinar todos los $(a,b)$ que simultáneamente $4 \divide a,\, 8 \divide b \y 33a + 9b = 120$.

\separadorCorto
Si $(33:9) \divide 120 \entonces 33a + 9b = 120$ tiene solución. $(33:9)=3,\, 3 \divide 120$\Tilde\\
$
	\llave{l}{
		4 \divide a \to a = 4k_1\\
		8 \divide b \to b = 8k_2
	}
	\flecha{meto en}[$33a + 9b = 120$]
	132 k_1 + 72 k_2 = 120
	\flecha{$(132:72) = 12 \divide 120$}[coprimizo]
	11 k_1 + 6 k_2 = 10$ \\

\noindent Busco solución particular con algo parecido a Euclides:\\
$\llaves{l}{
		11 = 6 \cdot 1 + 5 \\
		6 = 5 \cdot 1 + 1 \Tilde \\
	}
	\flecha{escribo al 1 como}[combinación entera de \blue{11} y \blue{6}]
	1 =  \blue{11} \cdot -1 + \blue{6} \cdot -2
	\flecha{solución}[particular]
	10 = \blue{11}\cdot(\ub{-10}{k_1})   + \blue{6} \cdot \ub{20}{k_2} \\
$
Para $11 k_1 + 6 k_2 = 10$ tengo la solución general $(k_1, k_2) = (-10 + (-6)k, 20 + 11k)$ con $k \en \enteros$\\
Pero quiero los valores de $a$ y $b$:\\
La solución general será $\boxed{(a,b) = (4k_1, 8k_2) = (-40 + 24 k, 160 + (-88)k)}$\\

\noindent Otra respuesta con solución a ojo menos falopa, esta recta es la misma que la anterior:\\
$(a,b) = (2+3k, 6-11k)$ con $\congruencia{k}{2}{8} $\\

%3
\ejercicio
Si se sabe que cada unidad de un cierto producto $A$ cuesta $39$ pesos y que cada unidad de un cierto
producto $B$ cuesta 48 pesos, ¿cuántas unidades de cada producto se pueden comprar gastando exactamente
135 pesos?

\separadorCorto

$
	\llave{l}{
		A \geq 0 \y B \geq 0 \text{. Dado que son productos.}\\
		(A:B) = 3 \entonces 39A + 28B = 135
		\flecha{coprimizar}
		13A + 16B = 45\\
		\text{ A ojo } \to (A,B) = (1,2)
	}
$

%4
\ejercicio
Hallar, cuando existan, todas las soluciones de las siguientes ecuaciones de congruencia:

\separadorCorto

\begin{enumerate}[label=\roman*)]
	\item $\congruencia{17X}{3}{11} \flecha{respuesta} \congruencia{X}{6}{11} $\\
	      \red{pasar}

	\item $\congruencia{56X}{28}{35}\\
		      \llave{l}{
			      \congruencia{56X}{28}{35} \sisolosi
			      \congruencia{7X}{21}{35} \stackrel{\red{?}}\sisolosi
			      7X - 35K = 21\\
			      \flecha{a}[ojo] (X,K) = (-2,-1) + q \cdot (-5, 1)\\
			      \congruencia{X}{-2}{5} \sisolosi \congruencia{X}{3}{5} = \set{\dots, -2, 3, 8, \dots, 5q+3}
		      }\\
		      \flecha{respuesta} \congruencia{X}{3}{5} $
	      \red{corroborar}

	\item
	\item $\congruencia{78X}{30}{12126}  \to 78X - 12126 Y = 30 \flecha{$(78:12126) = 6$}[coprimizando] 13X - 2021 Y = 5$\\
	      Busco solución particular con algo parecido a Euclides:\\
	      $\llave{l}{
			      \blue{2021} =  \blue{13} \cdot 155 + 6\\
			      13 =  6 \cdot 2 + 1
		      }\flecha{Escribo al 1 como}[combinación de \blue{13} y\blue{2021}]
		      1 = \blue{13} \cdot 311 + \blue{2021} \cdot (-2)
		      \flecha{quiero}[al 5]
		      5 =  \blue{13}\cdot 1555 + \blue{2021} \cdot (-10)$\\

	      Respuesta: $\boxed{\congruencia{78X}{30}{12126}
			      \stackrel{\red{?}} \sisolosi \congruencia{X}{1555}{2021}}$

\end{enumerate}

%5
\ejercicio
Hallar todos los $(a,b) \en \enteros^2$ tales que $\congruencia{b}{2a}{5}$ y $28a + 10b = 26$.

\separadorCorto
Parecido al \textbf{2.}.\\
$ \congruencia{b}{2a}{5} \sisolosi b = 5k + 2a
	\flecha{meto en}[$28a + 10b = 26$]
	48a + 50k = 26
	\flecha{$(48:59) = 2$}[$2 \divide 26$]
	24a + 25k = 13
	\flecha{a}[ojo]
	\llaves{l}{
		a = -13 + (-25)q\\
		k = 13 + 24 q
	}
$\\
\textit{Let's corroborate:}\\
$b = 5\cdot \ub{(13 + 24q)}{k} + 2\cdot \ub{(-13 + (-25)q)}{a} = 39 + 70q
	\llave{l}{
		b = \congruencia{39 + 70 q}{4}{5}\Tilde  \\
		2a = \congruencia{-26-50q}{-1}{5} \congruente 4\ (5)\Tilde
	}$

%10
\setcounter{ejercicio}{9}
\ejercicio Hallar, cuando existan, todos los enteros $a$ que satisfacen simultáneamente:\\

\begin{enumerate}[label=\roman*)]
	\item
	      $
		      \llave{l}{
			      \llamada1 \congruencia{a}{3}{10} \\
			      \llamada2 \congruencia{a}{2}{7} \\
			      \llamada3 \congruencia{a}{5}{9} \\
		      }
	      $\\
	      El sistema tiene solución dado que 10, 7 y 9 son coprimos dos a dos. Resuelvo:\\
	      $\flecha{Arranco}[ en $\llamada1$]
		      a = 10k + 3 \conga{7}
		      3k + 3 \conga{\llamada2}
		      2\ (7)
		      \flecha{usando que}[$3\cop 7$] \congruencia{k}{2}{7}
		      \to k = 7q + 2.\\
		      \flecha{actualizo}[$a$]
		      a = 10 \cdot \ub{(7q + 2)}{k} + 3 = 70 q + 23 \conga9
		      7q \conga{\llamada3}
		      5\ (9)
		      \flecha{usando que}[$7\cop 9$] \congruencia{q}{0}{9}
		      \to q = 9j\\
		      \flecha{actualizo}[$a$]
		      a = 70 \ub{(9j)}{q} + 23 = 680j + 23 \to \boxed{\congruencia{a}{23}{630}} \Tilde
	      $\\
	      La solución hallada es la que el Teorema chino del Resto me garantiza que tengo en el
	      intervalo $[0, 10\cdot 7 \cdot 9)$

	\item

	\item $
		      \llave{l}{
			      \llamada1 \congruencia{a}{1}{12} \\
			      \llamada2 \congruencia{a}{7}{10} \\
			      \llamada3 \congruencia{a}{4}{9}
		      }
	      $


\end{enumerate}



%15
\setcounter{ejercicio}{14}
\ejercicio
Hallar el resto de la división de $a$ por $p$ en los casos.

\separadorCorto
\begin{enumerate}[label=\roman*)]
	\item $a = 71^{22283},\, p=11$

	      \separadorCorto
	      $a = 71^{22283} = 71^{10 \cdot 2228 +2 + 1} = \ub{(71^{10})^{2228}}{\stackrel{11 \noDivide p}{\congruente} 1^{2228}\ (11)} \cdot 71^2 \cdot 71^1 \congruente 71^3 \ (11) \to \congruencia{a}{5^3}{11} $\Tilde\\
	      Usando corolario con $p$ primo y $p \cop 71$,  $\to \congruencia{71^{22283}}{71^{r_{10}(22283)}}{11} \congruente 71^3 \ (11) \to  \congruencia{a}{5^3}{11} \Tilde$

	\item $a = 5 \cdot 7^{2451} + 3 \cdot 65^{2345} - 23 \cdot 8^{138}, \, p = 13$

	      \separadorCorto
	      $\congruencia{a}{5 \cdot 7^{204 \cdot 12 + 3} + 3 \cdot 8^{11 \cdot 12 + 6}}{13}
		      \to
		      \congruencia{a}{5 \cdot (7^{12})^{204} \cdot 7^3 + 3 \cdot (8^{12})^{11} \cdot 8^6}{13}\\
		      \flecha{$p \noDivide 7$}[$p \noDivide 8$]
		      \congruencia{a}{5\cdot 7^3 + 3 \cdot 8^6 }{13}
		      \to
		      \congruencia{a}{5\cdot(- 6^3 + 3 \cdot 5^5)}{13} $ \red{consultar}
\end{enumerate}

%16
%Macro local
\def\cong97{\stackrel{(97)}\congruente}

\ejercicio
Resolver en $\enteros$ las siguientes eecuaciones de congruencia:

\separadorCorto

\begin{enumerate}[label=\roman*)]
	\item $\congruencia{2^{194}X}{7}{97}$

	      \separadorCorto
	      $\flecha{$2 \cop 97$}  2^{194} = (2^{96})^{2} \cdot 2^2 \congruente 4\ (97)
		      \to
		      \congruencia{4X}{7}{97}
		      \flecha{$\times 24$}
		      \congruencia{-X}{\ub{168}{\cong97 71}}{97}
		      \flecha{$-71 \cong97 26$} \congruencia{X}{26}{97} $\Tilde

	\item $\congruencia{5^{86}X}{3}{89}$\\

	      \separadorCorto
	      \hacer
\end{enumerate}

\setcounter{ejercicio}{19}
\ejercicio
% macro local

%Macro local
\def\sumLocal{\sumatoria{i=1}{1759}}
% fin macro local

Hallar el resto de la división de:
\begin{enumerate}[label=\roman*)]
	\item $43 \cdot 7^{135} + 24^{78} + 11^{222}$ por 70
	\item $\sumLocal i^{42}$ por 56
\end{enumerate}

\separadorCorto

\begin{enumerate}[label=\roman*)]
	\item \hacer

	\item Calcular el resto pedido equivale a resolver la ecuaición de equivalencia:\\
	      $ \congruencia{X}{\sumLocal i^{42}}{56} $ que será aún más simple en la forma:
	      $\llave{l}{
			      \congruencia{X}{\sumLocal i^{42}}{7}\\
			      \congruencia{X}{\sumLocal i^{42}}{8}
		      }\\
		      \text{Primerlo estudio la ecuación de módulo 7: }\\
		      \llave{l}{
			      \congruencia{\sumLocal i^{42}}{X}{7} \llamada1

			      \flecha{7 es primo, uso \magenta{Fermat}}[si $p \noDivide i \to i^{42} = \congruencia{(i^6)^7}{1}{7}$]
			      \sumLocal i^{42} = \sumLocal (i^6)^7
			      \flecha{$251 \cdot 7 + 2 = 1759$}\\
			      \sumLocal (i^6)^7 \conga7 251 \cdot \parentesis{(1^6)^7 + (2^6)^7 + (3^6)^7 + (4^6)^7 + (5^6)^7 + (6^6)^7 + (7^6)^7} + \parentesis{(1^6)^7 + (2^6)^7 + (3^6)^7 + (4^6)^7}\\
			      \sumLocal (i^6)^7 \magenta{$\conga7$ } 251 \cdot \parentesis{1 + 1 + 1 + 1 + 1 + 1 + 0} + \parentesis{1 + 1 + 1 + 1} =
			      251 \cdot  6 + 4 \conga7 3\\
			      \flecha{$\llamada1$ } \boxed{\congruencia{X}{3}{7}}\\
		      }$\\
	      Ahora se labura el módulo 8.\\

	      $\llave{l}
		      {
			      \congruencia{\sumLocal i^{42}}{X}{8}
			      \flecha{8 no es primo}[no uso Fermat]
			      \text{Analizo a mano}
			      \flecha{$219 \cdot 8 + 7 = 1759$}
			      \congruencia{X}{\sumLocal i^{42}}{8} \conga8
			      \\
			      \conga8 219 \cdot \ub{(1^{42} + 2^{42} + 3^{42} + 4^{42} + 5^{42} + 6^{42} + 7^{42} + 0^{42})}{\text{8 términos: } r_8(i^{42}) = (r_8(i))^{42} }
			      + (1^{42} + 2^{42} + 3^{42} + 4^{42} + 5^{42} + 6^{42} + 7^{42}) \\
			      \to
			      \llaves{l}{
				      2^{42} = (2^3)^{14}\conga8 0\\
				      4^{42} = (2^3)^{14} \cdot (2^3)^{14}\conga8 0\\
				      6^{42} = (2^3)^{14} \cdot 3^{42} \conga8 0\\
				      1^{42} = 1\\
				      3^{42} = (3^2)^{21} \conga8 1^{21} = 1\\
				      5^{42} = (5^2)^{21} \conga8 1^{21} = 1\\
				      7^{42} = (7^2)^{21} \conga8 1^{21} = 1
			      }\\
			      \flecha{reemplazo}[esa en]
			      \sumLocal i^{42} \conga8 219 \cdot 4 + 4  = 880 \conga8 0 \to \boxed{\congruencia{X}{0}{8}}\\
		      }$\\
	      El sistema
	      $\llave{l}{
			      \congruencia{X}{3}{7} \\
			      \congruencia{X}{0}{8} \\
		      }$ tiene solución $\congruencia{X}{24}{56}$, por lo tanto el \textit{resto pedido}: \boxed{ r_{56}\parentesis{\sumLocal i^{42}} = 24}
\end{enumerate}




\setcounter{ejercicio}{21}
%22
\ejercicio
Resolver en $\enteros$ la ecuación de congruencia $\congruencia{7X^{45}}{1}{46}$.

\separadorCorto
$\congruencia{7X^{45}}{1}{46}
	\flecha{multiplico por}[13]
	\congruencia{91X^{45}}{13}{46}
	\to \congruencia{X^{45}}{-13}{46}
	\to \congruencia{X^{45}}{33}{46}\\
	\to
	\llave{l}{
		\congruencia{X^{45}}{33}{23} \to\congruencia{X^{45}}{10}{23} 
		\flecha{23 primo y $23\noDivide X$}[$\congruencia{X^{22}}{1}{23}$]
		X^{22} X^{22} X^{1} \conga{23} \congruencia{X}{10}{23}\\

		\congruencia{X^{45}}{10}{2} \to  \congruencia{X^{45}}{0}{2}
		\flecha{$X$ multiplicado por}[si mismo impar veces]
		\congruencia{X}{0}{2}\\
	}
$\\
  La ecuación de congruencia \boxed{\congruencia{X}{10}{46}} cumple las condiciones encontradas.

% 23
\ejercicio Hallar todos los divisores positivos de $25^{70}$ que sean congruentes a 2 módulo 9 y 3 módulo 11.
\separadorCorto
Quiero que ocurra algo así:
$\llave{l}{
		\congruencia{25^{70}}{0}{d} \to \congruencia{5^{140}}{0}{d}  \\
		\congruencia{d}{2}{9} \\
		\congruencia{d}{3}{11}
	}$.
De la primera ecuación queda que el divisor $d = 5^\alpha$ con $\alpha$ compatible
con las otras ecuaciones.
$\to
	\llave{l}{
		\congruencia{5^\alpha}{2}{9} \\
		\congruencia{5^\alpha}{3}{11}
	}$\\

$\to \text{Usaré viejo truco de exponenciales de módulo periódicas:}\\
	\flecha{Busco}[$5^{a} \conga{d} 1$]
	\llave{l}{
		\congruencia{5^\alpha}{2}{9} \\
		\congruencia{5^3}{-1}{9} \flecha{al}[cuadrado] \congruencia{5^6}{1}{9}
		\flecha{$5^\alpha = 5^{6k+r_6(\alpha)} = (\ob{5^6}{\conga9 1})^k 5^{r_6(\alpha)}$}[tabla de restos]
		\begin{array}{|l|l|l|l|l|l|l|}
			\hline
			r_6(\alpha)   & 0 & 1 & 2 & 3 & 4 & 5       \\ \hline
			r_9(5^\alpha) & 1 & 5 & 7 & 8 & 4 & \red{2} \\ \hline
		\end{array}\\
		\flecha{por lo}[tanto] \text{para que } \congruencia{5^\alpha}{2}{9} \entonces \boxed{\congruencia{\alpha}{5}{9}}\Tilde     \\

		\separadorCorto

		\congruencia{5^\alpha}{3}{11}\flecha{a ojo}[$\alpha = 2$] \congruencia{\magenta{$5^2$}}{3}{11} \\
		\flecha{fermateo}[11 es primo, $11\noDivide 5$]
		\congruencia{\green{$5^{10}$}}{1}{11}
		\flecha{noto que tengo otro}[cuando hago $5^{12}$]
		\congruencia{\magenta{$5^2$} \cdot \green{$5^{10}$}}{3}{11}\\
		\flecha{para no perder soluciones de $\congruencia{5^\alpha}{3}{11} $}[tabla de restos por las dudas]
		\begin{array}{|l|l|l|l|l|l|l|l|l|l|l|}
			\hline
			r_{10}(\alpha)   & 0 & 1 & 2       & 3 & 4 & 5        \\ \hline
			r_{11}(5^\alpha) & 1 & 5 & \red{3} & 4 & 9 & \cyan{1} \\ \hline
		\end{array}\\
		\flecha{por lo tanto hay}[periodicidad de 5] \text{para que } \congruencia{5^\alpha}{3}{11}
		\entonces
		\boxed{\congruencia{\alpha}{2}{5}} \Tilde\\
	}$\\
El sistema
$\llave{l}{
		\congruencia{\alpha}{5}{9} \\
		\congruencia{\alpha}{2}{5}
	}$
se resuelve para $\congruencia{\alpha}{32}{45}$ y además $0<\alpha \leq 140$ lo que se
cumple para $\alpha = 45k + 32 =
	\llaves{lcr}{
		32  &  \text{ si } & k = 0\\
		77  &  \text{ si } & k = 1 \\
		122 &  \text{ si } & k = 2
	} \to \boxed{\divsetP{25^{70}}{ 5^{32}, 5^{77}, 5^{122} }}  $

\end{document}
