\documentclass[12pt,a4paper, spanish]{article}
% Sacar draft para que aparezcan las imagenes.
% Opciones: 12pt, 10pt, 11pt, landscape, twocolumn, fleqn, leqno...
% Opciones de clase: article, report, letter, beamer...

% Paquetes:
% =========
\usepackage[headheight=110pt, top = 2cm, bottom = 2cm, left=1cm, right=1cm]{geometry} %modifico márgenes
\usepackage[T1]{fontenc} % tildes
\usepackage[utf8]{inputenc} % Para poder escribir con tildes en el editor.
\usepackage[english]{babel} % Para cortar las palabras en silabas, creo.
\usepackage[ddmmyyyy]{datetime}
\usepackage{amsmath} % Soporte de mathmatics
\usepackage{amssymb} % fuentes de mathmatics
\usepackage{array} % Para tablas y eso
\usepackage{caption} % Configuracion de figuras y tablas
\usepackage[dvipsnames]{xcolor} % Para colorear el texto: black, blue, brown, cyan, darkgray, gray, green, lightgray, lime, magenta, olive, orange, pink, purple, red, teal, violet, white, yellow.
\usepackage{graphicx} % Necesario para poner imagenes
\usepackage{enumitem} % Cambiar labels y más flexibilidad para el enumerate
\usepackage{multicol} 
\usepackage{tikz} % para graficar
\usepackage{cancel}
\usepackage{titlesec} % para editar titulos y hacer secciones con formato a medida
\usepackage{ulem}
\usepackage{centernot} % tacha cosas
\usepackage{bbding} % símbolos de donde uso FiveStar
% \usepackage{lipsum}

% para hacer los graficos tipo grafos
\usetikzlibrary{shapes,arrows.meta, chains, matrix, calc, trees, positioning, fit}
\usetikzlibrary{external}

% Definiciones y nuevos comandos:
% =============
\def\partes{\mathcal P}
\def\relacion{\,\mathcal{R}\,}
\def\norelacion{\,\cancel{\relacion}\,}
\def\universo{\mathcal U}
\def\reales{\mathbb R}
\def\naturales{\mathbb N}
\def\enteros{\mathbb Z}
\def\complejos{\mathbb C}
\def\i{\text{i}}
\def\vacio{\varnothing}
\def\union{\cup}
\def\inter{\cap}
\def\y{\land}
\def\o{\lor}
\def\neg{\sim}
\def\entonces{\Rightarrow}
\def\sisolosi{\iff}
\def\clase{\overline}


\def\existe{\,\exists\,}
\def\noexiste{\,\nexists\,}
\def\paratodo{\forall}
\def\distinto{\neq}
\def\en{\in}
\def\talque{\;|\;}

% =====
\def\qvq{\text{ quiero ver que }}

%funciones
\def\imagen{\text{Im}}
\def\dominio{$\text{Dom}$}
\def\comp{\circ}
\def\inv{^{-1}}
\def\infinito{\infty}

% Llaves, paréntesis, contenedores
\newcommand{\llave}[2]{ \left\{ \begin{array}{#1} #2 \end{array}\right. }
\newcommand{\llaves}[2]{ \left\{ \begin{array}{#1} #2 \end{array} \right\} }
\newcommand{\matriz}[2]{\left( \begin{array}{#1} #2 \end{array} \right)}
\newcommand{\deter}[2]{\left| \begin{array}{#1} #2 \end{array} \right|}
\newcommand{\lista}[2][(1)]{\begin{enumerate}[\bf #1]\setlength\itemsep{-0.6ex} #2 \end{enumerate}}
\newcommand{\listal}[2][-0.6ex]{\begin{enumerate}[\bf(a)]\setlength\itemsep{#1} #2 \end{enumerate}}

% naturales
\newcommand{\sumatoria}[2]{\sum\limits_{#1}^{#2}}
\newcommand{\productoria}[2]{\prod\limits_{#1}^{#2}}
\newcommand{\kmasuno}[1]{\underbrace{#1}_{k+1\text{-ésimo}}}
\newcommand{\HI}[1]{\underbrace{#1}_{\text{HI}}}

% enteros
\def\divide{\,|\,}
\def\congruente{\, \equiv \,}
\newcommand{\congruencia}[3]{#1 \equiv #2 \;(\text{mod}\;#3)}
\newcommand{\divset}[2]{\mathcal{D}(#1) = \set{#2}}



% =====
% Miscelanea
% =====
\newcommand{\estabien}{{\color{blue} Consultado, está bien. \checkmark}}
\newcommand{\hacer}{{\color{black!30!red}Hacer!}}
\newcommand{\Hacer}{{\color{black!30!red}\Large Hacer!}}

\def\llamadaI{\stackrel{\cyan{$*^1$}}}
\def\llamadaII{\stackrel{\cyan{$*^2$}}}
\def\llamadaIII{\stackrel{\cyan{$*^3$}}}

% separador
\def\separador{\noindent\rule{\linewidth}{0.4pt}\\}
\def\separadorCorto{\noindent\rule{0.5\linewidth}{0.4pt}\\}

% sección ejercicio con su respectivo formato y contador
\newcounter{ejercicio}[subsubsection] % contador que se resetea en cada sección
\renewcommand{\theejercicio}{\arabic{ejercicio}} % el contador es un número arabic
\newcommand{\ejercicio}{%
	\stepcounter{ejercicio}% incremento en uno
	\titleformat{\section}[runin]{\normalfont\bfseries}{\theejercicio}{1em}{}%
	\section*{\noindent\theejercicio. \noindent}%
}

% Colores
\newcommand{\red}[1]{ {\color{red} \text{#1}}}
\newcommand{\green}[1]{ {\color{olive} \text{#1}}}
\newcommand{\blue}[1]{ {\color{blue} \text{#1}}}
\newcommand{\cyan}[1]{ {\color{cyan} \text{#1}}}
\newcommand{\magenta}[1]{ {\color{magenta} \text{#1}}}

% Conjuntos entre llaves
\newcommand{\set}[1] { \left\{ #1 \right\} }
\newcommand{\parentesis}[1] { \left( #1 \right) }

% Stackrel text
\newcommand{\stacktext}[2]{ \stackrel{\text{#1}}{#2} }
\def\eq?{\stackrel{\text{?}}}

% Flecha con texto
\NewDocumentCommand{\flecha}{m o}{%
	\IfNoValueTF{#2}{%
		\xrightarrow[]{\text{#1}}
	}{
		\xrightarrow[\text{#2}]{\text{#1}}
	}
}
 % idem con las definiciones

\begin{document}

\pagestyle{empty} % Para que no muestre el número en pie de página

% Info para armar título.
\title{Práctica 5 de álgebra 1} % título
\author{D. Garraz} % autor
\date{last update: \today} % Cambiar de ser necesario

\maketitle  % para que aprezca el título en el documento

\section{Definiciones y fórmulas útiles}

\def\mcd{(a:b)}

\begin{itemize}
	\item Sea $aX + bY = c \text{ con } a,\, b,\, c \en \enteros,\, a \distinto 0 \y b \distinto 0$ y sea
    $S = \set{ (x,y) \en \enteros^2 : aX + bY = C }$.\\
    Entonces $S \distinto \vacio \sisolosi (a:b) \divide c$
	      % \sisolosi
	      %  \ub{A}{\frac{a}{\mcd}} x + \ub{B}{\frac{b}{\mcd}} y = \ub{C}{\frac{c}{(a:b)}} \en \enteros$,
	      % \text{también tiene solución para $x$ y $y$.}

  \item Las socluciones al sistema: $S = \set{(x,y) \en \enteros^2 \text{ con }
    \llaves{l}{
     x = x_0 + kb'\\ 
     y = y_0 + kb' 
    }
    , k \en \enteros}
    $

  \item $\congruencia{aX}{c}{b}$ con $ a,\, b \distinto 0$ tiene solución $\sisolosi \mcd \divide c$ tiene solución $\sisolosi \mcd \divide c$. En ese caso, coprimizando:
\end{itemize}

\textit\underline{Ecuaciones de congruencia}

\begin{itemize}
  \item Algoritmo de solución:
    \begin{enumerate}[label=\arabic*)] 
  \item reducir $a,\, c$ módulo $m$. Podemos suponer $0 \leq a, c < m$
  \item tiene solución $\sisolosi (a:m) \divide c$. Y en ese caso coprimizo:
    \[
      \congruencia{aX}{c}{m} \sisolosi \congruencia{a'X}{c'}{m},\ \text{ con } a' = \frac{a}{(a:m)},\, m' = \frac{m}{(a:m)} \text{ y } c' = \frac{c}{(a:m)}
    \]
    \item   Ahora que $a' \cop m'$, puedo limpiar los factores comunes entre $a'$ y $c'$ (los puedo simplificar)
      \[
        \congruencia{a'X}{c'}(m') \sisolosi \congruencia{a''X}{c''}{m'} \text{ con } a'' = \frac{a'}(a':c') \text{ y } c'' = \frac{c'}{(a':c')}
      \]
    \item Encuentro una solución particular $X_0$ con $ 0\leq X_0 < m'$ y tenemos
      \[
        \congruencia{aX}{c}{m} \sisolosi \congruencia{X}{X_0}{m'}
      \]
     \end{enumerate}

\end{itemize}

\textit\underline{Ecuaciones de congruencia}
Sean $m_1,\dots m_n \en \enteros$ \underline{coprimos dos a dos} ($\paratodo i \distinto j$, se tiene $m_i \cop m_j$). \\
Entonces, dados $c_1,\dots, c_n \en \enteros$ cualesquiera, el sistema de ecuaciones de congruencia. 
\[
  \llave{c}{
  \congruencia{X}{c_1}{m_1}\\
  \congruencia{X}{c_2}{m_2}\\
  \vdots\\
  \congruencia{X}{c_n}{m_n}
  }
\] 

es equivalente al sistema (tienen misma soluciones)

\[
  \congruencia{X}{x_0}{m_1\cdot m_2 \cdots m_n}
\]
para algún $x_0$ con $0\leq x_0 < m_1\cdot m_2 \cdots m_n$

\textit{\underline{Pequeño teorema de Fermat}}
\begin{itemize}
  \item Sea $p$ primo, y sea $a \en \enteros$. Entonces:
    \begin{enumerate}[label=\arabic*.)]
        \item $ \congruencia{a^p}{a}{p} $
        \item $ p \noDivide a \entonces \congruencia{a^{p-1}}{1}{p} $
      \end{enumerate}
  \item Sea $p$ primo, entonces $ \paratodo a \en \enteros$ tal que $ p \noDivide a$ se tiene:
    \[
      \congruencia{a^n}{a^{r_{p-1}(n)}}{p} ,\, \paratodo n \en \naturales 
    \]
  \item Sea $a \en \enteros$ y $p > 0$ primo tal que $(a:p) = 1$, y sea  $d \en \naturales$ con $d \leq p-1$
    el mínimo tal que:
    \[
      \congruencia{a^d}{1}{p} \entonces d \divide (p-1)
    \]  
\end{itemize}




\subsubsection*{Ejercicios dados en clase:}

\newpage

%=========================
% Ejercicio guia
%=========================

\section*{Ejercicios de la guía:}
\setcounter{ejercicio}{0} % Reset the custom counter
%1
\ejercicio
%2
\ejercicio
Determinar todos los $(a,b)$ que simultáneamente $4 \divide a,\, 8 \divide b \y 33a + 9b = 120$.

\separadorCorto
respuesta: $(a,b) = (2+3k, 6-11k)$ con $\congruencia{k}{2}{8}$\\
\red{Pasar ejercicio}

%3
\ejercicio
Si se sabe que cada unidad de un cierto producto $A$ cuesta $39$ pesos y que cada unidad de un cierto
producto $B$ cuesta 48 pesos, ¿cuántas unidades de cada producto se pueden comprar gastando exactamente
135 pesos?

\separadorCorto

$
	\llave{l}{
		A \geq 0 \y B \geq 0 \text{. Dado que son productos.}\\
		(A:B) = 3 \entonces 39A + 28B = 135
		\flecha{coprimizar}
		13A + 16B = 45\\
		\text{ A ojo } \to (A,B) = (1,2)
	}
$

%4
\ejercicio
Hallar, cuando existan, todas las soluciones de las siguientes ecuaciones de congruencia:

\separadorCorto

\begin{enumerate}[label=\roman*)]
	\item $\congruencia{17X}{3}{11} \flecha{respuesta} \congruencia{X}{6}{11} $\\
	      \red{pasar}

	\item $\congruencia{56X}{28}{35}\\
		      \llave{l}{
			      \congruencia{56X}{28}{35} \sisolosi
			      \congruencia{7X}{21}{35} \stackrel{\red{?}}\sisolosi
			      7X - 35K = 21\\
			      \flecha{a}[ojo] (X,K) = (-2,-1) + q \cdot (-5, 1)\\
			      \congruencia{X}{-2}{5} \sisolosi \congruencia{X}{3}{5} = \set{\dots, -2, 3, 8, \dots, 5q+3}
		      }\\
		      \flecha{respuesta} \congruencia{X}{3}{5} $
	      \red{corroborar}
\end{enumerate}

\separadorCorto



\end{document}
